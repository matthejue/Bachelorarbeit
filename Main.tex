%!Tex Root = ../Main.tex
\documentclass{report}
\usepackage[showframe, margin=1.5cm]{geometry}
\usepackage[ngerman]{babel}
\usepackage{lipsum}
\usepackage[parfill, ]{parskip}
\setlength{\parskip}{0.4cm} % space between paragraphs
\usepackage{setspace}
\usepackage{graphicx}
\usepackage[colorlinks=true, allcolors=blue]{hyperref} %hidelinks
\usepackage{csquotes}
\usepackage[style=authortitle]{biblatex}
\addbibresource{./library/library.bib}
\usepackage{pdfpages}
\usepackage{booktabs} % for table rules
\usepackage{tabulary}
% \usepackage{tabularx}
\usepackage{array}
\usepackage{multirow}
\usepackage{amssymb}

% colorbox stuff
\usepackage{tcolorbox}
\usepackage{tikz}
\tcbuselibrary{skins}
\usetikzlibrary{patterns}
\usetikzlibrary{shadings}
\tcbuselibrary{theorems}
\usepackage{cleveref}

\usepackage{xcolor}
\definecolor{PrimaryColor}{HTML}{2A1F59}
\definecolor{SecondaryColor}{HTML}{4D2875}
\definecolor{TertiaryColor}{HTML}{FED32F}
\definecolor{gray75}{gray}{0.75}

\newcommand{\smalltt}[1]{{\small\texttt{#1}}}

% bold with color
\newcommand\colorbold[1]{\textcolor{SecondaryColor}{\textbf{#1}}}

% footer and header
\usepackage{fancyhdr}
\pagestyle{fancy}					% custom headers and footers
	%	footers
	\fancyfoot{}						% clear footers
	\rfoot[]{\thepage}					% right align page numbers

	%	headers
	\fancyheadoffset{1cm}
	\lhead{\nouppercase{\leftmark}} 	%	adding section on left side of the header
\rhead{\nouppercase{\rightmark}} 	%	adding subsection on right side of the header

% spacing after section
\usepackage{titlesec}

% \titlespacing*{\section}
% {0pt}{5.5ex plus 1ex minus .2ex}{4.3ex plus .2ex}
% \titlespacing*{\subsection}
% {0pt}{5.5ex plus 1ex minus .2ex}{4.3ex plus .2ex}
\titlespacing*{\section}{0cm}{*3}{*3}
\titlespacing*{\subsection}{0cm}{*3}{*2}

\usepackage{fix-cm}
\newcommand{\hsp}{\hspace{0pt}}
\titleformat{\chapter}[hang]{\bfseries}{\fontsize{100}{0}\selectfont \textcolor{SecondaryColor}\thechapter\hsp}{0.5cm}{\Huge}[]

\titlespacing*{\chapter}{0cm}{*0}{*4}

\includeonly{
  ./content/Motivation,
  ./content/Einführung,
  ./content/Implementierung,
  ./content/Ergebnisse_und_Ausblick,
  ./content/Appendix
}
 % ./content/Packete_und_Deklarationen.tex

\includeonly{
  ./content/Motivation,
  ./content/Einführung,
  ./content/Implementierung1,
  ./content/Implementierung2,
  ./content/Ergebnisse_und_Ausblick,
  ./content/Appendix
}


\begin{document}
  \sloppy

  \newtcolorbox{titlebox}[1]{skin=enhanced, arc=0mm, boxrule=0mm,
      title style={preaction={fill=PrimaryColor}, pattern=fivepointed stars, pattern color=white, opacity=0.1},
      interior style={preaction={fill=SecondaryColor}, pattern=fivepointed stars, pattern  color=white, opacity=0.3},
      frame style={color=white},
      % segmentation style={black,solid,opacity=0.2,line width=1pt}
      title={#1}
    }

  %!Tex Root = ../Main.tex
% ./Packete_und_Deklarationen.tex
% ./Motivation.tex
% ./Einführung.tex
% ./Implementierung.tex
% ./Ergebnisse_und_Ausblick.tex

\begin{titlepage}
  \vspace{1cm}
  \center
  \textsc{\LARGE Albert Ludwigs Universität Freiburg}\\[0.5cm]
  \textsc{\Large Technische Fakultät}\\[2.0cm]

  \vspace{1cm}

  \begin{titlebox}{\center \huge \bfseries PicoC-Compiler}
    \center
    % \\
    % \tcblower
    {\bfseries \center \LARGE \setstretch{1.1} Übersetzung einer Untermenge von C in den Befehlssatz der RETI-CPU\par}
  \end{titlebox}
    \textsc{\large Bachelorarbeit}\\
    \rule{\linewidth}{0.1mm}\\[0.5cm]

  {\large \emph{Abgabedatum:} 28\textsuperscript{th} April 2022}\\[2.5cm]

  \begin{minipage}{0.45\textwidth}
    \begin{flushleft} \large
      \emph{Author:}\\
      Jürgen Mattheis\\
      \hspace{1cm}\\
      \hspace{1cm}\\
      \hspace{1cm}\\
      \hspace{1cm}
    \end{flushleft}
  \end{minipage}
  ~
  \begin{minipage}{0.45\textwidth}
    \begin{flushright} \large
      \emph{Gutachter:}\\
      Prof. Dr. Scholl\\[0.64cm]
      \emph{Betreung:}\\
      M.Sc. Seufert\\
    \end{flushright}
  \end{minipage}

  \vspace{8.cm}
  \rule{11cm}{0.1mm}\\[0.25cm]
  \large{Eine Bachelorarbeit am Lehrstuhl für}\\
  \large{Betriebssysteme}
\end{titlepage}
                 % ./content/Titlepage.tex
  \newgeometry{margin=2.5cm}
  \setlength{\footskip}{30pt}                 % move pagenumber up and down
  \includepdf[pages=-]{./ErklrungfrdieAbschlussarbeit_unterschrieben.pdf}

  \tableofcontents
  \listoffigures
  \listofcodecaptions
  \listoftables
  % https://tex.stackexchange.com/questions/538528/tcolorbox-newtcbtheorem-index-with-tcolorbox
  \tcblistof[\chapter*]{theorem_list}{Definitionsverzeichnis}
  % https://tex.stackexchange.com/questions/49155/how-can-i-list-items-created-with-the-float-package-in-the-toc
  \listof{floatgrammar}{Grammatikverzeichnis}

  \numberwithin{codecaption}{chapter}

  \newtcbtheorem[list inside={theorem_list},crefname={definition}{definitions}, number within=chapter]{Definition}{Definition}%
  {enhanced, arc=0mm,top=3mm,bottom=3mm, boxrule=0mm, borderline south={1mm}{0pt}{PrimaryColor}, title style={color=PrimaryColor},
  interior style={opacity=0.2, fill=PrimaryColor},
  frame style={color=white}, fonttitle=\bfseries, fontupper=\itshape,
  before upper=\setlength{\parskip}{1em}
  }{def}

  \newtcolorbox{Special_Paragraph}{enhanced, breakable, sharpish corners, notitle, arc=0mm, left=3mm, right=3mm, boxrule=1mm, borderline vertical={1mm}{0pt}{PrimaryColor},
  interior style={fill=SecondaryColor},
  frame style={color=white},
  % https://tex.stackexchange.com/questions/459870/paragraph-breaks-inside-tcolorbox, maybe also parbox=false
  before upper=\setlength{\parskip}{1em}
  }

  % https://tex.stackexchange.com/questions/319355/tcolorbox-breakable-option-not-working
  \newtcbinputlisting{\codebox}[2][]{
  listing file={#2},
  enhanced, colframe=PrimaryColor,colback=SecondaryColor, fonttitle=\bfseries, arc=0mm, #1, listing only, listing engine=minted, minted style=colorful, minted options={fontsize=\small,breaklines,autogobble}, halign title=center, sharpish corners, drop fuzzy shadow, minted options={linenos=false, numbersep=0mm}
  }
% drop fuzzy shadow, drop lifted shadow, listing engine=listings


  \newtcbinputlisting{\numberedcodebox}[2][]{
  listing file={#2},
  enhanced, breakable, colframe=PrimaryColor,colback=white, fonttitle=\bfseries, arc=0mm, #1, listing only, listing engine=minted, minted style=colorful, halign title=center, sharpish corners, drop fuzzy shadow, overlay={\begin{tcbclipinterior}\fill[PrimaryColor] (frame.south west) rectangle ([xshift=5mm]frame.north west);\end{tcbclipinterior}}
  }

  % \newtcbinputlisting{\treebox}[2][]{
  % listing file={#2},
  % enhanced, colframe=PrimaryColor, colback=SecondaryColor, fonttitle=\bfseries, arc=0mm, #1, listing only, , halign title=center, sharpish corners, drop fuzzy shadow
  % }

  % ../Main.tex
% ./Packete_und_Deklarationen.tex
\chapter{Motivation}
\section{PicoC und RETI}
\section{Problemstellung}
\section{Compiler und Interpreter}
              % ./content/Motivation.tex
  %!Tex Root = ../Main.tex
% ./Packete_und_Deklarationen.tex
\chapter{Einführung}
\label{ch:einführung}

\section{Compiler und Interpreter}
\begin{Definition}{Compiler}{compiler}
\end{Definition}
\begin{Definition}{Interpreter}{Interpreter}
% TODO: Bild semantisch gleiche Bedeutung
\end{Definition}
\subsection{T-Diagramme}
\begin{Definition}{T-Diagram}{t_diagram}
\end{Definition}
\section{Grammatiken}
\section{Grundlagen}
\begin{Definition}{Sprache}{Sprache}
\end{Definition}
\begin{Definition}{Chromsky Hierarchie}{chromsky_hierarchie}
\end{Definition}
\begin{Definition}{Grammatik}{grammatik}
\end{Definition}
\begin{Definition}{Reguläre Sprachen}{reguläre_sprachen}
\end{Definition}
\begin{Definition}{Kontextfreie Sprachen}{kontextfreie_sprachen}
\end{Definition}
\subsection{Mehrdeutige Grammatiken}
\begin{Definition}{Ableitungsbaum}{ableitungsbaum}
% TODO: Bild hierfür
\end{Definition}
\begin{Definition}{Mehrdeutige Grammatik}{mehrdeutige_grammatik}
% TODO: (Bild hierfür)
\end{Definition}
\subsection{Präzidenz und Assoziativität}
\begin{Definition}{Assoziativität}{assoziativität}
\end{Definition}
\begin{Definition}{Präzidenz}{präzidenz}
\end{Definition}
% \subsection{Linksrekursiv und Rechtrekursiv}
\section{Lexikalische Analyse}
\label{sec:lexikalische_analyse}

Die \colorbold{Lexikalische Analyse} bildet üblicherweise die erste Ebene innerhalb der \colorbold{Pipe Architektur} bei der Implementierung von Compilern. Die Aufgabe der lexikalischen Analyse ist vereinfacht gesagt, in einem Inputstring, z.B. dem Inhalt einer Datei, welche in \colorbold{UTF-8} codiert ist, Folgen endlicher Symbole (auch \colorbold{Wörter} genannt) zu finden, die bestimmte \colorbold{Pattern} (Definition \ref{def:pattern}) matchen, die durch eine \colorbold{reguläre Grammatik} spezifiziert sind.

\begin{Definition}{Pattern}{pattern}
  \colorbold{Beschreibung} aller möglichen \colorbold{Lexeme} einer Menge $\mathbb{P}_{T}$, die einem bestimmten \colorbold{Token} $T$ zugeordnet werden.
  Die Menge $\mathbb{P}_{T}$ ist eine möglicherweise unendliche Menge von \colorbold{Wörtern}, die sich mit den Regeln einer \colorbold{regulären Grammatik} ${G}_{Lex}$ einer \colorbold{regulären Sprache} ${L}_{Lex}$ beschreiben lassen \footnote{Als Beschreibungswerkzeug können aber auch z.B. reguläre Ausdrücke hergenommen werden.}, die für die Beschreibung eines \colorbold{Tokens} $T$ zuständig sind.\footcite{noauthor_what_nodate}
\end{Definition}

Diese Folgen endlicher Symoble werden auch \colorbold{Lexeme} (Definition \ref{def:lexeme}) genannt.

\begin{Definition}{Lexeme}{lexeme}
  Ein \colorbold{Lexeme} ist ein \colorbold{Wort} aus dem Inputstring, welches das \colorbold{Pattern} für eines der \colorbold{Token} $T$ einer \colorbold{Sprache} ${L}_{Lex}$ matched.
\footcite{noauthor_what_nodate}
\end{Definition}

Diese \colorbold{Lexeme} werden vom \colorbold{Lexer} im \colorbold{Inputstring} identifziert und \colorbold{Tokens} $T$ zugeordnet (Definition \ref{def:lexer}). Die \colorbold{Tokens} sind es, die letztendlich an die \colorbold{Syntaktische Analyse} weitergegeben werden.

\begin{Definition}{Lexer (bzw. Scanner)}{lexer}
  Ein \colorbold{Lexer} ist eine \colorbold{partielle} Funktion \hspace{0.2cm}$lex: \Sigma^{*} \rightharpoonup (N \times W)^{*}$, welche ein \colorbold{Wort} aus $\Sigma^{*}$ auf ein \colorbold{Token} $T$ mit einem \colorbold{Tokennamen} $N$ und einem \colorbold{Tokenwert} $W$ abbildet, falls diese Folge von Symbolen sich unter der \colorbold{regulären Grammatik} ${G}_{Lex}$, der \colorbold{regulären Sprache} ${L_{Lex}}$ abbleiten lässt.\footcite{noauthor_lecture-notes-2021_2022}
\end{Definition}

Ein \colorbold{Lexer} ist im Allgemeinen eine \colorbold{partielle Funktion}, da es Zeichenfolgen geben kann, die kein \colorbold{Pattern} eines \colorbold{Tokens} der Sprache $L_{Lex}$ matchen. In Bezug auf eine Implementierung, wird, wenn der Lexer Teil der Implementierung eines Compilers ist, in diesem Fall eine \colorbold{Fehlermeldung} ausgegeben.

Eine weitere Aufgabe der \colorbold{Lekikalischen Analyse} ist es jegliche für die Weiterverarbeitung unwichtigen Symbole, wie Leerzeichen \,\textvisiblespace\,, Newline \verb|\n|\footnote{In Unix Systemen wird für Newline das ASCII Symbol \colorbold{line feed}, in Windows hingegen die ASCII Symbole \colorbold{carriage return} und \colorbold{line feed} nacheinander verwendet. Das wird aber meist durch die verwendete Porgrammiersprache, die man zur Inplementierung des Lexers nutzt wegabstrahiert.} und Tabs \verb|\t| aus dem Inputstring herauszufiltern. Das geschieht mittels des \colorbold{Lexers}, der allen für die \colorbold{Syntaktische Analyse} unwichtige Zeichen das leere Wort $\epsilon$ zuordnet. Das ist auch im Sinne der Definition, denn $\epsilon \in \Sigma^{*}$.

Nur das, was für die \colorbold{Syntaktische Analyse} wichtig ist, soll weiterverarbeitet werden, alles andere wird herausgefiltert.


% In den  $G_{Lex}$ Grammatiken einiger Programmiersprachen sind allerdinds alle möglichen Zeichenfolgen allein dadurch schon möglich, weil diese Programmiesprachen das Konzept eines \colorbold{Identifiers} o.ä. umsetzen, der alle möglichen Zeichenfolgen abfängt\footnote{Bei der Grammatik von C und auch PicoC ist das allerdings nicht der Fall, weil Identifier dort nicht mit einer Zahl anfangen dürfen.}. Wodurch der Lexer wiederum doch eine linkstotale partielle Funktion ist, die man im Allgemeinen einfach als \colorbold{Funktion} bezeichnet: $lex: \Sigma^{*} \rightarrow (N \times W)^{*}$.

Der Grund warum nicht einfach nur die \colorbold{Lexeme} an die \colorbold{Syntaktische Analyse} weitergegeben werden und der Grund für die Aufteilung des \colorbold{Tokens} in \colorbold{Tokenname} und \colorbold{Tokenwert} ist, weil z.B. die Bezeichner von Variablen, Konstanten und Funktionen beliebige Zeichenfolgen sein können, wie \smalltt{my\_fun}, \smalltt{my\_var} oder \smalltt{my\_const} und es auch viele verschiedenen Zahlen gibt, wie \smalltt{42}, \smalltt{314} oder \smalltt{12}. Die Überbegriffe bzw. Tokennamen für beliebige Bezeichner von Variablen, Konstanten und Funktionen und beliebige Zahlen sind aber trotz allem z.B. \smalltt{Zahl} und \smalltt{Bezeichner}.

Ein \colorbold{Lexeme} ist damit aber nicht das gleiche, wie der \colorbold{Tokenwert}, denn z.B. im Falle von PicoC kann z.B. der Wert $99$ durch zwei verschiedene Literale darstellt werden, einmal als ASCII-Zeichen \smalltt{'c'} und des Weiteren auch in Dezimalschreibweise als \smalltt{99}\footnote{Die Programmiersprache Python erlaubt es z.B. diesern Wert auch mit den Literalen \smalltt{0b1100011} und \smalltt{0x63} darzustellen.}. Der \colorbold{Tokenwert} ist jedoch der letztendliche Wert an sich, unabhängig von der Darstellungsform.

  Die \colorbold{Grammatik} $G_{Lex}$, die zur Beschreibung der Token $T$ einer regulären Sprache $L_{Lex}$ verwendet wird, ist üblicherweise \colorbold{regulär}, da ein typischer \colorbold{Lexer} immer nur \colorbold{ein Symbol} vorausschaut\footnote{Man nennt das auch einem \colorbold{Lookahead} von $1$}, unabhängig davon, was für Symbole davor aufgetaucht sind. Die übliche Implementierung eines \colorbold{Lexers} merkt sich nicht, was für Symbole davor aufgetaucht sind.

% TODO: später erwähnen, dass alle Regeln der Grammatik G_lex eine reguläre Form haben, was der Beweis ist

\begin{Special_Paragraph}
  Um Verwirrung verzubäugen ist es wichtig folgende Unterscheidung hervorzuheben: Wenn von \colorbold{Symbolen} die Rede ist, so werden in der \colorbold{Lexikalischen Analyse}, der \colorbold{Syntaktische Analyse} und der \colorbold{Code Generierung}, auf diesen verschiedenen Ebenen unterschiedliche Konzepte als Symbole bezeichnet.

  In der Lexikalischen Analyse sind einzelne \colorbold{Zeichen eines Zeichensatzes} die Symbole.

  In der Syntaktischen Analyse sind die \colorbold{Tokennamen} die Symbole.

  In der Code Generierung sind die \colorbold{Bezeichner von Variablen, Konstanten und Funktionnen} die Symbole\footnote{Das ist der Grund, warum die Tabelle, in der Informationen zu Identifiern gespeichert werden aus Kapitel \ref{ch:implementierung} Symboltabelle genannt wird.}.
\end{Special_Paragraph}

\begin{Definition}{Literal}{literal}
  Eine von möglicherweise vielen weiteren \colorbold{Darstellungsformen} für ein und denselben \colorbold{Wert}.
\end{Definition}

% TODO: zusammenfassendes Bild

\section{Syntaktische Analyse}

In der \colorbold{Syntaktischen Analyse} ist für einige Sprachen eine \colorbold{Kontextfreie Grammatik} $G_{Parse}$ notwendig, um die diese Sprache zu beschreiben, da viele Programmiersprachen z.B. für \colorbold{Funktionsaufrufe} \verb|fun(arg)| und \colorbold{Codeblöcke} \verb|if(1){}| syntaktische Mittel verwenden, die es notwendig machen sich zu merken wieviele öffnende Klammern \verb|'('| bzw. öffnende geschweifte Klammern \verb|'{'| es momentan gibt, die noch nicht durch eine enstsprechende schließende Klammer \verb|')'| bzw. schließende geschweifte Klammer \verb|'}'| geschlossen wurden.

% TODO: später erwähnen, dass alle Regeln der Grammatik G_parse eine kontexfreie Form haben, was der Beweis ist

Die vom \colorbold{Lexer} im Inputstring identifizierten \colorbold{Token} werden in der \colorbold{Syntaktischen Analyse} vom \colorbold{Parser} (Definition \ref{def:parser}) als \colorbold{Wegweiser} verwendet, da je nachdem, in welcher Reihenfolge die \colorbold{Token} auftauchen, dies einer anderen Ableitung nach der \colorbold{Grammatik} $G_{Parse}$ entspricht. Dabei wird in der Grammatik nach dem \colorbold{Tokennamen} unterschieden und nicht nach dem Tokenwert, da es nur von Interesse ist, ob an einer bestimmten Stelle z.B. eine \verb|Zahl| steht und nicht, welchen konkretten Wert diese \verb|Zahl| hat. Der \colorbold{Tokenwert} ist erst später in der \colorbold{Code Generierung} relevant.

Die \colorbold{Syntax}, in welcher der Inputstring aufgeschrieben ist, wird auch als \colorbold{konkrette Syntax} (Definition \ref{def:konkrette_syntax}) bezeichnet.

\begin{Definition}{Parser}{parser}
  Ein Programm, dass eine \colorbold{Eingabe} in eine für die \colorbold{Weiterverbeitung} taugliche Form bringt.
\end{Definition}

In Bezug auf Compilerbau hat ein \colorbold{Parser} meist die Aufgabe aus einem \colorbold{Inputstring} einen \colorbold{Derivation Tree} (Definition \ref{def:derivation_tree}) zu generieren.

\begin{Special_Paragraph}
  An dieser Stelle könnte möglicherweise eine Begriffsverwirrung enstehen, ob ein \colorbold{Lexer} nach der obigen Definition nicht auch ein \colorbold{Parser} ist.

  In Bezug auf Compilerbau ist ein \colorbold{Lexer} ein Teil eines Parsers und der Parser vereinigt sowohl die \colorbold{Lexikalische Analyse}, als auch einen Teil der \colorbold{Syntaktischen Analyse} in sich, aber für sich isoliert betrachtet ist ein Lexer nach Definition \ref{def:parser} ebenfalls ein Parser. Aber im Compilerbau überwiegt seine Funktionalität, dass er den Inputstring lexikalisch weiterverarbeitet, um ihn als Lexer zu bezeichnen, der Teil eines Parsers ist.
\end{Special_Paragraph}

Ein \colorbold{Parser} ist aber auch ein erweiterter \colorbold{Recognizer}, denn einmal hat der \colorbold{Parser} die Aufgabe eines \colorbold{Recognizers} (Definition \ref{def:recognizer}), nämlich zu überprüfen, ob ein Inputstring sich den Regeln der Grammatik $G_Parse$ ableiten lässt und ein \colorbold{Wort} der Sprache $L_{Parse}$ ist.

\begin{Definition}{Recognizer}{recognizer}

\end{Definition}

\begin{Definition}{Konkrette Syntax}{konkrette_syntax}
  \colorbold{Syntax} einer \colorbold{Sprache}, die durch die \colorbold{Grammatiken} $G_{Lex}$ und $G_{Parse}$ zusammengenommen beschrieben wird.

  Ein \colorbold{Programm} in seiner \colorbold{Textrepräsentation}, wie es in einer Textdatei nach den Regeln der \colorbold{Grammatiken} $G_{Lex}$ und $G_{Parse}$ abgeleitet steht, bevor man es kompiliert, ist in \colorbold{konkretter Syntax} aufgeschrieben.
\end{Definition}

\begin{Definition}{Derivation Tree (bzw. Parse Tree)}{derivation_tree}
\end{Definition}

\begin{Definition}{Abstrakte Syntax}{abstrakte_syntax}

\end{Definition}

\begin{Definition}{Abstrakte Syntax Tree}{abstrakte_syntax_tree}
\end{Definition}

\begin{Definition}{Transformer}{transformer}
\end{Definition}

\begin{Definition}{Visitor}{visitor}
\end{Definition}

% TODO: zusammenfassendes Bild
\section{Code Generierung}
\begin{Definition}{Pass}{pass}
% TODO: Bild semantisch gleiche Bedeutung
% TODO: auf T-Diagramme zurückkommen
\end{Definition}
\section{Fehlermeldungen}
\begin{Definition}{Fehlermeldung}{fehlermeldung}
\end{Definition}
% Kategorien von Fehlermeldungen
              % ./content/Einführung.tex
  %!Tex Root = ../Main.tex
% ./Packete_und_Deklarationen.tex
% ./Titlepage.tex
% ./Motivation.tex
% ./Einführung.tex
% ./Implementierung2.tex
% ./Ergebnisse_und_Ausblick.tex

\chapter{Implementierung}
\label{ch:implementierung}
\section{Architektur}
% Unterscheid zur Architektur aus dem Bachelorprojekt

% Cross Compiler
% https://tex.stackexchange.com/questions/8625/force-figure-placement-in-text
\begin{figure}[H]
  \centering
  \includegraphics[width=0.5\linewidth]{./figures/summarized_cross_compiler.png}
  \caption{Cross-Compiler Kompiliervorgang ausgeschrieben}
\end{figure}

\begin{figure}[H]
  \centering
  \includegraphics[width=0.33\linewidth]{./figures/compiliervorang_mit_machiene.png}
  \caption{Cross-Compiler Kompiliervorgang Kurzform}
\end{figure}

\begin{figure}[H]
  \centering
  \includegraphics[width=\linewidth]{./figures/passes.png}
  \caption{Architektur mit allen Passes ausgeschrieben}
\end{figure}

\section{Lexikalische Analyse}
\subsection{Verwendung von Lark}
\numberwithin{floatgrammar}{section}

\label{sec:lex_analyse_verwendung_von_lark}
% ./concrete_syntax_picoc.lark
\begin{grammar}[Konkrette Syntax des Lexers][H][gr:concrete_syntax_lex]
  \toprule
  \firstcase{COMMENT}{\dq //\dq\enspace /[{\wedge}\backslash n]{*}/\gralt \dq {/*}\dq\enspace  /(.\mid \setminus n)*?/\enspace \dq {*/}\dq }{L\_Comment}
  \firstcase{RETI\_COMMENT.2}{\dq {//}\dq \dq \text{\textvisiblespace} \dq ? \dq \#\dq /[\wedge\backslash n]{*}/}{}
  \midrule
  \firstcase{DIG\_NO\_0}{\dq 1\dq \gralt \dq 2\dq \gralt \dq 3\dq \gralt \dq 4\dq \gralt \dq 5\dq}{L\_Arith}
  \otherform{\dq 6\dq \gralt \dq 7\dq \gralt \dq 8\dq \gralt \dq 9\dq}{}
  \firstcase{DIG\_WITH\_0}{\dq 0\dq \gralt DIG\_NO\_0}{}
  \firstcase{NUM}{\dq 0\dq \gralt DIG\_NO\_0 DIG\_WITH\_0*}{}
  \firstcase{ASCII\_CHAR}{\dq\text{\textvisiblespace} \dq ..\dq \sim\dq }{}
  \firstcase{CHAR}{\dq '\dq ASCII\_CHAR\dq '\dq }{}
  \firstcase{FILENAME}{ASCII\_CHAR+\dq .picoc\dq }{}
  \firstcase{LETTER}{\dq {a}\dq ..\dq {z}\dq \gralt \dq {A}\dq ..\dq {Z}\dq}{}
  \firstcase{NAME}{(LETTER\gralt \dq \_\dq )}{}
  & & (LETTER\gralt DIG\_WITH\_0\gralt \dq \_\dq )* & \\
  \firstcase{name}{NAME\gralt INT\_NAME\gralt CHAR\_NAME}{}
  \otherform{VOID\_NAME}{}
  \firstcase{NOT}{\dq \sim\dq }{}
  \firstcase{REF\_AND}{\dq \&\dq }{}
  \firstcase{un\_op}{SUB\_MINUS\gralt LOGIC\_NOT\gralt NOT}{}
  \otherform{MUL\_DEREF\_PNTR \gralt REF\_AND}{}
  \firstcase{MUL\_DEREF\_PNTR}{\dq {*}\dq }{}
  \firstcase{DIV}{\dq /\dq }{}
  \firstcase{MOD}{\dq \%\dq }{}
  \firstcase{prec1\_op}{MUL\_DEREF\_PNTR\gralt DIV\gralt MOD}{}
  \firstcase{ADD}{\dq {+}\dq }{}
  \firstcase{SUB\_MINUS}{\dq {-}\dq }{}
  \firstcase{prec2\_op}{ADD\gralt SUB\_MINUS}{}
  \midrule
  \firstcase{LT}{\dq {<}\dq }{L\_Logic}
  \firstcase{LTE}{\dq {<=}\dq }{}
  \firstcase{GT}{\dq {>}\dq }{}
  \firstcase{GTE}{\dq {>=}\dq }{}
  \firstcase{rel\_op}{LT\gralt LTE\gralt GT\gralt GTE}{}
  \firstcase{EQ}{\dq {==}\dq }{}
  \firstcase{NEQ}{\dq {!=}\dq }{}
  \firstcase{eq\_op}{EQ\gralt NEQ}{}
  \firstcase{LOGIC\_NOT}{\dq !\dq }{}
  \midrule
  \firstcase{INT\_DT.2}{\dq int\dq }{}
  \firstcase{INT\_NAME.3}{\dq int\dq\enspace (LETTER\gralt DIG\_WITH\_0\gralt \dq \_\dq )+}{L\_Assign\_Alloc}
  \firstcase{CHAR\_DT.2}{\dq char\dq }{}
  \firstcase{CHAR\_NAME.3}{\dq char\dq\enspace (LETTER\gralt DIG\_WITH\_0\gralt \dq \_\dq )+}{}
  \firstcase{VOID\_DT.2}{\dq void\dq }{}
  \firstcase{VOID\_NAME.3}{\dq void\dq\enspace (LETTER\gralt DIG\_WITH\_0\gralt \dq \_\dq )+}{}
  \firstcase{prim\_dt}{INT\_DT\gralt CHAR\_DT\gralt VOID\_DT}{}
  \bottomrule
\end{grammar}

\begin{grammar}[\(\lambda\) calculus syntax][p][gr:ex1]
	\firstcase{T}{\nonterm{V}}{Variable}
	\otherform{(\nonterm{T}\ \nonterm{T})}{Application}
	\otherform{\lambda \nonterm{V}\cdot\nonterm{T}}{Abstraction}
	\firstcase{V}{x, y, \dots}{Variables}
\end{grammar}
\begin{grammar}[Advanced capabilities of \texttt{grammar.sty}][p][gr:ex2]
	\firstcase{A}{\nonterm{T} \gralt \nonterm{V}}{Multiple option on a single line}
	\highlight
	\otherform{\nonterm{A}}{Highlighted form}
	\downplay
	\otherform{\nonterm{B}\gralt \nonterm{C}}{Downplayed form}
	\otherform{\lochighlight{\nonterm{A}} \gralt \nonterm{B}}{Emphasize part of the line}
\end{grammar}

% erwähnen, dass in Lark die Grammatiken L_Lex und L_Parse gemischt sind
% EBNF erwähnen
% (erwähnen, dass finalle Grammatik im Appendix)
\subsection{Basic Parser}
\section{Syntaktische Analyse}
\subsection{Verwendung von Lark}
% ./concrete_syntax_picoc.lark
% https://tex.stackexchange.com/questions/851/removing-spaces-between-words-in-math-mode
In \ref{gr:concrete_syntax_parser}

\begin{grammar}[Konkrette Syntax des Parsers, Teil 1][H][gr:concrete_syntax_parser]
  \toprule
	\downplay
  \firstcase{prim\_exp}{name\gralt NUM\gralt CHAR\gralt "("logic\_or")"}{L\_Arith +}
	\downplay
  \firstcase{post\_exp}{array\_subscr\gralt struct\_attr\gralt fun\_call}{L\_Array +}
	\downplay
  \otherform{input\_exp\gralt print\_exp\gralt prim\_exp}{L\_Pntr +}
	\downplay
  \firstcase{un\_exp}{un\_op un\_exp\gralt post\_exp}{L\_Struct + L\_Fun}
  \midrule
	\downplay
  \firstcase{input\_exp}{\dq input\dq\dq(\dq\dq)\dq}{L\_Arith}
	\downplay
  \firstcase{print\_exp}{\dq print\dq\dq(\dq logic\_or\dq)\dq}{}
	\downplay
  \firstcase{arith\_prec1}{arith\_prec1\enspace prec1\_op\enspace un\_exp\gralt un\_exp}{}
	\downplay
  \firstcase{arith\_prec2}{arith\_prec2\enspace prec2\_op\enspace arith\_prec1\gralt arith\_prec1}{}
	\downplay
  \firstcase{arith\_and}{arith\_and\enspace \dq\&\dq\enspace arith\_prec2\gralt arith\_prec2}{}
	\downplay
  \firstcase{arith\_oplus}{arith\_oplus\enspace \dq {\wedge{}}\dq\enspace arith\_and\gralt arith\_and}{}
	\downplay
  \firstcase{arith\_or}{arith\_or\enspace \dq{\mid} \dq\enspace arith\_oplus\gralt arith\_oplus}{}
  \midrule
  \downplay
  \firstcase{rel\_exp}{rel\_exp\enspace rel\_op\enspace arith\_or\gralt arith\_or}{L\_Logic}
  \downplay
  \firstcase{eq\_exp}{eq\_exp\enspace eq\_op rel\_exp\gralt rel\_exp}{}
  \downplay
  \firstcase{logic\_and}{logic\_and\enspace \dq{\&\&}\dq\enspace eq\_exp\gralt eq\_exp}{}
  \downplay
  \firstcase{logic\_or}{logic\_or\enspace \dq{\mid\mid}\dq\enspace logic\_and\gralt logic\_and}{}
  \midrule
	\downplay
  \firstcase{type\_spec}{prim\_dt\gralt struct\_spec}{L\_Assign\_Alloc}
	\downplay
  \firstcase{alloc}{type\_spec\enspace pntr\_decl}{}
	\downplay
  \firstcase{assign\_stmt}{un\_exp\enspace \dq {=}\dq\enspace logic\_or\dq ;\dq }{}
  \firstcase{initializer}{logic\_or\gralt array\_init\gralt struct\_init}{}
	\downplay
  \firstcase{init\_stmt}{alloc\enspace \dq {=}\dq\enspace initializer\dq ;\dq }{}
	\downplay
  \firstcase{const\_init\_stmt}{\dq const\dq\enspace type\_spec\enspace name\enspace \dq {=}\dq\enspace NUM\dq ;\dq }{}
  \midrule
  \firstcase{pntr\_deg}{\dq {*}\dq *}{L\_Pntr}
  \firstcase{pntr\_decl}{pntr\_deg\enspace array\_decl\gralt array\_decl}{}
  \midrule
  \firstcase{array\_dims}{(\dq [\dq NUM\dq ]\dq )*}{L\_Array}
  \firstcase{array\_decl}{name\enspace array\_dims\gralt \dq (\dq pntr\_decl\dq )\dq  array\_dims}{}
  \firstcase{array\_init}{\dq \{\dq initializer(\dq ,\dq\enspace initializer)*\dq \}\dq }{}
  \firstcase{array\_subscr}{post\_exp\dq [\dq logic\_or\dq ]\dq }{}
  \midrule
  \firstcase{struct\_spec}{\dq struct\dq\enspace name}{L\_Struct}
  \firstcase{struct\_params}{(alloc\dq ;\dq )+}{}
  \firstcase{struct\_decl}{\dq struct\dq\enspace name\enspace \dq \{\dq struct\_params\dq \}\dq }{}
  \firstcase{struct\_init}{\dq \{\dq \dq .\dq name\dq {=}\dq initializer}{}
  & & (\dq ,\dq\enspace \dq .\dq name\dq {=}\dq initializer)*\dq \}\dq & \\
  \firstcase{struct\_attr}{post\_exp\dq .\dq name}{}
  \midrule
	\downplay
  \firstcase{if\_stmt}{\dq if\dq \dq (\dq logic\_or\dq )\dq\enspace exec\_part}{L\_If\_Else}
	\downplay
  \firstcase{if\_else\_stmt}{\dq if\dq \dq (\dq logic\_or\dq )\dq\enspace exec\_part\enspace \dq else\dq\enspace exec\_part}{}
  \midrule
	\downplay
  \firstcase{while\_stmt}{\dq while\dq \dq (\dq logic\_or\dq )\dq\enspace exec\_part}{L\_Loop}
	\downplay
  \firstcase{do\_while\_stmt}{\dq do\dq\enspace exec\_part\enspace \dq while \dq \dq (\dq logic\_or\dq )\dq \dq ;\dq }{}
  \bottomrule
\end{grammar}

\begin{grammar}[Konkrette Syntax des Parsers, Teil 2][H]
  \toprule
	\downplay
  \firstcase{decl\_exp\_stmt}{alloc\dq ;\dq }{L\_Stmt}
	\downplay
  \firstcase{decl\_direct\_stmt}{ assign\_stmt\gralt init\_stmt\gralt const\_init\_stmt}{}
  \firstcase{decl\_part}{ decl\_exp\_stmt\gralt decl\_direct\_stmt\gralt RETI\_COMMENT}{}
  \\[-0.2cm]
	\downplay
  \firstcase{compound\_stmt}{ \dq \{\dq exec\_part* \dq \}\dq }{}
	\downplay
  \firstcase{exec\_exp\_stmt}{logic\_or\dq ;\dq }{}
	\downplay
  \firstcase{exec\_direct\_stmt}{if\_stmt\gralt if\_else\_stmt\gralt while\_stmt\gralt do\_while\_stmt}{}
	\downplay
  \otherform{assign\_stmt\gralt fun\_return\_stmt}{}
  \firstcase{exec\_part}{compound\_stmt\gralt exec\_exp\_stmt\gralt exec\_direct\_stmt}{}
  \otherform{RETI\_COMMENT}{}
  \\[-0.2cm]
  \firstcase{decl\_exec\_stmts}{decl\_part* exec\_part*}{}
  \midrule
  \firstcase{fun\_args}{[logic\_or(\dq ,\dq\enspace logic\_or)*]}{L\_Fun}
  \firstcase{fun\_call}{name\dq (\dq fun\_args\dq )\dq }{}
  \firstcase{fun\_return\_stmt}{\dq return\dq\enspace [logic\_or]\dq ;\dq }{}
  \firstcase{fun\_params}{[alloc(\dq ,\dq\enspace alloc)*]}{}
  \firstcase{fun\_decl}{type\_spec\enspace pntr\_deg\enspace name\dq (\dq fun\_params\dq )\dq }{}
  \firstcase{fun\_def}{type\_spec\enspace pntr\_deg\enspace name\dq (\dq fun\_params\dq )\dq\enspace \dq \{\dq  decl\_exec\_stmts \dq \}\dq }{}
  \midrule
  \firstcase{decl\_def}{(struct\_decl\gralt fun\_decl)\dq ;\dq \gralt fun\_def}{L\_File}
  \firstcase{decls\_defs}{decl\_def*}{}
  \firstcase{file}{FILENAME\enspace decls\_defs}{}
  \bottomrule
\end{grammar}
% Vorteile von Lark
\subsection{Umsetzung von Präzidenz}
Die \colorbold{PicoC} Programmiersprache hat dieselben \colorbold{Präzidenzregeln} implementiert, wie die Programmiersprache \colorbold{C} \footcite{noauthor_c_nodate}. Die \colorbold{Präzidenzregeln} von \colorbold{PicoC} sind in Tabelle~\ref{tab:reference_table} aufgelistet.

% \rowcolors{2}{SecondaryColor}{white}
\begin{table}[H]
  \center
  % \Block{2}{=}{Links, dann rechts $\rightarrow$} \\
  \begin{NiceTabular}{X[1,c]X[2,c]X[3,l]X[2,l]}[rules/color=PrimaryColor] % {\linewidth}{|C|C|L|L|}
  \CodeBefore
  \rowcolor{PrimaryColor}{1}
  \rowcolors{2-18}{SecondaryColor}{}[cols={1-3}]
  \rowcolors{2-18}{SecondaryColor}{}[cols={4}, respect-blocks, restart]
  \Body
  \textbf{\textcolor{white}{Präzidenz}} &	\textbf{\textcolor{white}{Operator}} & \textbf{\textcolor{white}{Beschreibung}} &	\textbf{\textcolor{white}{Assoziativität}} \\
  1	& \verb|a()|	& Funktionsaufruf & \Block{3-1}{Links, dann rechts $\rightarrow$} \\
    & \verb|a[]|	& Indexzugriff & \\
    & \verb|a.b| & Attributzugriff & \\
  2	&	\verb|-a| & Unäres Minus & \Block{3-1}{Rechts, dann links $\leftarrow$} \\
    & \smalltt{!a $\thicksim$a}	& Logisches NOT und Bitweise NOT & \\
    & \verb|*a &a| & Dereferenz und Referenz, auch Adresse-von & \\
  3	& \smalltt{a*b a/b a\%b} &	Multiplikation, Division und Modulo & \Block{9-1}{Links, dann rechts $\rightarrow$} \\
  4	& \verb|a+b a-b|	& Addition und Subtraktion & \\
  5	& \verb|a<b a<=b| \verb|a>b a>=b| & Kleiner, Kleiner Gleich, Größer, Größer gleich & \\
  6 &	\verb|a==b a!=b| & Gleichheit und Ungleichheit & \\
  7 &	\verb|a&b| & Bitweise UND & \\
  8 &	\verb|a^b| & Bitweise XOR (exclusive or) & \\
  9 & \smalltt{a$\mid$b} & Bitweise ODER (inclusive or) & \\
  10	& \verb|a&&b| &	Logiches UND & \\
  11	& $a{\mid\mid} b$	& Logisches ODER & \\
  12 & \verb|a=b| & Zuweisung & Rechts, dann links $\leftarrow$ \\
  13 &	\verb|a,b|& Komma	& Links, dann rechts $\rightarrow$ \\
  \bottomrule
\end{NiceTabular}
\caption{Präzidenzregeln von PicoC}
\label{tab:reference_table}
\end{table}
% erwähnen von Mehrdeutigkeit und Assoziativität
% finalle Grammatik im Appendix
% Crafting Compilers Quelle benennen
\subsection{Derivation Tree Generierung}
\subsection{Early Parser}
\subsection{Derivation Tree Vereinfachung}
% Visitor erwähnen
\subsection{Abstrakt Syntax Tree Generierung}
\subsubsection{ASTNode}
\subsubsection{PicoC Nodes}
% Tabelle aller PicoC Nodes
% möglichst kurze und leicht verständliche Bezeichner für Nodes
\subsubsection{RETI Nodes}
% Tabelle aller RETI Nodes
% Transformer erwähnen
         % ./content/Implementierung1.tex
  %!Tex Root = ../Main.tex
% ./Packete_und_Deklarationen.tex
% ./Titlepage.tex
% ./Motivation.tex
% ./Einführung.tex
% ./Implementierung1.tex
% ./Ergebnisse_und_Ausblick.tex

\subsection{Umsetzung von Pointern}
\subsubsection{Referenzierung}
Die \colorbold{Referenzierung} \verb|&<var>| wird im Folgenden anhand des Beispiels in Code~\ref{code:picoc_code_für_pointer_referenzierung} erklärt.

\begin{code}
  \centering
  \numberedcodebox[minted language=c, minted options={highlightlines={3}}]{./code_examples/example_pntr_ref.picoc}
  \caption{PicoC Code für Pointer Referenzierung}
  \label{code:picoc_code_für_pointer_referenzierung}
\end{code}

Der Knoten \smalltt{Ref(Name('var')))} repräsentiert im \colorbold{Abstrakt Syntax Tree} in Code~\ref{code:abstract_syntax_tree_für_pointer_referenzierung} eine \colorbold{Referenzierung} \verb|&<var>|.

\begin{code}
  \centering
  \numberedcodebox[minted language=text, minted options={highlightlines={10}}]{./code_examples/example_pntr_ref.ast}
  \caption{Abstract Syntax Tree für Pointer Referenzierung}
  \label{code:abstract_syntax_tree_für_pointer_referenzierung}
\end{code}

Im \colorbold{PicoC-Mon Pass} in Code~\ref{code:picoc_mon_für_pointer_referenzierung} wird der Knoten \smalltt{Ref(Name('var')))} durch die Knoten \smalltt{Ref(GlobalRead(Num('0')))} und \smalltt{Assign(GlobalWrite(Num('1')), Tmp(Num('1')))} ersetzt. Im Fall, dass in \smalltt{Ref(exp))} das \smalltt{exp} vielleicht nicht direkt ein \smalltt{Name('var')} enthält und \smalltt{exp} z.B. ein \smalltt{Subscr(Attr(Name('var')))} ist, sind noch weitere Anweisungen zwischen den Zeilen \smalltt{11} und  \smalltt{12} nötig, die sich in diesem Beispiel um das Übersetzen von \smalltt{Subscr(exp)} und \smalltt{Attr(exp)} nach dem Schema in Subkapitel~\ref{mittelteil_für_die_verschiedenen_derived_datatypes} kümmern.

\begin{code}
  \centering
  \numberedcodebox[minted language=text, minted options={highlightlines={11-13}}]{./code_examples/example_pntr_ref.picoc_mon}
  \caption{PicoC Mon Pass für Pointer Referenzierung}
  \label{code:picoc_mon_für_pointer_referenzierung}
\end{code}

Im \colorbold{PicoC-Blocks Pass} in Code~\ref{code:reti_blocks_für_pointer_referenzierung} werden die \colorbold{PicoC-Knoten} \smalltt{ Ref(Global(Num('0')))} und \smalltt{Assign(Global(Num('1')), Stack(Num('1')))} durch ihre entsprechenden \colorbold{RETI-Knoten} ersetzt.

\begin{code}
  \centering
  \numberedcodebox[minted language=text, minted options={highlightlines={18-21,23-25}}]{./code_examples/example_pntr_ref.reti_blocks}
  \caption{RETI Blocks Pass für Pointer Referenzierung}
  \label{code:reti_blocks_für_pointer_referenzierung}
\end{code}
% Initialisierung eines Pointers
\subsubsection{Dereferenzierung durch Zugriff auf Arrayindex ersetzen}
Die \colorbold{Dereferenzierung} \smalltt{*<var>} wird im Folgenden anhand des Beispiels in Code~\ref{code:picoc_code_für_pointer_dereferenzierung} erklärt.

\begin{code}
  \centering
  \numberedcodebox[minted language=c, minted options={highlightlines={4}}]{./code_examples/example_pntr_deref.picoc}
  \caption{PicoC Code für Pointer Dereferenzierung}
  \label{code:picoc_code_für_pointer_dereferenzierung}
\end{code}

Der Knoten \smalltt{Deref(Name('var')))} repräsentiert im \colorbold{Abstrakt Syntax Tree} in Code~\ref{code:abstract_syntax_tree_für_pointer_dereferenzierung} eine \colorbold{Dereferenzierung} \smalltt{*<var>}.

\begin{code}
  \centering
  \numberedcodebox[minted language=text, minted options={highlightlines={11}}]{./code_examples/example_pntr_deref.ast}
  \caption{Abstract Syntax Tree für Pointer Dereferenzierung}
  \label{code:abstract_syntax_tree_für_pointer_dereferenzierung}
\end{code}

Im \colorbold{PicoC-Shrink Pass} in Code~\ref{code:picoc_shrink_für_pointer_dereferenzierung} wird ein Trick angewandet, bei dem jeder Knoten \smalltt{Deref(Name('pntr'), Num('0'))} einfach durch den Knoten \smalltt{Subscr(Name('pntr'), Num('0'))} ersetzt wird. Der Trick besteht darin, dass der \colorbold{Dereferenzoperator} \smalltt{*(<var> + <i>)} sich identisch zum \colorbold{Operator für den Zugriff auf einen Arrayindex} \smalltt{<var>[<i>]} verhält\footnote{In der Sprache $L_{C}$ gibt es einen Unterschied bei der Initialisierung bei z.B. \smalltt{<datatype> *<var> = \dq string\dq} und \smalltt{<datatype> <var>[<i>] = \dq string\dq}, der allerdings nichts mit den beiden Operatoren zu tuen hat, sondern mit der \colorbold{Initialisierung}, bei der die Sprache $L_{C}$ verwirrenderweise die eckigen Klammern \smalltt{[]} genauso, wie beim \colorbold{Operator für den Zugriff auf einen Arrayindex}, vor den Bezeichner schreibt: \smalltt{<var>[<i>]}, obwohl es ein \colorbold{Derived Datatype} ist.}. Damit sparrt man sich viele vermeidbare \colorbold{Fallunterscheidungen} und \colorbold{doppelten Code} und kann die \colorbold{Derefenzierung} \smalltt{*(<var> + <i>)} einfach von den Routinen für einen \colorbold{Zugriff auf einen Arrayindex} \smalltt{<var>[<i>]} übernehmen lassen.

\begin{code}
  \centering
  \numberedcodebox[minted language=text, minted options={highlightlines={11}}]{./code_examples/example_pntr_deref.picoc_shrink}
  \caption{PicoC Shrink Pass für Pointer Dereferenzierung}
  \label{code:picoc_shrink_für_pointer_dereferenzierung}
\end{code}

\subsection{Umsetzung von Arrays}
\subsubsection{Initialisierung von Arrays}

Die \colorbold{Initialisierung} eines \colorbold{Arrays} (\smalltt{<datatype> <var>[2][1] = \{\{3+1\}, \{4\}\}}) wird im Folgenden anhand des Beispiels in Code~\ref{code:picoc_code_für_array_initialisierung} erklärt.

% Stack und Globale Statische Daten
\begin{code}
  \centering
  \numberedcodebox[minted language=c, minted options={highlightlines={2, 6}}]{./code_examples/example_array_init.picoc}
  \caption{PicoC Code für Array Initialisierung}
  \label{code:picoc_code_für_array_initialisierung}
\end{code}

Die \colorbold{Initialisierung} eines \colorbold{Arrays} \smalltt{<datatype> <var>[2][1] = \{\{3+1\}, \{4\}\}} wird im \colorbold{Abstrakt Syntax Tree} in Code~\ref{code:abstract_syntax_tree_für_array_initialisierung} mithilfe der Komposition \smalltt{Assign(Alloc(Writeable(), ArrayDecl([Num('2'), Num('1')], IntType('int')), Name('ar')), Array([Array([BinOp(Num('3'), Add('+'), Num('1'))]), Array([Num('4')])]))} dargestellt.

\begin{code}
  \centering
  \numberedcodebox[minted language=text, minted options={highlightlines={9, 16}}]{./code_examples/example_array_init.ast}
  \caption{Abstract Syntax Tree für Array Initialisierung}
  \label{code:abstract_syntax_tree_für_array_initialisierung}
\end{code}

Bei der \colorbold{Initialisierung} eines \colorbold{Arrays} wird zuerst \smalltt{Alloc(Writeable(), ArrayDecl([Num('2'), Num('1')], IntType('int')))} ausgewertet, da eine Variable zuerst definiert sein muss, bevor man sie verwenden kann\footnote{Das Widerspricht der üblichen Auswertungsreihenfolge beim \colorbold{Zuweisungsoperator} \smalltt{=}, der \colorbold{rechtsassoziativ} ist. Der \colorbold{Zuweisungsoperator} \smalltt{=} tritt allerdings erst später in Aktion.}. Das \colorbold{Definieren} der Variable \smalltt{ar} erfolgt mittels der \colorbold{Symboltabelle}, die in Code~\ref{code:symboltabelle_für_array_initialisierung} dargestellt ist.

Bei Variablen auf dem \colorbold{Stackframe} wird ein Array \colorbold{rückwärts} auf das Stackframe geschrieben und auch die \colorbold{Adresse des ersten Elements} als Adresse des Arrays genommen. Dies macht den \colorbold{Zugriff auf ein Arrayelement} in Subkapitel~\ref{sec:zugriff_auf_ein_arrayelement} deutlich unkomplizierter, da man so nicht mehr zwischen \colorbold{Stackframe} und \colorbold{Globalen Statischen Daten} beim \colorbold{Zugriff auf ein Arrayelement} unterscheiden muss, da es Probleme macht, dass ein \colorbold{Stackframe} in die Entgegengesetzt Richtung der \colorbold{Globalen Statischen Daten} wächst\footnote{Wenn man beim \colorbold{GCC}~\cite{noauthor_gcc_nodate} einen Stackframe mittels des \colorbold{GDB}~\cite{noauthor_gcc_nodate} beobachtet, sieht man, dass dieser es genauso macht.}.

\begin{code}
  \centering
  \numberedcodebox[minted language=text, minted options={highlightlines={14-19,32-37}}]{./code_examples/example_array_init.st}
  \caption{Symboltabelle für Array Initialisierung}
  \label{code:symboltabelle_für_array_initialisierung}
\end{code}

Im \colorbold{PiocC-Mon Pass} in Code~\ref{code:picoc_mon_für_array_initialisierung} werden zuerst die \colorbold{Ausdrücke} im \colorbold{Array-Initializer} \smalltt{Array([Array([BinOp(Num('3'), Add('+'), Num('1'))]), Array([Num('4')])])} nach dem \colorbold{Depth-First-Search} Schema, von \colorbold{links-nach-rechts} ausgewertet und auf den \colorbold{Stack} geschrieben.

Im finalen Schritt muss zwischen \colorbold{Globalen Statischen Daten} bei der \smalltt{main}-Funktion und \colorbold{Stackframe} bei der Funktion \smalltt{fun} unterschieden werden. Die auf den Stack ausgewerteten Expressions werden mittels der Komposition \smalltt{Assign(Global(Num('0')), Stack(Num('2')))} bzw. \smalltt{Assign(Stackframe(Num('3')), Stack(Num('4')))}, versetzt in der selben Reihenfolge zu den \colorbold{Globalen Statischen Daten} bzw. auf den \colorbold{Stackframe} geschrieben.

In die Knoten \smalltt{Global('0')} und  \smalltt{Stackframe('3')} wurde hierbei die \colorbold{Startadresse} des jeweiligen Arrays geschrieben, sodass man nach dem \colorbold{PicoC-Mon Pass} nie mehr Variablen in der  \colorbold{Symboltabelle} nachsehen muss und gleich weiß, ob sie bei den \colorbold{Globalen Statischen Daten} oder auf dem \colorbold{Stackframe} liegen.

\begin{code}
  \centering
  \numberedcodebox[minted language=text, minted options={highlightlines={8-12,19-23}}]{./code_examples/example_array_init.picoc_mon}
  \caption{PicoC Mon Pass für Array Initialisierung}
  \label{code:picoc_mon_für_array_initialisierung}
\end{code}

Im \colorbold{PicoC-Blocks Pass} in Code~\ref{code:reti_blocks_für_array_initialisierung} werden die \colorbold{PicoC-Knoten} für die Ausdrücke \smalltt{Exp(exp)} und \smalltt{Assign(Global(Num('0')), Stack(Num('2')))} bzw. \smalltt{Assign(Stackframe(Num('3')), Stack(Num('4')))} durch ihre entsprechenden \colorbold{RETI-Knoten} ersetzt.

\begin{code}
  \centering
  \numberedcodebox[minted language=text, minted options={highlightlines={9-11,13-15,17-21,23-25,27-31,40-42,44-46,48-50,52-54,56-64}}]{./code_examples/example_array_init.reti_blocks}
  \caption{RETI Blocks Pass für Array Initialisierung}
  \label{code:reti_blocks_für_array_initialisierung}
\end{code}


% kleines Extra
\subsubsection{Zugriff auf ein Arrayelement}
\label{sec:zugriff_auf_ein_arrayelement}

Der \colorbold{Zugriff auf ein Arrayelement} \smalltt{ar[0]} wird im Folgenden anhand des Beispiels in Code~\ref{code:picoc_code_für_zugriff_auf_arrayindex} erklärt.

\begin{code}
  \centering
  \numberedcodebox[minted language=c, minted options={highlightlines={3,8}}]{./code_examples/example_array_access.picoc}
  \caption{PicoC-Code für Zugriff auf ein Arrayelement}
  \label{code:picoc_code_für_zugriff_auf_arrayindex}
\end{code}

Der \colorbold{Zugriff auf ein Arrayelement} \smalltt{ar[0]} wird im  \colorbold{Abstract Syntx Tree} in Code~\ref{code:abstract_syntax_tree_für_zugriff_auf_arrayindex} mithilfe des \colorbold{Container-Knotens} \smalltt{Subscr(Name('ar'), Num('0'))} dargestellt.

\begin{code}
  \centering
  \numberedcodebox[minted language=text, minted options={highlightlines={10,18}}]{./code_examples/example_array_access.ast}
  \caption{Abstract Syntax Tree für Zugriff auf ein Arrayelement}
  \label{code:abstract_syntax_tree_für_zugriff_auf_arrayindex}
\end{code}

Im \colorbold{PicoC-Mon Pass} in Code~\ref{code:picoc_mon_für_zugriff_auf_arrayindex} wird beim \colorbold{Container-Knoten} \smalltt{Subscr(Name('ar'), Num('0'))} zuerst im \colorbold{Einleitungsteil} die \colorbold{Adresse} der Variable \smalltt{Name('ar')} auf den \colorbold{Stack} geschrieben. Bei den \colorbold{Globalen Statischen Daten} der \smalltt{main}-Funktion wird das durch die Komposition \smalltt{Ref(Global(Num('0')))} dargestellt und beim \colorbold{Stackframe} der Funktionm \smalltt{fun} wird das durch die Komposition \smalltt{Ref(Stackframe(Num('2')))} dargestellt.

In nächsten Schritt, dem \colorbold{Mittelteil} wird die Adresse des \colorbold{Index}, des Arrays auf das Zugegriffen werden soll berechnet. Da der \colorbold{Index} auf den Zugegriffen werden soll auch durch das Ergebnis eines \colorbold{komplexeren Ausdrucks}, z.B. \smalltt{<ar>[1 + <var>]} bestimmt sein kann, indem auch \colorbold{Variablen} vorkommen können, kann dieser nicht während des \colorbold{Kompilierens} berechnet werden, sondern muss zur \colorbold{Laufzeit} berechnet werden.

Daher muss zuerst der Wert des \colorbold{Index}, dessen Adresse berechnet werden soll bestimmt werden, z.B. im einfachen Fall durch \smalltt{Exp(Num('0'))} und dann muss die \colorbold{Adresse des Index} berechnet werden, was durch die Komposition \smalltt{Ref(Subscr(Stack(Num('2')), Stack(Num('1'))))} dargestellt wird. Die Bedeutung der Komposition \smalltt{\smalltt{Ref(Subscr(Stack(Num('2')), Stack(Num('1'))))}} ist in Tabelle~\ref{tab:kompositionen_von_picoc_knoten_und_reti_knoten_mit_besonderer_bedeutung} dokumentiert.

Der Sc

\begin{code}
  \centering
  \numberedcodebox[minted language=text, minted options={highlightlines={11-14,26}}]{./code_examples/example_array_access.picoc_mon}
  \caption{PicoC-Mon Pass für Zugriff auf ein Arrayelement}
  \label{code:picoc_mon_für_zugriff_auf_arrayindex}
\end{code}

\begin{code}
  \centering
  \numberedcodebox[minted language=text, minted options={highlightlines={18-21,23-25,27-32,34-36,66-69}}]{./code_examples/example_array_access.reti_blocks}
  \caption{RETI-Blocks Pass für Zugriff auf ein Arrayelement}
  \label{code:reti_blocks_für_zugriff_auf_arrayindex}
\end{code}

\subsubsection{Zuweisung an Arrayindex}
% Formel aus der Vorlesung, wo ist die hier?
\begin{code}
  \centering
  \numberedcodebox[minted language=c]{./code_examples/example_array_assignment.picoc}
  \caption{PicoC Code für Zuweisung an Arrayindex}
  \label{code:picoc_code_für_array_assignment}
\end{code}

\begin{code}
  \centering
  \numberedcodebox[minted language=text]{./code_examples/example_array_assignment.ast}
  \caption{Abstract Syntax Tree für Zuweisung an Arrayindex}
  \label{code:abstract_syntax_tree_für_array_assignment}
\end{code}

\begin{code}
  \centering
  \numberedcodebox[minted language=text]{./code_examples/example_array_assignment.picoc_mon}
  \caption{PicoC Mon Pass für Zuweisung an Arrayindex}
  \label{code:picoc_mon_für_array_assignment}
\end{code}

\begin{code}
  \centering
  \numberedcodebox[minted language=text]{./code_examples/example_array_assignment.reti_blocks}
  \caption{RETI Blocks Pass für Zuweisung an Arrayindex}
  \label{code:reti_blocks_für_array_assignment}
\end{code}

\subsection{Umsetzung von Structs}
\subsubsection{Deklaration von Structs}
\begin{code}
  \centering
  \numberedcodebox[minted language=c]{./code_examples/example_struct_decl.picoc}
  \caption{PicoC Code für Deklaration von Structs}
  \label{code:picoc_code_für_deklaration_von_structs}
\end{code}

\begin{code}
  \centering
  \numberedcodebox[minted language=text]{./code_examples/example_struct_decl.st}
  \caption{Symboltabelle für Deklaration von Structs}
  \label{code:symboltabelle_für_deklaration_von_structs}
\end{code}

\subsubsection{Initialisierung von Structs}
\begin{code}
  \centering
  \numberedcodebox[minted language=c]{./code_examples/example_struct_init.picoc}
  \caption{PicoC Code für Initialisierung von Structs}
  \label{code:picoc_code_für_initialisierung_von_structs}
\end{code}

\begin{code}
  \centering
  \numberedcodebox[minted language=text]{./code_examples/example_struct_init.ast}
  \caption{Abstract Syntax Tree für Initialisierung von Structs}
  \label{code:abstract_syntax_tree_für_initialisierung_von_structs}
\end{code}

\begin{code}
  \centering
  \numberedcodebox[minted language=text]{./code_examples/example_struct_init.st}
  \caption{Symboltabelle für Initialisierung von Structs}
  \label{code:symboltabelle_für_initialisierung_von_structs}
\end{code}

\begin{code}
  \centering
  \numberedcodebox[minted language=text]{./code_examples/example_struct_init.picoc_mon}
  \caption{PicoC Mon Pass für Initialisierung von Structs}
  \label{code:picoc_mon_pass_für_initialisierung_von_structs}
\end{code}

\begin{code}
  \centering
  \numberedcodebox[minted language=text]{./code_examples/example_struct_init.reti_blocks}
  \caption{RETI Blocks Pass für Initialisierung von Structs}
  \label{code:reti_blocks_pass_für_initialisierung_von_structs}
\end{code}

% Stack und Globale Statische Daten
\subsubsection{Zugriff auf Structattribut}
% Formel aus der Vorlesung, wo ist die hier?
\begin{code}
  \centering
  \numberedcodebox[minted language=c]{./code_examples/example_struct_attr_access.picoc}
  \caption{PicoC Code für Zugriff auf Structattribut}
  \label{code:picoc_code_für_zugriff_auf_structattribut}
\end{code}

\begin{code}
  \centering
  \numberedcodebox[minted language=text]{./code_examples/example_struct_attr_access.ast}
  \caption{Abstract Syntax Tree für Zugriff auf Structattribut}
  \label{code:abstract_syntax_tree_für_zugriff_auf_structattribut}
\end{code}

\begin{code}
  \centering
  \numberedcodebox[minted language=text]{./code_examples/example_struct_attr_access.picoc_mon}
  \caption{PicoC Mon Pass für Zugriff auf Structattribut}
  \label{code:picoc_mon_pass_für_zugriff_auf_structattribut}
\end{code}

\begin{code}
  \centering
  \numberedcodebox[minted language=text]{./code_examples/example_struct_attr_access.reti_blocks}
  \caption{RETI Blocks Pass für Zugriff auf Structattribut}
  \label{code:reti_blocks_pass_für_zugriff_auf_structattribut}
\end{code}

\subsubsection{Zuweisung an Structattribut}
\begin{code}
  \centering
  \numberedcodebox[minted language=c]{./code_examples/example_struct_attr_assignment.picoc}
  \caption{PicoC Code für Zuweisung an Structattribut}
  \label{code:picoc_code_für_zuweisung_an_structattribut}
\end{code}

\begin{code}
  \centering
  \numberedcodebox[minted language=text]{./code_examples/example_struct_attr_assignment.ast}
  \caption{Abstract Syntax Tree für Zuweisung an Structattribut}
  \label{code:abstract_syntax_tree_für_zuweisung_an_structattribut}
\end{code}

\begin{code}
  \centering
  \numberedcodebox[minted language=text]{./code_examples/example_struct_attr_assignment.picoc_mon}
  \caption{PicoC Mon Pass für Zuweisung an Structattribut}
  \label{code:picoc_mon_pass_für_zuweisung_an_structattribut}
\end{code}

\begin{code}
  \centering
  \numberedcodebox[minted language=text]{./code_examples/example_struct_attr_assignment.reti_blocks}
  \caption{RETI Blocks Pass für Zuweisung an Structattribut}
  \label{code:reti_blocks_pass_für_zuweisung_an_structattribut}
\end{code}

\subsection{Umsetzung der Derived Datatypes im Zusammenspiel}
\subsubsection{Einleitungsteil für Globale Statische Daten und Stackframe}
% Stack und Globale Statische Daten, unterschieldihe Berechnung der Adressen
% unterschiedliche Adressberechnung
\begin{code}
  \centering
  \numberedcodebox[minted language=c]{./code_examples/example_derived_dts_introduction_part.picoc}
  \caption{PicoC Code für den Einleitungsteil}
  \label{code:picoc_code_einleitungsteil}
\end{code}

% spezielles Vorgehen bei PntrDecl

\begin{code}
  \centering
  \numberedcodebox[minted language=text]{./code_examples/example_derived_dts_introduction_part.ast}
  \caption{Abstract Syntax Tree für den Einleitungsteil}
  \label{code:abstract_syntax_tree_einleitungsteil}
\end{code}

\begin{code}
  \centering
  \numberedcodebox[minted language=text]{./code_examples/example_derived_dts_introduction_part.picoc_mon}
  \caption{PicoC Mon Pass für den Einleitungsteil}
  \label{code:picoc_mon_pass_einleitungsteil}
\end{code}

\begin{code}
  \centering
  \numberedcodebox[minted language=text]{./code_examples/example_derived_dts_introduction_part.reti_blocks}
  \caption{RETI Blocks Pass für den Einleitungsteil}
  \label{code:reti_blocks_pass_einleitungsteil}
\end{code}

\subsubsection{Mittelteil für die verschiedenen Derived Datatypes}
\label{mittelteil_für_die_verschiedenen_derived_datatypes}

\begin{code}
  \centering
  \numberedcodebox[minted language=c]{./code_examples/example_derived_dts_main_part.picoc}
  \caption{PicoC Code für den Mittelteil}
  \label{code:picoc_code_mittelteil}
\end{code}

% spezielles Vorgehen bei PntrDecl

\begin{code}
  \centering
  \numberedcodebox[minted language=text]{./code_examples/example_derived_dts_main_part.ast}
  \caption{Abstract Syntax Tree für den Mittelteil}
  \label{code:abstract_syntax_tree_mittelteil}
\end{code}

\begin{code}
  \centering
  \numberedcodebox[minted language=text]{./code_examples/example_derived_dts_main_part.picoc_mon}
  \caption{PicoC Mon Pass für den Mittelteil}
  \label{code:picoc_mon_pass_mittelteil}
\end{code}

\begin{code}
  \centering
  \numberedcodebox[minted language=text]{./code_examples/example_derived_dts_main_part.reti_blocks}
  \caption{RETI Blocks Pass für den Mittelteil}
  \label{code:reti_blocks_pass_mittelteil}
\end{code}
% spezielles Vorgehen bei PntrDecl

\subsubsection{Schlussteil für die verschiedenen Derived Datatypes}
\begin{code}
  \centering
  \numberedcodebox[minted language=c]{./code_examples/example_derived_dts_final_part.picoc}
  \caption{PicoC Code für den Schlussteil}
  \label{code:picoc_code_schlussteil}
\end{code}

\begin{code}
  \centering
  \numberedcodebox[minted language=text]{./code_examples/example_derived_dts_final_part.ast}
  \caption{Abstract Syntax Tree für den Schlussteil}
  \label{code:abstract_syntax_tree_schlussteil}
\end{code}

\begin{code}
  \centering
  \numberedcodebox[minted language=text]{./code_examples/example_derived_dts_final_part.picoc_mon}
  \caption{PicoC Mon Pass für den Schlussteil}
  \label{code:picoc_mon_pass_schlussteil}
\end{code}

\begin{code}
  \centering
  \numberedcodebox[minted language=text]{./code_examples/example_derived_dts_final_part.reti_blocks}
  \caption{RETI Blocks Pass für den Schlussteil}
  \label{code:reti_blocks_pass_schlussteil}
\end{code}

% Umgang, wenn Datentyp abrubt aufhört am Ende

\subsection{Umsetzung von Funktionen}
\subsubsection{Funktionen auflösen zu RETI Code}
\begin{code}
  \centering
  \numberedcodebox[minted language=c]{./code_examples/example_3_funs.picoc}
  \caption{PicoC Code für 3 Funktionen}
  \label{code:picoc_code_für_3_Funktionen}
\end{code}

\begin{code}
  \centering
  \numberedcodebox[minted language=text]{./code_examples/example_3_funs.ast}
  \caption{Abstract Syntax Tree für 3 Funktionen}
  \label{code:abstract_syntax_tree_für_3_Funktionen}
\end{code}

\begin{code}
  \centering
  \numberedcodebox[minted language=text]{./code_examples/example_3_funs.picoc_blocks}
  \caption{PicoC Blocks Pass für 3 Funktionen}
  \label{code:picoc_blocks_pass_für_3_Funktionen}
\end{code}

\begin{code}
  \centering
  \numberedcodebox[minted language=text]{./code_examples/example_3_funs.picoc_mon}
  \caption{PicoC Mon Pass für 3 Funktionen}
  \label{code:picoc_mon_pass_für_3_Funktionen}
\end{code}

\begin{code}
  \centering
  \numberedcodebox[minted language=text]{./code_examples/example_3_funs.reti_blocks}
  \caption{RETI Blocks Pass für 3 Funktionen}
  \label{code:reti_blocks_pass_für_3_Funktionen}
\end{code}

% einfügen unsichtbarer Returns bei void
\newlineparagraph{Sprung zur Main Funktion}

\begin{code}
  \centering
  \numberedcodebox[minted language=c]{./code_examples/example_3_funs_main.picoc}
  \caption{PicoC Code für Funktionen, wobei die main Funktion nicht die erste Funktion ist}
  \label{code:picoc_code_für_funktionen_wobei_die_main_funktion_nicht_die_erste_Funktion_ist}
\end{code}

\begin{code}
  \centering
  \numberedcodebox[minted language=text]{./code_examples/example_3_funs_main.picoc_mon}
  \caption{PicoC Mon Pass für Funktionen, wobei die main Funktion nicht die erste Funktion ist}
  \label{code:picoc_mon_pass_für_funktionen_wobei_die_main_funktion_nicht_die_erste_Funktion_ist}
\end{code}

\begin{code}
  \centering
  \numberedcodebox[minted language=text]{./code_examples/example_3_funs_main.reti_blocks}
  \caption{PicoC Blocks Pass für Funktionen, wobei die main Funktion nicht die erste Funktion ist}
  \label{code:picoc_blocks_pass_für_funktionen_wobei_die_main_funktion_nicht_die_erste_Funktion_ist}
\end{code}

\begin{code}
  \centering
  \numberedcodebox[minted language=text]{./code_examples/example_3_funs_main.reti_patch}
  \caption{PicoC Patch Pass für Funktionen, wobei die main Funktion nicht die erste Funktion ist}
  \label{code:picoc_patch_pass_für_funktionen_wobei_die_main_funktion_nicht_die_erste_Funktion_ist}
\end{code}

\subsubsection{Funktionsdeklaration und -definition und Umsetzung von Scopes}
\begin{code}
  \centering
  \numberedcodebox[minted language=c]{./code_examples/example_3_funs_fun_decl.picoc}
  \caption{PicoC Code für Funktionen, wobei eine Funktion vorher deklariert werden muss}
  \label{code:picoc_code_für_funktionen_picoc_code_für_funktionen_wobei_eine_funktion_vorher_deklariert_werden_muss}
\end{code}

Bei mehreren Funktionen werden die \colorbold{Scopes} der unterschiedlichen \colorbold{Funktionen} mittels eines \colorbold{Suffix} \smalltt{\dq <fun\_name>@\dq} umgesetzt, der an den \colorbold{Variablennamen} \smalltt{<var>} drangehängt wird: \smalltt{<var>@<fun\_name>}. Dieser \colorbold{Suffix} wird geändert sobald beim \colorbold{Top-Down}\footnote{D.h. von der Wurzel zu den Blättern eines Baumes} Durchiterieren über den \colorbold{Abstract Syntax Tree} des aktuellen \colorbold{Passes} nach dem \colorbold{Depth-First-Search} Schema über den

\begin{code}
  \centering
  \numberedcodebox[minted language=text]{./code_examples/example_3_funs_fun_decl.st}
  \caption{Symboltabelle für Funktionen, wobei eine Funktion vorher deklariert werden muss}
  \label{code:symboltabelle_für_funktionen_picoc_code_für_funktionen_wobei_eine_funktion_vorher_deklariert_werden_muss}
\end{code}

% Allocation von Variablen
% Stack und Globale Statische Daten
% die Sache mit Assign(Tmp, Global) und Assign(Global, Tmp)
% erwähnen, das Main Funktion keinen Stackframe hat
% zählen der Größe der lokalen Daten und Parameter
% TODO: Signatur zu Parameter umbenennen
\subsubsection{Funktionsaufruf}

\newlineparagraph{Ohne Rückgabewert}

% Unsichtbares return
\begin{code}
  \centering
  \numberedcodebox[minted language=c]{./code_examples/example_fun_call_no_return_value.picoc}
  \caption{PicoC Code für Funktionsaufruf ohne Rückgabewert}
  \label{code:picoc_code_für_funktionsaufruf_ohne_rückgabewert}
\end{code}

\begin{code}
  \centering
  \numberedcodebox[minted language=text]{./code_examples/example_fun_call_no_return_value.picoc_mon}
  \caption{PicoC Mon Pass für Funktionsaufruf ohne Rückgabewert}
  \label{code:picoc_mon_pass_für_funktionsaufruf_ohne_rückgabewert}
\end{code}

\begin{code}
  \centering
  \numberedcodebox[minted language=text]{./code_examples/example_fun_call_no_return_value.reti_blocks}
  \caption{RETI Blocks Pass für Funktionsaufruf ohne Rückgabewert}
  \label{code:reti_blocks_pass_für_funktionsaufruf_ohne_rückgabewert}
\end{code}

\begin{code}
  \centering
  \numberedcodebox[minted language=text]{./code_examples/example_fun_call_no_return_value.reti}
  \caption{RETI Pass für Funktionsaufruf ohne Rückgabewert}
  \label{code:reti_pass_für_funktionsaufruf_ohne_rückgabewert}
\end{code}

\newlineparagraph{Mit Rückgabewert}

\begin{code}
  \centering
  \numberedcodebox[minted language=c]{./code_examples/example_fun_call_with_return_value.picoc}
  \caption{PicoC Code für Funktionsaufruf mit Rückgabewert}
  \label{code:picoc_code_für_funktionsaufruf_mit_rückgabewert}
\end{code}

\begin{code}
  \centering
  \numberedcodebox[minted language=text]{./code_examples/example_fun_call_with_return_value.picoc_mon}
  \caption{PicoC Mon Pass für Funktionsaufruf mit Rückgabewert}
  \label{code:picoc_mon_pass_für_funktionsaufruf_mit_rückgabewert}
\end{code}

\begin{code}
  \centering
  \numberedcodebox[minted language=text]{./code_examples/example_fun_call_with_return_value.reti_blocks}
  \caption{RETI Blocks Pass für Funktionsaufruf mit Rückgabewert}
  \label{code:reti_blocks_pass_für_funktionsaufruf_mit_rückgabewert}
\end{code}

\begin{code}
  \centering
  \numberedcodebox[minted language=text]{./code_examples/example_fun_call_with_return_value.reti}
  \caption{RETI Pass für Funktionsaufruf mit Rückgabewert}
  \label{code:reti_pass_für_funktionsaufruf_mit_rückgabewert}
\end{code}

\newlineparagraph{Umsetzung von Call by Sharing für Arrays}

\begin{code}
  \centering
  \numberedcodebox[minted language=c, minted options={highlightlines={1,7}}]{./code_examples/example_fun_call_by_sharing_array.picoc}
  \caption{PicoC Code für Call by Sharing für Arrays}
  \label{code:picoc_code_für_call_by_sharing_für_arrays}
\end{code}

\begin{code}
  \centering
  \numberedcodebox[minted language=text, minted options={highlightlines={15-20}}]{./code_examples/example_fun_call_by_sharing_array.picoc_mon}
  \caption{PicoC Mon Pass für Call by Sharing für Arrays}
  \label{code:picoc_mon_pass_für_call_by_sharing_für_arrays}
\end{code}


% https://tex.stackexchange.com/questions/298383/how-to-highlight-color-draw-attention-to-a-particular-snippet-in-minted/498614#498614
\begin{code}
  \centering
  \numberedcodebox[minted language=text, minted options={highlightlines={15,24}}]{./code_examples/example_fun_call_by_sharing_array.st}
  \caption{Symboltabelle für Call by Sharing für Arrays}
  \label{code:symboltabelle_für_call_by_sharing_für_arrays}
\end{code}

\begin{code}
  \centering
  \numberedcodebox[minted language=text, minted options={highlightlines={13-20}}]{./code_examples/example_fun_call_by_sharing_array.reti_blocks}
  \caption{RETI Block Pass für Call by Sharing für Arrays}
  \label{code:reti_blocks_pass_für_call_by_sharing_für_arrays}
\end{code}

% die Sache mit dem erstetzen von ArryDecl durch PntrDecl

\newlineparagraph{Umsetzung von Call by Value für Structs}

\begin{code}
  \centering
  \numberedcodebox[minted language=c, minted options={highlightlines={8}}]{./code_examples/example_fun_call_by_value_struct.picoc}
  \caption{PicoC Code für Call by Value für Structs}
  \label{code:picoc_code_für_call_by_value_für_structs}
\end{code}

% argmode für Struct Call by Value

\begin{code}
  \centering
  \numberedcodebox[minted language=text, minted options={highlightlines={15-19}}]{./code_examples/example_fun_call_by_value_struct.picoc_mon}
  \caption{PicoC Mon Pass für Call by Value für Structs}
  \label{code:picoc_mon_pass_für_call_by_value_for_structs}
\end{code}

% hier könnte man anmerken, dass die Adressen unterschiedlich berechnet werden für Stack und Globale...

\begin{code}
  \centering
  \numberedcodebox[minted language=text, minted options={highlightlines={13-19}}]{./code_examples/example_fun_call_by_value_struct.reti_blocks}
  \caption{RETI Block Pass für Call by Value für Structs}
  \label{code:reti_blocks_pass_für_call_by_value_for_structs}
\end{code}

% Struct wird wirklich kopiert durch speziellen Argmode

\subsection{Umsetzung kleinerer Details}
% langen Sprüngen, großen Konstanten, Division durch 0
\section{Fehlermeldungen}
\subsection{Error Handler}
\subsection{Arten von Fehlermeldungen}
\subsubsection{Syntaxfehler}
\subsubsection{Laufzeitfehler}
% Fehlermeldung ist, wenn der Lexer (partielle Funktion) oder Parser nicht matcht
% Token und Nodes enthalten Position, im Transformer wird die Position von den Token auf die Nodes übertragen und auch die Symboltabelle speichert Position
         % ./content/Implementierung2.tex
  %!Tex Root = ../Main.tex
% ./Packete_und_Deklarationen.tex
% ./Titlepage.tex
% ./Motivation.tex
% ./Einführung.tex
% ./Implementierung1_Tables_DT_AST.tex,
% ./Implementierung2_Pntr_Array.tex,
% ./Implementierung3_Struct_Derived.tex,
% ./Implementierung4_Fun.tex,

\chapter{Ergebnisse und Ausblick}
\label{ch:ergebnisse_und_ausblick}

\section{Compiler}
\subsection{Überblick über Funktionen}
% Beschreiben des Shell Mode und der Commandline Options

\subsection{Vergleich mit GCC}
% Prinzipien eingehalten
% ähnliche Fehlermeldungen use. Veriablen auf dem Stack

\subsection{Showmode}
% erwähnen, dass das Bonus ist, Interpreter Bonus
% schönes Bildchen, wo RETI States erklärt wird
% Startprogram im EPROM
%
% (Projekt Open source)
\section{Qualitätssicherung}
\label{sec:qualitätssicherung}
% GCC + Execution entspricht einem einzigen großen Interpreter und beweist somit den linke Edge in 2.1
% Testsuite erklären, vielleicht Snippets usw. erwähnen
% Größe des Datensegments erklären
% Interpreter verhält sich identisch zu Spezifikation in Vorlesung, nur ohne INT und RTI
% Finale Anzahl Tests, Alte Tests und neue Tests überscnneiden sich teiweise in der Sache, die sie testen
% Verwendung der Testsuite inklusive Commandline Tutorial
% Aufteilung der Tests, Hart zu jedem Großtema ein oder zwei harte Tests
  \numberwithin{equation}{\tcbcounter}
  \begin{equation}
    \begin{tikzpicture}[auto, baseline=(current  bounding  box.center)]
      \node (reti) [align=center] at (135:3) {$L_{RETI}$-\\Maschinencode};
      \node (x_86) [align=center] at (45:3) {$L_{X_{86\_64}}$-\\Maschinencode};
      \node (picoc) [above=of reti] {Test in $L_{PicoC}$};
      \node (c) [above=of x_86] {Test in $L_C$};
      \node (output)  at (270:0) {$Output$};

      % https://tex.stackexchange.com/questions/24372/how-to-add-newline-within-node-using-tikz
      \draw[->] (picoc) to node[above] {convert\_to\_c} (c);
      \draw[->] (picoc) to node[left] {PicoC-Compiler} (reti);
      \draw[->] (c) to node[right] {GCC} (x_86);
      \draw[->] (reti) to[bend right] node[left] {RETI-Interpreter} (output);
      \draw[->] (x_86) to[bend left] node[right] {$X_{86\_64}$-CPU} (output);
    \end{tikzpicture}
    \label{eq:compiler_beziehungen}
  \end{equation}
% RETI-Interpreter erwähnen und erwähnen, dass das Bonus ist
% TODO: zusammenfassendes Bild
% \section{Kommentierter Kompiliervorgang}
\section{Erweiterungsideen}
Wenn eines Tages eine \colorbold{RETI-CPU} auf einem \colorbold{FPGA} implementiert werden sollte, sodass ein \colorbold{provisorisches Betriebssystem} darauf laufen könnte, dann wäre der nächste Schritt einen \colorbold{Self-Compiling Compiler} $C_{RETI\_PicoC}^{PicoC}$ (Defintion~\ref{def:self_compiling_compiler}) zu schreiben. Dadurch kann die \colorbold{Unabhängigkeit} von der Programmiersprache $L_Python$, in der der momentane Compiler $C_{PicoC}$ für $L_{PicoC}$ implementiert ist und die Unabhängigkeit von einer \colorbold{anderen Maschiene}, die bisher immer für das Cross-Compiling notwendig war erreicht werden.

\begin{Definition}{Self-compiling Compiler}{self_compiling_compiler}
  Compiler $C_w^w$, der in der Sprache $L_w$ \colorbold{geschrieben} ist, die er \colorbold{selbst} kompiliert. Also ein Compiler, der sich \colorbold{selbst} kompilieren kann.\footcite{earley_formalism_1970}
\end{Definition}

Will man nun für eine Maschiene $M_{RETI}$, auf der bisher keine anderen Programmiersprachen mittels \colorbold{Bootstrapping} (Definition~\ref{def:bootstrapping}) zum laufen gebracht wurden, den gerade beschriebenen \colorbold{Self-compiling Compiler} $C_{RETI\_PicoC}^{PicoC}$ implementieren und hat bereits den gesamtem \colorbold{Self-compiling Compiler} $C_{RETI\_PicoC}^{PicoC}$ in der Sprache  $L_{PicoC}$ geschrieben, so stösst man auf ein Problem, dass auf das \colorbold{Henne-Ei-Problem}\footnote{Beschreibt die Situation, wenn ein System sich selbst als \colorbold{Abhängigkeit} hat, damit es überhaupt einen \colorbold{Anfang} für dieses System geben kann. Dafür steht das Problem mit der \colorbold{Henne} und dem \colorbold{Ei} sinnbildlich, da hier die Frage ist, wie das ganze seinen Anfang genommen hat, da beides \colorbold{zirkular} voneinander abhängt.} reduziert werden kann. Man bräuchte, um den \colorbold{Self-compiling Compiler} $C_{RETI\_PicoC}^{PicoC}$ auf der \colorbold{Maschiene} $M_{RETI}$ zu kompilieren bereits einen kompilierten \colorbold{Self-compiling Compiler} $C_{RETI\_PicoC}^{PicoC}$, der mit der Maschienensprache $B_{RETI}$ läuft. Es liegt eine \colorbold{zirkulare Abhängigkeit} vor, die man nur auflösen kann, indem eine \colorbold{externe Entität} zur Hilfe nimmt.

Da man den gesamten \colorbold{Self-compiling Compiler} $C_{RETI\_PicoC}^{PicoC}$ nicht selbst komplett in der Maschienensprache $B_{RETI}$ schreiben will, wäre eine Möglichkeit, dass man den \colorbold{Cross-Compiler} $C_{PicoC}^{Python}$, den man bereits in der Programmiersprache $L_{Python}$ implementiert hat, der in diesem Fall einen \colorbold{Bootstrapping Compiler} (Definition~\ref{def:bootstrap_compiler}) darstellt, auf einer anderen Maschiene $M_{other}$ dafür nutzt, damit dieser den \colorbold{Self-compiling Compiler} $C_{RETI\_PicoC}^{PicoC}$ für die Maschiene $M_{RETI}$ kompiliert bzw. \colorbold{bootstraped} und man den kompilierten \colorbold{RETI-Maschiendencode} dann einfach von der Maschiene $M_{other}$ auf die Maschiene $M_{RETI}$ kopiert.\footnote{Im Fall, dass auf der Maschiene $M_{RETI}$ die Programmiersprache $L_{Python}$ bereits mittels \colorbold{Bootstrapping} zum Laufen gebracht wurde, könnte der \colorbold{Self-compiling Compiler} $C_{RETI\_PicoC}^{PicoC}$ auch mithife des \colorbold{Cross-Compilers} $C_{PicoC}^{Python}$ als \colorbold{externe Entität} und der Programmiersprache $L_{Python}$ auf der Maschiene $M_{RETI}$ selbst kompiliert werden.}

\begin{figure}[H]
  \centering
  \includegraphics[width=0.5\linewidth]{./figures/cross_compiling.png}
  \caption{Cross-Compiler als Bootstrap Compiler}
\end{figure}

\begin{Special_Paragraph}
  Einen ersten \colorbold{minimalen Compiler} $C_{2\_w\_min}$ für eine Maschiene $M_2$ und Wunschsprache $L_w$ kann man entweder mittels eines \colorbold{externen} \colorbold{Bootstrap Compilers} $C_w^o$ kompilieren\footnote{In diesem Fall, dem \colorbold{Cross-Compiler} $C_{PicoC}^{Python}$.} oder man schreibt ihn direkt in der \colorbold{Maschienensprache} $B_2$ bzw. wenn ein \colorbold{Assembler} vorhanden ist, in der \colorbold{Assemblesprache} $A_2$.

  Die letzte Option wäre allerdings nur beim allerersten Compiler $C_{first}$ für eine allererste \colorbold{abstraktere Programmiersprache} $L_{first}$ mit Schleifen, Verzweigungen usw. notwendig gewesen. Ansonsten hätte man immer eine Kette, die beim allersten Compiler $C_{first}$ anfängt fortführen können, in der ein Compiler einen anderen Compiler kompiliert bzw. einen ersten minimalen Compiler kompiliert und dieser minimale Compiler dann eine umfangreichere Version von sich kompiliert usw.
\end{Special_Paragraph}

\begin{Definition}{Minimaler Compiler}{minimaler_compiler}
  Compiler $C_{w\_min}$, der nur die \colorbold{notwendigsten Funktionalitäten} einer Wunschsprache $L_w$, wie \colorbold{Schleifen},  \colorbold{Verzweigungen} kompiliert, die für die Implementierung eines \colorbold{Self-compiling Compilers} $C_{w}^{w}$ oder einer \colorbold{ersten Version} $C_{w_i}^{w_i}$ des Self-compiling Compilers $C_w^w$ wichtig sind.\footnote{Den \colorbold{PicoC-Compiler} könnte man auch als einen \colorbold{minimalen Compiler} ansehen.}\footcite{thiemann_compilerbau_2021}
\end{Definition}

\begin{Definition}{Boostrap Compiler}{bootstrap_compiler}
  Compiler $C_w^o$, der es ermöglicht einen \colorbold{Self-compiling Compiler} $C_w^w$ zu \colorbold{boostrapen}, indem der Self-compiling Compiler $C_w^w$ mit dem \colorbold{Bootstrap Compiler} $C_w^o$ \colorbold{kompiliert} wird\footnote{Dabei kann es sich um einen \colorbold{lokal} auf der Maschiene selbst laufenden Compiler oder auch um einen \colorbold{Cross-Compiler} handeln.}. Der Bootstrapping Compiler stellt die  \colorbold{externe Entität} dar, die es ermöglicht die \colorbold{zirkulare Abhängikeit}, dass initial ein \colorbold{Self-compiling Compiler} $C_w^w$ bereits kompiliert vorliegen müsste, um sich selbst kompilieren zu können, zu brechen.\footcite{thiemann_compilerbau_2021}
\end{Definition}

Aufbauend auf dem \colorbold{Self-compiling Compiler} $C_{RETI\_PicoC}^{PicoC}$, der einen \colorbold{minimalen Compiler} (Definition~\ref{def:minimaler_compiler}) für eine Teilmenge der \colorbold{Programmiersprache} C bzw. $L_C$ darstellt, könnte man auch noch weitere Teile der Programmiersprache $C$ bzw. $L_C$ für die Maschiene $M_{RETI}$ mittels \colorbold{Bootstrapping} implementieren.\footnote{Natürlich könnte man aber auch einfach den \colorbold{Cross-Compiler} $C_{PicoC}^{Python}$ um weitere Funktionalitäten von $L_C$ erweitern, hat dann aber weiterhin eine \colorbold{Abhängigkeit} von der Programmiersprache $L_{Python}$.}

Das bewerkstelligt man, indem man \colorbold{iterativ} auf der Zielmaschine $M_{RETI}$ selbst, aufbauend auf diesem \colorbold{minimalen Compiler} $C_{RETI\_PicoC}^{PicoC}$, wie in Subdefinition~\ref{def:bootstrapping}{.1} den minimalen Compiler schrittweise zu einem immer vollständigeren \colorbold{C-Compiler} $C_C$ weiterentwickelt.

\begin{Definition}{Bootstrapping}{bootstrapping}
  Wenn man einen \colorbold{Self-compiling Compiler} $C_{w}^{w}$ einer Wunschsprache $L_w$ auf einer \colorbold{Zielmaschine} $M$ zum laufen bringt\footnote{Z.B. mithilfe eines \colorbold{Bootstrap Compilers}.}\footnote{Der Begriff hat seinen Ursprung in der englischen \colorbold{Redewendung} \glqq pulling yourself up by your own bootstraps\grqq, was im deutschen ungefähr der aus den \colorbold{Lügengeschichten des Freiherrn von Münchhausen} bekannten Redewendung \glqq sich am eigenen Schopf aus dem Sumpf ziehen\grqq entspricht.}\footnote{Hat man einmal einen solchen \colorbold{Self-compiling Compiler} $C_{w}^{w}$ auf der Maschiene $M$ zum laufen gebracht, so kann man den Compiler auf der Maschiene $M$ weiterentwicklern, ohne von externen Entitäten, wie einer bestimmten Sprache $L_o$, in der der Compiler oder eine frühere Version des Compilers ursprünglich geschrieben war abhängig zu sein.}\footnote{Einen Compiler in der Sprache zu schreiben, die er selbst kompiliert und diesen Compiler dann sich selbst kompilieren zu lassen, kann eine gute \colorbold{Probe aufs Exempel} darstellen, dass der Compiler auch wirklich funktioniert.}. Dabei ist die Art von \colorbold{Bootstrapping} in \ref{def:bootstrapping}{.1} nochmal gesondert hervorzuheben:

  {\normalfont\bfseries \thetcbcounter{.1}:\:\ignorespaces}
  Wenn man die \colorbold{aktuelle Version} eines \colorbold{Self-compiling Compilers} $C_{w_i}^{w_i}$ der Wunschsprache $L_{w_i}$ mithilfe von \colorbold{früheren Versionen} seiner selbst kompiliert. Man schreibt also z.B. die aktuelle Version des Self-compiling Compilers in der Sprache $L_{w_{i-1}}$, welche von der früheren Version des Compilers, dem Self-compiling Compiler $C_{w_{i-1}}^{w_{i-1}}$ kompiliert wird und schafft es so \colorbold{iterativ} immer umfangreichere Compiler zu bauen.\footnote{Es ist hierbei theoretisch nicht notwendig den \colorbold{letzten} Self-compiling Compiler $C_{w_{i-1}}^{w_{i-1}}$ für das Kompilieren des \colorbold{neuen} Self-compiling Compilers $C_{w_i}^{w_i}$ zu verwenden, wenn z.B. der \colorbold{Self-compiling Compiler} $C_{w_{i-3}}^{w_{i-3}}$ auch bereits alle Funktionalitäten, die beim Schreiben des \colorbold{Self-compiling Compilers} $C_w^w$ verwendet werden kompilieren kann.}\footnote{Der Begriff ist sinnverwandt mit dem \colorbold{Booten} eines Computers, wo die wichtigste Software, der \colorbold{Kernel} zuerst in den Speicher geladen wird und darauf aufbauend von diesem dann das Betriebssysteme, welches bei Bedarf dann \colorbold{Systemsoftware}, Software, die das Ausführen von Anwendungssoftware ermöglicht oder unterstützt, wie z.B. Treiber. und \colorbold{Anwendungssoftware}, Software, deren Anwendung darin besteht, dass sie dem Benutzer unmittelbar eine Dienstleistung zur Verfügung stellt, lädt.}\footcite{earley_formalism_1970}
\end{Definition}

\begin{figure}[H]
  \centering
  \includegraphics[width=0.66\linewidth]{./figures/bootstrapping.png}
  \caption{Iteratives Bootstrapping}
\end{figure}

\begin{Special_Paragraph}
  Auch wenn ein \colorbold{Self-compiling Compiler} $C_{w_i}^{w_i}$ in der Subdefinition~\ref{def:bootstrapping}{.1} selbst in einer früheren Version $L_{w_{i-1}}$ der Programmiersprache $L_{w_i}$ geschrieben wird, wird dieser nicht mit $C_{w_i}^{w_{i-1}}$ bezeichnet, sondern mit $C_{w_i}^{w_i}$, da es bei \colorbold{Self-compiling Compilern} darum geht, dass diese zwar in der Subdefinition~\ref{def:bootstrapping}{.1} eine frühere Version $C_{w_{i-1}}^{w_{i-1}}$ nutzen, um sich selbst kompilieren zu lassen, aber sie auch in der Lage sind sich selber zu kompilieren.
\end{Special_Paragraph}

% tail call
% partial evaluator, ohne besser zur Anschauung
% Garbage Collector
% Array Länge vorne speichern
% super einfach ein PicoPython zu machen von der Syntax her durch auswechseln der Grammatik
% richtigen Compiler mit Graph Coloring machen
% Debugger Informationen rein machen
% Linker und weitere Dateien
 % ./content/Ergebnisse_und_Ausblick.tex

  \appendix
  %!Tex Root=../Main.tex
% ./Packete_und_Deklarationen.tex
\chapter{Appendix}
                % ./content/Appendix.tex
  %!Tex Root = ../Main.tex
% ./Packete_und_Deklarationen.tex
\chapter{Danksagungen}
% kein Standardtext, wie er z.B. in Arbeitsurkunden steht
% reichlich b, meine wie sage, sonst würde Zeit nicht wert sein
% Vertrauen
% Arten von Menschen
% Kunde
% keine Selbverständlichkeit
            % ./content/Danksagungen.tex

  \printbibheading
  % \printbibliography[type=book,heading=subbibliography,title={Bücher}]
  % \printbibliography[type=article,heading=subbibliography,title={Artikel}]
  \printbibliography[type=online,heading=subbibliography,title={Online}]
  \printbibliography[type=book,heading=subbibliography,title={Bücher}]
  \printbibliography[type=article,heading=subbibliography,title={Artikel}]
  \printbibliography[type=unpublished,heading=subbibliography,title={Vorlesungen}]
  \printbibliography[nottype=book, nottype=article, nottype=online, nottype=unpublished,heading=subbibliography,title={Sonstige Quellen}]
  % ./Library.bib
\end{document}
