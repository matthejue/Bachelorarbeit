%!Tex Root = ../Main.tex
\documentclass{report}
\usepackage[showframe, margin=1.5cm]{geometry}
\usepackage[ngerman]{babel}
\usepackage{lipsum}
\usepackage[parfill, ]{parskip}
\setlength{\parskip}{0.4cm} % space between paragraphs
\usepackage{setspace}
\usepackage{graphicx}
\usepackage[colorlinks=true, allcolors=blue]{hyperref} %hidelinks
\usepackage{csquotes}
\usepackage[style=authortitle]{biblatex}
\addbibresource{./library/library.bib}
\usepackage{pdfpages}
\usepackage{booktabs} % for table rules
\usepackage{tabulary}
% \usepackage{tabularx}
\usepackage{array}
\usepackage{multirow}
\usepackage{amssymb}

% colorbox stuff
\usepackage{tcolorbox}
\usepackage{tikz}
\tcbuselibrary{skins}
\usetikzlibrary{patterns}
\usetikzlibrary{shadings}
\tcbuselibrary{theorems}
\usepackage{cleveref}

\usepackage{xcolor}
\definecolor{PrimaryColor}{HTML}{2A1F59}
\definecolor{SecondaryColor}{HTML}{4D2875}
\definecolor{TertiaryColor}{HTML}{FED32F}
\definecolor{gray75}{gray}{0.75}

\newcommand{\smalltt}[1]{{\small\texttt{#1}}}

% bold with color
\newcommand\colorbold[1]{\textcolor{SecondaryColor}{\textbf{#1}}}

% footer and header
\usepackage{fancyhdr}
\pagestyle{fancy}					% custom headers and footers
	%	footers
	\fancyfoot{}						% clear footers
	\rfoot[]{\thepage}					% right align page numbers

	%	headers
	\fancyheadoffset{1cm}
	\lhead{\nouppercase{\leftmark}} 	%	adding section on left side of the header
\rhead{\nouppercase{\rightmark}} 	%	adding subsection on right side of the header

% spacing after section
\usepackage{titlesec}

% \titlespacing*{\section}
% {0pt}{5.5ex plus 1ex minus .2ex}{4.3ex plus .2ex}
% \titlespacing*{\subsection}
% {0pt}{5.5ex plus 1ex minus .2ex}{4.3ex plus .2ex}
\titlespacing*{\section}{0cm}{*3}{*3}
\titlespacing*{\subsection}{0cm}{*3}{*2}

\usepackage{fix-cm}
\newcommand{\hsp}{\hspace{0pt}}
\titleformat{\chapter}[hang]{\bfseries}{\fontsize{100}{0}\selectfont \textcolor{SecondaryColor}\thechapter\hsp}{0.5cm}{\Huge}[]

\titlespacing*{\chapter}{0cm}{*0}{*4}

\includeonly{
  ./content/Motivation,
  ./content/Einführung,
  ./content/Implementierung,
  ./content/Ergebnisse_und_Ausblick,
  ./content/Appendix
}
 % ./content/Packete_und_Deklarationen.tex
%!Tex Root = ../Main.tex
% ./Packete_und_Deklarationen.tex
% ./Titlepage.tex
% ./Motivation.tex
% ./Einführung.tex
% ./Implementierung2_Pntr_Array.tex,
% ./Implementierung3_Struct_Derived.tex,
% ./Implementierung4_Fun.tex,
% ./Ergebnisse_und_Ausblick.tex

% https://tex.stackexchange.com/questions/479632/newcommand-combine-optional-star-and-optional-parameter

\NewDocumentCommand{\commentsecond}{s}{%
  \ignorespaces
  \IfBooleanT{#1}{%
    \downplay
  }
  \firstcase{stmt}{SingleLineComment(str, str)\gralt RETIComment()}{L\_Comment}
}

\NewDocumentCommand{\arith}{s}{%
  \ignorespaces
  \IfBooleanT{#1}{%
    \downplay
  }
  \firstcase{un\_op}{Minus() \gralt Not()}{L\_Arith}
  \IfBooleanT{#1}{%
    \downplay
  }
  \firstcase{bin\_op}{Add() \gralt Sub() \gralt Mul() \gralt Div() \gralt Mod()}{}
  \IfBooleanT{#1}{%
    \downplay
  }
  \otherform{Oplus() \gralt And() \gralt Or()}{}
  \IfBooleanT{#1}{%
    \downplay
  }
  \firstcase{exp}{Name(str) \gralt Num(str) \gralt Char(str)}{}
  \IfBooleanT{#1}{%
    \downplay
  }
  \otherform{BinOp(\langle exp\rangle , \langle bin\_op\rangle , \langle exp\rangle )}{}
  \IfBooleanT{#1}{%
    \downplay
  }
  \otherform{UnOp(\langle un\_op\rangle , \langle exp\rangle ) \gralt Call(Name('input'), Empty())}{}
  \IfBooleanT{#1}{%
    \downplay
  }
  \otherform{Call(Name('print'), \langle exp\rangle)}{}
  \IfBooleanT{#1}{%
    \downplay
  }
  \firstcase{stmt}{Exp(\langle exp\rangle)}{}
}

\NewDocumentCommand{\arithanf}{s}{%
  \ignorespaces
  \downplay
  \firstcase{un\_op}{Minus() \gralt Not()}{L\_Arith}
  \downplay
  \firstcase{bin\_op}{Add() \gralt Sub() \gralt Mul() \gralt Div() \gralt Mod()}{}
  \downplay
  \otherform{Oplus() \gralt And() \gralt Or()}{}
  \IfBooleanT{#1}{%
    \downplay
  }
  \firstcase{exp}{\textcolor{red}{Name(str)} \gralt \textcolor{gray!90!black}{Num(str) \gralt Char(str)} \gralt Global(Num(str))}{}
  \IfBooleanT{#1}{%
    \downplay
  }
  \otherform{Stackframe(Num(str)) \gralt Stack(Num(str))}{}
  \IfBooleanT{#1}{%
    \downplay
  }
  \otherform{BinOp(Stack(Num(str)), \langle bin\_op\rangle, Stack(Num(str)))}{}
  \IfBooleanT{#1}{%
    \downplay
  }
  \otherform{UnOp(\langle un\_op\rangle, Stack(Num(str))) \gralt Call(Name('input'), Empty())}{}
  \IfBooleanT{#1}{%
    \downplay
  }
  \otherform{Call(Name('print'), \langle exp\rangle)}{}
  \IfBooleanT{#1}{%
    \downplay
  }
  \IfBooleanT{#1}{%
    \downplay
  }
  \otherform{Exp(\langle exp\rangle)}{}
}

\NewDocumentCommand{\logic}{s}{%
  \ignorespaces
  \IfBooleanT{#1}{%
    \downplay
  }
  \firstcase{un\_op}{LogicNot()}{L\_Logic}
  \IfBooleanT{#1}{%
    \downplay
  }
  \firstcase{rel}{Eq() \gralt NEq() \gralt Lt() \gralt LtE() \gralt Gt() \gralt GtE()}{}
  \IfBooleanT{#1}{%
    \downplay
  }
  \firstcase{bin\_op}{LogicAnd() \gralt LogicOr()}{}
  \IfBooleanT{#1}{%
    \downplay
  }
  \firstcase{exp}{Atom(\langle exp\rangle, \langle rel\rangle, \langle exp\rangle) \gralt ToBool(\langle exp\rangle)}{}
}

\NewDocumentCommand{\logicanf}{s}{%
  \ignorespaces
  \downplay
  \firstcase{un\_op}{LogicNot()}{L\_Logic}
  \downplay
  \firstcase{rel}{Eq() \gralt NEq() \gralt Lt() \gralt LtE() \gralt Gt() \gralt GtE()}{}
  \downplay
  \firstcase{bin\_op}{LogicAnd() \gralt LogicOr()}{}
  \IfBooleanT{#1}{%
    \downplay
  }
  \firstcase{exp}{Atom(Stack(Num(str)), \langle rel\rangle, Stack(Num(str)))}{}
  \IfBooleanT{#1}{%
    \downplay
  }
  \otherform{ToBool(Stack(Num(str)))}{}
}

\NewDocumentCommand{\assign}{s}{%
  \ignorespaces
  \IfBooleanT{#1}{%
    \downplay
  }
  \firstcase{type\_qual}{Const() \gralt Writeable()}{L\_Assign\_Alloc}
  \IfBooleanT{#1}{%
    \downplay
  }
  \firstcase{datatype}{IntType() \gralt CharType() \gralt VoidType()}{}
  \IfBooleanT{#1}{%
    \downplay
  }
  \firstcase{exp}{Alloc(\langle type\_qual\rangle , \langle datatype\rangle , Name(str))}{}
  \IfBooleanT{#1}{%
    \downplay
  }
  \firstcase{stmt}{Assign(\langle exp\rangle, \langle exp\rangle)}{}
}

\NewDocumentCommand{\assignanf}{s}{%
  \ignorespaces
  \removed
  \firstcase{type\_qual}{Const() \gralt Writeable()}{L\_Assign\_Alloc}
  \removed
  \firstcase{datatype}{IntType() \gralt CharType() \gralt VoidType()}{}
  \removed
  \firstcase{exp}{Alloc(\langle type\_qual\rangle, \langle datatype\rangle, Name(str))}{}
  \IfBooleanT{#1}{%
    \downplay
  }
  \firstcase{stmt}{Assign(Global(Num(str)), Stack(Num(str))) }{}
  \otherform{Assign(Stackframe(Num(str)), Stack(Num(str)))}{}
  \otherform{Assign(Stack(Num(str)), Global(Num(str))) }{}
  \otherform{Assign(Stack(Num(str)), Stackframe(Num(str)))}{}
}

\NewDocumentCommand{\pntr}{s}{%
  \ignorespaces
  \IfBooleanT{#1}{%
    \downplay
  }
  \firstcase{datatype}{PntrDecl(Num(str), \langle datatype\rangle )}{L\_Pntr}
  \IfBooleanT{#1}{%
    \downplay
  }
  \firstcase{exp}{Deref(\langle exp\rangle , \langle exp\rangle) \gralt Ref(\langle exp\rangle)}{}
}

\NewDocumentCommand{\pntrshrink}{s}{%
  \ignorespaces
  \IfBooleanT{#1}{%
    \downplay
  }
  \firstcase{datatype}{PntrDecl(Num(str), \langle datatype\rangle )}{L\_Pntr}
  \downplay
  \firstcase{exp}{\lochighlight{Deref(\langle exp\rangle , \langle exp\rangle)} \gralt Ref(\langle exp\rangle)}{}
}

\NewDocumentCommand{\pntrshrinkafter}{s}{%
  \ignorespaces
  \IfBooleanT{#1}{%
    \downplay
  }
  \firstcase{datatype}{PntrDecl(Num(str), \langle datatype\rangle )}{L\_Pntr}
  \IfBooleanT{#1}{%
    \downplay
  }
  \firstcase{exp}{Ref(\langle exp\rangle )}{}
}

\NewDocumentCommand{\pntranf}{s}{%
  \ignorespaces
  \IfBooleanT{#1}{%
    \downplay
  }
  \firstcase{datatype}{PntrDecl(Num(str), \langle datatype\rangle )}{L\_Pntr}
  \otherform{Ref(Global(str)) \gralt Ref(Stackframe(str))}{}
  \otherform{Ref(Subscr(\langle exp\rangle, Num(str)) \gralt Ref(Attr(\langle exp\rangle, Name(str))))}{}
}

\NewDocumentCommand{\arraysecond}{s}{%
  \ignorespaces
  \IfBooleanT{#1}{%
    \downplay
  }
  \firstcase{datatype}{ArrayDecl(Num(str)+, \langle datatype\rangle )}{L\_Array}
  \IfBooleanT{#1}{%
    \downplay
  }
  \firstcase{exp}{Subscr(\langle exp\rangle , \langle exp\rangle ) \gralt Array(\langle exp\rangle +)}{}
}

\NewDocumentCommand{\arrayanf}{s}{%
  \ignorespaces
  \IfBooleanT{#1}{%
    \downplay
  }
  \firstcase{datatype}{ArrayDecl(Num(str)+, \langle datatype\rangle )}{L\_Array}
  \IfBooleanT{#1}{%
    \downplay
  }
  \firstcase{exp}{Subscr(\langle exp\rangle, Stack(Num(str))) \gralt Array(\langle exp\rangle +)}{}
}

\NewDocumentCommand{\struct}{s}{%
  \ignorespaces
  \IfBooleanT{#1}{%
    \downplay
  }
  \firstcase{datatype}{StructSpec(Name(str))}{L\_Struct}
  \IfBooleanT{#1}{%
    \downplay
  }
  \firstcase{exp}{Attr(\langle exp\rangle , Name(str))}{}
  \IfBooleanT{#1}{%
    \downplay
  }
  \otherform{Struct(Assign(Name(str), \langle exp\rangle )+)}{}
  \IfBooleanT{#1}{%
    \downplay
  }
  \firstcase{decl\_def}{StructDecl(Name(str),}{}
  \IfBooleanT{#1}{%
    \downplay
  }
  & & \qquad $Alloc(Writeable(), \langle datatype\rangle , Name(str))+)$ & \\
}

\NewDocumentCommand{\structanf}{s}{%
  \ignorespaces
  \IfBooleanT{#1}{%
    \downplay
  }
  \firstcase{datatype}{StructSpec(Name(str))}{L\_Struct}
  \IfBooleanT{#1}{%
    \downplay
  }
  \firstcase{exp}{Attr(\langle exp\rangle , Name(str))}{}
  \IfBooleanT{#1}{%
    \downplay
  }
  \otherform{Struct(Assign(Name(str), \langle exp\rangle )+)}{}
  \IfBooleanT{#1}{%
    \downplay
  }
  \firstcase{decl\_def}{StructDecl(Name(str),}{}
  \IfBooleanT{#1}{%
    \downplay
  }
  & & \qquad $Alloc(Writeable(), \langle datatype\rangle , Name(str))+)$ & \\
}

\NewDocumentCommand{\ifelse}{s}{%
  \ignorespaces
  \IfBooleanT{#1}{%
    \downplay
  }
  \firstcase{stmt}{If(\langle exp\rangle , \langle stmt\rangle *)}{L\_If\_Else}
  \IfBooleanT{#1}{%
    \downplay
  }
  \otherform{IfElse(\langle exp\rangle , \langle stmt\rangle *, \langle stmt\rangle *)}{}
}

\NewDocumentCommand{\ifelseblocks}{s}{%
  \ignorespaces
  \IfBooleanT{#1}{%
    \removed
  }
  \firstcase{stmt}{If(\langle exp\rangle , \langle stmt\rangle *)}{L\_If\_Else}
  \IfBooleanT{#1}{%
    \downplay
  }
  \otherform{IfElse(\langle exp\rangle , \langle stmt\rangle *, \langle stmt\rangle *)}{}
}

\NewDocumentCommand{\ifelseafter}{s}{%
  \ignorespaces
  \IfBooleanT{#1}{%
    \downplay
  }
  \firstcase{stmt}{IfElse(\langle exp\rangle , \langle stmt\rangle *, \langle stmt\rangle *)}{L\_If\_Else}
}

\NewDocumentCommand{\ifelseanf}{s}{%
  \ignorespaces
  \IfBooleanT{#1}{%
    \downplay
  }
  \firstcase{stmt}{IfElse(Stack(Num(str)), \langle stmt\rangle *, \langle stmt\rangle*)}{L\_If\_Else}
}

\NewDocumentCommand{\loopsecond}{s}{%
  \ignorespaces
  \IfBooleanT{#1}{%
    \downplay
  }
  \firstcase{stmt}{While(\langle exp\rangle , \langle stmt\rangle *)}{L\_Loop}
  \IfBooleanT{#1}{%
    \downplay
  }
  \otherform{DoWhile(\langle exp\rangle , \langle stmt\rangle *)}{}
}

\NewDocumentCommand{\loopblocks}{s}{%
  \ignorespaces
  \IfBooleanT{#1}{%
    \removed
  }
  \firstcase{stmt}{While(\langle exp\rangle , \langle stmt\rangle *)}{L\_Loop}
  \IfBooleanT{#1}{%
    \removed
  }
  \otherform{DoWhile(\langle exp\rangle , \langle stmt\rangle *)}{}
}

\NewDocumentCommand{\fun}{s}{%
  \ignorespaces
  \IfBooleanT{#1}{%
    \downplay
  }
  \firstcase{exp}{Call(Name(str), \langle exp\rangle *)}{L\_Fun}
  \IfBooleanT{#1}{%
    \downplay
  }
  \firstcase{stmt}{Return(\langle exp\rangle)}{}
  \IfBooleanT{#1}{%
    \downplay
  }
  \firstcase{decl\_def}{FunDecl(\langle datatype\rangle , Name(str),}{}
  \IfBooleanT{#1}{%
    \downplay
  }
  & & \qquad $Alloc(Writeable(), \langle datatype\rangle , Name(str))*)$ & \\
  \IfBooleanT{#1}{%
    \downplay
  }
  \otherform{FunDef(\langle datatype\rangle , Name(str),}{}
  \IfBooleanT{#1}{%
    \downplay
  }
  & & \qquad $Alloc(Writeable(), \langle datatype\rangle , Name(str))*, \langle stmt\rangle *)$ & \\
}

\NewDocumentCommand{\funafter}{s}{%
  \ignorespaces
  \IfBooleanT{#1}{%
    \downplay
  }
  \firstcase{exp}{Call(Name(str), \langle exp\rangle *)}{L\_Fun}
  \IfBooleanT{#1}{%
    \downplay
  }
  \firstcase{stmt}{Return(\langle exp\rangle )}{}
  \IfBooleanT{#1}{%
    \downplay
  }
  \firstcase{decl\_def}{FunDecl(\langle datatype\rangle , Name(str),}{}
  \IfBooleanT{#1}{%
    \downplay
  }
  & & \qquad $Alloc(Writeable(), \langle datatype\rangle , Name(str))*)$ & \\
  \IfBooleanT{#1}{%
    \downplay
  }
  \otherform{FunDef(\langle datatype\rangle , Name(str),}{}
  \IfBooleanT{#1}{%
    \downplay
  }
  & & \qquad $Alloc(Writeable(), \langle datatype\rangle , Name(str))*, \lochighlight{\langle block\rangle} *)$ & \\
}

\NewDocumentCommand{\funanf}{s}{%
  \ignorespaces
  \removed
  \firstcase{exp}{Call(Name(str), \langle exp\rangle *)}{L\_Fun}
  \firstcase{stmt}{StackMalloc(Num(str)) \gralt NewStackframe(Name(str), GoTo(str))}{}
  \otherform{Exp(GoTo(Name(str))) \gralt RemoveStackframe()}{}
  \otherform{Return(Empty()) \gralt \textcolor{gray!90!black}{Return(\langle exp\rangle)}}{}
  \removed
  \firstcase{decl\_def}{FunDecl(\langle datatype\rangle , Name(str)}{}
  \removed
  & & \qquad $Alloc(Writeable(), \langle datatype\rangle , Name(str))*)$ & \\
  \removed
  \otherform{FunDef(\langle datatype\rangle , Name(str),}{}
  \removed
  & & \qquad $Alloc(Writeable(), \langle datatype\rangle , Name(str))*, \langle block\rangle *)$ & \\
}

\NewDocumentCommand{\block}{s}{%
  \ignorespaces
  \IfBooleanT{#1}{%
    \downplay
  }
  \firstcase{block}{Block(Name(str), \langle stmt\rangle *)}{L\_Blocks}
  \IfBooleanT{#1}{%
    \downplay
  }
  \firstcase{stmt}{GoTo(Name(str))}{}
}

\NewDocumentCommand{\picocblocksleftover}{s}{%
  \ignorespaces
  \IfBooleanTF{#1}{%
    \firstcase{instr}{\textcolor{gray!90!black}{Exp(GoTo(str))} \gralt Exit(Num(str))}{L\_PicoC}
  }{
    \downplay
    \firstcase{instr}{Exp(GoTo(str))}{L\_PicoC}
  }
  \IfBooleanTF{#1}{%
    \downplay
    \firstcase{block}{Block(Name(str), \langle instr\rangle *)}{}
  }{
    \downplay
    \firstcase{block}{Block(Name(str), \lochighlight{\langle instr\rangle} *)}{}
  }
  \downplay
  \firstcase{file}{File(Name(str), \langle block\rangle *)}{}
}

\NewDocumentCommand{\picocremovedleftover}{s}{%
  \ignorespaces
  \removed
  \firstcase{instr}{Exp(GoTo(str)) \gralt Exit(Num(str))}{L\_PicoC}
  \removed
  \firstcase{block}{Block(Name(str), \langle instr\rangle *)}{}
  \removed
  \firstcase{block}{Block(Name(str), \lochighlight{\langle instr\rangle} *)}{}
  \removed
  \firstcase{file}{File(Name(str), \langle block\rangle *)}{}
}

\NewDocumentCommand{\file}{s}{%
  \ignorespaces
  \IfBooleanT{#1}{%
    \downplay
  }
  \firstcase{file}{File(Name(str), \langle decl\_def\rangle *)}{L\_File}
}

\NewDocumentCommand{\fileanf}{s}{%
  \ignorespaces
  \downplay
  \firstcase{file}{File(Name(str), \lochighlight{\langle block\rangle} *)}{L\_File}
}

\NewDocumentCommand{\symbolsecond}{s}{%
  \ignorespaces
  \firstcase{symbol\_table}{SymbolTable(\langle symbol\rangle *)}{L\_Symbol\_Table}
  \firstcase{symbol}{Symbol(\langle type\_qual\rangle, \langle datatype\rangle, \langle name\rangle, \langle val\rangle, \langle pos\rangle, \langle size\rangle)}{}
  \firstcase{type\_qual}{Empty()}{}
  \firstcase{datatype}{BuiltIn() \gralt SelfDefined()}{}
  \firstcase{name}{Name(str)}{}
  \firstcase{val}{Num(str) \gralt Empty()}{}
  \firstcase{pos}{Pos(Num(str), Num(str)) \gralt Empty()}{}
  \firstcase{size}{Num(str) \gralt Empty()}{}
}

\NewDocumentCommand{\retiblocks}{s}{%
  \ignorespaces
  \IfBooleanT{#1}{%
    \downplay
  }
  \firstcase{reg}{ACC() \gralt IN1() \gralt IN2() \gralt PC() \gralt SP() \gralt BAF()}{L\_RETI}
  \IfBooleanT{#1}{%
    \downplay
  }
  \otherform{CS() \gralt DS()}{}
  \IfBooleanT{#1}{%
    \downplay
  }
  \firstcase{arg}{Reg(\langle reg\rangle ) \gralt Num(str)}{}
  \IfBooleanT{#1}{%
    \downplay
  }
  \firstcase{rel}{Eq() \gralt NEq() \gralt Lt() \gralt LtE() \gralt Gt() \gralt GtE()}{}
  \IfBooleanT{#1}{%
    \downplay
  }
  \otherform{Always() \gralt NOp()}{}
  \IfBooleanT{#1}{%
    \downplay
  }
  \firstcase{op}{Add() \gralt Addi() \gralt Sub() \gralt Subi() \gralt Mult() \gralt Multi()}{}
  \IfBooleanT{#1}{%
    \downplay
  }
  \otherform{Div() \gralt Divi() \gralt Mod() \gralt Modi() \gralt Oplus() \gralt Oplusi()}{}
  \IfBooleanT{#1}{%
    \downplay
  }
  \otherform{Or() \gralt Ori() \gralt And() \gralt Andi()}{}
  \IfBooleanT{#1}{%
    \downplay
  }
  \otherform{Load() \gralt Loadin() \gralt Loadi() \gralt Store() \gralt Storein() \gralt Move()}{}
  \IfBooleanT{#1}{%
    \downplay
  }
  \firstcase{instr}{Instr(\langle op\rangle , \langle arg\rangle +) \gralt Jump(\langle rel\rangle , Num(str)) \gralt Int(Num(str))}{}
  \IfBooleanT{#1}{%
    \downplay
  }
  \otherform{RTI() \gralt Call(Name('print'), \langle reg\rangle ) \gralt Call(Name('input'), \langle reg\rangle )}{}
  \IfBooleanT{#1}{%
    \downplay
  }
  \otherform{SingleLineComment(str, str)}{}
  \IfBooleanT{#1}{%
    \downplay
  }
  \otherform{Instr(Loadi(), [Reg(Acc()), GoTo(Name(str))]) \gralt Jump(Eq(), GoTo(Name(str)))}{}
}

\NewDocumentCommand{\reti}{s}{%
  \ignorespaces
  \downplay
  \firstcase{reg}{ACC() \gralt IN1() \gralt IN2() \gralt PC() \gralt SP() \gralt BAF()}{L\_RETI}
  \downplay
  \otherform{CS() \gralt DS()}{}
  \downplay
  \firstcase{arg}{Reg(\langle reg\rangle ) \gralt Num(str)}{}
  \downplay
  \firstcase{rel}{Eq() \gralt NEq() \gralt Lt() \gralt LtE() \gralt Gt() \gralt GtE()}{}
  \downplay
  \otherform{Always() \gralt NOp()}{}
  \downplay
  \firstcase{op}{Add() \gralt Addi() \gralt Sub() \gralt Subi() \gralt Mult() \gralt Multi()}{}
  \downplay
  \otherform{Div() \gralt Divi() \gralt Mod() \gralt Modi() \gralt Oplus() \gralt Oplusi()}{}
  \downplay
  \otherform{Or() \gralt Ori() \gralt And() \gralt Andi()}{}
  \downplay
  \otherform{Load() \gralt Loadin() \gralt Loadi() \gralt Store() \gralt Storein() \gralt Move()}{}
  \downplay
  \firstcase{instr}{Instr(\langle op\rangle , \langle arg\rangle +) \gralt Jump(\langle rel\rangle , Num(str)) \gralt Int(Num(str))}{}
  \downplay
  \otherform{RTI() \gralt Call(Name('print'), \langle reg\rangle ) \gralt Call(Name('input'), \langle reg\rangle )}{}
  \downplay
  \otherform{SingleLineComment(str, str)}{}
  \removed
  \otherform{Instr(Loadi(), [Reg(Acc()), GoTo(Name(str))]) \gralt Jump(Eq(), GoTo(Name(str)))}{}
  \downplay
  \firstcase{program}{Program(Name(str), \langle instr\rangle *)}{}
}
 % ./content/Grammar.tex

\includeonly{
  % ./content/Motivation,
  % ./content/Einführung,
  ./content/Implementierung1_Tables_DT_AST,
  % ./content/Implementierung2_Pntr_Array,
  % ./content/Implementierung3_Struct_Derived,
  % ./content/Implementierung4_Fun,
  % ./content/Ergebnisse_und_Ausblick,
  % ./content/Appendix
  % ./content/Danksagungen
}


\begin{document}
  \sloppy

  \newtcolorbox{titlebox}[1]{skin=enhanced, arc=0mm, boxrule=0mm,
      title style={preaction={fill=PrimaryColor}, pattern=fivepointed stars, pattern color=white, opacity=0.1},
      interior style={preaction={fill=SecondaryColor}, pattern=fivepointed stars, pattern  color=white, opacity=0.3},
      frame style={color=white},
      % segmentation style={black,solid,opacity=0.2,line width=1pt}
      title={#1}
    }

  %!Tex Root = ../Main.tex
% ./Packete_und_Deklarationen.tex
% ./Motivation.tex
% ./Einführung.tex
% ./Implementierung.tex
% ./Ergebnisse_und_Ausblick.tex

\begin{titlepage}
  \vspace{1cm}
  \center
  \textsc{\LARGE Albert Ludwigs Universität Freiburg}\\[0.5cm]
  \textsc{\Large Technische Fakultät}\\[2.0cm]

  \vspace{1cm}

  \begin{titlebox}{\center \huge \bfseries PicoC-Compiler}
    \center
    % \\
    % \tcblower
    {\bfseries \center \LARGE \setstretch{1.1} Übersetzung einer Untermenge von C in den Befehlssatz der RETI-CPU\par}
  \end{titlebox}
    \textsc{\large Bachelorarbeit}\\
    \rule{\linewidth}{0.1mm}\\[0.5cm]

  {\large \emph{Abgabedatum:} 28\textsuperscript{th} April 2022}\\[2.5cm]

  \begin{minipage}{0.45\textwidth}
    \begin{flushleft} \large
      \emph{Author:}\\
      Jürgen Mattheis\\
      \hspace{1cm}\\
      \hspace{1cm}\\
      \hspace{1cm}\\
      \hspace{1cm}
    \end{flushleft}
  \end{minipage}
  ~
  \begin{minipage}{0.45\textwidth}
    \begin{flushright} \large
      \emph{Gutachter:}\\
      Prof. Dr. Scholl\\[0.64cm]
      \emph{Betreung:}\\
      M.Sc. Seufert\\
    \end{flushright}
  \end{minipage}

  \vspace{8.cm}
  \rule{11cm}{0.1mm}\\[0.25cm]
  \large{Eine Bachelorarbeit am Lehrstuhl für}\\
  \large{Betriebssysteme}
\end{titlepage}
                 % ./content/Titlepage.tex
  \newgeometry{margin=2.5cm}
  \setlength{\footskip}{30pt}                 % move pagenumber up and down
  \includepdf[pages=-]{./ErklrungfrdieAbschlussarbeit_unterschrieben.pdf}

  \tableofcontents
  \listoffigures
  \listofcodecaptions
  \listoftables
  % https://tex.stackexchange.com/questions/538528/tcolorbox-newtcbtheorem-index-with-tcolorbox
  \tcblistof[\chapter*]{theorem_list}{Definitionsverzeichnis}
  % https://tex.stackexchange.com/questions/49155/how-can-i-list-items-created-with-the-float-package-in-the-toc
  \listof{floatgrammar}{Grammatikverzeichnis}

  \numberwithin{codecaption}{chapter}

  \newtcbtheorem[list inside={theorem_list},crefname={definition}{definitions}, number within=chapter]{Definition}{Definition}%
  {enhanced, arc=0mm,top=3mm,bottom=3mm, boxrule=0mm, borderline south={1mm}{0pt}{PrimaryColor}, title style={color=PrimaryColor},
  interior style={opacity=0.2, fill=PrimaryColor},
  frame style={color=white}, fonttitle=\bfseries, fontupper=\itshape,
  before upper=\setlength{\parskip}{1em}
  }{def}

  \newtcolorbox{Special_Paragraph}{enhanced, breakable, sharpish corners, notitle, arc=0mm, left=3mm, right=3mm, boxrule=1mm, borderline vertical={1mm}{0pt}{PrimaryColor},
  interior style={fill=SecondaryColor},
  frame style={color=white},
  % https://tex.stackexchange.com/questions/459870/paragraph-breaks-inside-tcolorbox, maybe also parbox=false
  before upper=\setlength{\parskip}{1em}
  }

  % https://tex.stackexchange.com/questions/319355/tcolorbox-breakable-option-not-working
  \newtcbinputlisting{\codebox}[2][]{
  listing file={#2},
  enhanced, colframe=PrimaryColor,colback=SecondaryColor, fonttitle=\bfseries, arc=0mm, bottom=1mm, top=1mm, left=1mm, right=1mm, #1, listing only, listing engine=minted, minted style=colorful, halign title=center, sharpish corners, drop fuzzy shadow, minted options={fontsize=\small,linenos=false,breaklines,breakafter={_}, breakbefore={(}, breakaftersymbolpre={}, breakaftersymbolpost={}, breakbeforesymbolpre={}, breakbeforesymbolpost={}, breaksymbolsepleft=2mm, breaksymbolsepright=0mm, breakindent=0mm, breaksymbolindentleft=5mm, breaksymbolindentright=0mm, numbersep=0mm}
  }
% drop fuzzy shadow, drop lifted shadow, listing engine=listings

  \newtcolorbox{Code_Frame}[2][]{enhanced, sharpish corners, drop fuzzy shadow, arc=0mm, bottom=1mm, top=1mm, left=1mm, right=1mm, boxrule=1mm, halign title=center, fonttitle=\bfseries, interior style={fill=white}, frame style={color=PrimaryColor}, title={#2}, #1}

  % https://tex.stackexchange.com/questions/616575/how-to-draw-tcolorbox-without-a-frame
  \newtcolorbox{File}[1][]{enhanced, hbox, sharpish corners, drop fuzzy shadow, arc=0mm, notitle, interior style={fill=PrimaryColor}, frame empty, halign=center, fontupper=\color{white}\bfseries, #1}

  \newtcbinputlisting{\numberedcodebox}[2][]{
  listing file={#2},
  enhanced, breakable, colframe=PrimaryColor,colback=white, fonttitle=\bfseries, arc=0mm, #1, listing only, listing engine=minted, minted style=colorful, halign title=center, sharpish corners, drop fuzzy shadow, overlay={\begin{tcbclipinterior}\fill[PrimaryColor] (frame.south west) rectangle ([xshift=5mm]frame.north west);\end{tcbclipinterior}}
  }

  \DeclareTotalTCBox{\inlinebox}{ s v }
  {verbatim,colback=SecondaryColor,colframe=PrimaryColor}
  {\IfBooleanTF{#1}%
  {\textcolor{PrimaryColor}{\setBold >\enspace\ignorespaces}\lstinline[keywordstyle=\color{blue!35!white}\bfseries]^#2^}%
  {\lstinline[keywordstyle=\color{blue!35!white}\bfseries]^#2^}}
  % {verbatim,IfBooleanTF={#1}{colback=white, colframe=PrimaryColor, colupper=PrimaryColor}{colback=SecondaryColor,colframe=PrimaryColor}}

% maybe useful, add to toc: https://tex.stackexchange.com/questions/220059/list-of-figures-and-tables-not-in-contents
% maybe useful, no numbering: https://tex.stackexchange.com/questions/28333/continuous-v-per-chapter-section-numbering-of-figures-tables-and-other-docume

  % ../Main.tex
% ./Packete_und_Deklarationen.tex
\chapter{Motivation}
\section{PicoC und RETI}
\section{Problemstellung}
\section{Compiler und Interpreter}
                      % ./content/Motivation.tex
  %!Tex Root = ../Main.tex
% ./Packete_und_Deklarationen.tex
\chapter{Einführung}
\label{ch:einführung}

\section{Compiler und Interpreter}
\begin{Definition}{Compiler}{compiler}
\end{Definition}
\begin{Definition}{Interpreter}{Interpreter}
% TODO: Bild semantisch gleiche Bedeutung
\end{Definition}
\subsection{T-Diagramme}
\begin{Definition}{T-Diagram}{t_diagram}
\end{Definition}
\section{Grammatiken}
\section{Grundlagen}
\begin{Definition}{Sprache}{Sprache}
\end{Definition}
\begin{Definition}{Chromsky Hierarchie}{chromsky_hierarchie}
\end{Definition}
\begin{Definition}{Grammatik}{grammatik}
\end{Definition}
\begin{Definition}{Reguläre Sprachen}{reguläre_sprachen}
\end{Definition}
\begin{Definition}{Kontextfreie Sprachen}{kontextfreie_sprachen}
\end{Definition}
\subsection{Mehrdeutige Grammatiken}
\begin{Definition}{Ableitungsbaum}{ableitungsbaum}
% TODO: Bild hierfür
\end{Definition}
\begin{Definition}{Mehrdeutige Grammatik}{mehrdeutige_grammatik}
% TODO: (Bild hierfür)
\end{Definition}
\subsection{Präzidenz und Assoziativität}
\begin{Definition}{Assoziativität}{assoziativität}
\end{Definition}
\begin{Definition}{Präzidenz}{präzidenz}
\end{Definition}
% \subsection{Linksrekursiv und Rechtrekursiv}
\section{Lexikalische Analyse}
\label{sec:lexikalische_analyse}

Die \colorbold{Lexikalische Analyse} bildet üblicherweise die erste Ebene innerhalb der \colorbold{Pipe Architektur} bei der Implementierung von Compilern. Die Aufgabe der lexikalischen Analyse ist vereinfacht gesagt, in einem Inputstring, z.B. dem Inhalt einer Datei, welche in \colorbold{UTF-8} codiert ist, Folgen endlicher Symbole (auch \colorbold{Wörter} genannt) zu finden, die bestimmte \colorbold{Pattern} (Definition \ref{def:pattern}) matchen, die durch eine \colorbold{reguläre Grammatik} spezifiziert sind.

\begin{Definition}{Pattern}{pattern}
  \colorbold{Beschreibung} aller möglichen \colorbold{Lexeme} einer Menge $\mathbb{P}_{T}$, die einem bestimmten \colorbold{Token} $T$ zugeordnet werden.
  Die Menge $\mathbb{P}_{T}$ ist eine möglicherweise unendliche Menge von \colorbold{Wörtern}, die sich mit den Regeln einer \colorbold{regulären Grammatik} ${G}_{Lex}$ einer \colorbold{regulären Sprache} ${L}_{Lex}$ beschreiben lassen \footnote{Als Beschreibungswerkzeug können aber auch z.B. reguläre Ausdrücke hergenommen werden.}, die für die Beschreibung eines \colorbold{Tokens} $T$ zuständig sind.\footcite{noauthor_what_nodate}
\end{Definition}

Diese Folgen endlicher Symoble werden auch \colorbold{Lexeme} (Definition \ref{def:lexeme}) genannt.

\begin{Definition}{Lexeme}{lexeme}
  Ein \colorbold{Lexeme} ist ein \colorbold{Wort} aus dem Inputstring, welches das \colorbold{Pattern} für eines der \colorbold{Token} $T$ einer \colorbold{Sprache} ${L}_{Lex}$ matched.
\footcite{noauthor_what_nodate}
\end{Definition}

Diese \colorbold{Lexeme} werden vom \colorbold{Lexer} im \colorbold{Inputstring} identifziert und \colorbold{Tokens} $T$ zugeordnet (Definition \ref{def:lexer}). Die \colorbold{Tokens} sind es, die letztendlich an die \colorbold{Syntaktische Analyse} weitergegeben werden.

\begin{Definition}{Lexer (bzw. Scanner)}{lexer}
  Ein \colorbold{Lexer} ist eine \colorbold{partielle} Funktion \hspace{0.2cm}$lex: \Sigma^{*} \rightharpoonup (N \times W)^{*}$, welche ein \colorbold{Wort} aus $\Sigma^{*}$ auf ein \colorbold{Token} $T$ mit einem \colorbold{Tokennamen} $N$ und einem \colorbold{Tokenwert} $W$ abbildet, falls diese Folge von Symbolen sich unter der \colorbold{regulären Grammatik} ${G}_{Lex}$, der \colorbold{regulären Sprache} ${L_{Lex}}$ abbleiten lässt.\footcite{noauthor_lecture-notes-2021_2022}
\end{Definition}

Ein \colorbold{Lexer} ist im Allgemeinen eine \colorbold{partielle Funktion}, da es Zeichenfolgen geben kann, die kein \colorbold{Pattern} eines \colorbold{Tokens} der Sprache $L_{Lex}$ matchen. In Bezug auf eine Implementierung, wird, wenn der Lexer Teil der Implementierung eines Compilers ist, in diesem Fall eine \colorbold{Fehlermeldung} ausgegeben.

Eine weitere Aufgabe der \colorbold{Lekikalischen Analyse} ist es jegliche für die Weiterverarbeitung unwichtigen Symbole, wie Leerzeichen \,\textvisiblespace\,, Newline \verb|\n|\footnote{In Unix Systemen wird für Newline das ASCII Symbol \colorbold{line feed}, in Windows hingegen die ASCII Symbole \colorbold{carriage return} und \colorbold{line feed} nacheinander verwendet. Das wird aber meist durch die verwendete Porgrammiersprache, die man zur Inplementierung des Lexers nutzt wegabstrahiert.} und Tabs \verb|\t| aus dem Inputstring herauszufiltern. Das geschieht mittels des \colorbold{Lexers}, der allen für die \colorbold{Syntaktische Analyse} unwichtige Zeichen das leere Wort $\epsilon$ zuordnet. Das ist auch im Sinne der Definition, denn $\epsilon \in \Sigma^{*}$.

Nur das, was für die \colorbold{Syntaktische Analyse} wichtig ist, soll weiterverarbeitet werden, alles andere wird herausgefiltert.


% In den  $G_{Lex}$ Grammatiken einiger Programmiersprachen sind allerdinds alle möglichen Zeichenfolgen allein dadurch schon möglich, weil diese Programmiesprachen das Konzept eines \colorbold{Identifiers} o.ä. umsetzen, der alle möglichen Zeichenfolgen abfängt\footnote{Bei der Grammatik von C und auch PicoC ist das allerdings nicht der Fall, weil Identifier dort nicht mit einer Zahl anfangen dürfen.}. Wodurch der Lexer wiederum doch eine linkstotale partielle Funktion ist, die man im Allgemeinen einfach als \colorbold{Funktion} bezeichnet: $lex: \Sigma^{*} \rightarrow (N \times W)^{*}$.

Der Grund warum nicht einfach nur die \colorbold{Lexeme} an die \colorbold{Syntaktische Analyse} weitergegeben werden und der Grund für die Aufteilung des \colorbold{Tokens} in \colorbold{Tokenname} und \colorbold{Tokenwert} ist, weil z.B. die Bezeichner von Variablen, Konstanten und Funktionen beliebige Zeichenfolgen sein können, wie \smalltt{my\_fun}, \smalltt{my\_var} oder \smalltt{my\_const} und es auch viele verschiedenen Zahlen gibt, wie \smalltt{42}, \smalltt{314} oder \smalltt{12}. Die Überbegriffe bzw. Tokennamen für beliebige Bezeichner von Variablen, Konstanten und Funktionen und beliebige Zahlen sind aber trotz allem z.B. \smalltt{Zahl} und \smalltt{Bezeichner}.

Ein \colorbold{Lexeme} ist damit aber nicht das gleiche, wie der \colorbold{Tokenwert}, denn z.B. im Falle von PicoC kann z.B. der Wert $99$ durch zwei verschiedene Literale darstellt werden, einmal als ASCII-Zeichen \smalltt{'c'} und des Weiteren auch in Dezimalschreibweise als \smalltt{99}\footnote{Die Programmiersprache Python erlaubt es z.B. diesern Wert auch mit den Literalen \smalltt{0b1100011} und \smalltt{0x63} darzustellen.}. Der \colorbold{Tokenwert} ist jedoch der letztendliche Wert an sich, unabhängig von der Darstellungsform.

  Die \colorbold{Grammatik} $G_{Lex}$, die zur Beschreibung der Token $T$ einer regulären Sprache $L_{Lex}$ verwendet wird, ist üblicherweise \colorbold{regulär}, da ein typischer \colorbold{Lexer} immer nur \colorbold{ein Symbol} vorausschaut\footnote{Man nennt das auch einem \colorbold{Lookahead} von $1$}, unabhängig davon, was für Symbole davor aufgetaucht sind. Die übliche Implementierung eines \colorbold{Lexers} merkt sich nicht, was für Symbole davor aufgetaucht sind.

% TODO: später erwähnen, dass alle Regeln der Grammatik G_lex eine reguläre Form haben, was der Beweis ist

\begin{Special_Paragraph}
  Um Verwirrung verzubäugen ist es wichtig folgende Unterscheidung hervorzuheben: Wenn von \colorbold{Symbolen} die Rede ist, so werden in der \colorbold{Lexikalischen Analyse}, der \colorbold{Syntaktische Analyse} und der \colorbold{Code Generierung}, auf diesen verschiedenen Ebenen unterschiedliche Konzepte als Symbole bezeichnet.

  In der Lexikalischen Analyse sind einzelne \colorbold{Zeichen eines Zeichensatzes} die Symbole.

  In der Syntaktischen Analyse sind die \colorbold{Tokennamen} die Symbole.

  In der Code Generierung sind die \colorbold{Bezeichner von Variablen, Konstanten und Funktionnen} die Symbole\footnote{Das ist der Grund, warum die Tabelle, in der Informationen zu Identifiern gespeichert werden aus Kapitel \ref{ch:implementierung} Symboltabelle genannt wird.}.
\end{Special_Paragraph}

\begin{Definition}{Literal}{literal}
  Eine von möglicherweise vielen weiteren \colorbold{Darstellungsformen} für ein und denselben \colorbold{Wert}.
\end{Definition}

% TODO: zusammenfassendes Bild

\section{Syntaktische Analyse}

In der \colorbold{Syntaktischen Analyse} ist für einige Sprachen eine \colorbold{Kontextfreie Grammatik} $G_{Parse}$ notwendig, um die diese Sprache zu beschreiben, da viele Programmiersprachen z.B. für \colorbold{Funktionsaufrufe} \verb|fun(arg)| und \colorbold{Codeblöcke} \verb|if(1){}| syntaktische Mittel verwenden, die es notwendig machen sich zu merken wieviele öffnende Klammern \verb|'('| bzw. öffnende geschweifte Klammern \verb|'{'| es momentan gibt, die noch nicht durch eine enstsprechende schließende Klammer \verb|')'| bzw. schließende geschweifte Klammer \verb|'}'| geschlossen wurden.

% TODO: später erwähnen, dass alle Regeln der Grammatik G_parse eine kontexfreie Form haben, was der Beweis ist

Die vom \colorbold{Lexer} im Inputstring identifizierten \colorbold{Token} werden in der \colorbold{Syntaktischen Analyse} vom \colorbold{Parser} (Definition \ref{def:parser}) als \colorbold{Wegweiser} verwendet, da je nachdem, in welcher Reihenfolge die \colorbold{Token} auftauchen, dies einer anderen Ableitung nach der \colorbold{Grammatik} $G_{Parse}$ entspricht. Dabei wird in der Grammatik nach dem \colorbold{Tokennamen} unterschieden und nicht nach dem Tokenwert, da es nur von Interesse ist, ob an einer bestimmten Stelle z.B. eine \verb|Zahl| steht und nicht, welchen konkretten Wert diese \verb|Zahl| hat. Der \colorbold{Tokenwert} ist erst später in der \colorbold{Code Generierung} relevant.

Die \colorbold{Syntax}, in welcher der Inputstring aufgeschrieben ist, wird auch als \colorbold{konkrette Syntax} (Definition \ref{def:konkrette_syntax}) bezeichnet.

\begin{Definition}{Parser}{parser}
  Ein Programm, dass eine \colorbold{Eingabe} in eine für die \colorbold{Weiterverbeitung} taugliche Form bringt.
\end{Definition}

In Bezug auf Compilerbau hat ein \colorbold{Parser} meist die Aufgabe aus einem \colorbold{Inputstring} einen \colorbold{Derivation Tree} (Definition \ref{def:derivation_tree}) zu generieren.

\begin{Special_Paragraph}
  An dieser Stelle könnte möglicherweise eine Begriffsverwirrung enstehen, ob ein \colorbold{Lexer} nach der obigen Definition nicht auch ein \colorbold{Parser} ist.

  In Bezug auf Compilerbau ist ein \colorbold{Lexer} ein Teil eines Parsers und der Parser vereinigt sowohl die \colorbold{Lexikalische Analyse}, als auch einen Teil der \colorbold{Syntaktischen Analyse} in sich, aber für sich isoliert betrachtet ist ein Lexer nach Definition \ref{def:parser} ebenfalls ein Parser. Aber im Compilerbau überwiegt seine Funktionalität, dass er den Inputstring lexikalisch weiterverarbeitet, um ihn als Lexer zu bezeichnen, der Teil eines Parsers ist.
\end{Special_Paragraph}

Ein \colorbold{Parser} ist aber auch ein erweiterter \colorbold{Recognizer}, denn einmal hat der \colorbold{Parser} die Aufgabe eines \colorbold{Recognizers} (Definition \ref{def:recognizer}), nämlich zu überprüfen, ob ein Inputstring sich den Regeln der Grammatik $G_Parse$ ableiten lässt und ein \colorbold{Wort} der Sprache $L_{Parse}$ ist.

\begin{Definition}{Recognizer}{recognizer}

\end{Definition}

\begin{Definition}{Konkrette Syntax}{konkrette_syntax}
  \colorbold{Syntax} einer \colorbold{Sprache}, die durch die \colorbold{Grammatiken} $G_{Lex}$ und $G_{Parse}$ zusammengenommen beschrieben wird.

  Ein \colorbold{Programm} in seiner \colorbold{Textrepräsentation}, wie es in einer Textdatei nach den Regeln der \colorbold{Grammatiken} $G_{Lex}$ und $G_{Parse}$ abgeleitet steht, bevor man es kompiliert, ist in \colorbold{konkretter Syntax} aufgeschrieben.
\end{Definition}

\begin{Definition}{Derivation Tree (bzw. Parse Tree)}{derivation_tree}
\end{Definition}

\begin{Definition}{Abstrakte Syntax}{abstrakte_syntax}

\end{Definition}

\begin{Definition}{Abstrakte Syntax Tree}{abstrakte_syntax_tree}
\end{Definition}

\begin{Definition}{Transformer}{transformer}
\end{Definition}

\begin{Definition}{Visitor}{visitor}
\end{Definition}

% TODO: zusammenfassendes Bild
\section{Code Generierung}
\begin{Definition}{Pass}{pass}
% TODO: Bild semantisch gleiche Bedeutung
% TODO: auf T-Diagramme zurückkommen
\end{Definition}
\section{Fehlermeldungen}
\begin{Definition}{Fehlermeldung}{fehlermeldung}
\end{Definition}
% Kategorien von Fehlermeldungen
                      % ./content/Einführung.tex
  %!Tex Root = ../Main.tex
% ./Packete_und_Deklarationen.tex
% ./Titlepage.tex
% ./Motivation.tex
% ./Einführung.tex
% ./Implementierung2_Pntr_Array.tex,
% ./Implementierung3_Struct_Derived.tex,
% ./Implementierung4_Fun.tex,
% ./Ergebnisse_und_Ausblick.tex

\chapter{Implementierung}
\label{ch:implementierung}

\section{Lexikalische Analyse}
\subsection{Konkrette Syntax für die Lexikalische Analyse}
\numberwithin{floatgrammar}{section}

\label{sec:lex_analyse_verwendung_von_lark}
% ./concrete_syntax_picoc.lark
\begin{grammar}[Konkrette Syntax der Sprache $L_{PicoC}$ für die Lexikalische Analyse in EBNF, Teil 1][H][gr:concrete_syntax_lex_teil_1]
  \toprule
  \firstcase{COMMENT}{\dq //\dq\enspace /[{\wedge}\backslash n]{*}/\gralt \dq {/*}\dq\enspace  /(.\mid \setminus n)*?/\enspace \dq {*/}\dq }{L\_Comment}
  \firstcase{RETI\_COMMENT.2}{\dq {//}\dq \dq \text{\textvisiblespace} \dq ? \dq \#\dq /[\wedge\backslash n]{*}/}{}
  \midrule
  \firstcase{DIG\_NO\_0}{\dq 1\dq \gralt \dq 2\dq \gralt \dq 3\dq \gralt \dq 4\dq \gralt \dq 5\dq}{L\_Arith}
  \otherform{\dq 6\dq \gralt \dq 7\dq \gralt \dq 8\dq \gralt \dq 9\dq}{}
  \firstcase{DIG\_WITH\_0}{\dq 0\dq \gralt DIG\_NO\_0}{}
  \firstcase{NUM}{\dq 0\dq \gralt DIG\_NO\_0 DIG\_WITH\_0*}{}
  \firstcase{ASCII\_CHAR}{\dq\text{\textvisiblespace} \dq ..\dq \sim\dq }{}
  \firstcase{CHAR}{\dq '\dq ASCII\_CHAR\dq '\dq }{}
  \firstcase{FILENAME}{ASCII\_CHAR+\dq .picoc\dq }{}
  \firstcase{LETTER}{\dq {a}\dq ..\dq {z}\dq \gralt \dq {A}\dq ..\dq {Z}\dq}{}
  \firstcase{NAME}{(LETTER\gralt \dq \_\dq )}{}
  & & \qquad(LETTER\gralt DIG\_WITH\_0\gralt \dq \_\dq )* & \\
  \firstcase{name}{NAME\gralt INT\_NAME\gralt CHAR\_NAME}{}
  \otherform{VOID\_NAME}{}
  \firstcase{LOGIC\_NOT}{\dq !\dq }{}
  \firstcase{NOT}{\dq \sim\dq }{}
  \firstcase{REF\_AND}{\dq \&\dq }{}
  \firstcase{un\_op}{SUB\_MINUS\gralt LOGIC\_NOT\gralt NOT}{}
  \otherform{MUL\_DEREF\_PNTR \gralt REF\_AND}{}
  \firstcase{MUL\_DEREF\_PNTR}{\dq {*}\dq}{}
  \firstcase{DIV}{\dq /\dq}{}
  \firstcase{MOD}{\dq \%\dq}{}
  \firstcase{prec1\_op}{MUL\_DEREF\_PNTR\gralt DIV\gralt MOD}{}
  \firstcase{ADD}{\dq {+}\dq }{}
  \firstcase{SUB\_MINUS}{\dq {-}\dq }{}
  \firstcase{prec2\_op}{ADD\gralt SUB\_MINUS}{}
  \midrule
  \firstcase{LT}{\dq {<}\dq }{L\_Logic}
  \firstcase{LTE}{\dq {<=}\dq }{}
  \firstcase{GT}{\dq {>}\dq }{}
  \firstcase{GTE}{\dq {>=}\dq }{}
  \firstcase{rel\_op}{LT\gralt LTE\gralt GT\gralt GTE}{}
  \firstcase{EQ}{\dq {==}\dq }{}
  \firstcase{NEQ}{\dq {!=}\dq }{}
  \firstcase{eq\_op}{EQ\gralt NEQ}{}
  \bottomrule
\end{grammar}

\begin{grammar}[Konkrette Syntax der Sprache $L_{PicoC}$ für die Lexikalische Analyse in EBNF, Teil 2][H][gr:concrete_syntax_lex_teil_2]
  \toprule
  \firstcase{INT\_DT.2}{\dq int\dq }{L\_Assign\_Alloc}
  \firstcase{INT\_NAME.3}{\dq int\dq\enspace (LETTER\gralt DIG\_WITH\_0\gralt \dq \_\dq )+}{}
  \firstcase{CHAR\_DT.2}{\dq char\dq }{}
  \firstcase{CHAR\_NAME.3}{\dq char\dq\enspace (LETTER\gralt DIG\_WITH\_0\gralt \dq \_\dq )+}{}
  \firstcase{VOID\_DT.2}{\dq void\dq }{}
  \firstcase{VOID\_NAME.3}{\dq void\dq\enspace (LETTER\gralt DIG\_WITH\_0\gralt \dq \_\dq )+}{}
  \firstcase{prim\_dt}{INT\_DT\gralt CHAR\_DT\gralt VOID\_DT}{}
  \bottomrule
\end{grammar}

% \begin{grammar}[\(\lambda\) calculus syntax][p][gr:ex1]
%   \firstcase{T}{\nonterm{V}}{Variable}
%   \otherform{(\nonterm{T}\ \nonterm{T})}{Application}
%   \otherform{\lambda \nonterm{V}\cdot\nonterm{T}}{Abstraction}
%   \firstcase{V}{x, y, \dots}{Variables}
% \end{grammar}
% \begin{grammar}[Advanced capabilities of \texttt{grammar.sty}][p][gr:ex2]
%   \firstcase{A}{\nonterm{T} \gralt \nonterm{V}}{Multiple option on a single line}
%   \highlight
%   \otherform{\nonterm{A}}{Highlighted form}
%   \downplay
%   \otherform{\nonterm{B}\gralt \nonterm{C}}{Downplayed form}
%   \otherform{\lochighlight{\nonterm{A}} \gralt \nonterm{B}}{Emphasize part of the line}
% \end{grammar}

% erwähnen, dass in Lark die Grammatiken L_Lex und L_Parse gemischt sind
% EBNF erwähnen
% (erwähnen, dass finalle Grammatik im Appendix)
\subsection{Basic Lexer}
\section{Syntaktische Analyse}
\label{sec:syntaktische_analyse}

\subsection{Umsetzung von Präzidenz und Assoziativität}
\label{sec:umsetzung_von_präzidenz}

Die Programmiersprache $L_{PicoC}$ hat dieselben \colorbold{Präzidenzregeln} implementiert, wie die Programmiersprache $L_C$ \footcite{noauthor_c_nodate}. Die \colorbold{Präzidenzregeln} der Programmiersprache $L_{PicoC}$ sind in Tabelle~\ref{tab:reference_table} aufgelistet.

% \rowcolors{2}{SecondaryColor}{white}
\begin{table}[H]
  \center
  % \Block{2}{=}{Links, dann rechts $\rightarrow$} \\
  \begin{NiceTabular}{X[1,c]X[2,c]X[3,l]X[2,c]}[rules/color=PrimaryColor] % {\linewidth}{|C|C|L|L|}
  \CodeBefore
  \rowcolor{PrimaryColor}{1}
  \rowcolors{2-18}{SecondaryColor}{}[cols={2-3}]
  \rowcolors{2-18}{SecondaryColor}{}[cols={1,4}, respect-blocks, restart]
  \Body
  \textbf{\textcolor{white}{Präzidenz}} &	\textbf{\textcolor{white}{Operator}} & \textbf{\textcolor{white}{Beschreibung}} &	\textbf{\textcolor{white}{Assoziativität}} \\
  13	& \verb|a()|	& Funktionsaufruf & \Block{3-1}{Links, dann rechts $\rightarrow$} \\
    & \verb|a[]|	& Indexzugriff & \\
    & \verb|a.b| & Attributzugriff & \\
  12	&	\verb|-a| & Unäres Minus & \Block{3-1}{Rechts, dann links $\leftarrow$} \\
    & \smalltt{!a $\thicksim$a}	& Logisches NOT und Bitweise NOT & \\
    & \verb|*a &a| & Dereferenz und Referenz, auch Adresse-von & \\
  11	& \smalltt{a*b a/b a\%b} &	Multiplikation, Division und Modulo & \Block{9-1}{Links, dann rechts $\rightarrow$} \\
  10	& \verb|a+b a-b|	& Addition und Subtraktion & \\
  9	& \verb|a<b a<=b| \verb|a>b a>=b| & Kleiner, Kleiner Gleich, Größer, Größer gleich & \\
  8 &	\verb|a==b a!=b| & Gleichheit und Ungleichheit & \\
  7 &	\verb|a&b| & Bitweise UND & \\
  6 &	\verb|a^b| & Bitweise XOR (exclusive or) & \\
  5 & \smalltt{a$\mid$b} & Bitweise ODER (inclusive or) & \\
  4 & \verb|a&&b| &	Logiches UND & \\
  3 & $a{\mid\mid} b$	& Logisches ODER & \\
  2 & \verb|a=b| & Zuweisung & Rechts, dann links $\leftarrow$ \\
  1 &	\verb|a,b|& Komma	& Links, dann rechts $\rightarrow$ \\
  \bottomrule
\end{NiceTabular}
\caption{Präzidenzregeln von PicoC}
\label{tab:reference_table}
\end{table}

Würde man einen Teil dieser \colorbold{Operationen} ohne Beachtung von \colorbold{Präzidenzreglen} und \colorbold{Assoziativität} in eine Grammatik verarbeiten wollen, so könnte eine Grammatik, wie Grammatik~\ref{gr:undurchdachte_konkrette_syntax} dabei rauskommen.

\begin{grammar}[Undurchdachte Konkrette Syntax der Sprache $L_{PicoC}$ für die Syntaktische Analyse in EBNF][H][gr:undurchdachte_konkrette_syntax]
  \toprule
  \firstcase{prim\_exp}{name\gralt NUM\gralt CHAR\gralt \dq(\dq exp\dq)\dq}{L\_Arith +}
  \firstcase{un\_op}{\dq -\dq\gralt \dq\sim\dq \gralt \dq!\dq \gralt \dq *\dq \gralt \dq \&\dq}{}
  \firstcase{un\_exp}{un\_op\enspace exp}{}
  \firstcase{bin\_op}{\dq *\dq\gralt \dq /\dq\gralt \dq \%\dq \gralt \dq+\dq\gralt \dq-\dq\gralt \dq\&\dq \gralt \dq{\wedge{}}\dq \gralt \dq{\mid}\dq}{}
  \otherform{\dq{<}\dq \gralt \dq{<=}\dq \gralt \dq{>}\dq \gralt \dq{>=}\dq \gralt \dq{!=}\dq \gralt \dq{==}\dq \gralt \dq{\&\&}\dq \gralt \dq{\mid\mid}\dq}{}
  \firstcase{bin\_exp}{exp\enspace bin\_op\enspace exp}{}
  \firstcase{exp}{prim\_exp \gralt un\_exp \gralt bin\_exp}{}
  \bottomrule
\end{grammar}

Die Grammatik~\ref{gr:undurchdachte_konkrette_syntax} ist allerdings \colorbold{mehrdeutig}, verschiedene \colorbold{Ableitungen} in der Grammatik können zum selben \colorbold{Wort} führen. Z.B. kann das Wort \smalltt{3 * 1 >= -4 \&\& 1} über die Ableitung~\ref{eq:ableitung1} als auch über die Ableitung~\ref{eq:ableitung2}.

\begin{dmath}[compact]
  \smalltt{exp}\enspace\Rightarrow\enspace bin\_exp\Rightarrow\enspace exp\enspace bin\_op\enspace exp\enspace \Rightarrow^*\enspace exp\enspace bin\_op\enspace exp\enspace \&\&\enspace prim\_exp\enspace \Rightarrow^*\enspace bin\_exp\enspace >=\enspace un\_exp\enspace \&\&\enspace 1\enspace \Rightarrow^*\enspace exp\enspace bin\_op\enspace exp\enspace >=\enspace un\_op\enspace exp\enspace \&\&\enspace 1\enspace \Rightarrow\enspace prim\_exp\enspace ^*\enspace prim\_exp\enspace >=\enspace -prim\_exp\enspace \&\&\enspace 1\enspace \Rightarrow\enspace 3\enspace *\enspace 4\enspace >=\enspace -prim\_exp\enspace \&\&\enspace 1
    \label{eq:ableitung1}
\end{dmath}

\begin{equation}
   exp\Rightarrow \Rightarrow
    \label{eq:ableitung2}
\end{equation}

\subsection{Konkrette Syntax für die Syntaktische Analyse}
% ./concrete_syntax_picoc.lark
% https://tex.stackexchange.com/questions/851/removing-spaces-between-words-in-math-mode
In \ref{gr:concrete_syntax_parser}

\begin{grammar}[Konkrette Syntax der Sprache $L_{PicoC}$ für die Syntaktische Analyse in EBNF, Teil 1][H][gr:concrete_syntax_parser]
  \toprule
	\downplay
  \firstcase{prim\_exp}{name\gralt NUM\gralt CHAR\gralt "("logic\_or")"}{L\_Arith +}
  \firstcase{post\_exp}{array\_subscr\gralt struct\_attr\gralt fun\_call}{L\_Array +}
	\downplay
  \otherform{input\_exp\gralt print\_exp\gralt prim\_exp}{L\_Pntr +}
  \firstcase{un\_exp}{un\_op\enspace un\_exp\gralt post\_exp}{L\_Struct + L\_Fun}
  \midrule
	\downplay
  \firstcase{input\_exp}{\dq input\dq\dq(\dq\dq)\dq}{L\_Arith}
	\downplay
  \firstcase{print\_exp}{\dq print\dq\dq(\dq logic\_or\dq)\dq}{}
	\downplay
  \firstcase{arith\_prec1}{arith\_prec1\enspace prec1\_op\enspace un\_exp\gralt un\_exp}{}
	\downplay
  \firstcase{arith\_prec2}{arith\_prec2\enspace prec2\_op\enspace arith\_prec1\gralt arith\_prec1}{}
	\downplay
  \firstcase{arith\_and}{arith\_and\enspace \dq\&\dq\enspace arith\_prec2\gralt arith\_prec2}{}
	\downplay
  \firstcase{arith\_oplus}{arith\_oplus\enspace \dq {\wedge{}}\dq\enspace arith\_and\gralt arith\_and}{}
	\downplay
  \firstcase{arith\_or}{arith\_or\enspace \dq{\mid} \dq\enspace arith\_oplus\gralt arith\_oplus}{}
  \midrule
  \downplay
  \firstcase{rel\_exp}{rel\_exp\enspace rel\_op\enspace arith\_or\gralt arith\_or}{L\_Logic}
  \downplay
  \firstcase{eq\_exp}{eq\_exp\enspace eq\_op\enspace rel\_exp\gralt rel\_exp}{}
  \downplay
  \firstcase{logic\_and}{logic\_and\enspace \dq{\&\&}\dq\enspace eq\_exp\gralt eq\_exp}{}
  \downplay
  \firstcase{logic\_or}{logic\_or\enspace \dq{\mid\mid}\dq\enspace logic\_and\gralt logic\_and}{}
  \midrule
	\downplay
  \firstcase{type\_spec}{prim\_dt\gralt struct\_spec}{L\_Assign\_Alloc}
	\downplay
  \firstcase{alloc}{type\_spec\enspace pntr\_decl}{}
	\downplay
  \firstcase{assign\_stmt}{un\_exp\enspace \dq {=}\dq\enspace logic\_or\dq ;\dq }{}
  \firstcase{initializer}{logic\_or\gralt array\_init\gralt struct\_init}{}
	\downplay
  \firstcase{init\_stmt}{alloc\enspace \dq {=}\dq\enspace initializer\dq ;\dq }{}
	\downplay
  \firstcase{const\_init\_stmt}{\dq const\dq\enspace type\_spec\enspace name\enspace \dq {=}\dq\enspace NUM\dq ;\dq }{}
  \midrule
  \firstcase{pntr\_deg}{\dq {*}\dq *}{L\_Pntr}
  \firstcase{pntr\_decl}{pntr\_deg\enspace array\_decl\gralt array\_decl}{}
  \midrule
  \firstcase{array\_dims}{(\dq [\dq NUM\dq ]\dq )*}{L\_Array}
  \firstcase{array\_decl}{name\enspace array\_dims\gralt \dq (\dq pntr\_decl\dq )\dq  array\_dims}{}
  \firstcase{array\_init}{\dq \{\dq initializer(\dq ,\dq\enspace initializer)*\dq \}\dq }{}
  \firstcase{array\_subscr}{post\_exp\dq [\dq logic\_or\dq ]\dq }{}
  \midrule
  \firstcase{struct\_spec}{\dq struct\dq\enspace name}{L\_Struct}
  \firstcase{struct\_params}{(alloc\dq ;\dq )+}{}
  \firstcase{struct\_decl}{\dq struct\dq\enspace name\enspace \dq \{\dq struct\_params\dq \}\dq }{}
  \firstcase{struct\_init}{\dq \{\dq \dq .\dq name\dq {=}\dq initializer}{}
  & & \qquad(\dq ,\dq\enspace \dq .\dq name\dq {=}\dq initializer)*\dq \}\dq & \\
  \firstcase{struct\_attr}{post\_exp\dq .\dq name}{}
  \midrule
	\downplay
  \firstcase{if\_stmt}{\dq if\dq \dq (\dq logic\_or\dq )\dq\enspace exec\_part}{L\_If\_Else}
	\downplay
  \firstcase{if\_else\_stmt}{\dq if\dq \dq (\dq logic\_or\dq )\dq\enspace exec\_part\enspace \dq else\dq\enspace exec\_part}{}
  \midrule
	\downplay
  \firstcase{while\_stmt}{\dq while\dq \dq (\dq logic\_or\dq )\dq\enspace exec\_part}{L\_Loop}
	\downplay
  \firstcase{do\_while\_stmt}{\dq do\dq\enspace exec\_part\enspace \dq while \dq \dq (\dq logic\_or\dq )\dq \dq ;\dq }{}
  \bottomrule
\end{grammar}

\begin{grammar}[Konkrette Syntax der Sprache $L_{PicoC}$ für die Syntaktische Analyse in EBNF, Teil 2][H]
  \toprule
	\downplay
  \firstcase{decl\_exp\_stmt}{alloc\dq ;\dq }{L\_Stmt}
	\downplay
  \firstcase{decl\_direct\_stmt}{ assign\_stmt\gralt init\_stmt\gralt const\_init\_stmt}{}
  \firstcase{decl\_part}{ decl\_exp\_stmt\gralt decl\_direct\_stmt\gralt RETI\_COMMENT}{}
  \\[-0.2cm]
	\downplay
  \firstcase{compound\_stmt}{ \dq \{\dq exec\_part* \dq \}\dq }{}
	\downplay
  \firstcase{exec\_exp\_stmt}{logic\_or\dq ;\dq }{}
	\downplay
  \firstcase{exec\_direct\_stmt}{if\_stmt\gralt if\_else\_stmt\gralt while\_stmt\gralt do\_while\_stmt}{}
	\downplay
  \otherform{assign\_stmt\gralt fun\_return\_stmt}{}
  \firstcase{exec\_part}{compound\_stmt\gralt exec\_exp\_stmt\gralt exec\_direct\_stmt}{}
  \otherform{RETI\_COMMENT}{}
  \\[-0.2cm]
  \firstcase{decl\_exec\_stmts}{decl\_part* exec\_part*}{}
  \midrule
  \firstcase{fun\_args}{[logic\_or(\dq ,\dq\enspace logic\_or)*]}{L\_Fun}
  \firstcase{fun\_call}{name\dq (\dq fun\_args\dq )\dq }{}
  \firstcase{fun\_return\_stmt}{\dq return\dq\enspace [logic\_or]\dq ;\dq }{}
  \firstcase{fun\_params}{[alloc(\dq ,\dq\enspace alloc)*]}{}
  \firstcase{fun\_decl}{type\_spec\enspace pntr\_deg\enspace name\dq (\dq fun\_params\dq )\dq }{}
  \firstcase{fun\_def}{type\_spec\enspace pntr\_deg\enspace name\dq (\dq fun\_params\dq )\dq\enspace \dq \{\dq  decl\_exec\_stmts \dq \}\dq }{}
  \midrule
  \firstcase{decl\_def}{(struct\_decl\gralt fun\_decl)\dq ;\dq \gralt fun\_def}{L\_File}
  \firstcase{decls\_defs}{decl\_def*}{}
  \firstcase{file}{FILENAME\enspace decls\_defs}{}
  \bottomrule
\end{grammar}
% Vorteile von Lark


% erwähnen von Mehrdeutigkeit und Assoziativität
% finalle Grammatik im Appendix
% Crafting Compilers Quelle benennen

\subsection{Derivation Tree Generierung}
\subsubsection{Early Parser}
\subsubsection{Codebeispiel}
\label{sec:derivation_tree_generierung}
\begin{code}
  \centering
  \numberedcodebox[minted language=c]{./code_examples/example_dt_simple_ast_gen_array_decl_and_alloc.picoc}
  \caption{PicoC Code für Derivation Tree Generierung}
  \label{code:picoc_code_für_derivation_tree_generierung}
\end{code}

\begin{code}
  \centering
  \numberedcodebox[minted language=text]{./code_examples/example_dt_simple_ast_gen_array_decl_and_alloc.dt}
  \caption{Derivation Tree nach Derivation Tree Generierung}
  \label{code:derivation_tree_nach_derivation_tree_generierung}
\end{code}

\subsection{Derivation Tree Vereinfachung}
\subsubsection{Visitor}
\subsubsection{Codebeispiel}

Beispiel aus Subkapitel~\ref{sec:derivation_tree_generierung} wird fortgeführt.

\begin{code}
  \centering
  \numberedcodebox[minted language=text]{./code_examples/example_dt_simple_ast_gen_array_decl_and_alloc.dt_simple}
  \caption{Derivation Tree nach Derivation Tree Vereinfachung}
  \label{code:picoc_code_nach_derivation_tree_vereinfachung}
\end{code}

% Visitor erwähnen
\subsection{Abstrakt Syntax Tree Generierung}
\subsubsection{PicoC-Knoten}
% Tabelle aller PicoC Knoten
% möglichst kurze und leicht verständliche Bezeichner für Knoten

\begin{table}[H]
  \center
  \begin{NiceTabular}{X[7]X[10]}[rules/color=PrimaryColor]
  \CodeBefore
  \rowcolor{PrimaryColor}{1}
  \rowcolors{2-17}{SecondaryColor}{}[cols={1-2}]
  \Body
  \textbf{\textcolor{white}{PiocC-Knoten}} & \textbf{\textcolor{white}{Beschreibung}} \\
  \smalltt{Name(val)} & Ein \colorbold{Bezeichner}, z.B. \smalltt{my\_fun, my\_var} usw. , aber da es keine gute Kurzform für \smalltt{Identifier()} (englisches Wort für Bezeichner) gibt, wurde dieser Knoten \smalltt{Name()} genannt. \\
  \smalltt{Num(val)} & Eine \colorbold{Zahl}, z.B. \smalltt{42, -3} usw. \\
  \smalltt{Char(val)} & Ein \colorbold{Zeichen} der \colorbold{ASCII-Zeichenkodierung}, z.B. \smalltt{'c', '*'} usw. \\
  \smalltt{Minus(), Not(), DerefOp(), RefOp(), LogicNot()} & Die \colorbold{unären Operatoren} \smalltt{un\_op}: \smalltt{-a, {$\thicksim$}a, *a, \&a !a}. \\
  \smalltt{Add(), Sub(), Mul(), Div(), Mod(), Oplus(), And(), Or(), LogicAnd(), LogicOr()} & Die \colorbold{binären Operatoren} \smalltt{bin\_op}: \smalltt{a + b, a - b, a * b, a / b, a \% b, a $\wedge$ b, a \& b, a $\mid$ b, a \&\& b, a {$\mid\mid$} b}. \\
  \smalltt{Eq(), NEq(), Lt(), LtE(), Gt(), GtE()} & Die \colorbold{Relationen} \smalltt{rel}: \smalltt{a == b, a != b, a < b, a <= b, a > b, a >= b}. \\
  \smalltt{Const(), Writeable()} & Die \colorbold{Type Qualifier} \smalltt{type\_qual}: \smalltt{const}, was für ein \colorbold{nicht beschreibbare} \colorbold{Konstante} steht und das \colorbold{nicht} Angeben von \smalltt{const}, was für einen \colorbold{beschreibbare} Variable steht. \\
\smalltt{IntType(), CharType(), VoidType()} & Die \colorbold{Type Specifier} für \colorbold{Primitiven Datentypen}, die in der Abstrakten Syntax, um eine intuitive Bezeichnung zu haben einfach nur unter \colorbold{Datentypen} \smalltt{datatype} eingeordnet werden: \smalltt{int},  \smalltt{char},  \smalltt{void}. \\
\smalltt{Placeholder()} & \colorbold{Platzhalter} für einen Knoten, der diesen später \colorbold{ersetzt}. \\
  \smalltt{BinOp(exp, bin\_op, exp)} & Container für eine \colorbold{binäre Operation} mit $2$ Expressions: \smalltt{<exp1> <bin\_op> <exp2>}\\
  \smalltt{UnOp(un\_op, exp)} & Container für eine \colorbold{unäre Operation} mit einer Expression: \smalltt{<un\_op> <exp>}. \\
  \smalltt{Exit(num)} & Container für einen \colorbold{Exit Code}, der vor der Beendigung in das \smalltt{ACC} Register geschrieben wird und steht für die \colorbold{Beendigung} des laufenden Programmes. \\
  \smalltt{Atom(exp, rel, exp)} & Container für eine \colorbold{binäre Relation} mit $2$ Expressions: \smalltt{<exp1> <rel> <exp2>}\\
  \smalltt{ToBool(exp)} & Container für einen \colorbold{Arithmetischen Ausdruck}, wie z.B. \smalltt{1 + 3} oder einfach nur \smalltt{3}, der nicht nur $1$ oder $0$ als Ergebnis haben kann und daher bei einem Ergebnis $x > 1$ auf $1$ abgebildet wird. \\
  \smalltt{Alloc(type\_qual, datatype, name, \textcolor{gray!90!black}{local\_var\_or\_param})} & \colorbold{Container} für eine \colorbold{Allokation} \smalltt{<type\_qual> <datatype> <name>} mit den notwendigen Knoten \smalltt{type\_qual},  \smalltt{datatype} und  \smalltt{name}, die alle für einen Eintrag in der  \colorbold{Symboltabelle} notwendigen Informationen enthalten. Zudem besitzt er ein \textcolor{gray!90!black}{verstecktes Attribut}  \smalltt{local\_var\_or\_param}, dass die Information trägt, ob es sich bei der \colorbold{Variable} um eine \colorbold{Lokale Variable} oder einen \colorbold{Parameter} handelt. \\
  \smalltt{Assign(lhs, exp)} & Container für eine \colorbold{Zuweisung}, wobei \smalltt{lhs} ein  \smalltt{Subscr(exp1, exp2)}, \smalltt{Deref(exp1, exp2)}, \smalltt{Attr(exp, name)} oder  \smalltt{Name('var')} sein kann und  \smalltt{exp} ein beliebiger \colorbold{Logischer Ausdruck} sein kann: \smalltt{lhs = exp}. \\
  \bottomrule
\end{NiceTabular}
\caption{PicoC-Knoten Teil 1}
\label{tab:picoc_knoten_teil_1}
\end{table}

\begin{table}[H]
  \center
  \begin{NiceTabular}{X[7]X[10]}[rules/color=PrimaryColor]
  \CodeBefore
  \rowcolor{PrimaryColor}{1}
  \rowcolors{2-13}{SecondaryColor}{}[cols={1-2}]
  \Body
  \textbf{\textcolor{white}{PiocC-Knoten}} & \textbf{\textcolor{white}{Beschreibung}} \\
  \smalltt{Exp(exp, \textcolor{gray!90!black}{datatype}, \textcolor{gray!90!black}{error\_data})} & Container für einen \colorbold{beliebigen Ausdruck}, dessen Ergebnis auf den \colorbold{Stack} soll. Zudem besitzt er $2$ \textcolor{gray!90!black}{versteckte Attribute}, wobei \smalltt{datatype} im \colorbold{RETI Blocks Pass} wichtig ist und \smalltt{error\_data} für \colorbold{Fehlermeldungen} wichtig ist. \\
  \smalltt{Stack(num)} & Container, der für das \colorbold{temporäre} Ergebnis einer Berechnung, das \smalltt{num} Speicherzellen relativ zum  \colorbold{Stackpointer Register} \smalltt{SP} steht.\\
  \smalltt{Stackframe(num)} & Container, der für eine Variable steht, die \smalltt{num} Speicherzellen relativ zum \colorbold{Begin-Aktive-Funktion Register} \smalltt{BAF} steht. \\
  \smalltt{Global(num)} & Container, der für eine Variable steht, die \smalltt{num} Speicherzellen relativ zum \colorbold{Datensegment Register} \smalltt{DS} steht. \\
  \smalltt{StackMalloc(num)} & Container, der für das \colorbold{Allokieren} von \smalltt{num} Speicherzellen auf dem \colorbold{Stack} steht. \\
\smalltt{PntrDecl(num, datatype)} & Container, der für den \colorbold{Pointerdatentyp} steht: \smalltt{<prim\_dt> *<var>}, wobei das \colorbold{Attribut} \smalltt{num} die \colorbold{Anzahl zusammengefasster Pointer} angibt und \smalltt{datatype} der {Datentyp} ist, auf den der oder die \colorbold{Pointer} zeigen. \\
\smalltt{Ref(exp, \textcolor{gray!90!black}{datatype}, \textcolor{gray!90!black}{error\_data})} & Container, der für die Anwendung des \colorbold{Referenz-Operators} \smalltt{\&<var>} steht und die \colorbold{Adresse} einer \colorbold{Location} (Definition~\ref{def:location}) auf den Stack schreiben soll, die über \smalltt{exp} eingegrenzt wird. Zudem besitzt er $2$ \textcolor{gray!90!black}{versteckte Attribute}, wobei \smalltt{datatype} im \colorbold{RETI Blocks Pass} wichtig ist und \smalltt{error\_data} für \colorbold{Fehlermeldungen} wichtig ist. \\
\smalltt{Deref(lhs, exp)} & Container für den \colorbold{Indexzugriff} auf einen \colorbold{Array-} oder \colorbold{Pointerdatentyp}: \smalltt{<var>[<i>]}, wobei \smalltt{exp1} eine angehängte weitere \smalltt{Subscr(exp1, exp2)}, \smalltt{Deref(exp1, exp2)}, \smalltt{Attr(exp, name)} oder ein \smalltt{Name('var')} sein kann und \smalltt{exp2} der Index ist auf den zugegriffen werden soll. \\
\smalltt{ArrayDecl(nums, datatype)} & Container, der für den \colorbold{Arraydatentyp} steht: \smalltt{<prim\_dt> <var>[<i>]}, wobei das \colorbold{Attribut} \smalltt{nums} eine Liste von \smalltt{Num('x')} ist, die die \colorbold{Dimensionen} des Arrays angibt und \smalltt{datatype} der {Datentyp} ist, der über das Anwenden von \smalltt{Subscript()} auf das Array zugreifbar ist. \\
\smalltt{Array(exps, \textcolor{gray!90!black}{datatype})} & Container für den \colorbold{Initializer} eines \colorbold{Arrays}, dessen Einträge \smalltt{exps} weitere Initializer für eine \colorbold{Array-Dimension} oder ein Initializer für ein \colorbold{Struct} oder ein \colorbold{Logischer Ausdruck} sein können, z.B. \smalltt{\{\{1, 2\}, \{3, 4\}\}}. Des Weiteren besitzt er ein \textcolor{gray!90!black}{verstecktes Attribut} \smalltt{datatype}, welches für den \colorbold{PicoC-ANF Pass} Informationen transportiert, die für \colorbold{Fehlermeldungen} wichtig sind. \\
\smalltt{Subscr(exp1, exp2)} & Container für den \colorbold{Indexzugriff} auf einen \colorbold{Array-} oder \colorbold{Pointerdatentyp}: \smalltt{<var>[<i>]}, wobei \smalltt{exp1} eine angehängte weitere \smalltt{Subscr(exp1, exp2)}, \smalltt{Deref(exp1, exp2)} oder \smalltt{Attr(exp, name)} Operation sein kann oder ein \smalltt{Name('var')} sein kann und \smalltt{exp2} der Index ist auf den zugegriffen werden soll. \\
  \smalltt{StructSpec(name)} & Container für einen selbst definierten \colorbold{Structdatentyp}: \smalltt{struct <name>}, wobei das \colorbold{Attribut} \smalltt{name} festlegt, welchen \colorbold{selbst definierte} Structdatentyp dieser Container-Knoten repräsentiert. \\
  \smalltt{Attr(exp, name)} & Container für den \colorbold{Attributzugriff} auf einen \colorbold{Structdatentyp}: \smalltt{<var>.<attr>}, wobei \smalltt{exp1} eine angehängte weitere \smalltt{Subscr(exp1, exp2)}, \smalltt{Deref(exp1, exp2)} oder \smalltt{Attr(exp, name)} Operation sein kann oder ein \smalltt{Name('var')} sein kann und \smalltt{name} das Attribut ist, auf das zugegriffen werden soll. \\
  \bottomrule
\end{NiceTabular}
\caption{PicoC-Knoten Teil 2}
\label{tab:picoc_knoten_teil_2}
\end{table}

\begin{table}[H]
  \center
  \begin{NiceTabular}{X[7]X[10]}[rules/color=PrimaryColor]
  \CodeBefore
  \rowcolor{PrimaryColor}{1}
  \rowcolors{2-9}{SecondaryColor}{}[cols={1-2}]
  \Body
  \textbf{\textcolor{white}{PiocC-Knoten}} & \textbf{\textcolor{white}{Beschreibung}} \\
  \smalltt{Struct(assigns, \textcolor{gray!90!black}{datatype})} & Container für den \colorbold{Initializer} eines \colorbold{Structs}, z.B \smalltt{\{.<attr1>=\{1, 2\}, .<attr2>=\{3, 4\}\}}, dessen Eintrag \smalltt{assigns} eine Liste von \smalltt{Assign(lhs, exp)} ist mit einer Zuordnung eines \colorbold{Attributezeichners}, zu einem weiteren Initializer für eine \colorbold{Array-Dimension} oder zu einem Initializer für ein \colorbold{Struct} oder zu einem \colorbold{Logischen Ausdruck}. Des Weiteren besitzt er ein  \textcolor{gray!90!black}{verstecktes Attribut} \smalltt{datatype}, welches für den \colorbold{PicoC-ANF Pass} Informationen transportiert, die für \colorbold{Fehlermeldungen} wichtig sind. \\
  \smalltt{StructDecl(name, allocs)} & Container für die Deklaration eines \colorbold{selbstdefinierten Structdatentyps}, z.B. \smalltt{struct <var> \{<datatype> <attr1>; <datatype> <attr2>;\};}, wobei \smalltt{name} der \colorbold{Bezeichner} des Structdatentyps ist und \smalltt{allocs} eine Liste von Bezeichnern der \colorbold{Attribute} des Structdatentyps mit dazugehörigem \colorbold{Datentyp}, wofür sich der \colorbold{Container-Knoten} \smalltt{Alloc(type\_qual, datatype, name)} sehr gut als \colorbold{Container} eignet. \\
  \smalltt{If(exp, stmts)} & Container für ein \colorbold{If Statement} \smalltt{if(<exp>) \{ <stmts> \}} inklusive  \colorbold{Condition}  \smalltt{exp} und einem  \colorbold{Branch}  \smalltt{stmts}, indem eine \colorbold{Liste von Statements} stehen kann oder ein einzelnes \smalltt{GoTo(Name('block.xyz'))}. \\
  \smalltt{IfElse(exp, stmts1, stmts2)} & Container für ein \colorbold{If-Else Statement} \smalltt{if(<exp>) \{ <stmts2> \} else \{ <stmts2> \}} inklusive \colorbold{Codition} \smalltt{exp} und $2$ \colorbold{Branches} \smalltt{stmts1} und \smalltt{stmts2}, die zwei Alternativen Darstellen in denen jeweils \colorbold{Listen von Statements} oder  \smalltt{GoTo(Name('block.xyz'))}'s stehen können. \\
  \smalltt{While(exp, stmts)} & Container für ein \colorbold{While-Statement} \smalltt{while(<exp>) \{ <stmts> \}} inklusive  \colorbold{Condition}  \smalltt{exp} und einem \colorbold{Branch}  \smalltt{stmts}, indem eine \colorbold{Liste von Statements} stehen kann oder ein einzelnes \smalltt{GoTo(Name('block.xyz'))}. \\
  \smalltt{DoWhile(exp, stmts)} & Container für ein \colorbold{Do-While-Statement} \smalltt{do \{ <stmts> \} while(<exp>);} inklusive  \colorbold{Condition}  \smalltt{exp} und einem \colorbold{Branch}  \smalltt{stmts}, indem eine \colorbold{Liste von Statements} stehen kann oder ein einzelnes \smalltt{GoTo(Name('block.xyz'))}. \\
  \smalltt{Call(name, exps)} & Container für einen \colorbold{Funktionsaufruf}: \smalltt{fun\_name(exps)}, wobei \smalltt{name} der \colorbold{Bezeichner} der Funktion ist, die aufgerufen werden soll und \smalltt{exps} eine \colorbold{Liste von Argumenten} ist, die an die Funktion übergeben werden soll. \\
  \smalltt{Return(exp)} & Container für ein \smalltt{Return-Statement}: \smalltt{return <exp>}, wobei das  \colorbold{Attribut}  \smalltt{exp} einen  \colorbold{Logischen Ausdruck} darstellt, dessen Ergebnis vom  \colorbold{Return-Statement} zurückgegeben wird. \\
  \smalltt{FunDecl(datatype, name, allocs)} & Container für eine \colorbold{Funktionsdeklaration}, z.B. \smalltt{<datatype> <fun\_name>(<datatype> <param1>, <datatype> <param2>)}, wobei \smalltt{datatype} der  \colorbold{Rückgabewert} der Funktion ist,  \smalltt{name} der  \colorbold{Bezeichner} der Funktion ist und \smalltt{allocs} die \colorbold{Parameter} der Funktion sind, wobei der \colorbold{Container-Knoten} \smalltt{Alloc(type\_spec, datatype, name)} als Cotainer für die Parameter dient. \\
  \bottomrule
\end{NiceTabular}
\caption{PicoC-Knoten Teil 3}
\label{tab:picoc_knoten_teil_3}
\end{table}

\begin{table}[H]
  \center
  \begin{NiceTabular}{X[7]X[10]}[rules/color=PrimaryColor]
  \CodeBefore
  \rowcolor{PrimaryColor}{1}
  \rowcolors{2-9}{SecondaryColor}{}[cols={1-2}]
  \Body
  \textbf{\textcolor{white}{PiocC-Knoten}} & \textbf{\textcolor{white}{Beschreibung}} \\
  \smalltt{FunDef(datatype, name, allocs, stmts\_blocks)} & Container für eine \colorbold{Funktionsdefinition}, z.B. \smalltt{<datatype> <fun\_name>(<datatype> <param>) \{<stmts>\}}, wobei \smalltt{datatype} der  \colorbold{Rückgabewert} der Funktion ist,  \smalltt{name} der  \colorbold{Bezeichner} der Funktion ist, \smalltt{allocs} die \colorbold{Parameter} der Funktion sind, wobei der \colorbold{Container-Knoten} \smalltt{Alloc(type\_spec, datatype, name)} als Cotainer für die Parameter dient und \smalltt{stmts\_blocks} eine Liste von \colorbold{Statemetns} bzw.  \colorbold{Blöcken} ist, welche diese Funktion beinhaltet. \\
  \smalltt{NewStackframe(fun\_name, goto\_after\_call)} & Container für die \colorbold{Erstellung} eines neuen \colorbold{Stackframes} und Speicherung des Werts des \smalltt{BAF}-Registers der \colorbold{aufrufenden Funktion} und der \colorbold{Rücksprungadresse} nacheinander an den \colorbold{Anfang} des neuen \colorbold{Stackframes}. Das Attribut \smalltt{fun\_name} stehte dabei für den Bezeichner der Funktion, für die ein neuer \colorbold{Stackframe} erstellt werden soll. Das Attribut \smalltt{fun\_name} dient später dazu den \colorbold{Block} dieser Funktion zu finden, weil dieser für den weiteren Kompiliervorang wichtige Information in seinen \textcolor{gray!90!black}{versteckte Attributen} gespeichert hat. Des Weiteren ist das Attribut \smalltt{goto\_after\_call} ein \smalltt{GoTo(Name('addr@next\_instr'))}, welches später durch die \colorbold{Adresse} des Befehls, der direkt auf die \colorbold{Jump Instruction} folgt, ersetzt wird.\\
  \smalltt{RemoveStackframe()} & Container für das \colorbold{Entfernen} des aktuellen \colorbold{Stackframes} durch das \colorbold{Wiederherstellen} des im noch \colorbold{aktuellen Stackframe} gespeicherten Werts des \smalltt{BAF}-Registes der \colorbold{aufrufenden Funktion} und das Setzen des \smalltt{SP}-Registers auf den Wert des \smalltt{BAF}-Registesr \colorbold{vor} der Wiederherstellung. \\
  \smalltt{File(name, decls\_defs\_blocks)} & Container für alle \colorbold{Funkionen} oder \colorbold{Blöcke}, welche eine Datei als Ursprung haben, wobei \smalltt{name} der \colorbold{Dateiname} der Datei ist, die erstellt wird und \smalltt{decls\_defs\_blocks} eine Liste von \colorbold{Funktionen} bzw. \colorbold{Blöcken} ist. \\
  \smalltt{Block(name, stmts\_instrs, \textcolor{gray!90!black}{instrs\_before}, \textcolor{gray!90!black}{num\_instrs}, \textcolor{gray!90!black}{param\_size}, \textcolor{gray!90!black}{local\_vars\_size})} & Container für \colorbold{Statements}, der auch als \colorbold{Block} bezeichnet wird, wobei das Attribut \smalltt{name} der Bezeichners des \colorbold{Labels} (Definition~\ref{def:label}) des Blocks ist und \smalltt{stmts\_instrs} eine \colorbold{Liste von Statements} oder \colorbold{Instructions}. Zudem besitzt er noch $3$ \textcolor{gray!90!black}{versteckte Attribute}, wobei  \smalltt{instrs\_before} die Zahl der \colorbold{Instructions} vor diesem \colorbold{Block} zählt,  \smalltt{num\_instrs} die Zahl der Instructions ohne Kommentare in diesem Block zählt, \smalltt{param\_size} die voraussichtliche Anzahl an Speicherzellen aufaddiert, die für die \colorbold{Parameter} der Funktion belegt werden müssen und \smalltt{local\_vars\_size} die voraussichtliche Anzahl an Speicherzellen aufaddiert, die für die \colorbold{lokalen Variablen} der Funktion belegt werden müssen. \\
  \smalltt{GoTo(name)} & Container für ein \smalltt{Goto} zu einem anderen \colorbold{Block}, wobei das Attribut \smalltt{name} der Bezeichner des \colorbold{Labels} des Blocks ist zu dem Gesprungen werden soll. \\
  \smalltt{SingleLineComment(prefix, content)} & Container für einen \colorbold{Kommentar}, den der Compiler selber während des \colorbold{Kompiliervorangs} erstellt, der im \colorbold{RETI-Interpreter} selbst später \colorbold{nicht} sichtbar sein wird, aber in den \colorbold{Immediate}-Dateien, welche die \colorbold{Abstract Syntax Trees} nach den verschiedenen \colorbold{Passes} enthalten. \\
  \smalltt{RETIComment(value)} & Container für einen \colorbold{Kommentar} im Code der Form: \smalltt{// \# comment}, der im \colorbold{RETI-Intepreter} später sichtbar sein wird und zur Orientierung genutzt werden kann, allerdings in einer tatsächlichen Implementierung einer \colorbold{RETI-CPU} \colorbold{nicht umsetzbar} ist und auch nicht sinnvoll wäre umzusetzen. Der \colorbold{Kommentar} ist im Attribut \colorbold{value}, welches jeder Knoten besitzt gespeichert. \\
  \bottomrule
\end{NiceTabular}
\caption{PicoC-Knoten Teil 4}
\label{tab:picoc_knoten_teil_4}
\end{table}


\begin{Definition}{Label}{label}
  Durch einen \colorbold{Bezeichner} \colorbold{eindeutig} zuordenbares \colorbold{Sprungziel} im Programmcode.\footcite{thiemann_compilerbau_2021}
\end{Definition}

\begin{Special_Paragraph}
  Die \textcolor{gray!90!black}{ausgegrauten} Attribute der PicoC-Nodes sind \textcolor{gray!90!black}{versteckte Attribute}, die \colorbold{nicht} direkt bei der Erstellung der  \colorbold{PicoC-Nodes} mit einem Wert \colorbold{initialisiert} werden, sondern im \colorbold{Verlauf der Kompilierung} beim Durchlaufen der verschiedenen Passes etwas zugewiesen bekommen, dass im weiteren Kompiliervorgang \colorbold{Informationen} transportiert, die später im Kompiliervorgang nicht mehr so leicht zugänglich wären.

  Jeder \colorbold{Knoten} hat darüberhinaus auch noch $2$ \colorbold{Attribute} \smalltt{value} und \smalltt{position}, wobei \colorbold{value} bei einem \colorbold{Token-Knoten} (Definition~\ref{def:token_knoten}) dem \colorbold{Tokenwert} des Tokens, welches es ersetzt entspricht und bei \colorbold{Container-Knoten} (Definition~\ref{def:container_knoten}) unbesetzt ist. Das \colorbold{Attribut} \smalltt{position} wird später für Fehlermeldungen gebraucht.
\end{Special_Paragraph}

\begin{Definition}{Token-Knoten}{token_knoten}
  Ersetzt ein \colorbold{Token} bei der Generierung des \colorbold{Abstract Syntax Tree}, damit der Zugriff auf Knoten des Abstract Syntax Tree möglichst \colorbold{simpel} ist und keine vermeidbaren Fallunterscheidungen gemacht werden müssen.

  \colorbold{Token-Knoten} entsprechen im Abstract Syntax Tree \colorbold{Blättern}.\footcite{thiemann_compilerbau_2021}
\end{Definition}

\begin{Definition}{Container-Knoten}{container_knoten}
  Dient als \colorbold{Container} für andere \colorbold{Container-Knoten} und \colorbold{Token-Knoten}. Die \colorbold{Container-Knoten} werden optimalerweise immer so gewählt, dass sie \colorbold{mehrere Produktionen} der \colorbold{Konkretten Syntax} abdecken, die einen \colorbold{gleichen Aufbau} haben und sich auch unter einem \colorbold{Überbegriff} zusammenfassen lassen.\footnote{Wie z.B. die verschiedenen \colorbold{Arithmetischen Ausdrücke}, wie z.B. \smalltt{1 {\%} 3} und \colorbold{Logischen Ausdrücke}, wie z.B. \smalltt{1 \&\& 2 < 3}, die einen gleichen Aufbau haben mit immer einer \colorbold{Operation} in der Mitte haben und $2$ \colorbold{Operanden} auf beiden Seiten und sich unter dem Überbegriff \colorbold{Binäre Operationen} zusammenfassen lassen.}

  \colorbold{Container-Knoten} entsprechen im Abstract Syntax Tree \colorbold{Inneren Knoten}.\footcite{thiemann_compilerbau_2021}
\end{Definition}

\subsubsection{RETI-Knoten}
\begin{table}[H]
  \center
  \begin{NiceTabular}{X[8]X[10]}[rules/color=PrimaryColor]
  \CodeBefore
  \rowcolor{PrimaryColor}{1}
  \rowcolors{2-16}{SecondaryColor}{}[cols={1-2}]
  \Body
  \textbf{\textcolor{white}{RETI-Knoten}} &	\textbf{\textcolor{white}{Beschreibung}} \\
  \smalltt{Program(name, instrs)} & Container für alle \colorbold{Instructions}: \smalltt{<name> <instrs>}, wobei \smalltt{name} der \colorbold{Dateiname} der Datei ist, die erstellt wird und \smalltt{instrs} eine \colorbold{Liste von Instructions} ist. \\
  \smalltt{Instr(op, args)} & Container für eine \colorbold{Instruction}: \smalltt{<op> <args>}, wobei \smalltt{op} eine \colorbold{Operation} ist und  \smalltt{args} eine \colorbold{Liste von Argumenten} für dieser Operation. \\
  \smalltt{Jump(rel, im\_goto)} & Container für eine \colorbold{Jump-Instruction}: \smalltt{JUMP<rel> <im>}, wobei \smalltt{rel} eine \colorbold{Relation} ist und \smalltt{im\_goto} ein \colorbold{Immediate Value} \smalltt{Im(val)} für die \colorbold{Anzahl an Speicherzellen}, um die relativ zur \colorbold{Jump-Instruction} gesprungen werden soll oder ein \smalltt{GoTo(Name('block.xyz'))}, das später im \colorbold{RETI-Patch Pass} durch einen passenden \colorbold{Immediate Value} ersetzt wird. \\
  \smalltt{Int(num)} & Container für einen \colorbold{Interruptaufruf}: \smalltt{INT <im>}, wobei \smalltt{num} die \colorbold{Interrruptvektornummer} (IVN) für die passende Speicherzelle in der \colorbold{Interruptvektortabelle} ist, in der die Adresse der \colorbold{Interrupt-Service-Routine} (ISR) steht. \\
  \smalltt{Call(name, reg)} & Container für einen \colorbold{Prozeduraufruf}: \smalltt{CALL <name> <reg>}, wobei \smalltt{name} der \colorbold{Bezeichner} der  Prozedur, die aufgerufen werden soll ist und \smalltt{reg} ein \colorbold{Register} ist, das als \colorbold{Argument} an die Prozedur dient. Diese \colorbold{Operation} ist in der Betriebssysteme Vorlesung\tabularnote{\cite{scholl_betriebssysteme_2020}} nicht deklariert, sondern wurde dazuerfunden, um unkompliziert ein  \smalltt{CALL PRINT ACC} oder  \smalltt{CALL INPUT ACC} im RETI-Interpreter simulieren zu können. \\
  \smalltt{Name(val)} & Bezeichner für eine \colorbold{Prozedur}, z.B. \smalltt{PRINT} oder \smalltt{INPUT} oder den \colorbold{Programnamen}, z.B. \smalltt{PROGRAMNAME}. Dieses \colorbold{Argument} ist in der Betriebssysteme Vorlesung\tabularnote{\cite{scholl_betriebssysteme_2020}} nicht deklariert, sondern wurde dazuerfunden, um Bezeichner, wie \smalltt{PRINT}, \smalltt{INPUT} oder \smalltt{PROGRAMNAME} schreiben zu können. \\
  \smalltt{Reg(reg)} & Container für ein \colorbold{Register}. \\
  \smalltt{Im(val)} & Ein \colorbold{Immediate Value}, z.B. \smalltt{42, -3} usw. \\
  \smalltt{Add(), Sub(), Mult(), Div(), Mod(), Oplus(), Or(), And()} & \colorbold{Compute-Memory} oder \colorbold{Compute-Register} Operationen: \smalltt{ADD, SUB, MULT, DIV, OPLUS, OR, AND}.\\
  \smalltt{Addi(), Subi(), Multi(), Divi(), Modi(), Oplusi(), Ori(), Andi()} & \colorbold{Compute-Immediate} Operationen: \smalltt{ADDI, SUBI, MULTI, DIVI, MODI, OPLUSI, ORI, ANDI}.\\
  \smalltt{Load(), Loadin(), Loadi()} & \colorbold{Load} Operationen: \smalltt{LOAD, LOADIN, LOADI}.\\
  \smalltt{Store(), Storein(), Move()} & \colorbold{Store} Operationen: \smalltt{STORE, STOREIN, MOVE}.\\
  \smalltt{Lt(), LtE(), Gt(), GtE(), Eq(), NEq(), Always(), NOp()} & \colorbold{Relationen}: \smalltt{<, <=, >, >=, ==, !=, \_NOP}.\\
  \smalltt{Rti()} & \colorbold{R}eturn-\colorbold{F}rom-\colorbold{I}nterrupt Operation: \smalltt{RTI}.\\
  \smalltt{Pc(), In1(), In2(), Acc(), Sp(), Baf(), Cs(), Ds()} & \colorbold{Register}: \smalltt{PC, IN1, IN2, ACC, SP, BAF, CS, DS}.\\
  \bottomrule
\end{NiceTabular}
\caption{RETI-Knoten}
\label{tab:reti_knoten}
\end{table}
% Tabelle aller RETI Knoten
% Transformer erwähnen

\subsubsection{Kompositionen von PicoC-Knoten und RETI-Knoten mit besonderer Bedeutung}
Hier sind jegliche \colorbold{Kompositionen} von \colorbold{PicoC-Knoten} und \colorbold{RETI-Knoten} aufgelistet, die eine \colorbold{besondere Bedeutung} haben und nicht bereits in der \colorbold{Abstrakten Syntax}~\ref{gr:concrete_syntax_parser} enthalten sind.
% (generelle Aufgaben von Exp und Reg)

\begin{table}[H]
  \center
  \begin{NiceTabular}{X[16]X[20]}[rules/color=PrimaryColor]
  \CodeBefore
  \rowcolor{PrimaryColor}{1}
  \rowcolors{2-15}{SecondaryColor}{}[cols={1-2}]
  \Body
  \textbf{\textcolor{white}{Komposition}} &	\textbf{\textcolor{white}{Beschreibung}} \\
  \smalltt{Ref(Global(Num('addr')))}	& Speichert \colorbold{Adresse} der Speicherzelle, die \smalltt{Num('addr')} Speicherzellen \colorbold{relativ} zum \colorbold{Datensegment Register} \smalltt{DS} steht auf den \colorbold{Stack}. \\
  \smalltt{Ref(Stackframe(Num('addr')))} & Speichert \colorbold{Adresse} der Speicherzelle, die \smalltt{Num('addr')} Speicherzellen \colorbold{relativ} zum \colorbold{Begin-Aktive-Funktion Register} \smalltt{BAF} steht auf den \colorbold{Stack}. \\
  \smalltt{Ref(Subscr(Stack(Num('addr1')), Stack(Num('addr2'))))} & Berechnet die nächste \colorbold{Adresse} aus der \colorbold{Adresse}, die an Speicherzelle \smalltt{Stack(Num('addr1'))} steht und dem \colorbold{Subscript Index}, der an Speicherzelle \smalltt{Stack(Num('addr2'))} steht und speichert diese auf den \colorbold{Stack}. Die Berechnung ist abhängig davon ob der \colorbold{Datentyp} \smalltt{ArrayDecl(datatype)} oder \smalltt{PntrDecl(datatype)} ist. Der \colorbold{Datentyp} ist ein \textcolor{gray!90!black}{verstecktes Attribut} von \smalltt{Ref(exp)}. \\
  \smalltt{Ref(Attr(Stack(Num('addr1')), Name('attr')))} & Berechnet die nächste \colorbold{Adresse} aus der \colorbold{Adresse}, die an Speicherzelle \smalltt{Stack(Num('addr1'))} steht und dem \colorbold{Attributnamen} \smalltt{Name('attr')} und speichert diese auf den \colorbold{Stack}. Zur Berechnung ist der Name des \colorbold{Struct} in \smalltt{StructSpec(Name('st'))} notwendig, dessen \colorbold{Attribut} \smalltt{Name('attr')} ist. \smalltt{StructSpec(Name('st'))} ist ein \textcolor{gray!90!black}{verstecktes Attribut} von  \smalltt{Ref(exp)}. \\
  \smalltt{Assign(Stack(Num('size'))), Global(Num('addr')))} & Schreibt \smalltt{Num('size')} viele Speicherzellen, die ab \smalltt{Global(Num('addr'))} relativ zum \colorbold{Datensegment Register}  \smalltt{DS} stehen, versetzt genauso auf den \colorbold{Stack}. \\
  \smalltt{Assign(Stack(Num('size')), Stackframe(Num('addr')))} & Schreibt \smalltt{Num('size')} viele Speicherzellen, die ab \smalltt{Stackframe(Num('addr'))} relativ zum \colorbold{Begin-Aktive-Funktion Register} \smalltt{BAF} stehen, versetzt genauso auf den \colorbold{Stack}. \\
  \smalltt{Exp(Global(Num('addr'))} & Speichert \colorbold{Inhalt} der Speicherzelle, die \smalltt{Num('addr')} Speicherzellen \colorbold{relativ} zum \colorbold{Datensegment Register} \smalltt{DS} steht auf den \colorbold{Stack}. \\
  \smalltt{Exp(Stackframe(Num('addr'))} & Speichert \colorbold{Inhalt} der Speicherzelle, die \smalltt{Num('addr')} Speicherzellen \colorbold{relativ} zum \colorbold{Begin-Aktive-Funktion Register} \smalltt{BAF} steht auf den \colorbold{Stack}. \\
  \smalltt{Exp(Stack(Num('addr')))} & Speichert \colorbold{Inhalt} der Speicherzelle, die \smalltt{Num('addr')} Speicherzellen \colorbold{relativ} zum \colorbold{Stackpointer Register}  \smalltt{SP} steht auf den \colorbold{Stack}. \\
  \smalltt{Assign(Stack(Num('addr1')), Stack(Num('addr2')))} & Speichert \colorbold{Inhalt} der Speicherzelle \smalltt{Stack(Num('addr2'))}, die \smalltt{Num('addr2')} Speicherzellen \colorbold{relativ} zum \colorbold{Stackpointer Register}  \smalltt{SP} steht an der \colorbold{Adresse} in der Speicherzelle, die \smalltt{Num('addr1')} Speicherzellen \colorbold{relativ} zum \colorbold{Stackpointer Register}  \verb|SP| steht. \\
  \smalltt{Assign(Global(Num('addr')), Stack(Num('size')))} & Schreibt \smalltt{Num('size')} viele Speicherzellen, die auf dem \colorbold{Stack} stehen, versetzt genauso auf die Speicherzellen ab \smalltt{Num('addr')} \colorbold{relativ} zum \colorbold{Datensegment Register}  \smalltt{DS}. \\
  \smalltt{Assign(Stackframe(Num('addr')), Stack(Num('size')))} & Schreibt \smalltt{Num('size')} viele Speicherzellen, die auf dem \colorbold{Stack} stehen, versetzt genauso auf die Speicherzellen ab \smalltt{Num('addr')} \colorbold{relativ} zum \colorbold{Begin-Aktive-Funktion Register} \smalltt{BAF}. \\
  \smalltt{Exp(Reg(reg))} & Schreibt den aktuellen Wert des \colorbold{Registers} \smalltt{reg} auf den \colorbold{Stack}. \\
  \smalltt{Instr(Loadi(), [Reg(Acc()), GoTo(Name('addr@next\_instr'))])} & Lädt in das Register \verb|ACC| die \colorbold{Adresse} der Instruction, die in diesem Kontext direkt nach dem Sprung zum Block einer anderen Funktion steht.\\
  \bottomrule
\end{NiceTabular}
\caption{Kompositionen von PicoC-Knoten und RETI-Knoten mit besonderer Bedeutung}
\label{tab:kompositionen_von_picoc_knoten_und_reti_knoten_mit_besonderer_bedeutung}
\end{table}

\begin{Special_Paragraph}
  Um die obige Tabelle~\ref{tab:kompositionen_von_picoc_knoten_und_reti_knoten_mit_besonderer_bedeutung} nicht mit unnötig viel repetetiven Inhalt zu füllen, wurden die zahlreichen Kompostionen \colorbold{ausgelassen}, bei denen einfach nur \verb|exp| durch $\mathtt{Stack(Num('x')), x}\in\mathbb{N}$ ersetzt wurde.

  Zudem sind auch jegliche Kombinationen ausgelassen, bei denen einfach nur eine \colorbold{Expression} an ein \smalltt{Exp(exp)} bzw.  \smalltt{Ref(exp)} drangehängt wurde.
\end{Special_Paragraph}

\subsubsection{Abstrakte Syntax}
\label{sec:abstrakte_syntax}

% ./abstract_syntax.txt
\newpage

\begin{grammar}[Abstrakte Syntax der Sprache $L_{PiocC}$][H][gr:abstract_syntax_l_picoc]
  \toprule
  \commentsecond
  \midrule
  \arith
  \midrule
  \logic
  \midrule
  \assign
  \midrule
  \pntr
  \midrule
  \arraysecond
  \midrule
  \struct
  \midrule
  \ifelse
  \midrule
  \loopsecond
  \midrule
  \fun
  \midrule
  \file
  \bottomrule
\end{grammar}

\begin{Special_Paragraph}
  Man spricht hier von der \colorbold{\enquote{Abstrakten Syntax der Sprache $L_{PicoC}$}} und meint hier mit der Sprache $L_{PicoC}$ \colorbold{nicht} die Sprache, welche durch die \colorbold{Abstrakte Syntax} beschrieben wird. Es ist damit \colorbold{immer} die Sprache gemeint, die \colorbold{kompiliert} werden soll und zu deren Zweck die \colorbold{Abstrakt Syntax} überhaupt definiert wird. Für die tatsächliche Sprache, die durch die \colorbold{Abstrakt Syntax} beschrieben wird, interessiert man sich nie wirklich explizit. Diese \colorbold{Redeart} wurde aus der \colorbold{Quelle} \cite{g_siek_course_2022} übernommen.
\end{Special_Paragraph}

% $L_{X}$ ist nicht notwendig sich zu überlegen, hier so getann als gäbe es diese Sprache
% Abstrakte Syntax ist für den Programmierer als Orientierungshilfe bei der Implementierung

\subsubsection{Transformer}
% Vielleicht Sache mit ToBool erwähnen aber vielleicht reicht es schon in der entsprechenden Tabelle
% Sache mit Deref und (), leftmost node
% Umdrehen von ArrayDecl und PntrDecl

\subsubsection{Ausgabe von Abstract Syntax Trees}
\label{sec:ausgabe_von_abstract_syntax_trees}

Ein \colorbold{Knoten} eines \colorbold{Abstract Syntax Tree} kann entweder in der \colorbold{Konkretter Syntax} der Sprache, für dessen Kompilierung er generiert wurde oder in der \colorbold{Abstrakter Syntax}, die beschreibt, wie der Abstract Syntax Tree selbst aufgebaut sein darf ausgegeben werden.

Das Ausgeben eines \colorbold{Abstract Syntax Trees} wird im \colorbold{PicoC-Compiler} über die \colorbold{Magische Methode} \smalltt{\_\_repr\_\_()}\footnote{Spezielle Methode, die immer aufgerufen wird, wenn das \colorbold{Object}, dass in Besitz dieser Methode ist als \colorbold{String} mittels \smalltt{print()} oder zur \colorbold{Repräsentation} ausgegeben werden soll.} der Programmiersprache \colorbold{Python} umgesetzt. Sobald ein \colorbold{PicoC-Knoten} oder \colorbold{RETI-Knoten} ausgegeben werden soll, gibt seine Magische Methode \smalltt{\_\_repr\_\_()} eine nach der \colorbold{Abstrakten} oder \colorbold{Konkretten Syntax} aufgebaute \colorbold{Textrepräsentation} seiner selbst und all seiner Knoten mit an den richtigen Stellen passend gesetzten \colorbold{runden} \colorbold{öffnenden} \smalltt{(} und \colorbold{schließenden} \smalltt{)} \colorbold{Klammern}, sowie \colorbold{Kommas} \smalltt{','}, \colorbold{Semikolons} \smalltt{;} usw. zur Darstellung der \colorbold{Hierarchie} und zur \colorbold{Abtrennung} zurück. Dabei wird nach dem \colorbold{Depth-First-Search} Schema der gesamte \colorbold{Abstract Sybtax Tree} durchlaufen und die Magische \smalltt{\_\_repr\_\_()}-Methode der verschiedenen Knoten aufgerufen, die immer jeweils die \smalltt{\_\_repr\_\_()}-Methode ihrer Kinder aufrufen und die zurückgegebene \colorbold{Textrepräsentation} passend \colorbold{zusammenfügen} und selbst \colorbold{zurückgeben}.

Im \colorbold{PicoC-Compiler} wurden \colorbold{Abstrakte} und \colorbold{Konkrette Syntax} miteinander gemischt. Für \colorbold{PicoC-Knoten} wurde die \colorbold{Abstrakte Syntax} verwendet, da Passes schließlich auf \colorbold{Abstract Syntax Trees} operieren. Bei \colorbold{RETI-Knoten} wurde die \colorbold{Konkrette Syntax} verwendet, da \colorbold{Maschienenbefehle} in \colorbold{Konkretter Syntax} schließlich das \colorbold{Endprodukt} des Kompiliervorgangs sein sollen. Da die \colorbold{Abstrakte Syntax} von \colorbold{RETI-Knoten} so simpel ist, macht es kaum einen Unterschied in der Erkennbarkeit, bis auf fehlende gescheifte Klammern \smalltt{()} usw., ob man die \colorbold{RETI-Knoten} in \colorbold{Abstrakter} oder \colorbold{Konkretter} Syntax schreibt. Daher kann man auch einfach gleich die \colorbold{RETI-Knoten} in \colorbold{Konkretter Syntax} ausgeben und muss nicht beim letzten \colorbold{Pass} daran denken, am Ende die \colorbold{Konkrette}, statt der \colorbold{Abstrakten Syntax} für die \colorbold{RETI-Knoten} auszugeben.

\subsubsection{Codebeispiel}
Das Beispiel welches in Subkapitel~\ref{sec:derivation_tree_generierung} angefangen wurde, wird hier fortgeführt.
% Transformer erwähnen, viel zu viel um es hier reinzumachen

\begin{code}
  \centering
  \numberedcodebox[minted language=text]{./code_examples/example_dt_simple_ast_gen_array_decl_and_alloc.ast}
  \caption{Abstract Syntax Tree aus vereinfachtem Derivarion Tree generiert}
  \label{code:abstract_syntax_tree_aus_vereinfachtem_derivarion_tree_generiert}
\end{code}

\section{Code Generierung}
\subsection{Übersicht}
Nach der Generierung eines \colorbold{Abstract Syntax Tree} als Ergebnis der \colorbold{Lexikalischen} und \colorbold{Syntaktischen Analyse} in Unterkapitel~\ref{sec:syntaktische_analyse}, wird in diesem Kapitel mit den verschiedenen \colorbold{Kompositionen} von \colorbold{Container-Knoten} und \colorbold{Token-Knoten} im \colorbold{Abstract Syntax Tree} als Basis das gewünschte Endprodukt des \colorbold{PicoC-Compilers}, der \colorbold{RETI-Code} generiert.

Man steht nun dem Problem gegenüber einen \colorbold{Abstract Syntax Tree} der Sprache $L_{PicoC}$, der durch die \colorbold{Abstrakte Syntax} in Grammatik~\ref{gr:abstract_syntax_l_picoc} spezifiziert ist in einen entsprechenden \colorbold{Abstract Syntax Tree} der Sprache $L_{RETI}$ umzuformen. Das ganze lässt sich, wie in Unterkapitel~\ref{sec:code_generierung} bereits beschrieben vereinfachen, indem man dieses Problem in mehrere \colorbold{Passes} (Definition~\ref{def:pass}) herunterbricht.

% Unterscheid zur Architektur aus dem Bachelorprojekt

Beim \colorbold{PicoC-Compiler} handelt es sich um einen \colorbold{Cross-Compiler} (Definiton~\ref{def:cross_compiler}). Damit \colorbold{RETI-Code} erzeugt werden kann, der auf der \colorbold{RETI-Architektur} läuft, muss erst, wie im \colorbold{T-Diagram} (siehe Unterkapitel~\ref{sec:t_diagram}) in Abbildung~\ref{fig:cross_compiler_kompiliervorgang_ausgeschrieben} zu sehen ist, der \colorbold{Python-Code} des \colorbold{PicoC-Compilers} mittels eines Compilers, der z.B. auf einer $\mathtt{X_{86\_64}}$-Architektur laufen könnte zu \colorbold{Bytecode} kompiliert werden. Dieser \colorbold{Bytecode} wird dann von der \colorbold{Python-Virtual-Machine} (PVM) interpretiert, welche wiederum auf einer $\mathtt{X_{86\_64}}$-Architektur laufen könnte. Und selbst dieses \colorbold{T-Diagram} könnte noch ausführlicher ausgedrückt werden, indem nachgeforscht wird, in welcher Sprache eigentlich die \colorbold{Python-Virtual-Machine} geschrieben war, bevor sie zu $\mathtt{X_{86\_64}}$ kompiliert wurde usw.

% Cross Compiler
% https://tex.stackexchange.com/questions/8625/force-figure-placement-in-text
\begin{figure}[H]
  \centering
  \includegraphics[width=0.5\linewidth]{./figures/summarized_cross_compiler.png}
  \caption{Cross-Compiler Kompiliervorgang ausgeschrieben}
  \label{fig:cross_compiler_kompiliervorgang_ausgeschrieben}
\end{figure}

Dieses längliche \colorbold{T-Diagram} in Abbildung~\ref{fig:cross_compiler_kompiliervorgang_ausgeschrieben} lässt sich zusammenfassen, sodass man das \colorbold{T-Diagram} in Abbildung~\ref{fig:cross_compiler_kompiliervorgang_kurzform} erhält, in welcher direkt angegeben ist, dass der \colorbold{PicoC-Compiler} in $\mathtt{X_{86\_64}}$-Maschienensprache geschrieben ist.

\begin{figure}[H]
  \centering
  \includegraphics[width=0.33\linewidth]{./figures/compiliervorang_mit_machiene.png}
  \caption{Cross-Compiler Kompiliervorgang Kurzform}
  \label{fig:cross_compiler_kompiliervorgang_kurzform}
\end{figure}

Nachdem der Kompilierprozess des \colorbold{PicoC-Compiler} im \colorbold{vertikalen} nun genauer angesehen wurde, wird der Kompilierprozess im Folgenden im \colorbold{horinzontalen}, auf der Ebene der verschiedenen \colorbold{Passes} genauer betrachtet. Die Abbildung~\ref{fig:architektur_mit_allen_passes_ausgeschrieben} gibt einen guten Überblick über alle \colorbold{Passes} und wie diese in der \colorbold{Pipe-Architektur} (Definition~\ref{def:pipe_architektur}) des \colorbold{PicoC-Compilers} aufeinanderfolgen. In der \colorbold{Pipe-Architektur} nutzt der jeweils nächste \colorbold{Pass} den generierten \colorbold{Abstract Syntax Tree} des vorherigen Passes oder der Syntaktischen Analyse, um einen eigenen \colorbold{Abstract Syntax Tree} in seiner eigenen \colorbold{Sprache} zu generieren.

\begin{figure}[H]
  \centering
  \includegraphics[width=\linewidth]{./figures/passes.png}
  \caption{Architektur mit allen Passes ausgeschrieben}
  \label{fig:architektur_mit_allen_passes_ausgeschrieben}
\end{figure}

Im Unterkapitel~\ref{sec:passes} werden die unterschiedlichen \colorbold{Passes} des PicoC-Compilers erklärt. In den darauffolgenden Unterkapiteln~\ref{sec:umsetzung_von_pointern},~\ref{sec:umsetzung_von_arrays},~\ref{sec:umsetzung_von_structs}~und~\ref{sec:umsetzung_von_funktionen} zu \colorbold{Pointern},  \colorbold{Arrays}, \colorbold{Structs} und \colorbold{Funktionen} werden einzelne \colorbold{Aspekte}, die Thema dieser \colorbold{Bachelorarbeit} sind \colorbold{genauer betrachtet} und erklärt, die im Unterkapitel~\ref{sec:passes} nicht ausreichend vertieft wurden. Viele der verwendenten \colorbold{Ansätze} zur Lösung dieser Probleme basieren auf der Vorlesung~\cite{scholl_betriebssysteme_2020} und wurden in dieser Bachelorarbeit weiter ausgearbeitet, wo es nötig war, sodass diese mit dem \colorbold{PicoC-Compiler} auch in der \colorbold{Praxis} implementiert werden konnten.

% https://tex.stackexchange.com/questions/167380/how-to-refer-to-a-footnote
Um die verschiedenen Aspekte besser erklären zu können, werden \colorbold{Codebeispiele} verwendet, in welchen ein kleines repräsentatives \colorbold{PicoC-Programm} für einen spezifischen Aspekt in wichtigen \colorbold{Zwischenstadien der Kompilierung} gezeigt wird\footnote{Also die verschiedenen in den \colorbold{Passes} generierten \colorbold{Abstract Syntax Trees}, sofern der \colorbold{Pass} für den gezeigten Aspekt relevant ist.}. Die \colorbold{Codebeispiele} wurden alle mit dem \colorbold{PicoC-Compiler} kompiliert und danach \colorbold{nicht} mehr \colorbold{verändert}, also genauso, wie der \colorbold{PicoC-Compiler} sie kompiliert aus den Dateien in dieses Dokument eingelesen. Alle hier zur Repräsentation verwendeten \colorbold{PicoC-Programme} lassen sich unter dem \footnoteurl{https://github.com/matthejue/Bachelorarbeit/tree/master/code_examples} finden und mithilfe der im Ordner \inlinebox{/code_examples} beiliegenden \inlinebox{Makefile} und dem Befehl \inlinebox*{make compile-all} genauso \colorbold{kompilieren}, wie sie hier dargestellt sind\footnote{Es wurden zu diesem Zweck spezielle neue \colorbold{Command-line Optionen} erstellt, die bestimmte Kommentare \colorbold{herausfiltern} und manche Container-Knoten \colorbold{einzeilig} machen, damit die generierten \colorbold{Abstract Syntax Trees} in den verscchiedenen Zwischenstufen der Kompilierung \colorbold{nicht} zu langgestreckt und \colorbold{überfüllt} mit Kommentaren sind.}.

\subsection{Passes}
\label{sec:passes}

Im Folgenden werden die verschiedenen \colorbold{Passes} des \colorbold{PicoC-Compilers} für die Generierung von \colorbold{RETI-Code} besprochen. Viele dieser \colorbold{Passes} haben \colorbold{Aufgaben}, die eher unter die Themenbereiche des \colorbold{Bachelorprojekts} fallen. Allerdings ist das Verständnis der \colorbold{Passes} auch für das Verständnis der veschiedenen Aspekte\footnote{In kurz: \colorbold{Pointer}, \colorbold{Arrays}, \colorbold{Strcuts} und \colorbold{Funktionen}.} der \colorbold{Bachelorarbeit} wichtig.

Auf jedes Detail der einzelnen \colorbold{Passes} wird in diesem Unterkapitel allerdings nicht eingegangen, da diese einerseits in den Unterkapiteln~\ref{sec:umsetzung_von_pointern},~\ref{sec:umsetzung_von_arrays},~\ref{sec:umsetzung_von_structs}~und~\ref{sec:umsetzung_von_funktionen} zu \colorbold{Pointern},  \colorbold{Arrays}, \colorbold{Structs} und \colorbold{Funktionen} im Detail erklärt sind und andererseits viele Aufgaben dieser \colorbold{Passes} eher dem \colorbold{Bachelorprojekt} zuzurechnen sind.

\subsubsection{PicoC-Shrink Pass}
\label{picoc_shrink_pass}
\newlineparagraph{Aufgabe}
\label{sec:picoc_shrink_pass_zweck}
% dieser Pass existiert nur wegen der Erweiterungen

Der Aufgabe des \colorbold{PicoC-Shrink Pass} ist in Unterkapitel~\ref{dereferenzierung_durch_zugriff_auf_arrayindex_ersetzen} ausführlich an einem Beispiel erklärt. Kurzgefasst hat der \colorbold{PicoC-Shrink Pass} die Aufgabe, die Eigenheit auszunutzen, dass der \colorbold{Dereferenzierungoperator} \smalltt{*pntr} und die damit einhergehende \colorbold{Pointer Arithmetik} \smalltt{*(pntr + i)} sich in der Untermenge der Sprache $L_{C}$, welche die Sprache $L_{PicoC}$ darstellt genau gleich verhält, wie der \colorbold{Operator} für den \colorbold{Zugriff} auf den \colorbold{Index} eines \colorbold{Arrays} \smalltt{ar[i]}.

Daher wandelt der \colorbold{PicoC-Shrink Pass} alle Verwendungen des \colorbold{Knoten} \smalltt{Deref(exp, i)} im jeweiligen \colorbold{Abstract Syntax Tree} in \colorbold{Knoten} \smalltt{Subscr(exp, i)} um, sodass sich dadurch viele vermeidbare \colorbold{Fallunterscheidungen} und \colorbold{doppelter Code} bei der Implementierung vermeiden lassen. Man lässt die  \colorbold{Derefenzierung} \smalltt{*(var + i)} einfach von den Routinen für einen \colorbold{Zugriff auf einen Arrayindex} \smalltt{var[i]} übernehmen.

\newlineparagraph{Abstrakte Syntax}

Die \colorbold{Abstrakte Syntax} der Sprache $L_{PicoC\_Shrink}$ in Tabelle~\ref{gr:abstract_syntax_l_picoc_shrink} ist fast \colorbold{identisch} mit der \colorbold{Abstrakten Syntax} der Sprache $L_{PiocC}$ in Tabelle~\ref{gr:abstract_syntax_l_picoc}, nach welcher der \colorbold{erste} Abstract Syntax Tree in der \colorbold{Syntaktischen Analyse} generiert wurde. Der einzige \colorbold{Unterschied} liegt darin, dass es den Knoten \smalltt{Deref(exp, exp)} in Tabelle~\ref{gr:abstract_syntax_l_picoc_shrink} \colorbold{nicht} mehr gibt. Das liegt daran, dass dieser Pass alle \colorbold{Vorkommnisse} des Knoten \smalltt{Deref(exp, exp)} durch den Knoten \smalltt{Subscr(exp, exp)} auswechselt, der ebenfalls bereits in der \colorbold{Abstrakten Syntax} der Sprache $L_{PicoC}$ definiert ist.

\begin{grammar}[Abstrakte Syntax der Sprache $L_{PiocC\_Shrink}$][H][gr:abstract_syntax_l_picoc_shrink]
  \toprule
  \commentsecond*
  \midrule
  \arith*
  \midrule
  \logic*
  \midrule
  \assign*
  \midrule
  \pntrshrink*
  \midrule
  \arraysecond*
  \midrule
  \struct*
  \midrule
  \ifelse*
  \midrule
  \loopsecond*
  \midrule
  \fun*
  \midrule
  \file*
  \bottomrule
\end{grammar}

\begin{Special_Paragraph}
  Der \textcolor{red}{rot} markierte Knoten bedeutet, dass dieser im Vergleich zur voherigen \colorbold{Abstrakten Syntax} nicht mehr da ist.
\end{Special_Paragraph}

\newlineparagraph{Codebeispiel}

In den nächsten Unterkapiteln wird das Beispiel in Code~\ref{code:picoc_code_für_codebeispiel} zur \colorbold{Anschauung} der verschiedenen \colorbold{Passes} verwendet. Im Code~\ref{code:picoc_code_für_codebeispiel} ist in der Funktion \smalltt{faculty} ein \colorbold{iterativer} Algorithmus implementiert, der die \colorbold{Fakultät} eines übergebenen \colorbold{Arguments} berechnet. Der Algorithmus basiert auf einem \colorbold{Beispielprogramm} aus der Vorlesung~\cite{scholl_betriebssysteme_2020}, welcher in der Vorlesung allerdings \colorbold{rekursiv} implementiert ist.
% natürlich beide Beispiele als Tests verfügbar

Dieser \colorbold{rekursive} Algoirthmus ist allerdings \colorbold{kein} gutes \colorbold{Anschaungsbeispiel}, dass viele der Aufgaben der verschiedenen \colorbold{Passes} bei der Kompilierung veranschaulicht hätte. Viele Aufgaben der \colorbold{Passes}, wie z.B. bei der Kompilierung von \smalltt{if}-, \smalltt{if-else}-, \smalltt{while}- und \smalltt{do-while}-Statements wären im Beispiel aus der Vorlesung \colorbold{nicht} enthalten gewesen. Daher wurde das Beispiel aus der Vorlesung zu einem \colorbold{iterativen} Algorithmus~\ref{code:picoc_code_für_codebeispiel} umgeschrieben, um \smalltt{if}- und \smalltt{while}-Statemtens zu enthalten.

Beide Varianten des \colorbold{Algorithmus} wurden zum \colorbold{Testen} des PicoC-Compilers verwendet und sind als Tests im Ordner \inlinebox{/tests} unter \footnoteurl{https://github.com/matthejue/PicoC-Compiler/tree/new_architecture/tests}, unter den Testbezeichnungen \inlinebox{example_faculty_rec.picoc} und \inlinebox{example_faculty_it.picoc} zu finden.

Die Codebeispiele in diesem und den folgenden Unterkapiteln dienen allerdings nur als \colorbold{Anschauung} des jeweiligen \colorbold{Passes}, der in diesem Unterkapitel beschrieben wird und werden nicht im Detail erläutert, da viele Details der Passes später in den Unterkapiteln~\ref{sec:umsetzung_von_pointern},~\ref{sec:umsetzung_von_arrays},~\ref{sec:umsetzung_von_structs}~und~\ref{sec:umsetzung_von_funktionen} zu \colorbold{Pointern},  \colorbold{Arrays}, \colorbold{Structs} und \colorbold{Funktionen} mit eigenen \colorbold{Codebeispielen} erklärt werden und alle sonstigen Details dem \colorbold{Bachelorprojekt} zuzurechnen sind.

\begin{code}
  \centering
  \numberedcodebox[minted language=c]{./code_examples/example_faculty_it.picoc}
  \caption{PicoC Code für Codebespiel}
  \label{code:picoc_code_für_codebeispiel}
\end{code}

In Code~\ref{code:abstract_syntax_tree_für_codebeispiel} sieht man den \colorbold{Abstract Syntax Tree}, der in der \colorbold{Syntaktischen Analyse} generiert wurde.

\begin{code}
  \centering
  \numberedcodebox[minted language=text]{./code_examples/example_faculty_it.ast}
  \caption{Abstract Syntax Tree für Codebespiel}
  \label{code:abstract_syntax_tree_für_codebeispiel}
\end{code}

Im \colorbold{PicoC-Shrink-Pass} ändert sich nichts im Vergleich zum \colorbold{Abstract Syntax Tree} in Code~\ref{code:abstract_syntax_tree_für_codebeispiel}, da das Codebeispiel keine \colorbold{Dereferenzierung} enthält.

% TODO: nichts hinzugefügt zu Syntax

\subsubsection{PicoC-Blocks Pass}
\label{picoc_blocks_pass}
\newlineparagraph{Aufgabe}
\label{sec:picoc_blocks_pass_zweck}

Die Aufgabe des \colorbold{PicoC-Blocks Passes} ist es die Knoten \smalltt{If(exp, stmts)}, \smalltt{IfElse(exp, stmts1, stmts2)}, \smalltt{While(exp, stmts)} und \smalltt{DoWhile(exp, stmts)} mithilfe von \smalltt{Block(name, stmts\_instrs}-, \smalltt{GoTo(lable)}- und \smalltt{IfElse(exp, stmts1, stmts2)}-Knoten umzusetzen. Der \smalltt{IfElse(exp, stmts1, stmts2)}-Knoten wird zur Umsetzung der \colorbold{Bedingung} verwendet und es wird, je nachdem, ob die Bedingung \colorbold{wahr} oder \colorbold{falsch} ist mithilfe der \smalltt{GoTo(label)}-Knoten in einen von zwei \colorbold{alternativen Branches} gesprungen oder ein \colorbold{Branch} erneut aufgerufen usw.

\newlineparagraph{Abstrakte Syntax}

Zur Umsetzung dieses Passes ist es notwendig die \colorbold{Abstrakte Syntax} der Sprache $L_{PicoC\_Shrink}$ in Tabelle~\ref{gr:abstract_syntax_l_picoc_shrink} um die Knoten zu erweitern, die im Unterkapitel \ref{sec:picoc_blocks_pass_zweck} erwähnt wurden. Die Knoten \smalltt{If(exp, stmts)}, \smalltt{While(exp, stmts)} und \smalltt{DoWhile(exp, stmts)} gibt es nicht mehr, da sie durch \smalltt{Block(name, stmts\_instrs}-, \smalltt{GoTo(lable)}- und \smalltt{IfElse(exp, stmts1, stmts2)}-Knoten ersetzt wurden. Die \colorbold{Funktionsdefinition} \smalltt{FunDef(⟨datatype⟩, Name(str), Alloc(Writeable(), ⟨datatype⟩, Name(str))*, ⟨block⟩*)} ist nun ein Container für \colorbold{Blöcke} \smalltt{Block(Name(str), ⟨stmt⟩*)} und keine Statements \smalltt{stmt} mehr. Das resultiert in der \colorbold{Abstrakten Syntax} der Sprache $L_{PicoC\_Blocks}$ in Tabelle~\ref{gr:abstract_syntax_l_picoc_blocks}.

\begin{grammar}[Abstrakte Syntax der Sprache $L_{PiocC\_Blocks}$][H][gr:abstract_syntax_l_picoc_blocks]
  \toprule
  \commentsecond*
  \midrule
  \arith*
  \midrule
  \logic*
  \midrule
  \assign*
  \midrule
  \pntrshrinkafter*
  \midrule
  \arraysecond*
  \midrule
  \struct*
  \midrule
  \ifelseblocks*
  \midrule
  \loopblocks*
  \midrule
  \funafter*
  \midrule
  \block
  \midrule
  \file*
  \bottomrule
\end{grammar}

\begin{Special_Paragraph}
  Alles \textcolor{red}{rot} markierte bedeutet, es wurde \colorbold{entfernt} oder \colorbold{abgeändert}. Alles \textcolor{gray!90!black}{ausgegraute} bedeutet, es hat sich im Vergleich zur letzten Abstrakten Syntax \colorbold{nichts} geändert. Alle normal in \smalltt{schwarz} geschriebenen Knoten wurden \colorbold{neu} hinzugefügt.

  Die \colorbold{Abstrakte Syntax} soll im Gegensatz zur \colorbold{Konkretten Syntax} meist nur vom \colorbold{Programmierer} verstanden werden, der den Compiler implementiert und sollte daher vor allem \colorbold{einfach verständlich} sein und stellt daher eine \colorbold{Obermenge} aller tatsächlich möglichen \colorbold{Kompositionen} von \colorbold{Knoten} dar\footnote{D.h. auch wenn dort \smalltt{exp} als \colorbold{Attribut} steht, kann dort \colorbold{nicht} jeder Knoten, der sich aus der \colorbold{Produktion} \smalltt{exp} ergibt auch wirklich eingesetzt werden.}.

  Man bezeichnet hier die \colorbold{Abstrakte Syntax} als \colorbold{\enquote{Abstrakte Syntax der Sprache $L_{Picoc\_Blocks}$}}. Diese Sprache $L_{Picoc\_Blocks}$ wird durch eine \colorbold{Konkrette Syntax} beschrieben, die allerdings nicht weiter relevant ist, da in den \colorbold{Passes} nur \colorbold{Abstract Syntax Trees} umgeformt werden. Es ist hierbei nur wichtig zu wissen, dass die \colorbold{Abstrakte Syntax} theoretisch zur Kompilierung der Sprache $L_{Picoc\_Blocks}$ definiert ist, also die Sprache $L_{PicoC\_Blocks}$ nicht die Sprache ist, die von der \colorbold{Abstrakten Syntax} beschrieben ist.
\end{Special_Paragraph}
% TODO: man tut so als gäbe es Konkrette Syntax

\newlineparagraph{Codebeispiel}

In Code~\ref{code:picoc_blocks_pass_für_codebeispiel} sieht man den \colorbold{Abstract-Syntax-Tree} des \colorbold{PiocC-Blocks Passes} für das aus Unterkapitel~\ref{code:picoc_code_für_codebeispiel} weitergeführte Beispiel, indem nun eigene \colorbold{Blöcke} für die Funktion \smalltt{faculty} und die \smalltt{main}-Funktion erstellt werden, in denen die \colorbold{ersten} Statements der jeweiligen Funktionen bis zum \colorbold{letzten} Statement oder bis zum ersten \colorbold{Auftauchen} eines \smalltt{If(exp, stmts)}-, \smalltt{IfElse(exp, stmts1, stmts2)}-, \smalltt{While(exp, stmts)}- oder \smalltt{DoWhile(exp, stmts)}-Knoten stehen. Je nachdem, ob ein \smalltt{If(exp, stmts)}-, \smalltt{IfElse(exp, stmts1, stmts2)}-, \smalltt{While(exp, stmts)}- oder \smalltt{DoWhile(exp, stmts)}-Knoten auftaucht, werden für die \colorbold{Bedingung} und mögliche \colorbold{Branches} eigene \colorbold{Blöcke} erstellt.

\begin{code}
  \centering
  \numberedcodebox[minted language=text]{./code_examples/example_faculty_it.picoc_blocks}
  \caption{PicoC-Blocks Pass für Codebespiel}
  \label{code:picoc_blocks_pass_für_codebeispiel}
\end{code}

\subsubsection{PicoC-ANF Pass}
\label{picoc_mon_pass}

\newlineparagraph{Aufgabe}
\label{sec:picoc_mon_pass_zweck}

Die Aufgabe des \colorbold{PicoC-ANF Passes} ist es den \colorbold{Abstract Syntax Tree} der Sprache $L_{PicoC\_Blocks}$ in die \colorbold{Abstrakte Syntax} der Sprache $L_{PicoC\_ANF}$ umzuformen, welche in \colorbold{A-Normalform} (Definition~\ref{def:a_normal_form}) und damit auch in \colorbold{Monadischer Normalform} (Definition~\ref{def:monadische_normalform}) ist. Um Wiederholung zu vermeiden wird zur Erklärung der \colorbold{A-Normalform} auf Unterkapitel~\ref{sec:a_normalform} verwiesen.

Zudem wird eine \colorbold{Symboltabelle} (Definition~\ref{def:symboltabelle}) eingeführt. In der \colorbold{Symboltabelle} wird beim Anlegen eines \colorbold{neuen Eintrags} für eine Variable zunächst eine \colorbold{Adresse} zugewiesen, die dem Wert einer von zwei \colorbold{Countern}  \smalltt{rel\_global\_addr} und  \smalltt{rel\_stack\_addr} entspricht. Der Counter \smalltt{rel\_global\_addr} ist für Variablen in den \colorbold{Globalen Statischen Daten} und der \colorbold{Counter}  \smalltt{rel\_stack\_addr} ist für Variablen auf dem \colorbold{Stackframe}. Einer der beiden \colorbold{Counter} wird entsprechend der \colorbold{Größe} der angelegten Variable \colorbold{hochgezählt}.

Kommt im \colorbold{Programmcode} an einer späteren Stelle diese Variable \smalltt{Name('symbol')} vor, so wird mit dem \colorbold{Symbol}\footnote{Bzw. der \colorbold{Bezeichner}} als Schlüssel in der \colorbold{Symboltabelle} nachgeschlagen und anstelle des \smalltt{Name(str)}-Knotens die in der \colorbold{Symboltabelle} nachgeschlagene Adresse in einem \smalltt{Global(Num('addr'))}- bzw. \smalltt{Stackframe(Num('addr'))}-Knoten eingesetzt eingefügt. Ob der \smalltt{Global(Num('addr'))}- oder  der \smalltt{Stackframe(Num('addr'))}-Knoten zum Einsatz kommt, entscheidet sich anhand des \colorbold{Scopes} (z.B. \smalltt{@scope}), der in der \colorbold{Symboltabelle} an den \colorbold{Bezeichner} drangehängt ist (z.B. \smalltt{identifier@scope}).\footnote{Die Umsetzung von \colorbold{Scopes} wird in Unterkapitel~\ref{sec:funktionsdeklaration_und_definition_und_umsetzung_von_scopes} genauer beschrieben.}

\begin{Definition}{Symboltabelle}{symboltabelle}
  Eine über ein \colorbold{Assoziatives Feld} umgesetzte \colorbold{Datenstruktur}, die notwendig ist, um das Konzept einer \colorbold{Variablen} in einer Sprache umzusetzen. Diese Datenstruktur ordnet jedem \colorbold{Symbol}\footnote{In einer \colorbold{Symboltabelle} werden \colorbold{Bezeichner} als \colorbold{Symbole} bezeichnet.} einer \colorbold{Variablen}, \colorbold{Konstanten} oder \colorbold{Funktion} aus einem \colorbold{Programm}, Informationen, wie die \colorbold{Adresse}, die \colorbold{Position} im Programmcode oder den \colorbold{Datentyp} zu.

  Die \colorbold{Symboltabelle} muss nur während des \colorbold{Kompiliervorgangs} im \colorbold{Speicher} existieren, da die Einträge in der \colorbold{Symboltabelle} beeinflussen, was für \colorbold{Maschinencode} generiert wird und dadurch im \colorbold{Maschinencode} bereits die richtigen \colorbold{Adressen} usw. angesprochen werden und es die Symboltabelle selbst \colorbold{nicht} mehr braucht.
\end{Definition}

\newlineparagraph{Abstrakte Syntax}

Zur Umsetzung dieses Passes ist es notwendig die \colorbold{Abstrakte Syntax} der Sprache $L_{PicoC\_Blocks}$ in Tabelle~\ref{gr:abstract_syntax_l_picoc_blocks} in die \colorbold{A-Normalform} zu bringen. Darunter fällt es unter anderem, dafür zu sorgen, dass \colorbold{Komplexe Knoten}, wie z.B. \smalltt{BinOp(exp, bin\_op, exp)} nur \colorbold{Atomare Knoten}, wie z.B. \smalltt{Stack(Num(str))} enthalten können. Des Weiteren werden auch \colorbold{Funktionen} und \colorbold{Funktionsaufrufe} aufgelöst, sodass u.a. die \colorbold{Blöcke} \smalltt{Block(Name(str), stmt*)} nun direkt im \smalltt{File(Name(str), block*)}-Knoten liegen usw., was in Unterkapitel~\ref{sec:umsetzung_von_funktionen} genauer erklärt wird. Die \colorbold{Symboltabelle} ist ebenfalls als \colorbold{Abstract Syntax Tree} umgesetzt, wofür in der \colorbold{Abstrakten Syntax} der Sprache $L_{PicoC\_ANF}$ in Grammatik~\ref{gr:abstract_syntax_l_picoc_anf} neue Knoten eingeführt werden.

Das ganze resultiert in der \colorbold{Abstrakten Syntax} der Sprache $L_{PicoC\_ANF}$ in Grammatik~\ref{gr:abstract_syntax_l_picoc_anf}.

\begin{grammar}[Abstrakte Syntax der Sprache $L_{PiocC\_ANF}$][H][gr:abstract_syntax_l_picoc_anf]
  \toprule
  \commentsecond*
  \midrule
  \arithanf
  \midrule
  \logicanf
  \midrule
  \assignanf
  \midrule
  \pntranf*
  \midrule
  \arrayanf*
  \midrule
  \struct*
  \midrule
  \ifelseanf*
  \midrule
  \funanf*
  \midrule
  \block*
  \midrule
  \fileanf*
  \midrule
  \symbolsecond
  \bottomrule
\end{grammar}

\newlineparagraph{Codebeispiel}

In Code~\ref{code:picoc_mon_pass_für_codebeispiel} sieht man den \colorbold{Abstract-Syntax-Tree} des \colorbold{PiocC-ANF Passes} für das aus Unterkapitel~\ref{code:picoc_code_für_codebeispiel} weitergeführte Beispiel, indem alls Statements und Ausdrücke in \colorbold{A-Normalform} sind. Die \smalltt{IfElse(exp, stmts, stmts)}-Knoten sind hier in  \colorbold{A-Normalform} gebracht worden, indem ihre \colorbold{Komplexe Bedingung} vorgezogen wurde und das Ergebnis der \colorbold{Komplexen Bedingung} einer \colorbold{Location} zugewiesen ist und sie selbst das Ergebnis über den \colorbold{Atomaren Ausdruck} \smalltt{Stack(Num(str))} vom Stack lesen: \smalltt{IfElse(Stack(Num(str)), stmts, stmts)}. \colorbold{Funktionen} sind nur noch über die \colorbold{Labels} von Blöcken zu erkennen, die den gleichen Bezeichner haben, wie die ursprüngliche Funktion und es lässt sich nur durch das \colorbold{Nachverfolgen} der \smalltt{GoTo(Name('label'))}-Knoten nachvollziehen, was ursprünglich zur Funktion gehörte.

\begin{code}
  \centering
  \numberedcodebox[minted language=text]{./code_examples/example_faculty_it.picoc_mon}
  \caption{PicoC-ANF Pass für Codebespiel}
  \label{code:picoc_mon_pass_für_codebeispiel}
\end{code}

\subsubsection{RETI-Blocks Pass}
\label{reti_blocks_pass}

\newlineparagraph{Aufgabe}
\label{sec:reti_blocks_pass_zweck}

Die Aufgabe des \colorbold{RETI-Blocks Passes} ist es die \colorbold{Statements} in der \colorbold{Blöcken}, die durch \colorbold{PicoC-Knoten} im \colorbold{Abstract Syntax Tree} der Sprache $L_{PicoC\_ANF}$ dargestellt sind durch ihren entsprechenden \colorbold{RETI-Knoten} zu ersetzen.

\newlineparagraph{Abstrakte Syntax}

Die \colorbold{Abstrakte Syntax} der Sprache $L_{RETI\_Blocks}$ in Grammatik~\ref{gr:abstract_syntax_l_reti_blocks} ist verglichen mit der \colorbold{Abstrakten Syntax} der Sprache $L_{PicoC\_ANF}$ in Grammatik~\ref{gr:abstract_syntax_l_picoc_anf} stark verändert, denn der Großteil der \colorbold{PicoC-Knoten} wird in diesem Pass durch entsprechende \colorbold{RETI-Knoten} ersetzt. Die einzigen verbleibenden \colorbold{PicoC-Knoten} sind \smalltt{Exp(GoTo(str))}, \smalltt{Block(Name(str), ⟨instr⟩*)} und \smalltt{File(Name(str), ⟨block⟩*)}, da das gesamte Konzept mit den \colorbold{Blöcken} erst im \colorbold{RETI-Pass} in Unterkapitel~\ref{gr:abstract_syntax_l_reti} aufgelöst wird.

\begin{grammar}[Abstrakte Syntax der Sprache $L_{RETI\_Blocks}$][H][gr:abstract_syntax_l_reti_blocks]
  \toprule
  \retiblocks
  \midrule
  \picocblocksleftover
  \bottomrule
\end{grammar}

\newlineparagraph{Codebeispiel}

In Code~\ref{code:reti_blocks_pass_für_codebeispiel} sieht man den \colorbold{Abstract-Syntax-Tree} des \colorbold{RETI-Blocks Passes} für das aus Unterkapitel~\ref{code:picoc_code_für_codebeispiel} weitergeführte Beispiel, indem die \colorbold{Statements}, die durch entsprechende \colorbold{PicoC-Knoten} im \colorbold{Abstrakt Syntax Tree} der Sprache $L_{PicoC\_ANF}$ in Grammatik~\ref{gr:abstract_syntax_l_picoc_anf} repräsentiert waren nun durch ihre entsprechennden \colorbold{RETI-Knoten} ersetzt werden.

\begin{code}
  \centering
  \numberedcodebox[minted language=text]{./code_examples/example_faculty_it.reti_blocks}
  \caption{RETI-Blocks Pass für Codebespiel}
  \label{code:reti_blocks_pass_für_codebeispiel}
\end{code}

\begin{Special_Paragraph}
  Wenn der \colorbold{Abstract Syntax Tree} ausgegeben wird, ist die Darstellung nicht auschließlich in \colorbold{Abstrakter Syntax}, da die \colorbold{RETI-Knoten} aus bereits im Unterkapitel~\ref{sec:ausgabe_von_abstract_syntax_trees} vermitteltem Grund in \colorbold{Konkretter Syntax} ausgeben werden.
\end{Special_Paragraph}

\subsubsection{RETI-Patch Pass}
\label{reti_patch_pass}

\newlineparagraph{Aufgabe}
\label{sec:reti_patch_pass_zweck}

Die Aufgabe des \colorbold{RETI-Patch Passes} ist das \colorbold{Ausbessern} (engl. to patch) des \colorbold{Abstract Syntax Trees}, durch:
\begin{itemize}
  \item das \colorbold{Einfügen} eines \smalltt{start.<nummer>}-Blockes, welcher ein \smalltt{GoTo(Name('main'))} zur  \smalltt{main}-Funktion enthält, wenn in manchen Fällen die \smalltt{main}-Funktion \colorbold{nicht} die erste Funktion ist und daher am Anfang zur \smalltt{main}-Funktion gesprungen werden muss.
  \item das \colorbold{Entfernen} von \smalltt{GoTo()}'s, deren Sprung nur \colorbold{eine} Adresse weiterspringen würde.
  \item das \colorbold{Voranstellen} von \colorbold{RETI-Knoten}, die vor jeder \colorbold{Division} \smalltt{Instr(Div(), args)} prüfen, ob, nicht durch \smalltt{0} geteilt wird.\footnote{Das fällt unter die Themenbereiche des \colorbold{Bachelorprojekts} und wird daher \colorbold{nicht} genauer erläutert.}
  \item das Überprüfen darauf, ob bestimmte \colorbold{Immediates} \smalltt{Im(str)} in Befehlen, wie z.B. \smalltt{Jump(rel, Im(str))}, \smalltt{Instr(Loadin(), [reg, reg, Im(str)])}, \smalltt{Instr(Loadi(), [reg, Im(str)])} usw. \colorbold{kleiner} $-2^{21}$ oder \colorbold{größer} $2^{21}-1$ sind. Im Fall dessen, dass es so ist, muss der \colorbold{gewünschte Zahlenwert} durch \colorbold{Bitshiften} und Anwenden von \colorbold{bitweise Oder} berechnet werden. Im Fall, dessen, dass der \colorbold{Immediate} allerdings \colorbold{kleiner} $-(2^{31})$ oder \colorbold{größer} $2^{31} - 1$ ist, wird eine Fehlermeldung \smalltt{TooLargeLiteral} ausgegeben.
\end{itemize}

\newlineparagraph{Abstrakte Syntax}

Die \colorbold{Abstrakte Syntax} der Sprache $L_{RETI\_Patch}$ in Grammatik~\ref{gr:abstract_syntax_l_reti_patch} ist im Vergleich zur \colorbold{Abstrakten Syntax} der Sprache $L_{RETI\_Blocks}$ in Grammatik~\ref{gr:abstract_syntax_l_reti_blocks} kaum verändert. Es muss nur ein Knoten \smalltt{Exit()} hinzugefügt werden, der im Falle einer  \colorbold{Division durch $0$} die Ausführung des Programs beendet.

\begin{grammar}[Abstrakte Syntax der Sprache $L_{RETI\_Patch}$][H][gr:abstract_syntax_l_reti_patch]
  \toprule
  \retiblocks*
  \midrule
  \picocblocksleftover*
  \bottomrule
\end{grammar}

\newlineparagraph{Codebeispiel}

In Code~\ref{code:reti_patch_pass_für_codebeispiel} sieht man den \colorbold{Abstract-Syntax-Tree} des \colorbold{PiocC-Patch Passes} für das aus Unterkapitel~\ref{code:picoc_code_für_codebeispiel} weitergeführte Beispiel. Durch den \colorbold{RETI-Patch Pass} wurde hier ein \smalltt{start.<nummer>}-Block\footnote{Dieser \colorbold{Block} wurde im Code~\ref{code:picoc_blocks_pass_für_codebeispiel} markiert.} eingesetzt, da die \smalltt{main}-Funktion \colorbold{nicht} die \colorbold{erste} Funktion ist. Des Weiteren wurden durch diesen Pass einzelne \smalltt{GoTo(Name(str))}-\colorbold{Statements} entfernt\footnote{Diese \colorbold{entfernten} \smalltt{GoTo(Name(str))}'s' wurden ebenfalls im Code~\ref{code:picoc_blocks_pass_für_codebeispiel} markiert.}, die nur einen Sprung um \colorbold{eine} Position entsprochen hätten.

\begin{code}
  \centering
  \numberedcodebox[minted language=text, minted options={highlightlines={4-9,24,38,66}}]{./code_examples/example_faculty_it.reti_patch}
  \caption{RETI-Patch Pass für Codebespiel}
  \label{code:reti_patch_pass_für_codebeispiel}
\end{code}

\subsubsection{RETI Pass}
\label{reti_pass}

\newlineparagraph{Aufgabe}
\label{sec:reti_pass_zweck}

Die Aufgabe des \colorbold{RETI-Patch Passes} ist es die \smalltt{GoTo(Name(str))}-Knoten in den den Knoten \smalltt{Instr(Loadi(), [reg, GoTo(Name(str))])}, \smalltt{Jump(Eq(), GoTo(Name(str)))} und \smalltt{Exp(GoTo(Name(str)))} durch eine entsprechende \colorbold{Adresse} zu ersetzen, die entsprechende \colorbold{Distanz} oder einen entsprechenden \colorbold{Sprungbefehl} mit passender \colorbold{Distanz} \smalltt{Jump(Always(), Im(str(distance)))}. Die \colorbold{Distanz-} und \colorbold{Adressberechnung} wird in Unterkapitel~\ref{sec:funktionsaufruf} genauer mit \colorbold{Formeln} erklärt.

\newlineparagraph{Konkrette und Abstrakte Syntax}

Die \colorbold{Abstrakte Syntax} der Sprache $L_{RETI}$ in Grammatik~\ref{gr:abstract_syntax_l_reti} hat im Vergleich zur \colorbold{Abstrakten Syntax} der Sprache $L_{RETI\_Patch}$ in Grammatik~\ref{gr:abstract_syntax_l_reti_patch} nur noch auschließlich \colorbold{RETI-Knoten}. Alle \colorbold{RETI-Knoten} stehen nun einem \smalltt{Program(Name(str), instr)}-Knoten.

Ausgegeben wird der finale \colorbold{Maschinencode} allerdings in \colorbold{Konkretter Syntax}, die sich aus den Grammatiken~\ref{gr:konkrette_syntax_l_reti_lexer} und \ref{gr:konkrette_syntax_l_reti_parser} für jeweils die \colorbold{Lexikalische} und \colorbold{Syntaktische Analyse} zusammensetzt. Der Grund, warum die \colorbold{Konkrette Syntax} der Sprache  $L_{RETI}$ auch nochmal in einen Teil für die \colorbold{Lexikalische} und \colorbold{Syntaktische Analyse} unterteilt ist, hat den Grund, dass für die Bachelorarbeit zum \colorbold{Testen} des \colorbold{PicoC-Compilers} ein \colorbold{RETI-Interpreter} implementiert wurde, der den RETI-Code \colorbold{lexen} und \colorbold{parsen} muss, um ihn später \colorbold{interpretieren} zu können.

% dieser Pass entspricht Assembler bis auf die Sache mit binärer Repräsentation, was der PicoC-Compiler garnicht macht

\begin{grammar}[Konkrette Syntax der Sprache $L_{RETI}$ für die Lexikalische Analyse in EBNF][H][gr:konkrette_syntax_l_reti_lexer]
  \toprule
  \firstcase{dig\_no\_0}{ \dq 1\dq \gralt \dq 2\dq \gralt \dq 3\dq \gralt \dq 4\dq \gralt \dq 5\dq \gralt \dq 6\dq}{L\_Program}
  \otherform{\dq 7\dq \gralt \dq 8\dq \gralt \dq 9\dq}{}
  \firstcase{dig\_with\_0}{ \dq 0\dq \gralt dig\_no\_0}{}
  \firstcase{num}{ \dq 0\dq \gralt dig\_no\_0 \enspace dig\_with\_0*\gralt \dq {-}\dq dig\_no\_0*}{}
  \firstcase{letter}{ \dq a\dq ... \dq Z\dq }{}
  \firstcase{name}{ letter(letter \mid  dig\_with\_0 \mid  \_)*}{}
  \firstcase{reg}{ \dq ACC\dq \gralt \dq IN1\dq \gralt \dq IN2\dq \gralt \dq PC\dq \gralt \dq SP\dq}{}
  \otherform{\dq BAF\dq \gralt \dq CS\dq \gralt \dq DS\dq}{}
  \firstcase{arg}{ reg \gralt  num}{}
  \firstcase{rel}{ \dq {==}\dq \gralt \dq {!=}\dq \gralt \dq {<}\dq \gralt \dq {<=}\dq\gralt \dq {>}\dq}{}
  \otherform{\dq {>=}\dq \gralt \dq \_NOP\dq}{}
  \bottomrule
\end{grammar}

\begin{grammar}[Konkrette Syntax der Sprache $L_{RETI}$ für die Syntaktische Analyse in EBNF][H][gr:konkrette_syntax_l_reti_parser]
\toprule
\firstcase{instr}{\dq ADD\dq\enspace reg\enspace arg\gralt \dq ADDI\dq\enspace reg\enspace num\gralt \dq SUB\dq\enspace reg\enspace arg}{L\_Program}
\otherform{\dq SUBI\dq\enspace reg\enspace\enspace num\gralt \dq MULT\dq\enspace reg\enspace arg\gralt \dq MULTI\dq\enspace reg\enspace num}{}
\otherform{\dq DIV\dq\enspace reg\enspace arg\gralt \dq DIVI\dq\enspace reg\enspace num\gralt \dq MOD\dq\enspace reg\enspace arg}{}
\otherform{\dq MODI\dq\enspace reg\enspace num\gralt \dq OPLUS\dq\enspace reg\enspace arg\gralt \dq OPLUSI\dq\enspace reg\enspace num}{}
\otherform{\dq OR\dq\enspace reg\enspace arg\gralt \dq ORI\dq\enspace reg\enspace num}{}
\otherform{\dq AND\dq\enspace reg\enspace arg\gralt \dq ANDI\dq\enspace reg\enspace num}{}
\otherform{\dq LOAD\dq\enspace reg\enspace num\gralt \dq LOADIN\dq\enspace arg\enspace arg\enspace num}{}
\otherform{\dq LOADI\dq\enspace reg\enspace num}{}
\otherform{\dq STORE\dq\enspace reg\enspace num\gralt \dq STOREIN\dq\enspace arg\enspace arg num}{}
\otherform{\dq MOVE\dq\enspace reg\enspace reg}{}
\otherform{\dq JUMP\dq rel\enspace num\gralt INT\enspace num\gralt RTI}{}
\otherform{\dq CALL\dq\enspace \dq INPUT\dq\enspace  reg\gralt \dq CALL\dq\enspace \dq PRINT\dq\enspace reg}{}
\firstcase{program}{name\enspace (instr\dq ;\dq )*}{}
\bottomrule
\end{grammar}

% TODO: es braucht noch eine Konkrette Syntax dafür
\begin{grammar}[Abstrakte Syntax der Sprache $L_{RETI}$][H][gr:abstract_syntax_l_reti]
  \toprule
  \reti
  \midrule
  \picocremovedleftover
  \bottomrule
\end{grammar}

\newlineparagraph{Codebeispiel}

Nach dem \colorbold{RETI-Pass} ist das Programm komplett in \colorbold{RETI-Knoten} übersetzt, die allerdings in ihrer \colorbold{Konkretten Syntax} ausgegeben werden, wie in Code~\ref{code:reti_pass_für_codebeispiel} zu sehen ist. Es gibt \colorbold{keine Blöcke} mehr und die \colorbold{RETI-Befehle} in diesen Blöcken wurden \colorbold{zusammengesetzt}, wie sie in den \colorbold{Blöcken} angeordnet waren. Die letzten \colorbold{Nicht-RETI-Befehle} oder \colorbold{RETI-Befehle}, die \colorbold{nicht} auschließlich aus \colorbold{RETI-Ausdrücken} bestehen\footnote{Wie z.B. \seqtt{LOADI\; ACC\; GoTo(Name('addr@next\_instr'))}, \seqtt{Exp(GoTo(Name('main.0')))} und \seqtt{JUMP==\; GoTo(Name('if\_else\_after.2'))}.}, die sich in den \colorbold{Blöcken} befunden haben, wurden durch \colorbold{RETI-Befehle} ersetzt.

Der \smalltt{Program(Name(str), instr)}-Knoten, indem alle \colorbold{RETI-Knoten} stehen gibt alleinig die \colorbold{RETI-Knoten}, die er beinhaltet aus und fügt ansonsten nichts hinzu, wodurch der \colorbold{Abstract Syntax Tree}, wenn er in eine Datei ausgegeben wird, direkt \colorbold{RETI-Code} in \colorbold{menschenlesbarer Repräsentation} erzeugt.

\begin{code}
  \centering
  \numberedcodebox[minted language=text]{./code_examples/example_faculty_it.reti}
  \caption{RETI Pass für Codebespiel}
  \label{code:reti_pass_für_codebeispiel}
\end{code}

% TODO: zusammenfassendes Bild
  % ./content/Implementierung1_Tables_DT_AST.tex
  %!Tex Root = ../Main.tex
% ./Packete_und_Deklarationen.tex
% ./Titlepage.tex
% ./Motivation.tex
% ./Einführung.tex
% ./Implementierung1_Tables_DT_AST.tex,
% ./Implementierung3_Struct_Derived.tex,
% ./Implementierung4_Fun.tex,
% ./Ergebnisse_und_Ausblick.tex

\subsection{Umsetzung von Pointern}
\label{sec:umsetzung_von_pointern}
\subsubsection{Referenzierung}
Die \colorbold{Referenzierung} (z.B. \verb|&var|) wird im Folgenden anhand des Beispiels in Code~\ref{code:picoc_code_für_pointer_referenzierung} erklärt.

\begin{code}
  \centering
  \numberedcodebox[minted language=c, minted options={highlightlines={3}}]{./code_examples/example_pntr_ref.picoc}
  \caption{PicoC-Code für Pointer Referenzierung}
  \label{code:picoc_code_für_pointer_referenzierung}
\end{code}

Der Knoten \smalltt{Ref(Name('var')))} repräsentiert im \colorbold{Abstract Syntax Tree} in Code~\ref{code:abstract_syntax_tree_für_pointer_referenzierung} eine \colorbold{Referenzierung} \verb|&var| und der Knoten \smalltt{PntrDecl(Num('1'), IntType('int'))} repräsentiert einen Pointer \smalltt{*pntr}.

\begin{code}
  \centering
  \numberedcodebox[minted language=text, minted options={highlightlines={10}}]{./code_examples/example_pntr_ref.ast}
  \caption{Abstract Syntax Tree für Pointer Referenzierung}
  \label{code:abstract_syntax_tree_für_pointer_referenzierung}
\end{code}

Bevor man einem \colorbold{Pointer} eine eine \colorbold{Adresse} (z.B. \smalltt{\&var}) zuweisen kann, muss dieser erstmal \colorbold{definiert} sein. Dafür braucht es einen Eintrag in der \colorbold{Symboltabelle} in Code~\ref{code:symboltabelle_für_pointer_referenzierung}.

\begin{Special_Paragraph}
Die \colorbold{Größe} eines Pointers (z.B. eines Pointers auf ein Array von \smalltt{int}: \smalltt{pntr = int *pntr[3]}), die ihm \smalltt{size}-Attribut der \colorbold{Symboltabelle} eingetragen ist, ist dabei immer: $\mathtt{size(pntr) = 1}$.
\end{Special_Paragraph}

\begin{code}
  \centering
  \numberedcodebox[minted language=text, minted options={highlightlines={23-28}}]{./code_examples/example_pntr_ref.st}
  \caption{Symboltabelle für Pointer Referenzierung}
  \label{code:symboltabelle_für_pointer_referenzierung}
\end{code}

Im \colorbold{PicoC-ANF Pass} in Code~\ref{code:picoc_mon_für_pointer_referenzierung} wird der Knoten \smalltt{Ref(Name('var')))} durch die Knoten \smalltt{Ref(GlobalRead(Num('0')))} und \smalltt{Assign(GlobalWrite(Num('1')), Tmp(Num('1')))} ersetzt. Im Fall, dass in \smalltt{Ref(exp))} das \smalltt{exp} vielleicht nicht direkt ein \smalltt{Name('var')} enthält und \smalltt{exp} z.B. ein \smalltt{Subscr(Attr(Name('var')))} ist, sind noch weitere Anweisungen zwischen den Zeilen \smalltt{11} und  \smalltt{12} nötig, die sich in diesem Beispiel um das Übersetzen von \smalltt{Subscr(exp)} und \smalltt{Attr(exp)} nach dem Schema in Subkapitel~\ref{sec:mittelteil_für_die_verschiedenen_derived_datatypes} kümmern.

\begin{code}
  \centering
  \numberedcodebox[minted language=text, minted options={highlightlines={11-12}}]{./code_examples/example_pntr_ref.picoc_mon}
  \caption{PicoC-ANF Pass für Pointer Referenzierung}
  \label{code:picoc_mon_für_pointer_referenzierung}
\end{code}

Im \colorbold{RETI-Blocks Pass} in Code~\ref{code:reti_blocks_für_pointer_referenzierung} werden die \colorbold{PicoC-Knoten} \smalltt{ Ref(Global(Num('0')))} und \smalltt{Assign(Global(Num('1')), Stack(Num('1')))} durch ihre entsprechenden \colorbold{RETI-Knoten} ersetzt.

\begin{code}
  \centering
  \numberedcodebox[minted language=text, minted options={highlightlines={18-21,23-25}}]{./code_examples/example_pntr_ref.reti_blocks}
  \caption{RETI-Blocks Pass für Pointer Referenzierung}
  \label{code:reti_blocks_für_pointer_referenzierung}
\end{code}
% Initialisierung eines Pointers
\subsubsection{Dereferenzierung durch Zugriff auf Arrayindex ersetzen}
\label{dereferenzierung_durch_zugriff_auf_arrayindex_ersetzen}
Die \colorbold{Dereferenzierung} (z.B. \smalltt{*var}) wird im Folgenden anhand des Beispiels in Code~\ref{code:picoc_code_für_pointer_dereferenzierung} erklärt.

\begin{code}
  \centering
  \numberedcodebox[minted language=c, minted options={highlightlines={4}}]{./code_examples/example_pntr_deref.picoc}
  \caption{PicoC-Code für Pointer Dereferenzierung}
  \label{code:picoc_code_für_pointer_dereferenzierung}
\end{code}

Der Container-Knoten \smalltt{Deref(Name('var'), Num('0')))} repräsentiert im \colorbold{Abstract Syntax Tree} in Code~\ref{code:abstract_syntax_tree_für_pointer_dereferenzierung} eine \colorbold{Dereferenzierung} \smalltt{*var}. Es gibt herbei \colorbold{zwei} Fälle. Bei der Anwendung von \colorbold{Pointer Arithmetik}, wie z.B. \smalltt{*(var + 2 - 1)} übersetzt sich diese zu \smalltt{Deref(Name('var'), BinOp(Num('2'), Sub(), BinOp(Num('1'))))}. Bei einer normalen \colorbold{Dereferenzierung}, wie z.B. \smalltt{*var}, übersetzt sich diese zu \smalltt{Deref(Name('var'), Num('0'))}.

\begin{code}
  \centering
  \numberedcodebox[minted language=text, minted options={highlightlines={11}}]{./code_examples/example_pntr_deref.ast}
  \caption{Abstract Syntax Tree für Pointer Dereferenzierung}
  \label{code:abstract_syntax_tree_für_pointer_dereferenzierung}
\end{code}

Im \colorbold{PicoC-Shrink Pass} in Code~\ref{code:picoc_shrink_für_pointer_dereferenzierung} wird ein Trick angewandet, bei dem jeder Knoten \smalltt{Deref(Name('pntr'), Num('0'))} einfach durch den Knoten \smalltt{Subscr(Name('pntr'), Num('0'))} ersetzt wird. Der Trick besteht darin, dass der \colorbold{Dereferenzoperator} (z.B. \smalltt{*(var + 1)}) sich identisch zum \colorbold{Operator für den Zugriff auf einen Arrayindex} (z.B. \smalltt{var[1]}) verhält\footnote{In der Sprache $L_{C}$ gibt es einen Unterschied bei der Initialisierung bei z.B. \smalltt{int *var = \dq string\dq} und z.B. \smalltt{int var[1] = \dq string\dq}, der allerdings nichts mit den beiden Operatoren zu tuen hat, sondern mit der \colorbold{Initialisierung}, bei der die Sprache $L_{C}$ verwirrenderweise die eckigen Klammern \smalltt{[]} genauso, wie beim \colorbold{Operator für den Zugriff auf einen Arrayindex}, vor den Bezeichner schreibt (z.B. \smalltt{var[1]}), obwohl es ein \colorbold{Derived Datatype} ist.}. Damit sparrt man sich viele vermeidbare \colorbold{Fallunterscheidungen} und \colorbold{doppelten Code} und kann die \colorbold{Derefenzierung} (z.B. \smalltt{*(var + 1)}) einfach von den Routinen für einen \colorbold{Zugriff auf einen Arrayindex} (z.B. \smalltt{var[1]}) übernehmen lassen.

\begin{code}
  \centering
  \numberedcodebox[minted language=text, minted options={highlightlines={11}}]{./code_examples/example_pntr_deref.picoc_shrink}
  \caption{PicoC-Shrink Pass für Pointer Dereferenzierung}
  \label{code:picoc_shrink_für_pointer_dereferenzierung}
\end{code}

\subsection{Umsetzung von Arrays}
\label{sec:umsetzung_von_arrays}
\subsubsection{Initialisierung von Arrays}
\label{sec:initialisierung_von_arrays}

Die \colorbold{Initialisierung} eines \colorbold{Arrays} (z.B. \smalltt{int ar[2][1] = \{\{3+1\}, \{4\}\}}) wird im Folgenden anhand des Beispiels in Code~\ref{code:picoc_code_für_array_initialisierung} erklärt.

% Stack und Globale Statische Daten
\begin{code}
  \centering
  \numberedcodebox[minted language=c, minted options={highlightlines={2, 6}}]{./code_examples/example_array_init.picoc}
  \caption{PicoC-Code für Array Initialisierung}
  \label{code:picoc_code_für_array_initialisierung}
\end{code}

Die \colorbold{Initialisierung} eines \colorbold{Arrays} \seqtt{int ar[2][1] = \{\{3+1\}, \{4\}\}} wird im \colorbold{Abstract Syntax Tree} in Code~\ref{code:abstract_syntax_tree_für_array_initialisierung} mithilfe der Komposition \seqtt{Assign(Alloc(Writeable(), ArrayDecl([Num('2'), Num('1')], IntType('int')), Name('ar')), Array([Array([BinOp(Num('3'), Add('+'), Num('1'))]), Array([Num('4')])]))} dargestellt.

\begin{code}
  \centering
  \numberedcodebox[minted language=text, minted options={highlightlines={9, 16}}]{./code_examples/example_array_init.ast}
  \caption{Abstract Syntax Tree für Array Initialisierung}
  \label{code:abstract_syntax_tree_für_array_initialisierung}
\end{code}

Bei der \colorbold{Initialisierung} eines \colorbold{Arrays} wird zuerst \smalltt{Alloc(Writeable(), ArrayDecl([Num('2'), Num('1')], IntType('int')))} ausgewertet, da eine Variable zuerst definiert sein muss, bevor man sie verwenden kann\footnote{Das Widerspricht der üblichen Auswertungsreihenfolge beim \colorbold{Zuweisungsoperator} \smalltt{=}, der \colorbold{rechtsassoziativ} ist. Der \colorbold{Zuweisungsoperator} \smalltt{=} tritt allerdings erst später in Aktion.}. Das \colorbold{Definieren} der Variable \smalltt{ar} erfolgt mittels der \colorbold{Symboltabelle}, die in Code~\ref{code:symboltabelle_für_array_initialisierung} dargestellt ist.

Bei Variablen auf dem \colorbold{Stackframe} wird ein Array \colorbold{rückwärts} auf das Stackframe geschrieben und auch die \colorbold{Adresse des ersten Elements} als Adresse des Arrays genommen. Dies macht den \colorbold{Zugriff auf einen Arrayindex} in Subkapitel~\ref{sec:zugriff_auf_arrayindex} deutlich unkomplizierter, da man so nicht mehr zwischen \colorbold{Stackframe} und \colorbold{Globalen Statischen Daten} beim \colorbold{Zugriff auf einen Arrayindex} unterscheiden muss, da es Probleme macht, dass ein \colorbold{Stackframe} in die entgegengesetzte Richtung wächst, verglichen mit den \colorbold{Globalen Statischen Daten}\footnote{Wenn man beim \colorbold{GCC}~\cite{noauthor_gcc_nodate} einen Stackframe mittels des \colorbold{GDB}~\cite{noauthor_gcc_nodate} beobachtet, sieht man, dass dieser es genauso macht.}.

\begin{Special_Paragraph}
  Das \colorbold{Größe} des Arrays $\mathtt{datatype \enspace ar[dim_1]\ldots[dim_k]}$, die ihm \smalltt{size}-Attribut des \colorbold{Symboltabelleneintrags} eingetragen ist, berechnet sich dabei aus der \colorbold{Mächtigkeit} der einzelnen \colorbold{Dimensionen} des Arrays multipliziert mit der \colorbold{Größe} des \colorbold{grundlegenden Datentyps} der einzelnen \colorbold{Arrayelemente}: $\mathtt{size(datatype(ar)) = \left(\prod^n_{j=1} dim_j\right)\cdot size(datatype)}$\footnote{Die \colorbold{Funktion}  \smalltt{type} ordnet einer  \colorbold{Variable} ihren \colorbold{Datentyp} zu. Das ist notwendig, weil die \colorbold{Funktion} \smalltt{size} nur bei einem \colorbold{Datentyp} als \colorbold{Funktionsargument} die \colorbold{Größe dieses Datentyps} als \colorbold{Zielwert} liefert}.
\end{Special_Paragraph}

\begin{code}
  \centering
  \numberedcodebox[minted language=text, minted options={highlightlines={14-19,32-37}}]{./code_examples/example_array_init.st}
  \caption{Symboltabelle für Array Initialisierung}
  \label{code:symboltabelle_für_array_initialisierung}
\end{code}

Im \colorbold{PiocC-Mon Pass} in Code~\ref{code:picoc_mon_für_array_initialisierung} werden zuerst die \colorbold{Logischen Ausdrücke} in den Blättern des Teilbaums, der beim \colorbold{Array-Initializers} \colorbold{Container-Knoten} \smalltt{Array([Array([BinOp(Num('3'), Add('+'), Num('1'))]), Array([Num('4')])])} anfängt nach dem \colorbold{Depth-First-Search} Schema, von \colorbold{links-nach-rechts} ausgewertet und auf den \colorbold{Stack} geschrieben\footnote{Da der \colorbold{Zuweisungsoperator} \smalltt{=} \colorbold{rechtsassoziativ} ist und auch rein \colorbold{logisch}, weil man nichts zuweisen kann, was man noch nicht berechnet hat.}.

Im finalen Schritt muss zwischen \colorbold{Globalen Statischen Daten} bei der \smalltt{main}-Funktion und \colorbold{Stackframe} bei der Funktion \smalltt{fun} unterschieden werden. Die auf den Stack ausgewerteten Expressions werden mittels der Komposition \smalltt{Assign(Global(Num('0')), Stack(Num('2')))} bzw. \smalltt{Assign(Stackframe(Num('3')), Stack(Num('4')))}, die in Tabelle~\ref{tab:kompositionen_von_picoc_knoten_und_reti_knoten_mit_besonderer_bedeutung} genauer beschrieben ist, versetzt in der selben Reihenfolge zu den \colorbold{Globalen Statischen Daten} bzw. auf den \colorbold{Stackframe} geschrieben.

Der \colorbold{Trick} ist hier, dass egal wieviele Dimensionen und was für einen Datentyp das \colorbold{Array} hat, man letztendlich immer das gesamte Array erwischt, wenn man einfach die \colorbold{Größe des Arrays} viele \colorbold{Speicherzellen} mit z.B. der \colorbold{Komposition} \smalltt{Assign(Global(Num('0')), Stack(Num('2')))} verschiebt.

In die Knoten \smalltt{Global('0')} und  \smalltt{Stackframe('3')} wurde hierbei die \colorbold{Startadresse} des jeweiligen Arrays geschrieben, sodass man nach dem \colorbold{PicoC-ANF Pass} nie mehr Variablen in der  \colorbold{Symboltabelle} nachsehen muss und gleich weiß, ob sie in Bezug zu den \colorbold{Globalen Statischen Daten} oder dem \colorbold{Stackframe} stehen.

\begin{code}
  \centering
  \numberedcodebox[minted language=text, minted options={highlightlines={8-12,19-23}}]{./code_examples/example_array_init.picoc_mon}
  \caption{PicoC-ANF Pass für Array Initialisierung}
  \label{code:picoc_mon_für_array_initialisierung}
\end{code}

Im \colorbold{RETI-Blocks Pass} in Code~\ref{code:reti_blocks_für_array_initialisierung} werden die \colorbold{Kompositionen} \smalltt{Exp(exp)} und \smalltt{Assign(Global(Num('0')), Stack(Num('2')))} bzw. \smalltt{Assign(Stackframe(Num('3')), Stack(Num('4')))} durch ihre entsprechenden \colorbold{RETI-Knoten} ersetzt.

\begin{code}
  \centering
  \numberedcodebox[minted language=text, minted options={highlightlines={9-11,13-15,17-21,23-25,27-31,40-42,44-46,48-50,52-54,56-64}}]{./code_examples/example_array_init.reti_blocks}
  \caption{RETI-Blocks Pass für Array Initialisierung}
  \label{code:reti_blocks_für_array_initialisierung}
\end{code}


% kleines Extra
\subsubsection{Zugriff auf einen Arrayindex}
\label{sec:zugriff_auf_arrayindex}

Der \colorbold{Zugriff auf einen Arrayindex} (z.B. \smalltt{ar[0]}) wird im Folgenden anhand des Beispiels in Code~\ref{code:picoc_code_für_zugriff_auf_arrayindex} erklärt.

\begin{code}
  \centering
  \numberedcodebox[minted language=c, minted options={highlightlines={3,8}}]{./code_examples/example_array_access.picoc}
  \caption{PicoC-Code für Zugriff auf einen Arrayindex}
  \label{code:picoc_code_für_zugriff_auf_arrayindex}
\end{code}

Der \colorbold{Zugriff auf einen Arrayindex} \smalltt{ar[0]} wird im  \colorbold{Abstract Syntax Tree} in Code~\ref{code:abstract_syntax_tree_für_zugriff_auf_arrayindex} mithilfe des \colorbold{Container-Knotens} \smalltt{Subscr(Name('ar'), Num('0'))} dargestellt.

\begin{code}
  \centering
  \numberedcodebox[minted language=text, minted options={highlightlines={10,18}}]{./code_examples/example_array_access.ast}
  \caption{Abstract Syntax Tree für Zugriff auf einen Arrayindex}
  \label{code:abstract_syntax_tree_für_zugriff_auf_arrayindex}
\end{code}

Im \colorbold{PicoC-ANF Pass} in Code~\ref{code:picoc_mon_für_zugriff_auf_arrayindex} wird vom \colorbold{Container-Knoten} \smalltt{Subscr(Name('ar'), Num('0'))} zuerst im \colorbold{Anfangsteil}~\ref{sec:einleitungsteil_für_globale_statische_daten_und_stackframe} die \colorbold{Adresse} der Variable \smalltt{Name('ar')} auf den \colorbold{Stack} geschrieben. Bei den \colorbold{Globalen Statischen Daten} der \smalltt{main}-Funktion wird das durch die Komposition \smalltt{Ref(Global(Num('0')))} dargestellt und beim \colorbold{Stackframe} der Funktionm \smalltt{fun} wird das durch die Komposition \smalltt{Ref(Stackframe(Num('2')))} dargestellt.

In nächsten Schritt, dem \colorbold{Mittelteil}~\ref{sec:mittelteil_für_die_verschiedenen_derived_datatypes} wird die \colorbold{Adresse} ab der das \colorbold{Arrayelement} des Arrays auf das Zugegriffen werden soll anfängt berechnet. Dabei wurde im \colorbold{Anfangsteil} bereits die \colorbold{Anfangsadresse} des Arrays, in dem dieses \colorbold{Arrayelement} liegt auf den \colorbold{Stack} gelegt. Da ein \colorbold{Index} auf den Zugegriffen werden soll auch durch das Ergebnis eines \colorbold{komplexeren Ausdrucks}, z.B. \smalltt{ar[1 + var]} bestimmt sein kann, indem auch \colorbold{Variablen} vorkommen können, kann dieser nicht während des \colorbold{Kompilierens} berechnet werden, sondern muss zur \colorbold{Laufzeit} berechnet werden.

Daher muss zuerst der Wert des \colorbold{Index}, dessen Adresse berechnet werden soll bestimmt werden, z.B. im einfachen Fall durch \smalltt{Exp(Num('0'))} und dann muss die \colorbold{Adresse des Index} berechnet werden, was durch die Komposition \smalltt{Ref(Subscr(Stack(Num('2')), Stack(Num('1'))))} dargestellt wird. Die Bedeutung der Komposition \smalltt{\smalltt{Ref(Subscr(Stack(Num('2')), Stack(Num('1'))))}} ist in Tabelle~\ref{tab:kompositionen_von_picoc_knoten_und_reti_knoten_mit_besonderer_bedeutung} dokumentiert.

Zur \colorbold{Adressberechnung} ist es notwendig auf die \colorbold{Dimensionen} (z.B. \smalltt{[Num('3')]}) des Arrays, auf dessen \colorbold{Arrayelement} zugegriffen wird, zugreifen zu können. Daher ist der \colorbold{Arraydatentyp} (z.B. \smalltt{ArrayDecl([Num('3')], IntType('int'))}) dem \colorbold{Container-Knoten} \smalltt{Ref(exp, \textcolor{gray!90!black}{datatype})} als \textcolor{gray!90!black}{verstecktes Attribut} \smalltt{datatype} angehängt. Das \textcolor{gray!90!black}{versteckte Attribut} wird während des Kompiliervorgangs im \colorbold{PiocC-Mon Pass} dem \colorbold{Container-Knoten} \smalltt{Ref(exp, \textcolor{gray!90!black}{datatype})} angehängt.

Je nachdem, ob mehrere \smalltt{Subscr(exp, exp)} eine Komposition bilden (z.B. \smalltt{Subscr(Subscr(Name('var'), Num('1')), Num('1'))}) ist es notwendig mehrere \colorbold{Adressberechnungsschritte für den Index} \smalltt{Ref(Subscr(Stack(Num('2')), Stack(Num('1'))))} einzuleiten und es muss auch möglich sein, z.B. einen \colorbold{Attributzugriff} \smalltt{var.attr} und eine \colorbold{Zugriff auf einen Arryindex} \smalltt{var[1]} miteinander zu kombinieren, was in Subkapitel~\ref{sec:mittelteil_für_die_verschiedenen_derived_datatypes} allgemein erklärt ist.

Im letzten Schritt, dem \colorbold{Schlussteil}~\ref{sec:schlussteil_für_die_verschiedenen_derived_datatypes} wird der \colorbold{Inhalt} des \colorbold{Index}, dessen \colorbold{Adresse} in den vorherigen Schritten berechnet wurde, nun auf den \colorbold{Stack} geschrieben, wobei dieser die \colorbold{Adresse} auf dem Stack ersetzt, die es zum Finden des \colorbold{Index} brauchte. Dies wird durch den Knoten \smalltt{Exp(Stack(Num('1')))} dargestellt. Je nachdem, welchen \colorbold{Datentyp} die Variable \smalltt{ar} hat und auf welchen \colorbold{Unterdatentyp} folglich im \colorbold{Kontext} zuletzt zugegriffen wird, abhängig davon wird der \colorbold{Schlussteil} \smalltt{Exp(Stack(Num('1')))} auf eine andere Weise verarbeitet (siehe Subkapitel~\ref{sec:schlussteil_für_die_verschiedenen_derived_datatypes}). Der \colorbold{Unterdatentyp} ist dabei ein \textcolor{gray!90!black}{verstecktes Attribut} des \smalltt{Exp(Stack(Num('1')))}-Knoten.

Der einzige \colorbold{Unterschied}, je nachdem, ob der \colorbold{Zugriff auf einen Arrayindex} (z.B. \smalltt{ar[1]}) in der  \smalltt{main}-Funktion oder der Funktion \smalltt{fun} erfolgt, ist eigentlich nur beim \colorbold{Anfangsteil}, beim Schreiben der \colorbold{Adresse} der Variable \smalltt{ar} auf den \colorbold{Stack} zu finden, bei dem unterschiedliche \colorbold{RETI-Instructions} für eine Variable, die in den \colorbold{Globalen Statischen Daten} liegt und eine Variable, die auf dem \colorbold{Stackframe} liegt erzeugt werden müssen.

\begin{Special_Paragraph}
  Die Berechnung der \colorbold{Adresse}, ab der ein \colorbold{Arrayelement} eines Arrays $\mathtt{datatype\enspace ar[dim_1]\ldots[dim_n]}$ abgespeichert ist, kann mittels der Formel~\ref{eq:adresse_von_arrayelement}:

  \numberwithin{equation}{section}

  \begin{equation}
  \mathtt{ref(ar[idx_1]\ldots[idx_n]) = ref(ar) + \left(\sum_{i=1}^{n}\left(\prod_{j=i+1}^{n} dim_{j}\right) \cdot idx_{i}\right) \cdot \operatorname{size}(datatype)}
    \label{eq:adresse_von_arrayelement}
  \end{equation}
  aus der Betriebssysteme Vorlesung\footcite{scholl_betriebssysteme_2020} berechnet werden\footnote{\smalltt{ref(exp)} steht dabei für die Berechnung der \colorbold{Adresse} von \smalltt{exp}, wobei \smalltt{exp} z.B. \smalltt{ar[3][2]} sein könnte.}.

  Die Komposition \smalltt{Ref(Global(num))} bzw. \smalltt{Ref(Stackframe(num))} repräsentiert dabei den Summanden $\smalltt{ref(ar)}$ in der Formel.

  Die Komposition \smalltt{Exp(num)} repräsentiert dabei einen \colorbold{Subindex} (z.B. \smalltt{i} in \smalltt{a[i][j][k]}) beim \colorbold{Zugriff auf ein Arrayelement}, der als Faktor $\mathtt{idx_i}$ in der Formel auftaucht.

  Der Komposition \smalltt{Ref(Subscr(Stack(Num('2')), Stack(Num('1'))))} repräsentiert dabei einen ausmultiplizierten Summanden $\mathtt{\left(\prod_{j=i+1}^{n} dim_{j}\right) \cdot idx_{i} \cdot size(datatpye)}$ in der Formel.

Die Komposition \smalltt{Exp(Stack(Num('1')))} repräsentiert dabei das Lesen des \colorbold{Inhalts} $\mathtt{M\left[ref(ar[idx_1]\ldots[idx_n])\right]}$ der Speicherzelle an der finalen \colorbold{Adresse}  $\mathtt{ref(ar[idx_1]\ldots[idx_n])}$.
\end{Special_Paragraph}

\begin{code}
  \centering
  \numberedcodebox[minted language=text, minted options={highlightlines={11-14,26}}]{./code_examples/example_array_access.picoc_mon}
  \caption{PicoC-ANF Pass für Zugriff auf einen Arrayindex}
  \label{code:picoc_mon_für_zugriff_auf_arrayindex}
\end{code}

Im \colorbold{RETI-Blocks Pass} in Code~\ref{code:reti_blocks_für_zugriff_auf_arrayindex} werden die \colorbold{Kompositionen} \smalltt{Ref(Global(Num('0')))}, \smalltt{Ref(Subscr(Stack(Num('2')) und Stack(Num('1'))))} durch ihre entsprechenden \colorbold{RETI-Knoten} ersetzt.

\begin{code}
  \centering
  \numberedcodebox[minted language=text, minted options={highlightlines={18-21,23-25,27-32,34-36,66-69}}]{./code_examples/example_array_access.reti_blocks}
  \caption{RETI-Blocks Pass für Zugriff auf einen Arrayindex}
  \label{code:reti_blocks_für_zugriff_auf_arrayindex}
\end{code}

\subsubsection{Zuweisung an Arrayindex}
\label{sec:zuweisung_an_arrayindex}
% Formel aus der Vorlesung, wo ist die hier?

Die \colorbold{Zuweisung} eines Wertes an einen \colorbold{Arrayindex} (z.B. \smalltt{ar[2] = 42;}) wird im Folgenden anhand des Beispiels in Code~\ref{code:picoc_code_für_array_assignment} erläutert.

\begin{code}
  \centering
  \numberedcodebox[minted language=c, minted options={highlightlines={3}}]{./code_examples/example_array_assignment.picoc}
  \caption{PicoC-Code für Zuweisung an Arrayindex}
  \label{code:picoc_code_für_array_assignment}
\end{code}

Im \colorbold{Abstract Syntax Tree} in Code~\ref{code:abstract_syntax_tree_für_array_assignment} wird eine \colorbold{Zuweisung} an einen \colorbold{Arrayindex} \smalltt{ar[2] = 42;} durch die Komposition \smalltt{Assign(Subscr(Name('ar'), Num('2')), Num('42'))} dargestellt.

\begin{code}
  \centering
  \numberedcodebox[minted language=text, minted options={highlightlines={10}}]{./code_examples/example_array_assignment.ast}
  \caption{Abstract Syntax Tree für Zuweisung an Arrayindex}
  \label{code:abstract_syntax_tree_für_array_assignment}
\end{code}

Im \colorbold{PicoC-ANF Pass} in Code~\ref{code:picoc_mon_für_array_assignment} wird zuerst die \colorbold{rechte} Seite des \colorbold{rechtsassoziativen} Zuweisungsoperators \smalltt{=}, bzw. des \colorbold{Container-Knotens} der diesen darstellt ausgewertet: \smalltt{Exp(Num('42'))}.

Danach ist das Vorgehen, bzw. sind die Kompostionen, die dieses darauffolgende Vorgehen darstellen: \smalltt{Ref(Global(Num('0')))}, \smalltt{Exp(Num('2'))} und \smalltt{Ref(Subscr(Stack(Num('2')), Stack(Num('1'))))} identisch zum \colorbold{Anfangsteil} und \colorbold{Mittelteil} aus dem vorherigen Subkapitel~\ref{sec:zugriff_auf_arrayindex}. Es wird die \colorbold{Adresse} des \colorbold{Index}, dem das Ergebnis der Ausdrucks auf der rechten Seite des \colorbold{Zuweisungsoperators} \smalltt{=} zugewiesen wird berechet, wie in Subkapitel~\ref{sec:zugriff_auf_arrayindex}.

Zum Schluss stellt die \colorbold{Komposition} \smalltt{Assign(Stack(Num('1')), Stack(Num('2')))}\footnote{Ist in Tabelle~\ref{tab:kompositionen_von_picoc_knoten_und_reti_knoten_mit_besonderer_bedeutung} genauer beschrieben ist} die Zuweisung \smalltt{=} des Ergebnisses des Ausdrucks auf der \colorbold{rechten} Seite der Zuweisung zum \colorbold{Arrayindex}, dessen \colorbold{Adresse} im Schritt danach berechnet wurde dar.

\begin{code}
  \centering
  \numberedcodebox[minted language=text, minted options={highlightlines={9-13}}]{./code_examples/example_array_assignment.picoc_mon}
  \caption{PicoC-ANF Pass für Zuweisung an Arrayindex}
  \label{code:picoc_mon_für_array_assignment}
\end{code}

Im \colorbold{RETI-Blocks Pass} in Code~\ref{code:reti_blocks_für_array_assignment} werden die \colorbold{Kompositionen} \smalltt{Ref(Global(Num('0')))}, \smalltt{Ref(Subscr(Stack(Num('2')), Stack(Num('1'))))} und \smalltt{Assign(Stack(Num('1')), Stack(Num('2')))} durch ihre entsprechenden \colorbold{RETI-Knoten} ersetzt.

\begin{code}
  \centering
  \numberedcodebox[minted language=text, minted options={highlightlines={10-12,14-17,19-21,23-28,30-33}}]{./code_examples/example_array_assignment.reti_blocks}
  \caption{RETI-Blocks Pass für Zuweisung an Arrayindex}
  \label{code:reti_blocks_für_array_assignment}
\end{code}
     % ./content/Implementierung2_Pntr_Array.tex
  %!Tex Root = ../Main.tex
% ./Packete_und_Deklarationen.tex
% ./Titlepage.tex
% ./Motivation.tex
% ./Einführung.tex
% ./Implementierung1_Tables_DT_AST.tex,
% ./Implementierung2_Pntr_Array.tex,
% ./Implementierung4_Fun.tex,
% ./Ergebnisse_und_Ausblick.tex

\subsection{Umsetzung von Structs}
\subsubsection{Deklaration und Definition von Structtypen}

Die \colorbold{Deklaration} eines neuen \colorbold{Structtyps} (z.B. \smalltt{struct st \{int len; int ar[2];\};}) und die \colorbold{Definition} einer Variable mit diesem \colorbold{Structtyp} (z.B. \smalltt{struct st st\_var;}) wird im Folgenden anhand des Beispiels in Code~\ref{code:picoc_code_für_die_deklaration_eines_structtyps} erläutert.

\begin{code}
  \centering
  \numberedcodebox[minted language=c, minted options={highlightlines={1,4}}]{./code_examples/example_struct_decl_def.picoc}
  \caption{PicoC-Code für die Deklaration eines Structtyps}
  \label{code:picoc_code_für_die_deklaration_eines_structtyps}
\end{code}

Bevor irgendwas definiert werden kann, muss erstmal ein \colorbold{Structtyp} deklariert werden. Im \colorbold{Abstract Syntax Tree} in Code~\ref{code:symboltabelle_für_die_deklaration_eines_structtyps} wird die \colorbold{Deklaration eines Structtyps} \smalltt{struct st \{int len; int ar[2];\};} durch die Komposition \smalltt{StructDecl(Name('st'), [Alloc(Writeable(), IntType('int'), Name('len')) Alloc(Writeable(), ArrayDecl([Num('2')], IntType('int')), Name('ar'))])} dargestellt.

Die \colorbold{Definition} einer Variable mit diesem \colorbold{Structtyp}  \smalltt{struct st st\_var;}  wird durch die Komposition \smalltt{Alloc(Writeable(), StructSpec(Name('st')), Name('st\_var'))} dargestellt.

\begin{code}
  \centering
  \numberedcodebox[minted language=text, minted options={highlightlines={4-9,15}}]{./code_examples/example_struct_decl_def.ast}
  \caption{Abstract Syntax Tree für die Deklaration eines Structtyps}
  \label{code:abstract_syntax_tree_für_die_deklaration_eines_structtyps}
\end{code}

Für den \colorbold{Structtyp} selbst wird in der \colorbold{Symboltabelle}, die in Code~\ref{code:symboltabelle_für_die_deklaration_eines_structtyps} dargestellt ist ein Eintrag mit dem \colorbold{Schlüssel} \smalltt{st} erstellt. Die Felder dieses Eintrags \smalltt{type\_qualifier}, \smalltt{datatype},  \smalltt{name}, \smalltt{position} und \smalltt{size} sind wie üblich belegt, allerdings sind in dem \smalltt{value\_address}-Feld die Attribute des \colorbold{Structtyps} \smalltt{[Name('len@st'), Name('ar@st')]} aufgelistet, sodass man über den \colorbold{Structtyp} \smalltt{st} die  \colorbold{Attribute} des Structtyps in der  \colorbold{Symboltabelle} nachschlagen kann. Die Schlüssel der \colorbold{Attribute} haben einen \colorbold{Suffix} \smalltt{@st} angehängt, der eine Art \colorbold{Scope} innerhalb des \colorbold{Structtyps} für seine Attribut darstellt. Es gilt foglich, dass \colorbold{innerhalb} eines \colorbold{Structtyps} zwei Attribute nicht gleich benannt werden können, aber dafür zwei \colorbold{unterschiedliche} \colorbold{Structtypen} ihre Attribute gleich benennen können.

Jedes der \colorbold{Attribute} \smalltt{[Name('len@st'), Name('ar@st')]} erhält auch einen eigenen Eintrag in der \colorbold{Symboltabelle}, wobei die Felder \smalltt{type\_qualifier}, \smalltt{datatype},  \smalltt{name}, \smalltt{value\_address}, \smalltt{position} und \smalltt{size} wie üblich belegt werden. Die Felder \smalltt{type\_qualifier}, \smalltt{datatype} und \smalltt{name} werden z.B. bei \smalltt{Name('ar@st')} mithilfe der Attribute von \smalltt{Alloc(Writeable(), ArrayDecl([Num('2')], IntType('int')), Name('ar'))])} belegt.

Für die \colorbold{Definition} einer Variable \smalltt{st\_var@main} mit diesem \colorbold{Structtyp} \smalltt{st} wird ein Eintrag in der \colorbold{Symboltabelle} angelegt. Das \smalltt{datatyp}-Feld enthält dabei den Namen des \colorbold{Structtyps} als Komposition \smalltt{StructSpec(Name('st'))}, wodurch jederzeit alle wichtigen Informationen zu diesem \colorbold{Structyp}\footnote{Wie z.B. vor allem die \colorbold{Größe} bzw. \colorbold{Anzahl an Speicherzellen}, die dieser \colorbold{Structtyp} einnimmt.} und seinen \colorbold{Attributen} in der  \colorbold{Symboltabelle} nachgeschlagen werden können.

% https://tex.stackexchange.com/questions/1959/allowing-line-break-at-in-inline-math-mode
\begin{Special_Paragraph}
  Die \colorbold{Größe} einer Variable \smalltt{st\_var}, die ihm \smalltt{size}-Feld des \colorbold{Symboltabelleneintrags} eingetragen ist und mit dem \colorbold{Structtyp} $\mathtt{struct\enspace st\enspace \{datatype_1\enspace attr_1;\enspace\ldots\enspace datatype_n\enspace attr_n;\};}$\footnote{Hier wird es der Einfachheit halber so dargestellt, als hätte die Programmiersprache $L_{PicoC}$ nicht die Fragwürdige Designentscheidung, auch die eckigen Klammern \smalltt{[]} für die Definition eines Arrays \colorbold{vor} die Variable zu schreiben von $\mathtt{L_C}$ übernommen. Es wird so getann, als würde der komplette \colorbold{Datentyp} immer \colorbold{hinter} der Variable stehen: \smalltt{datatype var}.} definiert ist ($\mathtt{struct\enspace st\enspace st\_var;}$), berechnet sich dabei aus der Summe der \colorbold{Größen} der einzelnen \colorbold{Datentypen} $\mathtt{datatype_1\enspace \ldots\enspace datatpye_n}$ der \colorbold{Attribute} $\mathtt{attr_1,\enspace \ldots\enspace attr_n}$ des \colorbold{Structtyps}: $\mathtt{size(st) = \sum^n_{i=1} size(datatype_i)}$.
\end{Special_Paragraph}

\begin{code}
  \centering
  \numberedcodebox[minted language=text, minted options={highlightlines={5-10,14-19,23-28,41-46}}]{./code_examples/example_struct_decl_def.st}
  \caption{Symboltabelle für die Deklaration eines Structtyps}
  \label{code:symboltabelle_für_die_deklaration_eines_structtyps}
\end{code}

\subsubsection{Initialisierung von Structs}

Die \colorbold{Initialisierung eines Structs} wird im Folgenden mithilfe des Beispiels in Code~\ref{code:picoc_code_für_initialisierung_von_structs} erklärt.

\begin{code}
  \centering
  \numberedcodebox[minted language=c, minted options={highlightlines={7}}]{./code_examples/example_struct_init.picoc}
  \caption{PicoC-Code für Initialisierung von Structs}
  \label{code:picoc_code_für_initialisierung_von_structs}
\end{code}

Im \colorbold{Abstract Syntax Tree} in Code~\ref{code:abstract_syntax_tree_für_initialisierung_von_structs} wird die \colorbold{Initialisierung eines Structs} \smalltt{struct st1 st = \{.attr1=var, .attr2=\{.attr=\{\{\&var, \&var\}\}\}\};} mithilfe der \colorbold{Komposition} \smalltt{Assign(Alloc(Writeable(), StructSpec(Name('st1')), Name('st')), Struct(\ldots))} dargestellt.

\begin{code}
  \centering
  \numberedcodebox[minted language=text, minted options={highlightlines={21}}]{./code_examples/example_struct_init.ast}
  \caption{Abstract Syntax Tree für Initialisierung von Structs}
  \label{code:abstract_syntax_tree_für_initialisierung_von_structs}
\end{code}


Im \colorbold{PicoC-Mon Pass} in Code~\ref{code:picoc_mon_pass_für_initialisierung_von_structs} wird die \colorbold{Komposition} \smalltt{Assign(Alloc(Writeable(), StructSpec(Name('st1')), Name('st')), Struct(\ldots))} auf fast dieselbe Weise ausgewertet, wie bei der \colorbold{Initialisierung eines Arrays} in Subkapitel~\ref{sec:initialisierung_von_arrays} daher wird um keine Wiederholung zu betreiben auf Subkapitel~\ref{sec:initialisierung_von_arrays} verwiesen. Um das ganze interressanter zu gestalten wurde das Beispiel in Code~\ref{code:picoc_code_für_initialisierung_von_structs} so gewählt, dass sich daran eine komplexere, mehrstufige Initialisierung mit \colorbold{verschiedenen} Datentypen erklären lässt.

Der \colorbold{Struct-Initializer} Teilbaum \smalltt{Struct([Assign(Name('attr1'), Name('var')), Assign(Name('attr2'), Struct([Assign(Name('attr'), Array([Array([Ref(Name('var')), Ref(Name('var'))])]))]))])}, der beim \colorbold{Struct-Initializer} \colorbold{Container-Knoten} anfängt, wird auf dieselbe Weise nach dem \colorbold{Depth-First-Search} Prinzip von \colorbold{links-nach-rechts} ausgewertet, wie es bei der \colorbold{Initialisierung eines Arrays} in Subkapitel~\ref{sec:initialisierung_von_arrays} bereits erklärt wurde.

Beim \colorbold{Iterieren} über den \colorbold{Teilbaum}, muss beim \colorbold{Struct-Initializer} nur beachtet werden, dass bei den \smalltt{Assign(lhs, exp)}-Knoten, über welche die \colorbold{Attributzuweisung} dargestellt wird (z.B. \smalltt{Assign(Name('attr2'), Struct([Assign(Name('attr'), Array([Array([Ref(Name('var')), Ref(Name('var'))])]))]))}) der Teilbaum beim rechten \smalltt{exp} Attribut weitergeht.

Im Allgemeinen gibt es beim \colorbold{Initialisieren} eines \colorbold{Arrays} oder \colorbold{Structs} im Teilbaum auf der \colorbold{rechten Seite}, der beim jeweiligen obersten \colorbold{Initializer} anfängt immer nur $3$ Fällte, man hat es auf der \colorbold{rechten} Seite entweder mit einem \colorbold{Struct-Initialiser}, einem \colorbold{Array-Initialiser} oder einem \colorbold{Logischen Ausdruck} zu tuen. Bei \colorbold{Array-} und \colorbold{Struct-Initialisier} wird einfach über diese nach dem \colorbold{Depth-First-Search} Schema von \colorbold{links-nach-rechts} iteriert und die Ergebnisse der \colorbold{Logischen Ausdrücken} in den \colorbold{Blättern} auf den \colorbold{Stack} gespeichert. Der Fall, dass ein \colorbold{Logischer Ausdruck} vorliegt erübrigt sich damit.

\begin{code}
  \centering
  \numberedcodebox[minted language=text, minted options={highlightlines={11-14}}]{./code_examples/example_struct_init.picoc_mon}
  \caption{PicoC-Mon Pass für Initialisierung von Structs}
  \label{code:picoc_mon_pass_für_initialisierung_von_structs}
\end{code}

Im \colorbold{RETI-Blocks Pass} in Code~\ref{code:reti_blocks_pass_für_initialisierung_von_structs} werden die \colorbold{Kompositionen} \smalltt{Exp(exp)}, \smalltt{Ref(exp)} und \smalltt{Assign(Global(Num('1')), Stack(Num('3')))} durch ihre entsprechenden \colorbold{RETI-Knoten} ersetzt.

\begin{code}
  \centering
  \numberedcodebox[minted language=text, minted options={highlightlines={18-20,22-25,27-30,32-38}}]{./code_examples/example_struct_init.reti_blocks}
  \caption{RETI-Blocks Pass für Initialisierung von Structs}
  \label{code:reti_blocks_pass_für_initialisierung_von_structs}
\end{code}

% Stack und Globale Statische Daten
\subsubsection{Zugriff auf Structattribut}
% Formel aus der Vorlesung, wo ist die hier?

Der \colorbold{Zugriff auf ein Structattribut} wird im Folgenden mithilfe des Beispiels in Code~\ref{code:picoc_code_für_zugriff_auf_structattribut} erklärt.

\begin{code}
  \centering
  \numberedcodebox[minted language=c, minted options={highlightlines={5}}]{./code_examples/example_struct_attr_access.picoc}
  \caption{PicoC-Code für Zugriff auf Structattribut}
  \label{code:picoc_code_für_zugriff_auf_structattribut}
\end{code}

Im \colorbold{Abstract Syntax Tree} in Code~\ref{code:abstract_syntax_tree_für_zugriff_auf_structattribut} wird der \colorbold{Zugriff auf ein Structattribut} \smalltt{st.y} mithilfe der \colorbold{Komposition} \smalltt{Exp(Attr(Name('st'), Name('y')))} dargestellt.

\begin{code}
  \centering
  \numberedcodebox[minted language=text, minted options={highlightlines={16}}]{./code_examples/example_struct_attr_access.ast}
  \caption{Abstract Syntax Tree für Zugriff auf Structattribut}
  \label{code:abstract_syntax_tree_für_zugriff_auf_structattribut}
\end{code}

Im \colorbold{PicoC-Mon Pass} in Code~\ref{code:picoc_mon_pass_für_zugriff_auf_structattribut} wird die Komposition \smalltt{Exp(Attr(Name('st'), Name('y')))} auf ähnliche Weise ausgewertet, wie die Komposition, die einen \colorbold{Zugriff auf ein Arrayelement} \smalltt{Exp(Subscr(Name('ar'), Num('0')))} in Subkapitel~\ref{sec:zugriff_auf_arrayindex} darstellt. Daher wird hier, um Wiederholung zu vermeiden nur in Kürze das Vorgehen umrissen und ansonsnten auf das Subkapitel~\ref{sec:zugriff_auf_arrayindex} verwiesen.

Die Komposition \smalltt{Exp(Attr(Name('st'), Name('y')))} wird genauso, wie in Subkapitel~\ref{sec:zugriff_auf_arrayindex} durch Kompositionen ersetzt, die sich in \colorbold{Anfangsteil}~\ref{sec:einleitungsteil_für_globale_statische_daten_und_stackframe}, \colorbold{Mittelteil}~\ref{sec:mittelteil_für_die_verschiedenen_derived_datatypes} und \colorbold{Schlussteil}~\ref{sec:schlussteil_für_die_verschiedenen_derived_datatypes} aufteilen lassen. In diesem Fall sind es \smalltt{Ref(Global(Num('0')))} (\colorbold{Anfangsteil}), \smalltt{Ref(Attr(Stack(Num('1')), Name('y')))} (\colorbold{Mittelteil}) und \smalltt{Exp(Stack(Num('1')))} (\colorbold{Schlussteil}). Der \colorbold{Anfangsteil} und \colorbold{Schlussteil} sind genau gleich, wie in Subkapitel~\ref{sec:zugriff_auf_arrayindex}.

Nur für den \colorbold{Mittelteil} wird eine andere Komposition \smalltt{Ref(Attr(Stack(Num('1')), Name('y')))} gebraucht. Diese Komposition \smalltt{Ref(Attr(Stack(Num('1')), Name('y')))} erfüllt die Aufgabe die \colorbold{Adresse}, ab der das \colorbold{Attribut} auf das zugegriffen wird anfängt zu berechnen. Dabei wurde die \colorbold{Anfangsadresse} des \colorbold{Structs} indem dieses Attribut liegt bereits vorher auf den \colorbold{Stack} gelegt.

Im Gegensatz zur Komposition \smalltt{Ref(Subscr(Stack(Num('2')), Stack(Num('1'))))} beim \colorbold{Zugriff auf einen Arrayindex} in Subkapitel~\ref{sec:zugriff_auf_arrayindex}, muss hier vorher nichts anderes als die \colorbold{Anfangsadresse} des \colorbold{Structs} auf dem \colorbold{Stack} liegen. Das \colorbold{Structattribut} auf welches zugegriffen wird steht bereits in der Komposition \smalltt{Ref(Attr(Stack(Num('1')), Name('y')))}, nämlich \smalltt{Name('y')}. Den \colorbold{Structtyp}, dem dieses Attribut gehört, kann man aus dem \textcolor{gray!90!black}{versteckten Attribut} \smalltt{datatype} herauslesen. Das \textcolor{gray!90!black}{versteckte Attribut} wird während des Kompiliervorgangs im \colorbold{PiocC-Mon Pass} dem \colorbold{Container-Knoten} \smalltt{Ref(exp, \textcolor{gray!90!black}{datatype})} angehängt.

  % Die \colorbold{Größe} einer Variable \smalltt{st\_var}, die ihm \smalltt{size}-Feld des \colorbold{Symboltabelleneintrags} eingetragen ist und mit dem \colorbold{Structtyp} $\mathtt{struct\enspace st\enspace \{datatype_1\enspace attr_1;\enspace\ldots\enspace datatype_n\enspace attr_n;\};}$\footnote{Hier wird es der Einfachheit halber so dargestellt, als hätte die Programmiersprache $L_{PicoC}$ nicht die Fragwürdige Designentscheidung, auch die eckigen Klammern \smalltt{[]} für die Definition eines Arrays \colorbold{vor} die Variable zu schreiben von $\mathtt{L_C}$ übernommen. Es wird so getann, als würde der komplette \colorbold{Datentyp} immer \colorbold{hinter} der Variable stehen: \smalltt{datatype var}.} definiert ist ($\mathtt{struct\enspace st\enspace st\_var;}$), berechnet sich dabei aus der Summe der \colorbold{Größen} der einzelnen \colorbold{Datentypen} $\mathtt{datatype_1\enspace \ldots\enspace datatpye_n}$ der \colorbold{Attribute} $\mathtt{attr_1,\enspace \ldots\enspace attr_n}$ des \colorbold{Structtyps}: $\mathtt{size(st) = \sum^n_{i=1} size(datatype_i)}$.


\begin{Special_Paragraph}
  Die Berechnung der \colorbold{Adresse}, ab der ein \colorbold{Structattribut} $\mathtt{attr_i}$ eines \colorbold{Structs} \smalltt{struct st st\_var} des \colorbold{Structtyps} $\mathtt{struct\enspace st\enspace \{datatype_1\enspace attr_1;\enspace\ldots\enspace datatype_n\enspace attr_n;\};}$ abgespeichert ist, kann mittels der Formel~\ref{eq:adresse_von_structattribut}:

  \numberwithin{equation}{section}

  \begin{equation}
  \mathtt{ref(st.attr_1\ldots .attr_n) = ref(st) + \sum_{i=1}^{n}\sum_{j=1}^{i} \operatorname{size}(datatype_j)} \\
    \label{eq:adresse_von_structattribut}
  \end{equation}
  aus der Betriebssysteme Vorlesung\footcite{scholl_betriebssysteme_2020} berechnet werden\footnote{\smalltt{ref(exp)} steht dabei für die Berechnung der \colorbold{Adresse} von \smalltt{exp}, wobei \smalltt{exp} z.B. \smalltt{ar[3][2]} sein könnte.}.

  Die Kompositionen \smalltt{Ref(Global(Num('0')))} und \smalltt{Ref(Stackframe(Num('2')))} repräsentiert dabei den Summanden $\smalltt{ref(ar)}$ in der Formel.

  Die Komposition \smalltt{Exp(Num('2'))} repräsentiert dabei einen \colorbold{Subindex} (z.B. \smalltt{i} in \smalltt{a[i][j][k]}) beim \colorbold{Zugriff auf ein Arrayelement}, der als Faktor $\mathtt{idx_i}$ in der Formel auftaucht.

  Der Komposition \smalltt{Ref(Subscr(Stack(Num('2')), Stack(Num('1'))))} repräsentiert dabei einen ausmultiplizierten Summanden $\mathtt{\left(\prod_{j=i+1}^{n} dim_{j}\right) \cdot idx_{i} \cdot size(datatpye)}$ in der Formel.

Die Komposition \smalltt{Exp(Stack(Num('1')))} repräsentiert dabei das Lesen des \colorbold{Inhalts} $\mathtt{M\left[ref(ar[idx_1]\ldots[idx_n])\right]}$ der Speicherzelle an der finalen \colorbold{Adresse}  $\mathtt{ref(ar[idx_1]\ldots[idx_n])}$.
\end{Special_Paragraph}

\begin{code}
  \centering
  \numberedcodebox[minted language=text, minted options={highlightlines={12-14}}]{./code_examples/example_struct_attr_access.picoc_mon}
  \caption{PicoC-Mon Pass für Zugriff auf Structattribut}
  \label{code:picoc_mon_pass_für_zugriff_auf_structattribut}
\end{code}

\begin{code}
  \centering
  \numberedcodebox[minted language=text, minted options={highlightlines={24-27,29-31,33-35}}]{./code_examples/example_struct_attr_access.reti_blocks}
  \caption{RETI-Blocks Pass für Zugriff auf Structattribut}
  \label{code:reti_blocks_pass_für_zugriff_auf_structattribut}
\end{code}

\subsubsection{Zuweisung an Structattribut}
\begin{code}
  \centering
  \numberedcodebox[minted language=c, minted options={highlightlines={5}}]{./code_examples/example_struct_attr_assignment.picoc}
  \caption{PicoC-Code für Zuweisung an Structattribut}
  \label{code:picoc_code_für_zuweisung_an_structattribut}
\end{code}

\begin{code}
  \centering
  \numberedcodebox[minted language=text, minted options={highlightlines={16}}]{./code_examples/example_struct_attr_assignment.ast}
  \caption{Abstract Syntax Tree für Zuweisung an Structattribut}
  \label{code:abstract_syntax_tree_für_zuweisung_an_structattribut}
\end{code}

\begin{code}
  \centering
  \numberedcodebox[minted language=text, minted options={highlightlines={12-15}}]{./code_examples/example_struct_attr_assignment.picoc_mon}
  \caption{PicoC-Mon Pass für Zuweisung an Structattribut}
  \label{code:picoc_mon_pass_für_zuweisung_an_structattribut}
\end{code}

\begin{code}
  \centering
  \numberedcodebox[minted language=text, minted options={highlightlines={24-26,28-31,33-35,37-40}}]{./code_examples/example_struct_attr_assignment.reti_blocks}
  \caption{RETI-Blocks Pass für Zuweisung an Structattribut}
  \label{code:reti_blocks_pass_für_zuweisung_an_structattribut}
\end{code}

\subsection{Umsetzung der Derived Datatypes im Zusammenspiel}
\subsubsection{Anfangsteil für Globale Statische Daten und Stackframe}
\label{sec:einleitungsteil_für_globale_statische_daten_und_stackframe}
% Stack und Globale Statische Daten, unterschieldihe Berechnung der Adressen
% unterschiedliche Adressberechnung
\begin{code}
  \centering
  \numberedcodebox[minted language=c]{./code_examples/example_derived_dts_introduction_part.picoc}
  \caption{PicoC-Code für den Anfangsteil}
  \label{code:picoc_code_einleitungsteil}
\end{code}

% spezielles Vorgehen bei PntrDecl

\begin{code}
  \centering
  \numberedcodebox[minted language=text]{./code_examples/example_derived_dts_introduction_part.ast}
  \caption{Abstract Syntax Tree für den Anfangsteil}
  \label{code:abstract_syntax_tree_einleitungsteil}
\end{code}

\begin{code}
  \centering
  \numberedcodebox[minted language=text]{./code_examples/example_derived_dts_introduction_part.picoc_mon}
  \caption{PicoC-Mon Pass für den Anfangsteil}
  \label{code:picoc_mon_pass_einleitungsteil}
\end{code}

\begin{code}
  \centering
  \numberedcodebox[minted language=text]{./code_examples/example_derived_dts_introduction_part.reti_blocks}
  \caption{RETI-Blocks Pass für den Anfangsteil}
  \label{code:reti_blocks_pass_einleitungsteil}
\end{code}

\subsubsection{Mittelteil für die verschiedenen Derived Datatypes}
\label{sec:mittelteil_für_die_verschiedenen_derived_datatypes}

\begin{code}
  \centering
  \numberedcodebox[minted language=c]{./code_examples/example_derived_dts_main_part.picoc}
  \caption{PicoC-Code für den Mittelteil}
  \label{code:picoc_code_mittelteil}
\end{code}

% spezielles Vorgehen bei PntrDecl

\begin{code}
  \centering
  \numberedcodebox[minted language=text]{./code_examples/example_derived_dts_main_part.ast}
  \caption{Abstract Syntax Tree für den Mittelteil}
  \label{code:abstract_syntax_tree_mittelteil}
\end{code}

\begin{code}
  \centering
  \numberedcodebox[minted language=text]{./code_examples/example_derived_dts_main_part.picoc_mon}
  \caption{PicoC-Mon Pass für den Mittelteil}
  \label{code:picoc_mon_pass_mittelteil}
\end{code}

\begin{code}
  \centering
  \numberedcodebox[minted language=text]{./code_examples/example_derived_dts_main_part.reti_blocks}
  \caption{RETI-Blocks Pass für den Mittelteil}
  \label{code:reti_blocks_pass_mittelteil}
\end{code}
% spezielles Vorgehen bei PntrDecl

\subsubsection{Schlussteil für die verschiedenen Derived Datatypes}
\label{sec:schlussteil_für_die_verschiedenen_derived_datatypes}
\begin{code}
  \centering
  \numberedcodebox[minted language=c]{./code_examples/example_derived_dts_final_part.picoc}
  \caption{PicoC-Code für den Schlussteil}
  \label{code:picoc_code_schlussteil}
\end{code}

\begin{code}
  \centering
  \numberedcodebox[minted language=text]{./code_examples/example_derived_dts_final_part.ast}
  \caption{Abstract Syntax Tree für den Schlussteil}
  \label{code:abstract_syntax_tree_schlussteil}
\end{code}

\begin{code}
  \centering
  \numberedcodebox[minted language=text]{./code_examples/example_derived_dts_final_part.picoc_mon}
  \caption{PicoC-Mon Pass für den Schlussteil}
  \label{code:picoc_mon_pass_schlussteil}
\end{code}

\begin{code}
  \centering
  \numberedcodebox[minted language=text]{./code_examples/example_derived_dts_final_part.reti_blocks}
  \caption{RETI-Blocks Pass für den Schlussteil}
  \label{code:reti_blocks_pass_schlussteil}
\end{code}

% Umgang, wenn Datentyp abrubt aufhört am Ende
 % ./content/Implementierung3_Struct_Derived.tex
  %!Tex Root = ../Main.tex
% ./Packete_und_Deklarationen.tex
% ./Titlepage.tex
% ./Motivation.tex
% ./Einführung.tex
% ./Implementierung1_Tables_DT_AST.tex,
% ./Implementierung2_Pntr_Array.tex,
% ./Implementierung3_Struct_Derived.tex,
% ./Ergebnisse_und_Ausblick.tex

\subsection{Umsetzung von Funktionen}
\label{sec:umsetzung_von_funktionen}

\subsubsection{Mehrere Funktionen}

Die Umsetzung \colorbold{mehrerer Funktionen} wird im Folgenden mithilfe des Beispiels in Code~\ref{code:picoc_code_für_3_funktionen} erklärt. Dieses Beispiel soll nur zeigen, wie Funktionen in verschiedenen, für die Kompilierung von Funktionen relevanten \colorbold{Passes} kompiliert werden. Das Beispiel ist so gewählt, dass es möglichst \colorbold{isoliert} von weiterem möglicherweise störendem Code ist.

\begin{code}
  \centering
  \numberedcodebox[minted language=c]{./code_examples/verbose_3_funs.picoc}
  \caption{PicoC-Code für 3 Funktionen}
  \label{code:picoc_code_für_3_funktionen}
\end{code}

Im \colorbold{Abstract Syntax Tree} in Code~\ref{code:abstract_syntax_tree_für_3_Funktionen} wird eine \colorbold{Funktion}, wie z.B. \smalltt{void fun(int param;)\{\enspace return param;\enspace \}} mit der Komposition \smalltt{FunDef(IntType(), Name('fun'), [Alloc(Writeable(), IntType(), Name('fun'))], [Return(Exp(Name('param')))])} dargestellt. Die einzelnen \colorbold{Attribute} dieses Container-Knoten sind in Tabelle~\ref{tab:picoc_knoten_teil_4} erklärt.

\begin{code}
  \centering
  \numberedcodebox[minted language=text]{./code_examples/verbose_3_funs.ast}
  \caption{Abstract Syntax Tree für 3 Funktionen}
  \label{code:abstract_syntax_tree_für_3_Funktionen}
\end{code}

Im \colorbold{PicoC-Blocks Pass} in Code~\ref{code:picoc_blocks_pass_für_3_Funktionen} werden die \colorbold{Statements} der Funktion in \colorbold{Blöcke} \smalltt{Block(name, stmts\_instrs)} aufgeteilt. Dabei bekommt ein Block \smalltt{Block(name, stmts\_instrs)}, der die Statements der Funktion vom \colorbold{Anfang} bis zum \colorbold{Ende} oder bis zum Auftauchen eines \smalltt{If(exp, stmts)}, \smalltt{IfElse(exp, stmts1, stmts2)}, \smalltt{While(exp, stmts)} oder \smalltt{DoWhile(exp, stmts)}\footnote{Eine Erklärung dazu ist in Unterkapitel~\ref{sec:picoc_blocks_pass_zweck} zu finden.} beinhaltet den \colorbold{Bezeichner} bzw. den \smalltt{Name(str)}-Token-Knoten der Funktion an sein \colorbold{Label} bzw. an sein \smalltt{name}-Attribut zugewiesen. Dem \colorbold{Bezeichner} wird vor der Zuweisung allerdings noch eine \colorbold{Nummer} angehängt \smalltt{<name>.<nummer>}\footnote{Der \colorbold{Grund} dafür kann im Unterkapitel~\ref{sec:picoc_blocks_pass_zweck} nachgelesen werden.}.

Es werden parallel dazu neue Zuordnungen im \colorbold{Dictionary} \smalltt{fun\_name\_to\_block\_name} hinzugefügt. Das \colorbold{Dicionary} ordnet einem \colorbold{Funktionsnamen} den \colorbold{Blocknamen} des Blockes, der das erste \colorbold{Statement} der Funktion enthält und dessen \colorbold{Bezeichner} \smalltt{<name>.<nummer>} bis auf die angehängte \colorbold{Nummer} identisch zu dem der Funktion ist zu\footnote{Das ist der \colorbold{Block}, über den im \colorbold{obigen letzten Paragraph} gesprochen wurde.}. Diese Zuordnung ist nötig, da \colorbold{Blöcke} noch eine \colorbold{Nummer} an ihren Bezeichner \smalltt{<name>.<nummer>} angehängt haben.

% Dieser Pass hat allerdings keine direkte wichtige Bedeutung bei der Kompilierung von \colorbold{Funktionen}. Die Erwähnung dieses \colorbold{Pass} ist nur wichtig, weil nach diesem Pass die \colorbold{Statements} in \colorbold{Blöcke} aufgeteilt sind.

\begin{code}
  \centering
  \numberedcodebox[minted language=text]{./code_examples/verbose_3_funs.picoc_blocks}
  \caption{PicoC-Blocks Pass für 3 Funktionen}
  \label{code:picoc_blocks_pass_für_3_Funktionen}
\end{code}

Im \colorbold{PicoC-Mon Pass} in Code~\ref{code:picoc_mon_pass_für_3_Funktionen} werden die \smalltt{FunDef(datatype, name, allocs, stmts)}-Container-Knoten komplett aufgelöst, sodass sich im \smalltt{File(name, decls\_defs\_blocks)}-Container-Knoten nur noch Blöcke befinden.

\begin{code}
  \centering
  \numberedcodebox[minted language=text]{./code_examples/verbose_3_funs.picoc_mon}
  \caption{PicoC-Mon Pass für 3 Funktionen}
  \label{code:picoc_mon_pass_für_3_Funktionen}
\end{code}

Nach dem \colorbold{RETI Pass} in Code~\ref{code:reti_pass_für_3_Funktionen} gibt es nur noch \colorbold{RETI-Instructions}, die Blöcke wurden entfernt und die \colorbold{RETI-Instructions} in diesen Blöcken wurden genauso zusammengefügt, wie die Blöcke angeordnet waren. Ohne die \colorbold{Kommentare} könnte man die Funktionen nicht mehr direkt ausmachen, denn die \colorbold{Kommentare} enthalten die \colorbold{Labelbezeichner} \smalltt{<name>.<nummer>} der Blöcke, die in diesem Beispiel immer zugleich bis auf die Nummer, dem \colorbold{Namen} der jeweiligen \colorbold{Funktion} entsprechen.

Da es in der \smalltt{main}-Funktion keinen \colorbold{Funktionsaufruf} gab, wird der Code, der nach der \colorbold{Instruction} in der \colorbold{markierten Zeile} kommt nicht mehr betreten. Funktionen sind im \colorbold{RETI-Code} nur dadurch existent, dass im RETI-Code \colorbold{Sprünge} (z.B. \smalltt{JUMP<rel> <im>}) zu den jeweils richtigen Positionen gemacht werden, nämlich dorthin, wo die \colorbold{RETI-Instructions}, die aus den \colorbold{Statemtens} einer \colorbold{Funktion} kompiliert wurden anfangen.

\begin{code}
  \centering
  \numberedcodebox[minted language=text, minted options={highlightlines={6}}]{./code_examples/verbose_3_funs.reti}
  \caption{RETI-Blocks Pass für 3 Funktionen}
  \label{code:reti_pass_für_3_Funktionen}
\end{code}

% einfügen unsichtbarer Returns bei void
\newlineparagraph{Sprung zur Main Funktion}

Im vorherigen Beispiel in Code~\ref{code:picoc_code_für_3_funktionen} war die \smalltt{main}-Funktion die \colorbold{erste} Funktion, die im Code vorkam. Dadurch konnte die \smalltt{main}-Funktion direkt betreten werden, da die \colorbold{Ausführung} des Programmes immer ganz vorne im \colorbold{RETI-Code} beginnt. Man musste sich daher keine Gedanken darum machen, wie man die \colorbold{Ausführung}, die von der \smalltt{main}-Funktion ausgeht überhaupt startet.

Im Beispiel in Code~\ref{code:picoc_code_für_funktionen_wobei_die_main_funktion_nicht_die_erste_Funktion_ist} ist die \smalltt{main}-Funktion allerdings \colorbold{nicht} die \colorbold{erste} Funktion. Daher muss dafür gesorgt werden, dass die \smalltt{main}-Funktion die erste Funktion ist, die ausgeführt wird.

\begin{code}
  \centering
  \numberedcodebox[minted language=c, minted options={highlightlines={8-10}}]{./code_examples/verbose_3_funs_main.picoc}
  \caption{PicoC-Code für Funktionen, wobei die main Funktion nicht die erste Funktion ist}
  \label{code:picoc_code_für_funktionen_wobei_die_main_funktion_nicht_die_erste_Funktion_ist}
\end{code}

Im \colorbold{RETI-Blocks Pass} in Code~\ref{code:reti_blocks_pass_für_funktionen_wobei_die_main_funktion_nicht_die_erste_Funktion_ist} sind die \colorbold{Funktionen} nur noch durch \colorbold{Blöcke} umgesetzt.

\begin{code}
  \centering
  \numberedcodebox[minted language=text, minted options={highlightlines={23-28}}]{./code_examples/verbose_3_funs_main.reti_blocks}
  \caption{RETI-Blocks Pass für Funktionen, wobei die main Funktion nicht die erste Funktion ist}
  \label{code:reti_blocks_pass_für_funktionen_wobei_die_main_funktion_nicht_die_erste_Funktion_ist}
\end{code}

\colorbold{Eine simple Möglichkeit} ist es, die \smalltt{main-Funktion} einfach nach \colorbold{vorne} zu schieben, damit diese als \colorbold{erstes} ausgeführt wird. Im \smalltt{File(name, decls\_defs)}-Container-Knoten muss dazu im \smalltt{decls\_defs}-Attribut, welches eine \colorbold{Liste von Funktionen} ist, die \smalltt{main}-Funktion an Index $0$ geschoben werden.

\colorbold{Eine andere Möglichkeit} und die Möglichkeit für die sich in der \colorbold{Implementierung} des \colorbold{PicoC-Compilers} entschieden wurde, ist es, wenn die \smalltt{main}-Funktion nicht die erste auftauchende Funktion ist, einen \smalltt{start.<nummer>}-Block als ersten Block einzufügen, der einen \smalltt{GoTo(Name('main.<nummer>'))}-Container-Knoten enthält, der im \colorbold{RETI Pass}~\ref{code:reti_pass_für_funktionen_wobei_die_main_funktion_nicht_die_erste_Funktion_ist} in einen Sprung zur \smalltt{main}-Funktion übersetzt wird.

In der Implementierung des \colorbold{PicoC-Compilers} wurde sich für diese Möglichkeit entschieden, da es für \colorbold{Studenten}, welche die Verwender des \colorbold{PiocC-Compilers} sein werden vermutlich am \colorbold{intuitivsten} ist,  wenn der \colorbold{RETI-Code} für die Funktionen an denselben Stellen relativ zueinander verortet ist, wie die Funktionsdefinitionen im \colorbold{PicoC-Code}.

Das \colorbold{Einsetzen} des \smalltt{start.<nummer>}-Blockes erfolgt im \colorbold{RETI-Patch Pass} in Code~\ref{code:reti_patch_pass_für_funktionen_wobei_die_main_funktion_nicht_die_erste_Funktion_ist}, da der \colorbold{RETI-Patch}-Pass der Pass ist, der für das \colorbold{Ausbessern} (engl. to patch) zuständig ist, wenn z.B. in manchen Fällen die \smalltt{main}-Funktion nicht die erste Funktion ist.

\begin{code}
  \centering
  \numberedcodebox[minted language=text, minted options={highlightlines={4-9,29-34}}]{./code_examples/verbose_3_funs_main.reti_patch}
  \caption{RETI-Patch Pass für Funktionen, wobei die main Funktion nicht die erste Funktion ist}
  \label{code:reti_patch_pass_für_funktionen_wobei_die_main_funktion_nicht_die_erste_Funktion_ist}
\end{code}

Im \colorbold{RETI Pass} in Code~\ref{code:reti_pass_für_funktionen_wobei_die_main_funktion_nicht_die_erste_Funktion_ist} wird das \smalltt{GoTo(Name('main.<nummer>'))} durch den entsprechenden \smalltt{Sprung} \smalltt{JUMP <distanz\_zur\_main\_funktion>} ersetzt und die Blöcke entfernt.

\begin{code}
  \centering
  \numberedcodebox[minted language=text, minted options={highlightlines={3,19}}]{./code_examples/verbose_3_funs_main.reti}
  \caption{RETI Pass für Funktionen, wobei die main Funktion nicht die erste Funktion ist}
  \label{code:reti_pass_für_funktionen_wobei_die_main_funktion_nicht_die_erste_Funktion_ist}
\end{code}

\subsubsection{Funktionsdeklaration und -definition und Umsetzung von Scopes}

In der Programmiersprache $L_C$ und somit auch $L_{PicoC}$ ist es notwendig, dass eine Funktion \colorbold{deklariert} ist, bevor man einen \colorbold{Funktionsaufruf} zu dieser Funktion machen kann. Das ist notwendig, damit \colorbold{Fehlermeldungen} ausgegeben werden können, wenn der \colorbold{Prototyp} (Definition~\ref{def:funktionsprototyp}) der Funktion nicht mit den \colorbold{Datentypen} der \colorbold{Argumente} oder der \colorbold{Anzahl} \colorbold{Argumente} übereinstimmt, die beim \colorbold{Funktionsaufruf} an die Funktion in einer \colorbold{festen} Reihenfolge übergeben werden.

Die Dekleration einer Funktion kann explizit erfolgen (z.B. \smalltt{int fun2(int var);}), wie in der im Beispiel in Code~\ref{code:picoc_code_für_funktionen_picoc_code_für_funktionen_wobei_eine_funktion_vorher_deklariert_werden_muss} \colorbold{markierten Zeile} \smalltt{1} oder zusammen mit der \colorbold{Funktionsdefinition} (z.B. \smalltt{void fun1()\{\}}), wie in den \colorbold{markierten Zeilen} \smalltt{3-4}.

In dem Beispiel in Code~\ref{code:picoc_code_für_funktionen_picoc_code_für_funktionen_wobei_eine_funktion_vorher_deklariert_werden_muss} erfolgt ein \colorbold{Funktionsaufruf} zur Funktion \smalltt{fun2}, die allerdings erst nach der \smalltt{main}-Funktion definiert ist. Daher ist eine \colorbold{Funktionsdekleration}, wie in der \colorbold{markierten Zeile} \smalltt{1} notwendig. Beim \colorbold{Funktionsaufruf} zur Funktion \smalltt{fun1} ist das \colorbold{nicht} notwendig, da die Funktion vorher \colorbold{definiert} wurde, wie in den \colorbold{markierten Zeilen} \smalltt{3-4} zu sehen ist.

\begin{Definition}{Funktionsprototyp}{funktionsprototyp}
  \colorbold{Deklaration} einer Funktion, welche den \colorbold{Funktionsbezeichner}, die \colorbold{Datentypen} der einzelnen \colorbold{Funktionsparameter}, die \colorbold{Parametereihenfolge} und den \colorbold{Rückgabewert} einer Funktion spezifiziert. Es ist \colorbold{nicht} möglich zwei Funktiondeklarationen mit dem \colorbold{gleichen} Funktionsbezeichner zu haben.\footnote{Der \colorbold{Funktionsprototyp} ist von der \colorbold{Funktionsignatur} zu unterschieden, die in Programmiersprache wie \smalltt{C++} und \smalltt{Java} für die \colorbold{Auflösung} von \colorbold{Überladung} bei z.B. \colorbold{Methoden} verwendet wird und sich in manchen Sprachen für den \colorbold{Rückgabewert} interessiert und in manchen nicht, je nach Umsetzung. In solchen Sprachen ist es möglich mehrere \colorbold{Methoden} oder \colorbold{Funktionen} mit dem \colorbold{gleichen} Bezeichner zu haben, solange sie sich durch die \colorbold{Datentpyen} von \colorbold{Parametern}, die \colorbold{Parameterreihenfolge}, manchmal auch \colorbold{Scopes} und \colorbold{Klassentpyen} usw. unterschieden.}\footcite{noauthor_what_nodate-4}
\end{Definition}

\begin{code}
  \centering
  \numberedcodebox[minted language=c, minted options={highlightlines={1,3-4}}]{./code_examples/verbose_3_funs_fun_decl.picoc}
  \caption{PicoC-Code für Funktionen, wobei eine Funktion vorher deklariert werden muss}
  \label{code:picoc_code_für_funktionen_picoc_code_für_funktionen_wobei_eine_funktion_vorher_deklariert_werden_muss}
\end{code}

Die \colorbold{Deklaration} einer \colorbold{Funktion} erfolgt mithilfe der \colorbold{Symboltabelle}, die in Code~\ref{code:symboltabelle_für_funktionen_picoc_code_für_funktionen_wobei_eine_funktion_vorher_deklariert_werden_muss} für das Beispiel in Code~\ref{code:picoc_code_für_funktionen_picoc_code_für_funktionen_wobei_eine_funktion_vorher_deklariert_werden_muss} dargestellt ist. Die \colorbold{Attribute} des \colorbold{Symbols} \smalltt{Symbols(type\_qual, datatype, name, val\_addr, pos, size)} werden wie üblich gesetzt. Dem \smalltt{datatype}-Attribut wird dabei einfach die komplette Komposition der \colorbold{Funktionsdeklaration} \smalltt{FunDecl(IntType('int'), Name('fun2'), [Alloc(Writeable(), IntType('int'), Name('var'))])} zugewiesen.

Die Varaiblen \smalltt{var@main} und \smalltt{var@fun2} der \smalltt{main}-Funktion und der Funktion \smalltt{fun2} haben unterschiedliche \colorbold{Scopes} (Definition~\ref{def:scope}). Die \colorbold{Scopes} der \colorbold{Funktionen} werden mittels eines \colorbold{Suffix} \smalltt{\dq @<fun\_name>\dq} umgesetzt, der an den \colorbold{Bezeichner} \smalltt{var} drangehängt wird: \smalltt{var@<fun\_name>}. Dieser \colorbold{Suffix} wird geändert sobald beim \colorbold{Top-Down}\footnote{D.h. von der \colorbold{Wurzel} zu den \colorbold{Blättern} eines Baumes} Durchiterieren über den \colorbold{Abstract Syntax Tree} des aktuellen \colorbold{Passes} ein \colorbold{Funktionswechsel} eintritt und über die Statements der nächsten Funktion iteriert wird, für die der \colorbold{Suffix} der neuen Funktion \smalltt{FunDef(name, datatype, params, stmts)} angehängt wird, der aus dem \smalltt{name}-Attribut entnommen wird.

Ein Grund, warum \colorbold{Scopes} über das Anhängen eines \colorbold{Suffix} an den \colorbold{Bezeichner} gelöst sind, ist, dass auf diese Weise die \colorbold{Schlüssel}, die aus dem \colorbold{Bezeichner} einer Variable und einem angehängten \colorbold{Suffix} bestehen, in der als \colorbold{Dictionary} umgesetzten \colorbold{Symboltabelle} eindeutig sind. Damit man einer Variable direkt den \colorbold{Scope} ablesen kann in dem sie definiert wurde, ist der \colorbold{Suffix} ebenfalls im \smalltt{Name(str)}-Token-Knoten des \smalltt{name}-Attribubtes eines \colorbold{Symbols} der Symboltabelle angehängt. Zur beseren Vorstellung ist dies ist in Code~\ref{code:symboltabelle_für_funktionen_picoc_code_für_funktionen_wobei_eine_funktion_vorher_deklariert_werden_muss} \colorbold{markiert}.

Die Variable \smalltt{var@main}, bei der es sich um eine \colorbold{Lokale Variable} der \smalltt{main}-Funktion handelt, ist nur innerhalb des \colorbold{Codeblocks} \smalltt{\{\}} der \smalltt{main}-Funktion \colorbold{sichtbar} und die Variable \smalltt{var@fun2} bei der es sich im einen \colorbold{Parameter} handelt, ist nur innerhalb des \colorbold{Codeblocks} \smalltt{\{\}} der Funktion \smalltt{fun2} \colorbold{sichtbar}. Das ist dadurch umgesetzt, dass der \colorbold{Suffix}, der bei jedem \colorbold{Funktionswechsel} angepasst wird, auch beim Nachschlagen eines \colorbold{Symbols} in der \colorbold{Symboltabelle} an den \colorbold{Bezeichner} der Variablen, die man nachschlagen will angehängt wird. Und da die Zuordnungen im \colorbold{Dictionary} \colorbold{eindeutig} sind, kann eine Variable nur in genau der Funktion nachgeschlagen werden, in der sie \colorbold{definiert} wurde.

Das Zeichen \smalltt{'@'} wurde aus einem bestimmten Grund als \colorbold{Trennzeichen} verwendet, nämlich, weil kein Bezeichner das Zeichen \smalltt{'@'} jemals selbst enthalten kann. Damit ist ausgeschlossen, dass falls ein \colorbold{Benutzer} des \colorbold{PicoC-Compilers} zufällig auf die Idee kommt seine Funktion genauso zu nennen (z.B. \smalltt{var@fun2} als Funktionsname), es zu Problemen kommt, weil bei einem Nachschlagen der \colorbold{Variable} die \colorbold{Funktion} nachgeschlagen wird.

\begin{Definition}{Scope (bzw. Sichtbarkeitsbereich)}{scope}
  \colorbold{Bereich} in einem Programm, in dem eine Variable \colorbold{sichtbar} ist und \colorbold{verwendet} werden kann.\footcite{thiemann_einfuhrung_2018}
\end{Definition}

\begin{code}
  \centering
  \numberedcodebox[minted language=text, minted options={highlightlines={34, 43}}]{./code_examples/verbose_3_funs_fun_decl.st}
  \caption{Symboltabelle für Funktionen, wobei eine Funktion vorher deklariert werden muss}
  \label{code:symboltabelle_für_funktionen_picoc_code_für_funktionen_wobei_eine_funktion_vorher_deklariert_werden_muss}
\end{code}

% Allocation von Variablen
% Stack und Globale Statische Daten
% die Sache mit Assign(Tmp, Global) und Assign(Global, Tmp)
% erwähnen, das Main Funktion keinen Stackframe hat
% zählen der Größe der lokalen Daten und Parameter
% TODO: Signatur zu Parameter umbenennen
\subsubsection{Funktionsaufruf}

\newlineparagraph{Ohne Rückgabewert}

% TODO: Betriebssysteme Vorlesung erwähnen auch in anderen Kapiteln

% Unsichtbares return
\begin{code}
  \centering
  \numberedcodebox[minted language=c]{./code_examples/verbose_fun_call_no_return_value.picoc}
  \caption{PicoC-Code für Funktionsaufruf ohne Rückgabewert}
  \label{code:picoc_code_für_funktionsaufruf_ohne_rückgabewert}
\end{code}

\begin{code}
  \centering
  \numberedcodebox[minted language=text]{./code_examples/verbose_fun_call_no_return_value.picoc_mon}
  \caption{PicoC-Mon Pass für Funktionsaufruf ohne Rückgabewert}
  \label{code:picoc_mon_pass_für_funktionsaufruf_ohne_rückgabewert}
\end{code}

\begin{code}
  \centering
  \numberedcodebox[minted language=text]{./code_examples/verbose_fun_call_no_return_value.reti_blocks}
  \caption{RETI-Blocks Pass für Funktionsaufruf ohne Rückgabewert}
  \label{code:reti_blocks_pass_für_funktionsaufruf_ohne_rückgabewert}
\end{code}

\begin{code}
  \centering
  \numberedcodebox[minted language=text]{./code_examples/verbose_fun_call_no_return_value.reti}
  \caption{RETI-Pass für Funktionsaufruf ohne Rückgabewert}
  \label{code:reti_pass_für_funktionsaufruf_ohne_rückgabewert}
\end{code}

\newlineparagraph{Mit Rückgabewert}

% todo: unsichtbare return statements

\begin{code}
  \centering
  \numberedcodebox[minted language=c]{./code_examples/verbose_fun_call_with_return_value.picoc}
  \caption{PicoC-Code für Funktionsaufruf mit Rückgabewert}
  \label{code:picoc_code_für_funktionsaufruf_mit_rückgabewert}
\end{code}

\begin{code}
  \centering
  \numberedcodebox[minted language=text]{./code_examples/verbose_fun_call_with_return_value.picoc_mon}
  \caption{PicoC-Mon Pass für Funktionsaufruf mit Rückgabewert}
  \label{code:picoc_mon_pass_für_funktionsaufruf_mit_rückgabewert}
\end{code}

\begin{code}
  \centering
  \numberedcodebox[minted language=text]{./code_examples/verbose_fun_call_with_return_value.reti_blocks}
  \caption{RETI-Blocks Pass für Funktionsaufruf mit Rückgabewert}
  \label{code:reti_blocks_pass_für_funktionsaufruf_mit_rückgabewert}
\end{code}

\begin{code}
  \centering
  \numberedcodebox[minted language=text]{./code_examples/verbose_fun_call_with_return_value.reti}
  \caption{RETI-Pass für Funktionsaufruf mit Rückgabewert}
  \label{code:reti_pass_für_funktionsaufruf_mit_rückgabewert}
\end{code}

\newlineparagraph{Umsetzung von Call by Sharing für Arrays}

\begin{code}
  \centering
  \numberedcodebox[minted language=c, minted options={highlightlines={1,7}}]{./code_examples/verbose_fun_call_by_sharing_array.picoc}
  \caption{PicoC-Code für Call by Sharing für Arrays}
  \label{code:picoc_code_für_call_by_sharing_für_arrays}
\end{code}

\begin{code}
  \centering
  \numberedcodebox[minted language=text, minted options={highlightlines={15-20}}]{./code_examples/verbose_fun_call_by_sharing_array.picoc_mon}
  \caption{PicoC-Mon Pass für Call by Sharing für Arrays}
  \label{code:picoc_mon_pass_für_call_by_sharing_für_arrays}
\end{code}


% https://tex.stackexchange.com/questions/298383/how-to-highlight-color-draw-attention-to-a-particular-snippet-in-minted/498614#498614
\begin{code}
  \centering
  \numberedcodebox[minted language=text, minted options={highlightlines={15,24}}]{./code_examples/verbose_fun_call_by_sharing_array.st}
  \caption{Symboltabelle für Call by Sharing für Arrays}
  \label{code:symboltabelle_für_call_by_sharing_für_arrays}
\end{code}

\begin{code}
  \centering
  \numberedcodebox[minted language=text, minted options={highlightlines={13-20}}]{./code_examples/verbose_fun_call_by_sharing_array.reti_blocks}
  \caption{RETI-Block Pass für Call by Sharing für Arrays}
  \label{code:reti_blocks_pass_für_call_by_sharing_für_arrays}
\end{code}

% die Sache mit dem erstetzen von ArryDecl durch PntrDecl

\newlineparagraph{Umsetzung von Call by Value für Structs}

\begin{code}
  \centering
  \numberedcodebox[minted language=c, minted options={highlightlines={8}}]{./code_examples/verbose_fun_call_by_value_struct.picoc}
  \caption{PicoC-Code für Call by Value für Structs}
  \label{code:picoc_code_für_call_by_value_für_structs}
\end{code}

% argmode für Struct Call by Value

\begin{code}
  \centering
  \numberedcodebox[minted language=text, minted options={highlightlines={15-19}}]{./code_examples/verbose_fun_call_by_value_struct.picoc_mon}
  \caption{PicoC-Mon Pass für Call by Value für Structs}
  \label{code:picoc_mon_pass_für_call_by_value_for_structs}
\end{code}

% hier könnte man anmerken, dass die Adressen unterschiedlich berechnet werden für Stack und Globale...

\begin{code}
  \centering
  \numberedcodebox[minted language=text, minted options={highlightlines={13-19}}]{./code_examples/verbose_fun_call_by_value_struct.reti_blocks}
  \caption{RETI-Block Pass für Call by Value für Structs}
  \label{code:reti_blocks_pass_für_call_by_value_for_structs}
\end{code}

% Struct wird wirklich kopiert durch speziellen Argmode

% \subsection{Umsetzung kleinerer Details}
% langen Sprüngen, großen Konstanten, Division durch 0
\section{Fehlermeldungen}
\subsection{Error Handler}
\subsection{Arten von Fehlermeldungen}
\subsubsection{Syntaxfehler}
\subsubsection{Laufzeitfehler}
% Fehlermeldung ist, wenn der Lexer (partielle Funktion) oder Parser nicht matcht
% Token und Nodes enthalten Position, im Transformer wird die Position von den Token auf die Nodes übertragen und auch die Symboltabelle speichert Position
            % ./content/Implementierung4_Fun.tex
  %!Tex Root = ../Main.tex
% ./Packete_und_Deklarationen.tex
% ./Titlepage.tex
% ./Motivation.tex
% ./Einführung.tex
% ./Implementierung1_Tables_DT_AST.tex,
% ./Implementierung2_Pntr_Array.tex,
% ./Implementierung3_Struct_Derived.tex,
% ./Implementierung4_Fun.tex,

\chapter{Ergebnisse und Ausblick}
\label{ch:ergebnisse_und_ausblick}

\section{Compiler}
\subsection{Überblick über Funktionen}
% Beschreiben des Shell Mode und der Commandline Options

\subsection{Vergleich mit GCC}
% Prinzipien eingehalten
% ähnliche Fehlermeldungen use. Veriablen auf dem Stack

\subsection{Showmode}
% erwähnen, dass das Bonus ist, Interpreter Bonus
% schönes Bildchen, wo RETI States erklärt wird
% Startprogram im EPROM
%
% (Projekt Open source)
\section{Qualitätssicherung}
\label{sec:qualitätssicherung}
% GCC + Execution entspricht einem einzigen großen Interpreter und beweist somit den linke Edge in 2.1
% Testsuite erklären, vielleicht Snippets usw. erwähnen
% Größe des Datensegments erklären
% Interpreter verhält sich identisch zu Spezifikation in Vorlesung, nur ohne INT und RTI
% Finale Anzahl Tests, Alte Tests und neue Tests überscnneiden sich teiweise in der Sache, die sie testen
% Verwendung der Testsuite inklusive Commandline Tutorial
% Aufteilung der Tests, Hart zu jedem Großtema ein oder zwei harte Tests
  \numberwithin{equation}{\tcbcounter}
  \begin{equation}
    \begin{tikzpicture}[auto, baseline=(current  bounding  box.center)]
      \node (reti) [align=center] at (135:3) {$L_{RETI}$-\\Maschinencode};
      \node (x_86) [align=center] at (45:3) {$L_{X_{86\_64}}$-\\Maschinencode};
      \node (picoc) [above=of reti] {Test in $L_{PicoC}$};
      \node (c) [above=of x_86] {Test in $L_C$};
      \node (output)  at (270:0) {$Output$};

      % https://tex.stackexchange.com/questions/24372/how-to-add-newline-within-node-using-tikz
      \draw[->] (picoc) to node[above] {convert\_to\_c} (c);
      \draw[->] (picoc) to node[left] {PicoC-Compiler} (reti);
      \draw[->] (c) to node[right] {GCC} (x_86);
      \draw[->] (reti) to[bend right] node[left] {RETI-Interpreter} (output);
      \draw[->] (x_86) to[bend left] node[right] {$X_{86\_64}$-CPU} (output);
    \end{tikzpicture}
    \label{eq:compiler_beziehungen}
  \end{equation}
% RETI-Interpreter erwähnen und erwähnen, dass das Bonus ist
% TODO: zusammenfassendes Bild
% \section{Kommentierter Kompiliervorgang}
\section{Erweiterungsideen}
Wenn eines Tages eine \colorbold{RETI-CPU} auf einem \colorbold{FPGA} implementiert werden sollte, sodass ein \colorbold{provisorisches Betriebssystem} darauf laufen könnte, dann wäre der nächste Schritt einen \colorbold{Self-Compiling Compiler} $C_{RETI\_PicoC}^{PicoC}$ (Defintion~\ref{def:self_compiling_compiler}) zu schreiben. Dadurch kann die \colorbold{Unabhängigkeit} von der Programmiersprache $L_Python$, in der der momentane Compiler $C_{PicoC}$ für $L_{PicoC}$ implementiert ist und die Unabhängigkeit von einer \colorbold{anderen Maschiene}, die bisher immer für das Cross-Compiling notwendig war erreicht werden.

\begin{Definition}{Self-compiling Compiler}{self_compiling_compiler}
  Compiler $C_w^w$, der in der Sprache $L_w$ \colorbold{geschrieben} ist, die er \colorbold{selbst} kompiliert. Also ein Compiler, der sich \colorbold{selbst} kompilieren kann.\footcite{earley_formalism_1970}
\end{Definition}

Will man nun für eine Maschiene $M_{RETI}$, auf der bisher keine anderen Programmiersprachen mittels \colorbold{Bootstrapping} (Definition~\ref{def:bootstrapping}) zum laufen gebracht wurden, den gerade beschriebenen \colorbold{Self-compiling Compiler} $C_{RETI\_PicoC}^{PicoC}$ implementieren und hat bereits den gesamtem \colorbold{Self-compiling Compiler} $C_{RETI\_PicoC}^{PicoC}$ in der Sprache  $L_{PicoC}$ geschrieben, so stösst man auf ein Problem, dass auf das \colorbold{Henne-Ei-Problem}\footnote{Beschreibt die Situation, wenn ein System sich selbst als \colorbold{Abhängigkeit} hat, damit es überhaupt einen \colorbold{Anfang} für dieses System geben kann. Dafür steht das Problem mit der \colorbold{Henne} und dem \colorbold{Ei} sinnbildlich, da hier die Frage ist, wie das ganze seinen Anfang genommen hat, da beides \colorbold{zirkular} voneinander abhängt.} reduziert werden kann. Man bräuchte, um den \colorbold{Self-compiling Compiler} $C_{RETI\_PicoC}^{PicoC}$ auf der \colorbold{Maschiene} $M_{RETI}$ zu kompilieren bereits einen kompilierten \colorbold{Self-compiling Compiler} $C_{RETI\_PicoC}^{PicoC}$, der mit der Maschienensprache $B_{RETI}$ läuft. Es liegt eine \colorbold{zirkulare Abhängigkeit} vor, die man nur auflösen kann, indem eine \colorbold{externe Entität} zur Hilfe nimmt.

Da man den gesamten \colorbold{Self-compiling Compiler} $C_{RETI\_PicoC}^{PicoC}$ nicht selbst komplett in der Maschienensprache $B_{RETI}$ schreiben will, wäre eine Möglichkeit, dass man den \colorbold{Cross-Compiler} $C_{PicoC}^{Python}$, den man bereits in der Programmiersprache $L_{Python}$ implementiert hat, der in diesem Fall einen \colorbold{Bootstrapping Compiler} (Definition~\ref{def:bootstrap_compiler}) darstellt, auf einer anderen Maschiene $M_{other}$ dafür nutzt, damit dieser den \colorbold{Self-compiling Compiler} $C_{RETI\_PicoC}^{PicoC}$ für die Maschiene $M_{RETI}$ kompiliert bzw. \colorbold{bootstraped} und man den kompilierten \colorbold{RETI-Maschiendencode} dann einfach von der Maschiene $M_{other}$ auf die Maschiene $M_{RETI}$ kopiert.\footnote{Im Fall, dass auf der Maschiene $M_{RETI}$ die Programmiersprache $L_{Python}$ bereits mittels \colorbold{Bootstrapping} zum Laufen gebracht wurde, könnte der \colorbold{Self-compiling Compiler} $C_{RETI\_PicoC}^{PicoC}$ auch mithife des \colorbold{Cross-Compilers} $C_{PicoC}^{Python}$ als \colorbold{externe Entität} und der Programmiersprache $L_{Python}$ auf der Maschiene $M_{RETI}$ selbst kompiliert werden.}

\begin{figure}[H]
  \centering
  \includegraphics[width=0.5\linewidth]{./figures/cross_compiling.png}
  \caption{Cross-Compiler als Bootstrap Compiler}
\end{figure}

\begin{Special_Paragraph}
  Einen ersten \colorbold{minimalen Compiler} $C_{2\_w\_min}$ für eine Maschiene $M_2$ und Wunschsprache $L_w$ kann man entweder mittels eines \colorbold{externen} \colorbold{Bootstrap Compilers} $C_w^o$ kompilieren\footnote{In diesem Fall, dem \colorbold{Cross-Compiler} $C_{PicoC}^{Python}$.} oder man schreibt ihn direkt in der \colorbold{Maschienensprache} $B_2$ bzw. wenn ein \colorbold{Assembler} vorhanden ist, in der \colorbold{Assemblesprache} $A_2$.

  Die letzte Option wäre allerdings nur beim allerersten Compiler $C_{first}$ für eine allererste \colorbold{abstraktere Programmiersprache} $L_{first}$ mit Schleifen, Verzweigungen usw. notwendig gewesen. Ansonsten hätte man immer eine Kette, die beim allersten Compiler $C_{first}$ anfängt fortführen können, in der ein Compiler einen anderen Compiler kompiliert bzw. einen ersten minimalen Compiler kompiliert und dieser minimale Compiler dann eine umfangreichere Version von sich kompiliert usw.
\end{Special_Paragraph}

\begin{Definition}{Minimaler Compiler}{minimaler_compiler}
  Compiler $C_{w\_min}$, der nur die \colorbold{notwendigsten Funktionalitäten} einer Wunschsprache $L_w$, wie \colorbold{Schleifen},  \colorbold{Verzweigungen} kompiliert, die für die Implementierung eines \colorbold{Self-compiling Compilers} $C_{w}^{w}$ oder einer \colorbold{ersten Version} $C_{w_i}^{w_i}$ des Self-compiling Compilers $C_w^w$ wichtig sind.\footnote{Den \colorbold{PicoC-Compiler} könnte man auch als einen \colorbold{minimalen Compiler} ansehen.}\footcite{thiemann_compilerbau_2021}
\end{Definition}

\begin{Definition}{Boostrap Compiler}{bootstrap_compiler}
  Compiler $C_w^o$, der es ermöglicht einen \colorbold{Self-compiling Compiler} $C_w^w$ zu \colorbold{boostrapen}, indem der Self-compiling Compiler $C_w^w$ mit dem \colorbold{Bootstrap Compiler} $C_w^o$ \colorbold{kompiliert} wird\footnote{Dabei kann es sich um einen \colorbold{lokal} auf der Maschiene selbst laufenden Compiler oder auch um einen \colorbold{Cross-Compiler} handeln.}. Der Bootstrapping Compiler stellt die  \colorbold{externe Entität} dar, die es ermöglicht die \colorbold{zirkulare Abhängikeit}, dass initial ein \colorbold{Self-compiling Compiler} $C_w^w$ bereits kompiliert vorliegen müsste, um sich selbst kompilieren zu können, zu brechen.\footcite{thiemann_compilerbau_2021}
\end{Definition}

Aufbauend auf dem \colorbold{Self-compiling Compiler} $C_{RETI\_PicoC}^{PicoC}$, der einen \colorbold{minimalen Compiler} (Definition~\ref{def:minimaler_compiler}) für eine Teilmenge der \colorbold{Programmiersprache} C bzw. $L_C$ darstellt, könnte man auch noch weitere Teile der Programmiersprache $C$ bzw. $L_C$ für die Maschiene $M_{RETI}$ mittels \colorbold{Bootstrapping} implementieren.\footnote{Natürlich könnte man aber auch einfach den \colorbold{Cross-Compiler} $C_{PicoC}^{Python}$ um weitere Funktionalitäten von $L_C$ erweitern, hat dann aber weiterhin eine \colorbold{Abhängigkeit} von der Programmiersprache $L_{Python}$.}

Das bewerkstelligt man, indem man \colorbold{iterativ} auf der Zielmaschine $M_{RETI}$ selbst, aufbauend auf diesem \colorbold{minimalen Compiler} $C_{RETI\_PicoC}^{PicoC}$, wie in Subdefinition~\ref{def:bootstrapping}{.1} den minimalen Compiler schrittweise zu einem immer vollständigeren \colorbold{C-Compiler} $C_C$ weiterentwickelt.

\begin{Definition}{Bootstrapping}{bootstrapping}
  Wenn man einen \colorbold{Self-compiling Compiler} $C_{w}^{w}$ einer Wunschsprache $L_w$ auf einer \colorbold{Zielmaschine} $M$ zum laufen bringt\footnote{Z.B. mithilfe eines \colorbold{Bootstrap Compilers}.}\footnote{Der Begriff hat seinen Ursprung in der englischen \colorbold{Redewendung} \glqq pulling yourself up by your own bootstraps\grqq, was im deutschen ungefähr der aus den \colorbold{Lügengeschichten des Freiherrn von Münchhausen} bekannten Redewendung \glqq sich am eigenen Schopf aus dem Sumpf ziehen\grqq entspricht.}\footnote{Hat man einmal einen solchen \colorbold{Self-compiling Compiler} $C_{w}^{w}$ auf der Maschiene $M$ zum laufen gebracht, so kann man den Compiler auf der Maschiene $M$ weiterentwicklern, ohne von externen Entitäten, wie einer bestimmten Sprache $L_o$, in der der Compiler oder eine frühere Version des Compilers ursprünglich geschrieben war abhängig zu sein.}\footnote{Einen Compiler in der Sprache zu schreiben, die er selbst kompiliert und diesen Compiler dann sich selbst kompilieren zu lassen, kann eine gute \colorbold{Probe aufs Exempel} darstellen, dass der Compiler auch wirklich funktioniert.}. Dabei ist die Art von \colorbold{Bootstrapping} in \ref{def:bootstrapping}{.1} nochmal gesondert hervorzuheben:

  {\normalfont\bfseries \thetcbcounter{.1}:\:\ignorespaces}
  Wenn man die \colorbold{aktuelle Version} eines \colorbold{Self-compiling Compilers} $C_{w_i}^{w_i}$ der Wunschsprache $L_{w_i}$ mithilfe von \colorbold{früheren Versionen} seiner selbst kompiliert. Man schreibt also z.B. die aktuelle Version des Self-compiling Compilers in der Sprache $L_{w_{i-1}}$, welche von der früheren Version des Compilers, dem Self-compiling Compiler $C_{w_{i-1}}^{w_{i-1}}$ kompiliert wird und schafft es so \colorbold{iterativ} immer umfangreichere Compiler zu bauen.\footnote{Es ist hierbei theoretisch nicht notwendig den \colorbold{letzten} Self-compiling Compiler $C_{w_{i-1}}^{w_{i-1}}$ für das Kompilieren des \colorbold{neuen} Self-compiling Compilers $C_{w_i}^{w_i}$ zu verwenden, wenn z.B. der \colorbold{Self-compiling Compiler} $C_{w_{i-3}}^{w_{i-3}}$ auch bereits alle Funktionalitäten, die beim Schreiben des \colorbold{Self-compiling Compilers} $C_w^w$ verwendet werden kompilieren kann.}\footnote{Der Begriff ist sinnverwandt mit dem \colorbold{Booten} eines Computers, wo die wichtigste Software, der \colorbold{Kernel} zuerst in den Speicher geladen wird und darauf aufbauend von diesem dann das Betriebssysteme, welches bei Bedarf dann \colorbold{Systemsoftware}, Software, die das Ausführen von Anwendungssoftware ermöglicht oder unterstützt, wie z.B. Treiber. und \colorbold{Anwendungssoftware}, Software, deren Anwendung darin besteht, dass sie dem Benutzer unmittelbar eine Dienstleistung zur Verfügung stellt, lädt.}\footcite{earley_formalism_1970}
\end{Definition}

\begin{figure}[H]
  \centering
  \includegraphics[width=0.66\linewidth]{./figures/bootstrapping.png}
  \caption{Iteratives Bootstrapping}
\end{figure}

\begin{Special_Paragraph}
  Auch wenn ein \colorbold{Self-compiling Compiler} $C_{w_i}^{w_i}$ in der Subdefinition~\ref{def:bootstrapping}{.1} selbst in einer früheren Version $L_{w_{i-1}}$ der Programmiersprache $L_{w_i}$ geschrieben wird, wird dieser nicht mit $C_{w_i}^{w_{i-1}}$ bezeichnet, sondern mit $C_{w_i}^{w_i}$, da es bei \colorbold{Self-compiling Compilern} darum geht, dass diese zwar in der Subdefinition~\ref{def:bootstrapping}{.1} eine frühere Version $C_{w_{i-1}}^{w_{i-1}}$ nutzen, um sich selbst kompilieren zu lassen, aber sie auch in der Lage sind sich selber zu kompilieren.
\end{Special_Paragraph}

% tail call
% partial evaluator, ohne besser zur Anschauung
% Garbage Collector
% Array Länge vorne speichern
% super einfach ein PicoPython zu machen von der Syntax her durch auswechseln der Grammatik
% richtigen Compiler mit Graph Coloring machen
% Debugger Informationen rein machen
% Linker und weitere Dateien
         % ./content/Ergebnisse_und_Ausblick.tex

  \appendix
  %!Tex Root=../Main.tex
% ./Packete_und_Deklarationen.tex
\chapter{Appendix}
                % ./content/Appendix.tex
  %!Tex Root = ../Main.tex
% ./Packete_und_Deklarationen.tex
\chapter{Danksagungen}
% kein Standardtext, wie er z.B. in Arbeitsurkunden steht
% reichlich b, meine wie sage, sonst würde Zeit nicht wert sein
% Vertrauen
% Arten von Menschen
% Kunde
% keine Selbverständlichkeit
            % ./content/Danksagungen.tex

  \printbibheading
  % \printbibliography[type=book,heading=subbibliography,title={Bücher}]
  % \printbibliography[type=article,heading=subbibliography,title={Artikel}]
  \printbibliography[type=online,heading=subbibliography,title={Online}]
  \printbibliography[type=book,heading=subbibliography,title={Bücher}]
  \printbibliography[type=article,heading=subbibliography,title={Artikel}]
  \printbibliography[type=unpublished,heading=subbibliography,title={Vorlesungen}]
  \printbibliography[nottype=book, nottype=article, nottype=online, nottype=unpublished,heading=subbibliography,title={Sonstige Quellen}]
  % ./Library.bib
\end{document}
