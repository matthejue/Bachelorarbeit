\documentclass{scrreprt}
\usepackage[margin=1.5cm]{geometry}
\usepackage[ngerman]{babel}
\usepackage{lipsum}
\usepackage[parfill]{parskip}
\usepackage{setspace}
\usepackage{graphicx}
\usepackage[hidelinks]{hyperref}

\begin{document}
  \begin{titlepage}
    \vspace{1cm}
    \center
    \textsc{\LARGE Albert Ludwigs Universität Freiburg}\\[0.5cm]
    \textsc{\Large Technische Fakultät}\\[2.0cm]

    \rule{\linewidth}{0.5mm}\\[0.4cm]
      {\huge \bfseries PicoC-Compiler\\\LARGE - \\\setstretch{1.1} Übersetzung eines Subsets von C in den Assembler der RETI-CPU\par}
      \vspace{0.5cm}
      \textsc{\large Bachelorarbeit}\\
      \rule{\linewidth}{0.5mm}\\[0.5cm]

    {\large \emph{Due Date:} 28\textsuperscript{th} April 2022}\\[2.5cm]

    \begin{minipage}{0.45\textwidth}
      \begin{flushleft} \large
        \emph{Author:}\\
        Jürgen Mattheis\\
        \hspace{1cm}\\
        \hspace{1cm}\\
        \hspace{1cm}\\
        \hspace{1cm}
      \end{flushleft}
    \end{minipage}
    ~
    \begin{minipage}{0.45\textwidth}
      \begin{flushright} \large
        \emph{Gutachter:}\\
        Prof. Dr. Scholl\\
        Prof. Dr. XY\\[0.64cm]
        \emph{Betreung:}\\
        M.Sc. Seufert\\
      \end{flushright}
    \end{minipage}

    \vspace{9.5cm}
    \large{Eine Bachelorarbeit am Lehrstuhl für}\\
    \large{Betriebssysteme}
  \end{titlepage}
  \newgeometry{margin=1.5cm}
  \tableofcontents
  \chapter{Motivation}
  \section{PicoC und RETI}
  \section{Problemstellung}
  \section{Compiler und Interpreter}
  \chapter{Einführung}
  \section{Grammatiken}
  \subsection{Konkrete Syntax}
  \subsection{Chromsky Hierarchie}
  \subsection{Reguläre Sprachen}
  \subsection{Kontextfreie Sprachen}
  \subsection{Linksrekursiv und Rechtrekursiv}
  \subsection{Ableitungsbaum}
  \section{Lexikalische Analyse}
  \subsection{Lexer}
  \section{Syntax Analyse}
  \subsection{Derivation Tree}
  \subsection{Abtrakte Syntax}
  \subsection{Parser}
  \subsection{Descent Parsing}
  \subsection{First and Follow Set}
  \subsection{Lookahead}
  \subsection{Aktionen}
  \section{Code Generation}
  \subsection{Passes}
  \subsection{T-Diagramme}
  \chapter{Implementierung}
  \section{Grammatiken}
  \subsection{PicoC}
  \subsection{RETI}
  \section{Lexikalische Analyse}
  \subsection{LL(1) Recursive-Descent Lexer}
  \section{Syntax Analyse}
  \subsection{Normalized Heterogeneous ASTNode}
  \subsection{LL(k) Recursive-Descent Parser}
  \subsection{Backtracking Parser}
  \subsection{Visitor und Transformer}
  \subsection{Lark}
  \section{Code Generation}
  \subsection{Symbol Table for Nested Scopes}
  \subsection{PicoC-zu-PicoC-Blocks Pass}
  \subsection{PicoC-Blocks-zu-RETI-Blocks Pass}
  \subsection{RETI-Blocks-zu-RETI Pass}
  \chapter{Ergebnisse und Ausblick}
  \section{Vergleich eigener Parser und Lark}
  \section{Beispielhafte Auführung}
  \section{Qualitätskontrolle}
  \section{Erweiterungsideen}
  \lipsum

\end{document}
