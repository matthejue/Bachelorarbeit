\documentclass{scrreprt}
\usepackage[margin=1.5cm]{geometry}
\usepackage[ngerman]{babel}
\usepackage{lipsum}
\usepackage[parfill]{parskip}
\usepackage{setspace}
\usepackage{graphicx}

\begin{document}
  \begin{titlepage}
    \vspace{1cm}
    \center
    \textsc{\LARGE Albert Ludwigs Universität Freiburg}\\[0.5cm]
    \textsc{\Large Technische Fakultät}\\[2.0cm]

    \rule{\linewidth}{0.5mm}\\[0.4cm]
      {\huge \bfseries PicoC-Compiler\\\LARGE-\\\setstretch{1.1} Übersetzung eines Subsets von C in den Assembler der RETI-CPU\par}\\[0.4cm]
      \textsc{\large Bachelorarbeit}\\
      \rule{\linewidth}{0.5mm}\\[0.5cm]

    {\large \emph{Due Date:} 28\textsuperscript{th} April 2022}\\[2.5cm]

    \begin{minipage}{0.45\textwidth}
      \begin{flushleft} \large
        \emph{Author:}\\
        Jürgen Mattheis\\
        \hspace{1cm}\\
        \hspace{1cm}\\
        \hspace{1cm}\\
        \hspace{1cm}
      \end{flushleft}
    \end{minipage}
    ~
    \begin{minipage}{0.45\textwidth}
      \begin{flushright} \large
        \emph{Gutachter:}\\
        Prof. Dr. Scholl\\
        Prof. Dr. XY\\[0.64cm]
        \emph{Betreung:}\\
        M.Sc. Seufert\\
      \end{flushright}
    \end{minipage}

    \vspace{9.5cm}
    \large{Eine Bachelorarbeit am Lehrstuhl für}\\
    \large{Betriebssysteme}
  \end{titlepage}
  \newgeometry{margin=1.5cm}
  \tableofcontents
  \chapter{Motivation}
  \section{}
  \chapter{Einführung}
  \chapter{Implementierung}
  \chapter{Ergebnisse und Ausblick}
  \lipsum

\end{document}
