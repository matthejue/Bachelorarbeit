%!Tex Root = ../Main.tex
% ./Packete_und_Deklarationen.tex
\chapter{Einführung}
\label{ch:einführung}

\section{Compiler und Interpreter}
\begin{Definition}{Compiler}{compiler}


\end{Definition}
\begin{Definition}{Interpreter}{Interpreter}
% TODO: Bild semantisch gleiche Bedeutung
\end{Definition}
\subsection{T-Diagramme}
\begin{Definition}{T-Diagram}{t_diagram}
\end{Definition}
\section{Grammatiken}
\section{Grundlagen}
\begin{Definition}{Sprache}{Sprache}
\end{Definition}
\begin{Definition}{Chromsky Hierarchie}{chromsky_hierarchie}
\end{Definition}
\begin{Definition}{Grammatik}{grammatik}
\end{Definition}
\begin{Definition}{Reguläre Sprachen}{reguläre_sprachen}
\end{Definition}
\begin{Definition}{Ableitung}{ableitung}
\end{Definition}
\begin{Definition}{Links- und Rechtsableitung}{links_und_rechtsableitung}
\end{Definition}
\begin{Definition}{Linksrekursive Grammatiken}{linksrekursive_grammatiken}
Eine \colorbold{Grammatik} ist \colorbold{linksrekursiv}, wenn sie ein  \colorbold{Nicht-Terminalsymbol} enthält, dass \colorbold{linksrekursiv} ist.

Ein \colorbold{Nicht-Terminalsymbol} ist  \colorbold{linksrekursiv}, wenn das \colorbold{linkeste Symbol} in einer seiner \colorbold{Produktionen} es selbst ist oder zu sich selbst gemacht werden kann durch eine Folge von Ableitungen:
\begin{equation*}
  A \Rightarrow^{*} Aa,
\end{equation*}
wobei $a$ eine beliebige Folge von \colorbold{Terminalsymbolen} und \colorbold{Nicht-Terminalsymbolen} ist.
\end{Definition}
\subsection{Mehrdeutige Grammatiken}
\begin{Definition}{Ableitungsbaum}{ableitungsbaum}
% TODO: Bild hierfür
\end{Definition}
\begin{Definition}{Mehrdeutige Grammatik}{mehrdeutige_grammatik}
% TODO: (Bild hierfür)
\end{Definition}
\subsection{Präzidenz und Assoziativität}
\begin{Definition}{Assoziativität}{assoziativität}
\end{Definition}
\begin{Definition}{Präzidenz}{präzidenz}
\end{Definition}
\begin{Definition}{Wortproblem}{wortproblem}
  % zu Grammatiken schieben
\end{Definition}
\begin{Definition}{LL(k)-Grammatik}{llk_grammatik}
  Eine Grammatik ist \colorbold{LL(k)} für $k\in\mathbb{N}$, falls jeder Ableitungsschritt eindeutig durch die nächsten $k$ \colorbold{Symbole} des \colorbold{Eingabeworts} bzw. in Bezug zu Compilerbau \colorbold{Token} des \colorbold{Inputstrings} zu bestimmen ist\footnote{Das wird auch als \colorbold{Lookahead} von $k$ bezeichnet.}. Dabei steht \colorbold{LL} für \colorbold{l}eft-to-right und \colorbold{l}eftmost-derivation, da das \colorbold{Eingabewort} von \colorbold{links nach rechts} geparsed und immer \colorbold{Linksableitungen} genommen werden müssen\footnote{Wobei sich das mit den \colorbold{Linksableitungen} automatisch ergibt, wenn man das Eingabewort von  \colorbold{links-nach-rechts} parsed und jeder der nächsten $k$ \colorbold{Ableitungsschritte} eindeutig sein soll.}, damit die obige Bedingung mit den \colorbold{nächsten} $k$ Symbolen gilt.
\end{Definition}
\begin{Definition}{Kontextfreie Sprachen}{kontextfreie_sprachen}
\end{Definition}
% \subsection{Linksrekursiv und Rechtrekursiv}
\section{Lexikalische Analyse}
\label{sec:lexikalische_analyse}

Die \colorbold{Lexikalische Analyse} bildet üblicherweise die erste Ebene innerhalb der \colorbold{Pipe Architektur} bei der Implementierung von Compilern. Die Aufgabe der lexikalischen Analyse ist vereinfacht gesagt, in einem Inputstring, z.B. dem Inhalt einer Datei, welche in \colorbold{UTF-8} codiert ist, Folgen endlicher Symbole (auch \colorbold{Wörter} genannt) zu finden, die bestimmte \colorbold{Pattern} (Definition~\ref{def:pattern}) matchen, die durch eine \colorbold{reguläre Grammatik} spezifiziert sind.

\begin{Definition}{Pattern}{pattern}
  \colorbold{Beschreibung} aller möglichen \colorbold{Lexeme} einer Menge $\mathbb{P}_{T}$, die einem bestimmten \colorbold{Token} $T$ zugeordnet werden.
  Die Menge $\mathbb{P}_{T}$ ist eine möglicherweise unendliche Menge von \colorbold{Wörtern}, die sich mit den Produktionen einer \colorbold{regulären Grammatik} ${G}_{Lex}$ einer \colorbold{regulären Sprache} ${L}_{Lex}$ beschreiben lassen \footnote{Als Beschreibungswerkzeug können aber auch z.B. reguläre Ausdrücke hergenommen werden.}, die für die Beschreibung eines \colorbold{Tokens} $T$ zuständig sind.\footcite{noauthor_what_nodate}
\end{Definition}

Diese Folgen endlicher Symoble werden auch \colorbold{Lexeme} (Definition~\ref{def:lexeme}) genannt.

\begin{Definition}{Lexeme}{lexeme}
  Ein \colorbold{Lexeme} ist ein \colorbold{Wort} aus dem Inputstring, welches das \colorbold{Pattern} für eines der \colorbold{Token} $T$ einer \colorbold{Sprache} ${L}_{Lex}$ matched.
\footcite{noauthor_what_nodate}
\end{Definition}

Diese \colorbold{Lexeme} werden vom \colorbold{Lexer} im \colorbold{Inputstring} identifziert und \colorbold{Tokens} $T$ zugeordnet (Definition~\ref{def:lexer}). Die \colorbold{Tokens} sind es, die letztendlich an die \colorbold{Syntaktische Analyse} weitergegeben werden.

\begin{Definition}{Lexer (bzw. Scanner)}{lexer}
  Ein \colorbold{Lexer} ist eine \colorbold{partielle} Funktion \hspace{0.2cm}$lex: \Sigma^{*} \rightharpoonup (N \times W)^{*}$, welche ein \colorbold{Wort} aus $\Sigma^{*}$ auf ein \colorbold{Token} $T$ mit einem \colorbold{Tokennamen} $N$ und einem \colorbold{Tokenwert} $W$ abbildet, falls diese Folge von Symbolen sich unter der \colorbold{regulären Grammatik} ${G}_{Lex}$, der \colorbold{regulären Sprache} ${L_{Lex}}$ abbleiten lässt.\footcite{noauthor_lecture-notes-2021_2022}
\end{Definition}

Ein \colorbold{Lexer} ist im Allgemeinen eine \colorbold{partielle Funktion}, da es Zeichenfolgen geben kann, die kein \colorbold{Pattern} eines \colorbold{Tokens} der Sprache $L_{Lex}$ matchen. In Bezug auf eine Implementierung, wird, wenn der Lexer Teil der Implementierung eines Compilers ist, in diesem Fall eine \colorbold{Fehlermeldung} ausgegeben.

Eine weitere Aufgabe der \colorbold{Lekikalischen Analyse} ist es jegliche für die Weiterverarbeitung unwichtigen Symbole, wie Leerzeichen \,\textvisiblespace\,, Newline \verb|\n|\footnote{In Unix Systemen wird für Newline das ASCII Symbol \colorbold{line feed}, in Windows hingegen die ASCII Symbole \colorbold{carriage return} und \colorbold{line feed} nacheinander verwendet. Das wird aber meist durch die verwendete Porgrammiersprache, die man zur Inplementierung des Lexers nutzt wegabstrahiert.} und Tabs \verb|\t| aus dem Inputstring herauszufiltern. Das geschieht mittels des \colorbold{Lexers}, der allen für die \colorbold{Syntaktische Analyse} unwichtige Zeichen das leere Wort $\epsilon$ zuordnet. Das ist auch im Sinne der Definition, denn $\epsilon \in \Sigma^{*}$.
Nur das, was für die \colorbold{Syntaktische Analyse} wichtig ist, soll weiterverarbeitet werden, alles andere wird herausgefiltert.

Der Grund warum nicht einfach nur die \colorbold{Lexeme} an die \colorbold{Syntaktische Analyse} weitergegeben werden und der Grund für die Aufteilung des \colorbold{Tokens} in \colorbold{Tokenname} und \colorbold{Tokenwert} ist, weil z.B. die Bezeichner von Variablen, Konstanten und Funktionen beliebige Zeichenfolgen sein können, wie \smalltt{my\_fun}, \smalltt{my\_var} oder \smalltt{my\_const} und es auch viele verschiedenen Zahlen gibt, wie \smalltt{42}, \smalltt{314} oder \smalltt{12}. Die Überbegriffe bzw. Tokennamen für beliebige Bezeichner von Variablen, Konstanten und Funktionen und beliebige Zahlen sind aber trotz allem z.B. \smalltt{Zahl} und \smalltt{Bezeichner}.

Ein \colorbold{Lexeme} ist damit aber nicht das gleiche, wie der \colorbold{Tokenwert}, denn z.B. im Falle von PicoC kann z.B. der Wert $99$ durch zwei verschiedene Literale darstellt werden, einmal als ASCII-Zeichen \smalltt{'c'} und des Weiteren auch in Dezimalschreibweise als \smalltt{99}\footnote{Die Programmiersprache Python erlaubt es z.B. diesern Wert auch mit den Literalen \smalltt{0b1100011} und \smalltt{0x63} darzustellen.}. Der \colorbold{Tokenwert} ist jedoch der letztendliche Wert an sich, unabhängig von der Darstellungsform.

  Die \colorbold{Grammatik} $G_{Lex}$, die zur Beschreibung der Token $T$ einer regulären Sprache $L_{Lex}$ verwendet wird, ist üblicherweise \colorbold{regulär}, da ein typischer \colorbold{Lexer} immer nur \colorbold{ein Symbol} vorausschaut\footnote{Man nennt das auch einem \colorbold{Lookahead} von $1$}, unabhängig davon, was für Symbole davor aufgetaucht sind. Die übliche Implementierung eines \colorbold{Lexers} merkt sich nicht, was für Symbole davor aufgetaucht sind.

% TODO: später erwähnen, dass alle Produtkionen der Grammatik G_lex eine reguläre Form haben, was der Beweis ist

\begin{Special_Paragraph}
  Um Verwirrung verzubäugen ist es wichtig folgende Unterscheidung hervorzuheben: Wenn von \colorbold{Symbolen} die Rede ist, so werden in der \colorbold{Lexikalischen Analyse}, der \colorbold{Syntaktische Analyse} und der \colorbold{Code Generierung}, auf diesen verschiedenen Ebenen unterschiedliche Konzepte als Symbole bezeichnet.

  In der Lexikalischen Analyse sind einzelne \colorbold{Zeichen eines Zeichensatzes} die Symbole.

  In der Syntaktischen Analyse sind die \colorbold{Tokennamen} die Symbole.

  In der Code Generierung sind die \colorbold{Bezeichner von Variablen, Konstanten und Funktionnen} die Symbole\footnote{Das ist der Grund, warum die Tabelle, in der Informationen zu Identifiern gespeichert werden aus Kapitel~\ref{ch:implementierung} Symboltabelle genannt wird.}.
\end{Special_Paragraph}

\begin{Definition}{Literal}{literal}
  Eine von möglicherweise vielen weiteren \colorbold{Darstellungsformen} für ein und denselben \colorbold{Wert}.
  % TODO: vielleicht schönes Bildchen
\end{Definition}

Um eine Gesamtübersicht über die \colorbold{Lexikalische Analyse} zu geben, ist in Abbildung~\ref{fig:lexikalische_analyse_veranschaulichung} die Lexikalische Analyse an einem Beispiel veranschaulicht.

\begin{figure}[h]
  \codebox[title=Inputstring, remember as=inputstring, width=0.2\linewidth, nobeforeafter]{./code_examples/example1.picoc}
  \hfill
  \treebox[title=Tokenfolge, remember as=tokenfolge, width=0.6\linewidth, nobeforeafter]{./code_examples/example1.tokens}

  \begin{tikzpicture}[overlay,remember picture,line width=1mm,draw=PrimaryColor]
  \draw[->] (inputstring.east) to[bend right] node[above] {Lexer} (tokenfolge.west);
  \end{tikzpicture}
  \caption{Veranschaulichung der Lexikalischen Analyse}
  \label{fig:lexikalische_analyse_veranschaulichung}
\end{figure}

\section{Syntaktische Analyse}
In der \colorbold{Syntaktischen Analyse} ist für einige Sprachen eine \colorbold{Kontextfreie Grammatik} $G_{Parse}$ notwendig, um diese Sprache zu beschreiben, da viele Programmiersprachen z.B. für \colorbold{Funktionsaufrufe} \verb|fun(arg)| und \colorbold{Codeblöcke} \verb|if(1){}| syntaktische Mittel verwenden, die es notwendig machen sich zu merken wieviele öffnende Klammern \verb|'('| bzw. öffnende geschweifte Klammern \verb|'{'| es momentan gibt, die noch nicht durch eine enstsprechende schließende Klammer \verb|')'| bzw. schließende geschweifte Klammer \verb|'}'| geschlossen wurden.

% TODO: später erwähnen, dass alle Produktionen der Grammatik G_parse eine kontexfreie Form haben, was der Beweis ist

Die \colorbold{Syntax}, in welcher der \colorbold{Inputstring} aufgeschrieben ist, wird auch als \colorbold{Konkrette Syntax} (Definition~\ref{def:konkrette_syntax}) bezeichnet. In einem Zwischenschritt, dem \colorbold{Parsen} wird aus diesem Inputstring mithilfe eines \colorbold{Parsers} (Definition~\ref{def:parser}), ein \colorbold{Derivation Tree} (Definition~\ref{def:derivation_tree}) generiert, der als Zwischenstufe hin zum einem \colorbold{Abstrakt Syntax Tree} (Definition~\ref{def:abstrakte_syntax_tree}) dient. Für einen ordentlichen Code ist es vor allem im Compilerbau förderlich kleinschrittig vorzugehen, deshalb erst die Generierung des \colorbold{Derivation Tree} und dann der \colorbold{Abstrakt Syntax Tree}.

\begin{Definition}{Konkrette Syntax}{konkrette_syntax}
  \colorbold{Syntax} einer \colorbold{Sprache}, die durch die \colorbold{Grammatiken} $G_{Lex}$ und $G_{Parse}$ zusammengenommen beschrieben wird.

  Ein \colorbold{Programm} in seiner \colorbold{Textrepräsentation}, wie es in einer Textdatei nach den Produktionen der \colorbold{Grammatiken} $G_{Lex}$ und $G_{Parse}$ abgeleitet steht, bevor man es kompiliert, ist in \colorbold{Konkretter Syntax} aufgeschrieben.\footcite{noauthor_course_2022}
\end{Definition}

\begin{Definition}{Derivation Tree (bzw. Parse Tree)}{derivation_tree}
  \colorbold{Compilerinterne Darstellung} eines in \colorbold{Konkretter Syntax} geschriebenen Inputstrings als \colorbold{Baumdatenstruktur}, in der \colorbold{Nichtterminalsymbole} die \colorbold{Inneren Knoten} des Baumes und \colorbold{Terminalsymbole} die \colorbold{Blätter} des Baumes bilden. Jede \colorbold{Produktions} der \colorbold{Grammatik} $G_{Parse}$, die ein Teil der \colorbold{Konkrette Syntax} ist,  wird zu einem eigenen \colorbold{Knoten}.

  Der \colorbold{Derivation Tree} wird optimalerweise immer so konstruiert bzw. die \colorbold{Konkrette Syntax} immer so definiert, dass sich möglichst einfach ein \colorbold{Abstrakt Syntax Tree} daraus konstruieren lässt.

% TODO: vielleicht ein hübsches Bildchen
\end{Definition}

\begin{Definition}{Parser}{parser}
  Ein Programm, dass eine \colorbold{Eingabe} in eine für die \colorbold{Weiterverbeitung} taugliche Form bringt.

  % https://www.overleaf.com/learn/latex/Counters
  \newcounter{subdefcounter}
  \counterwithin{subdefcounter}{\tcbcounter}
  \setcounter{subdefcounter}{1}

  % https://tex.stackexchange.com/questions/7627/how-to-reference-paragraph
  \titleformat{\paragraph}[runin]{\normalfont\normalsize\bfseries}{}{0mm}{}[:]

  % https://tex.stackexchange.com/questions/7627/how-to-reference-paragraph
  \paragraph{\thesubdefcounter}\label{par:parser}
  In Bezug auf Compilerbau ist ein \colorbold{Parser} ein Programm, dass einen Inputstring von \colorbold{Konkretter Syntax} in die compilerinterne Darstellung eines \colorbold{Derivation Tree} übersetzt, was auch als  \colorbold{Parsen} bezeichnet wird\footnote{Es gibt allerdings auch alternative Definitionen, denen nach ein Parser in Bezug auf Compilerbau ein Programm ist, dass einen Inputstring von \colorbold{Konkretter Syntax} in  \colorbold{Abstrakte Syntax} übersetzt. Im Folgenden wird allerdings die obigte Definition \nameref{par:parser} verendet.}.\footcite{noauthor_compiler_nodate}
\end{Definition}

\begin{Special_Paragraph}
  An dieser Stelle könnte möglicherweise eine Begriffsverwirrung enstehen, ob ein \colorbold{Lexer} nach der obigen Definition nicht auch ein \colorbold{Parser} ist.

  In Bezug auf Compilerbau ist ein \colorbold{Lexer} ein Teil eines Parsers. Der Parser vereinigt sowohl die \colorbold{Lexikalische Analyse}, als auch einen Teil der \colorbold{Syntaktischen Analyse} in sich. Aber für sich isoliert, ohne Bezug zu Compilerbau betrachtet, ist ein Lexer nach Definition~\ref{def:parser} ebenfalls ein Parser. Aber im Compilerbau hat \colorbold{Parser} eine spezifischere Definition und hier überwiegt beim \colorbold{Lexer} seine Funktionalität, dass er den Inputstring lexikalisch weiterverarbeitet, um ihn als Lexer zu bezeichnen, der Teil eines Parsers ist.
\end{Special_Paragraph}

Die vom \colorbold{Lexer} im Inputstring identifizierten \colorbold{Token} werden in der \colorbold{Syntaktischen Analyse} vom \colorbold{Parser} (Definition~\ref{def:parser}) als \colorbold{Wegweiser} verwendet, da je nachdem, in welcher Reihenfolge die \colorbold{Token} auftauchen, dies einer anderen Ableitung in der \colorbold{Grammatik} $G_{Parse}$ entspricht. Dabei wird in der Grammatik nach dem \colorbold{Tokennamen} unterschieden und nicht nach dem Tokenwert, da es nur von Interesse ist, ob an einer bestimmten Stelle z.B. eine \verb|Zahl| steht und nicht, welchen konkretten Wert diese \verb|Zahl| hat. Der \colorbold{Tokenwert} ist erst später in der \colorbold{Code Generierung} in~\ref{sec:code_generierung} relevant.

Ein \colorbold{Parser} ist genauergesagt ein erweiterter \colorbold{Recognizer} (Definition~\ref{def:recognizer}), denn ein Parser löst das \colorbold{Wortproblem} (Definition~\ref{def:wortproblem}) für die \colorbold{Sprache}, die durch die \colorbold{Konkrette Syntax} beschrieben wird und konstruiert parallel dazu oder im Nachgang aus den Informationen, die während der Ausführung des Recognition Algorithmus gesichert wurden den \colorbold{Derivation Tree}.

\begin{Definition}{Recognizer (bzw. Erkenner)}{recognizer}
  Entspricht dem Maschinenmodell eines \colorbold{Automaten}. Im Bezug auf Compilerbau entspricht der \colorbold{Recognizer} einem \colorbold{Kellerautomaten}, in dem \colorbold{Wörter} bestimmter \colorbold{Kontextfreier Sprachen} erkannt werden. Der \colorbold{Recognizer} erkennt, ob ein Iputstring bzw. \colorbold{Wort} sich mit den Produktionen der \colorbold{Konkrette Syntax} ableiten lässt, also ob er bzw. es Teil der Sprache ist, die von der \colorbold{Konkretten Syntax} beschrieben wird oder nicht\footnote{Das vom \colorbold{Recognizer} gelöste Problem ist auch als \colorbold{Wortproblem} bekannt.}.
\end{Definition}

\begin{Special_Paragraph}
Für das \colorbold{Parsen} gibt es grundsätzlich \colorbold{zwei} verschiedene Ansätze:

% https://tex.stackexchange.com/questions/12373/how-to-change-the-space-between-the-itemize-items-in-latex
% https://stackoverflow.com/questions/1061112/eliminate-space-before-beginitemize
\begin{itemize}[itemsep=-1mm, topsep=-1mm]
  \coloritem[Top-Down Parsing] Der \colorbold{Derivation Tree} wird von \colorbold{oben-nach-unten} generiert, also von der \colorbold{Wurzel} zu den \colorbold{Blättern}. Dementsprechend fängt die Generierung des \colorbold{Derivation Tree} mit dem \colorbold{Startsymbol} der \colorbold{Grammatik} an und wendet in jedem Schritt eine \colorbold{Linksableitung} auf die \colorbold{Nicht-Terminalsymbole} an, bis man \colorbold{Terminalsymbole} hat und der gewünschte \colorbold{Inputstring} abgeleitet wurde oder es sich herausstellt, dass dieser nicht abgeleitet werden kann.\footcite{noauthor_what_nodate-2}

  Der Grund, warum die \colorbold{Linksableitung} verwendet wird und nicht z.B. die \colorbold{Rechtsableitung} ist, weil der das \colorbold{Eingabewert} bzw. der \colorbold{Inputstring} von \colorbold{links nach rechts} eingelesen wird, was gut damit zusammenpasst, dass die \colorbold{Linksableitung} die \colorbold{Blätter} von \colorbold{links-nach-rechts} generiert.

  Welche der \colorbold{Produktionen} für ein \colorbold{Nicht-Terminalsymbol} angewandt wird, wenn es mehrere Alternativen gibt, wird entweder durch \colorbold{Backtracking} oder durch \colorbold{Vorausschauen} gelöst.

  Eine sehr einfach zu implementierende Technik für \colorbold{Top-Down Parser} ist hierbei der \colorbold{Rekursive Abstieg}. Dabei wird jedem \colorbold{Nicht-Terminalsymbol} eine \colorbold{Prozedur} zugeordnet, welche die Produktionsregeln dieses \colorbold{Nicht-Terminalsymbols} umsetzt. Prozeduren rufen sich dabei wechselseitig gegenseitig entsprechend der \colorbold{Produktionsregeln} auf, falls eine entsprechende Produktionsregel eine \colorbold{Rekursion} enthält.

  \colorbold{Rekursiver Abstieg} kann mit \colorbold{Backtracking} verbunden werden, um auch Grammatiken parsen zu können, die nicht \colorbold{LL(k)} (Definition~\ref{def:llk_grammatik}) sind. Dabei werden meist nach dem \colorbold{Depth-First-Search Prinzip} alle \colorbold{Produktionen} für ein \colorbold{Nicht-Terminalsymbol} solange durchgegangen bis der gewüschte Inpustring abgeleitet ist oder alle \colorbold{Alternativen} für einen Schritt abgesucht sind, bis man wieder beim ersten Schritt angekommen ist und da auch alle \colorbold{Alternativen} abgesucht sind. Mit dieser Methode ist das Parsen \colorbold{Linksrekursiver Grammatiken} (Definition~\ref{def:linksrekursive_grammatiken}) allerdings nicht möglich, ohne die Grammatik vorher umgeformt zu haben und jegliche \colorbold{Linksrekursion} aus der \colorbold{Grammatik} entfernt zu haben, da diese zu \colorbold{Unendlicher Rekursion} führt\footnote{Diese Art von Parser wurde im \colorbold{PicoC-Compiler} implementiert, als dieser noch auf dem Stand des \colorbold{Bachelorprojektes} war, bevor er durch den nicht selbst implementierten \colorbold{Earley Parser} von \colorbold{Lark} (siehe \cite{noauthor_lark_2022}) ersetzt wurde.}

  Wenn man eine \colorbold{LL(k)} Grammatik hat, kann man auf \colorbold{Backtracking verzichten} und es reicht einfach nur immer $k$ \colorbold{Symbole} im \colorbold{Eingabewort} bzw. in Bezug auf Compilerbau \colorbold{Token} im \colorbold{Inpustring} vorauszuschauen. \colorbold{Mehrdeutige Grammatiken} sind dadurch ausgeschlossen, weil \colorbold{LL(k)} keine \colorbold{Mehrdeutigkeit} zulässt.\footnote{Diese Art von Parser ist im \colorbold{RETI-Interpreter} implementiert, da die \colorbold{RETI-Sprache} eine besonders simple \colorbold{LL(1) Grammatik} besitzt. Dieser \colorbold{Parser} wird auch als \colorbold{Predictive Parser} oder \colorbold{LL(k) Recursive Descent Parser} bezeichnet, wobei \colorbold{Recursive Descent} das englische Wort für \colorbold{Rekursiven Abstieg} ist.}
  \coloritem[Bottom-Up Parsing] Es wird mit dem \colorbold{Eingabewort} bzw. \colorbold{Inputstring} gestartet und versucht \colorbold{Rechtsableitungen}, entsprechend der \colorbold{Produktionen} der \colorbold{Konkretten Syntax} rückwärts anzuwenden bis man beim \colorbold{Startsymbol} landet.\footcite{noauthor_what_nodate-1}
  \coloritem[Chart Parser] Es wird \colorbold{Dynamische Programmierung} verwendet und partielle Zwischenergebnisse werden in einer \colorbold{Tabelle} (bzw. einem \colorbold{Chart}) gespeichert und können wiederverwendet werden. Das macht das Parsen \colorbold{Kontextfreier Grammatiken} effizienter, sodass es nur noch \colorbold{polynomielle} Zeit braucht, da \colorbold{Backtracking} nicht mehr notwendig ist.\footnote{Der \colorbold{Earley Parser}, den \colorbold{Lark} und damit der \colorbold{PicoC-Compiler} verwendet fällt unter diese Kategorie}
\end{itemize}
  % ein hübsches Bildchen, wo man einen Überblick bekommt
\end{Special_Paragraph}

% Problem mit Linksrekursion und Mehrdeutigkeit ansprechen

Der \colorbold{Abstrakt Syntax Tree} wird mithilfe von \colorbold{Transformern} (Definition~\ref{def:transformer}) und \colorbold{Visitors} (Definition~\ref{def:visitor}) generiert und ist das Endprodukt der \colorbold{Syntaktischen Analyse}. Wenn man die gesamte \colorbold{Syntaktische Analyse} betrachtet, so übersetzt diese einen Inpustring von der \colorbold{Konkretten Syntax} in die \colorbold{Abstrakte Syntax} (Definition~\ref{def:abstrakte_syntax}).

Die \colorbold{Baumdatenstruktur} des \colorbold{Derivation Tree} und  \colorbold{Abstrakt Syntax Tree} ermöglicht es die Operationen, die der Compiler bei der Weiterverarbeitung des Inputstrings ausführen muss möglichst \colorbold{effizient} auszuführen.

\begin{Definition}{Transformer}{transformer}
Ein Programm, dass von \colorbold{unten-nach-oben}, nach dem \colorbold{Breadth First Search} Prinzip alle Knoten des \colorbold{Derivation Tree} besucht und in Bezug zu Compilerbau, beim Antreffen eines bestimmten Knoten des \colorbold{Derivation Tree} einen entsprechenden Knoten des \colorbold{Abstrakt Syntax Tree} erzeugt und diesen anstelle des Knotens des \colorbold{Derivation Tree} setzt und so Stück für Stück den \colorbold{Abstrak Syntax Tree} konstruiert.
\end{Definition}

\begin{Definition}{Visitor}{visitor}
Ein Programm, dass von \colorbold{unten-nach-oben}, nach dem \colorbold{Breadth First Search} Prinzip alle Knoten des \colorbold{Derivation Tree} besucht und in Bezug zu Compilerbau, beim Antreffen eines bestimmten \colorbold{Knoten} des Derivation Tree, diesen \colorbold{in-place} mit anderen Knoten \colorbold{tauscht} oder \colorbold{manipuliert}, um den Derivation Tree für die weitere Verarbeitung durch z.B. einen Transformer zu vereinfachen.

Kann theoretisch auch zur Konstruktion eines \colorbold{Abstrakt Syntax Tree} verwendet werden, wenn z.B. eine externe Klasse verwendet wird, welches für die Konstruktion des \colorbold{Abstrakt Syntax Tree} verantwortlich ist, aber dafür ist ein \colorbold{Transformer} besser geeignet.
\end{Definition}

\begin{Definition}{Abstrakte Syntax}{abstrakte_syntax}
  \colorbold{Syntax}, die beschreibt, was für Arten von \colorbold{Komposition} bei den \colorbold{Knoten} eines \colorbold{Abstrakt Syntax Trees} möglich sind.\footcite{noauthor_course_2022}

  Jene Produktionen, die in der \colorbold{Konkretten Syntax} für die Umsetzung von \colorbold{Präzidenz} notwendig waren, sind in der \colorbold{Abstrakten Syntax} abgeflacht.
\end{Definition}

\begin{Definition}{Abstrakt Syntax Tree}{abstrakte_syntax_tree}
  \colorbold{Compilerinterne Darstellung} eines Programs, in welcher sich anhand der Knoten auf dem Pfad von der Wurzel zu einem \colorbold{Blatt} nicht mehr direkt nachvollziehen lässt, durch welche \colorbold{Produktionen} dieses Blatt abgeleitet wurde.

Der \colorbold{Abstrakt Syntax Tree} hat einmal den Zweck, dass die Kompositionen, die die Konten bilden können \colorbold{semantisch} näher an den \colorbold{Instructions eines Assemblers} dran sind und, dass man mit ihm bei der Betrachtung eines \colorbold{Knoten}, der für einen Teil des Programms steht, möglichst schnell die Frage beantworten kann, welche \colorbold{Funktionalität} der Sprache dieser umsetzt, welche \colorbold{Bestandteile} er hat und welche Funktionalität der Sprache diese Bestandteile umsetzen usw.\footcite{noauthor_course_2022}
\end{Definition}

Je weiter \colorbold{unten}\footnote{In der Informatik wachsen \colorbold{Bäume} von \colorbold{oben-nach-unten}. Die \colorbold{Wurzel} ist also \colorbold{oben}.} und \colorbold{links} ein Knoten im \colorbold{Abstrakt Syntax Tree} liegt, desto eher wird dieser Knoten komplett abgearbeitet sein, da in der \colorbold{Code Generierung} die Knoten nach dem \colorbold{Depth First Search} Prinzip abgearbeitet werden.

Um eine Gesamtübersicht über die \colorbold{Syntaktische Analyse} zu geben, ist in Abbildung~\ref{fig:syntaktische_analyse_veranschaulichung} die Syntaktische mit dem Beispiel aus Subkapitel~\ref{sec:lexikalische_analyse} fortgeführt.

\pagebreak
% https://tex.stackexchange.com/questions/8625/force-figure-placement-in-text
\begin{figure}[H]
  \centering
  \treebox[title=Tokenfolge, remember as=tokenfolge, width=0.6\linewidth, nobeforeafter]{./code_examples/example1.tokens}
  \hfill
  \treebox[title=Abstract Syntax Tree, remember as=abstract_sytax_tree, width=0.3\linewidth, nobeforeafter]{./code_examples/example1.ast}
  \vspace{2cm}
  \treebox[title=Derivation Tree, remember as=derivation_tree, width=0.6\linewidth]{./code_examples/example1.dt}

  \begin{tikzpicture}[overlay,remember picture,line width=1mm,draw=PrimaryColor]
    \draw[->] (tokenfolge.south) to node[left] {Parser} (derivation_tree.north);
    \draw[->] (derivation_tree.north) to node[right] {Visitor und Transformer} (abstract_sytax_tree.south);
  \end{tikzpicture}
  \caption{Veranschaulichung der Syntaktischen Analyse}
  \label{fig:syntaktische_analyse_veranschaulichung}
\end{figure}

\section{Code Generierung}
\label{sec:code_generierung}
\begin{Definition}{Pass}{pass}
% TODO: Bild semantisch gleiche Bedeutung
% TODO: auf T-Diagramme zurückkommen
\end{Definition}
\section{Fehlermeldungen}
\begin{Definition}{Fehlermeldung}{fehlermeldung}
\end{Definition}

% Kategorien von Fehlermeldungen
