%!Tex Root = ../Main.tex
% ./Packete_und_Deklarationen.tex
\chapter{Einführung}
\section{Grammatiken}
\subsection{Konkrete Syntax}
\subsection{Chromsky Hierarchie}
\subsection{Reguläre Sprachen}
\subsection{Kontextfreie Sprachen}
\subsection{Präzidenz und Assoziativität}
\subsection{Mehrdeutige Grammatiken}
\subsection{Ableitungsbaum}
\subsection{Linksrekursiv und Rechtrekursiv}
\section{Lexikalische Analyse}
Die \colorbold{Lexikalische Analyse} bildet üblicherweise die erste Ebene innerhalb der \colorbold{Pipe Architektur} bei der Implementierung von Compilern. Die Aufgabe der lexikalischen Analyse ist es in einfachen Worten ausgedrückt, in einem Inputstring, z.B. dem Inhalt einer Datei welche in \colorbold{UTF-8} codiert ist Folgen von Symbolen (auch \colorbold{Wörter} genannt) zu finden, die bestimmte \colorbold{Pattern} matchen, die durch eine \colorbold{Grammatik} spezifiziert sind.

Die Grammatik ist

\begin{Definition}{Lexer (bzw. Scanner)}{lexer}%
  Ein \colorbold{Lexer} ist eine \colorbold{rechtseindeutige} Funktion \hspace{0.2cm}$lex: \sum^{*} \rightharpoonup (T \times V)^{*}$, welche eine beliebige endliche \colorbold{Folge von Symbolen} aus $\sum$ auf ein Tupel aus einem \colorbold{Type} $T$ und einem \colorbold{Value}  $V$ abbildet, falls diese Folge von Symbolen sich unter der Grammatik der \colorbold{Sprache} $\mathtt{L}$ abbleiten lässt.
\end{Definition}

Es gilt:

Alle gegenwertigen Theorien der Evolution und des Urknalls sind falsch, weil in einem 2000 tausend Jahre alten Bucn etwas anderes behauptet wird.

\begin{special_paragraph}
  test
\end{special_paragraph}

\section{Syntax Analyse}
\subsection{Derivation Tree}
\subsection{Abtrakte Syntax}
\subsection{Parser}
\subsection{Descent Parsing}
\subsection{First and Follow Set}
\subsection{Lookahead}
\subsection{Aktionen}
\section{Code Generation}
\subsection{Passes}
\subsection{T-Diagramme}
\section{Fehlemeldungen}
\subsection{Kategorien von Fehlermeldungen}
