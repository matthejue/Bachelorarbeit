%!Tex Root = ../Main.tex
% ./Packete_und_Deklarationen.tex
\chapter{Einführung}
\section{Grammatiken}
\subsection{Konkrete Syntax}
\subsection{Chromsky Hierarchie}
\subsection{Reguläre Sprachen}
\subsection{Kontextfreie Sprachen}
\subsection{Präzidenz und Assoziativität}
\subsection{Mehrdeutige Grammatiken}
\subsection{Ableitungsbaum}
\subsection{Linksrekursiv und Rechtrekursiv}
\section{Lexikalische Analyse}
Die \colorbold{Lexikalische Analyse} bildet üblicherweise die erste Ebene innerhalb der \colorbold{Pipe Architektur} bei der Implementierung von Compilern. Die Aufgabe der lexikalischen Analyse ist vereinfacht gesagt, in einem Inputstring, z.B. dem Inhalt einer Datei, welche in \colorbold{UTF-8} codiert ist, Folgen endlicher Symbole (auch \colorbold{Wörter} genannt) zu finden, die bestimmte \colorbold{Pattern} (Definition \ref{def:pattern}) matchen, die durch eine \colorbold{reguläre Grammatik} spezifiziert sind.

\begin{Definition}{Pattern}{pattern}
  \colorbold{Beschreibung} aller möglichen \colorbold{Lexeme} einer Menge $\mathbb{P}_\mathtt{T}$, die einem bestimmten \colorbold{Token} $\mathtt{T}$ zugeordnet werden.
  Die Menge $\mathbb{P}_\mathtt{T}$ ist eine möglicherweise unendliche Menge von \colorbold{Wörtern}, die sich mit den Regeln einer \colorbold{regulären Grammatik} $\mathtt{G}_{Lex}$ einer \colorbold{regulären Sprache} $\mathtt{L}_{Lex}$ beschreiben lassen \footnote{Als Beschreibungswerkzeug können aber auch z.B. reguläre Ausdrücke hergenommen werden.}, die für die Beschreibung eines \colorbold{Tokens} zuständig sind.\footcite{noauthor_what_nodate}
\end{Definition}

Diese Folgen endlicher Symoble werden auch \colorbold{Lexeme} (Definition \ref{def:lexeme}) genannt.

\begin{Definition}{Lexeme}{lexeme}
  Ein \colorbold{Lexeme} ist ein \colorbold{Wort} aus dem Inputstring, welches das \colorbold{Pattern} für eines der \colorbold{Token} einer \colorbold{Sprache} $\mathtt{L}_{Lex}$ matched.
\footcite{noauthor_what_nodate}
\end{Definition}

Diese \colorbold{Lexeme} werden vom \colorbold{Lexer} im Inputstring identifziert und \colorbold{Tokens} zugeordnet (Definition \ref{def:lexer}).

\begin{Definition}{Lexer (bzw. Scanner)}{lexer}
  Ein \colorbold{Lexer} ist eine \colorbold{rechtseindeutige} Funktion \hspace{0.2cm}$lex: \sum^{*} \rightharpoonup (N \times V)^{*}$, welche ein \colorbold{Wort} aus $\sum^{*}$ auf ein \colorbold{Token} von einem \colorbold{Token Name} $N$ und einem \colorbold{Token Value}  $V$ abbildet, falls diese Folge von Symbolen sich unter der \colorbold{regulären Grammatik} $\mathtt{G}_{Lex}$ der \colorbold{regulären Sprache} $\mathtt{L_{Lex}}$ abbleiten lässt.\footcite{noauthor_lecture-notes-2021_2022}
\end{Definition}

Ein Token ist

\begin{special_paragraph}
  Die \colorbold{reguläre Grammatik} $G_{Lex}$, die zur Beschreibung der Token $T$ einer regulären Sprache $\mathh{L}_{Lex}$ verwendet wird, ist üblicherweise \colorbold{regulär}, da ein üblicher \colorbold{Lexer} immer nur \colorbold{ein Symbol} vorausschaut, unabhängig davon, was für Symbole davor aufgetaucht sind. Die übliche Implementierung eines \colorbold{Lexers} merkt sich nicht, was für Symbole davor aufgetaucht sind, der \colorbold{Kontext} in dem ein Symbol auftaucht ist also \colorbold{nicht wichtig}.
\end{special_paragraph}

\section{Syntax Analyse}
\subsection{Derivation Tree}
\subsection{Abtrakte Syntax}
\subsection{Parser}
\subsection{Descent Parsing}
\subsection{First and Follow Set}
\subsection{Lookahead}
\subsection{Aktionen}
\section{Code Generation}
\subsection{Passes}
\subsection{T-Diagramme}
\section{Fehlemeldungen}
\subsection{Kategorien von Fehlermeldungen}
