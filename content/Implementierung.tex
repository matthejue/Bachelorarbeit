% ../Main.tex
\chapter{Implementierung}
\section{Grammatiken}
\subsection{PicoC}
Die PicoC Subsprache von C, verfolgt dieselben Präzidenzregeln, wie die Sprache C \footcite{noauthor_c_nodate}, welche in Tabelle~\ref{tab:reference_table} dargestellt ist.

\begin{tabulary}{\linewidth}{LLLL}
  \toprule
  Precedence &	Operator & Description &	Associativity \\
  \midrule
  1	& \verb|a()|	& Function call & Left-to-right $\rightarrow$ \\
    & \verb|a[]|	& Subscript & \\
    & \verb|.| & 	Member access & \\
  \addlinespace
  2	&	\verb|-a| & Unary minus & Right-to-left $\leftarrow$ \\
    & \verb|! ~|	& Logical NOT and bitwise NOT & \\
    & \verb|*a| & Indirection (dereference) & \\
    & \verb|&| & Address-of & \\
  \addlinespace
  3	& \verb|a*b a/b a\%b| &	Multiplication, division, and remainder & Left-to-right $\rightarrow$ \\
  \addlinespace
  6	& \verb|a+b a-b|	& Addition and subtraction & \\
  \addlinespace
  7	& \verb|<< >>|	& Bitwise left shift and right shift & \\
  \addlinespace
  9	& \verb|< <= > >=|	& For relational operators < and ≤ and > and ≥ respectively & \\
  \addlinespace
  10 &	\verb|==|   !=	& For equality operators = and ≠ respectively & \\
  \addlinespace
  11 &	\verb|a&b|	Bitwise AND & \\
  \addlinespace
  12 &	\verb|^	&| Bitwise XOR (exclusive or) & \\
  \addlinespace
  13 &	\midbar \verb|&| Bitwise OR (inclusive or) & \\
  \addlinespace
  14	& \verb|&&| &	Logical AND & \\
  \addlinespace
  15	& \mid\mid	& Logical OR & \\
  \addlinespace
  16 & \verb|=| & Direct Assignment & Right-to-left $\leftarrow$ \\
  \addlinespace
  17 &	\verb|,|	& Comma	& Left-to-right $\rightarrow$ \\
  \bottomrule

  \label{tab:reference_table}
  \caption{Präzidenzregeln von PicoC}
\end{tabulary}

\subsection{RETI}
\subsection{Mehrdeutigkeit}
\subsection{Präzidenz und Assoziativität}
\subsection{Linksrekursivität}
\section{Lexikalische Analyse}
\subsection{Lark}
\subsection{LL(1) Recursive-Descent Lexer}
\section{Syntax Analyse}
\subsection{Lark}
\subsection{Normalized Heterogeneous ASTNode}
\subsection{Early Algorithmus}
Das LL(k) Recursive-Descent Parser \footcite{parr_language_2009} Pattern ist
\subsection{Visitor und Transformer}
\section{Code Generation}
\subsection{Symbol Table for Nested Scopes}
\subsection{PicoC-Shrink Pass}
\subsection{PicoC-Blocks Pass}
\subsection{PicoC-Mon Pass}
\subsection{RETI-Blocks Pass}
\subsection{RETI-Patch Pass}
\subsection{RETI Pass}
\section{Fehlermeldungen}
\subsection{Error Handler}
