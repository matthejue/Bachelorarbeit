% ../Main.tex
\chapter{Implementierung}
\section{Grammatiken}
\subsection{PicoC}
Die PicoC Sprache hat dieselben Präzidenzregeln implementiert, wie die Sprache C \footcite{noauthor_c_nodate}. Die Präzidenzregeln von PicoC sind in Tabelle~\ref{tab:reference_table} aufgelistet.

\begin{table}[h]
  \center
  \begin{tabulary}{\linewidth}{|C|C|L|L|}
  \toprule
  Präzidenz &	Operator & Beschreibung &	Assoziativität \\
  \midrule
  1	& \verb|a()|	& Funktionsaufruf & \multirow{2}{=}{Links, dann rechts $\rightarrow$} \\
    & \verb|a[]|	& Indexzugriff & \\
    & \verb|a.b| & Attributzugriff & \\
  \hline
  2	&	\verb|-a| & Unäres Minus & \multirow{2}{=}{Rechts, dann links $\leftarrow$} \\
    & \smalltt{!a $\thicksim$a}	& Logisches NOT und Bitweise NOT & \\
    & \verb|*a| & Dereferenz & \\
    & \verb|&a| & Referenz, Adresse-von & \\
  \hline
  3	& \smalltt{a*b a/b a\%b} &	Multiplikation, Division und Modulo & Links, dann rechts $\rightarrow$ \\
  \cline{1-3}
  4	& \verb|a+b a-b|	& Addition und Subtraktion & \\
  \cline{1-3}
  5	& \verb|a<b a<=b| \verb|a>b a>=b| & Kleiner, Kleiner Gleich, Größer, Größer gleich & \\
  \cline{1-3}
  6 &	\verb|a==b a!=b|	& Gleichheit und Ungleichheit & \\
  \cline{1-3}
  7 &	\verb|a&b| & Bitweise UND & \\
  \cline{1-3}
  8 &	\verb|a^b| & Bitweise XOR (exclusive or) & \\
  \cline{1-3}
  9 & \smalltt{a$\mid$b} & Bitweise ODER (inclusive or) & \\
  \cline{1-3}
  10	& \verb|a&&b| &	Logiches UND & \\
  \cline{1-3}
  11	& $a{\mid\mid} b$	& Logisches ODER & \\
  \hline
  12 & \verb|a=b| & Zuweisung & Rechts, dann links $\leftarrow$ \\
  \hline
  13 &	\verb|a,b|& Komma	& Links, dann rechts $\rightarrow$ \\
  \bottomrule
\end{tabulary}
\label{tab:reference_table}
\caption{Präzidenzregeln von PicoC}
\end{table}

\subsection{RETI}
\subsection{Mehrdeutigkeit}
\subsection{Präzidenz und Assoziativität}
\subsection{Linksrekursivität}
\section{Lexikalische Analyse}
\subsection{Lark}
\subsection{LL(1) Recursive-Descent Lexer}
\section{Syntax Analyse}
\subsection{Lark}
\subsection{Normalized Heterogeneous ASTNode}
\subsection{Early Algorithmus}
Das LL(k) Recursive-Descent Parser \footcite{parr_language_2009} Pattern ist
\subsection{Visitor und Transformer}
\section{Code Generation}
\subsection{Symbol Table for Nested Scopes}
\subsection{PicoC-Shrink Pass}
\subsection{PicoC-Blocks Pass}
\subsection{PicoC-Mon Pass}
\subsection{RETI-Blocks Pass}
\subsection{RETI-Patch Pass}
\subsection{RETI Pass}
\section{Fehlermeldungen}
\subsection{Error Handler}
