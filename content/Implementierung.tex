% ../Main.tex
\chapter{Implementierung}
\section{Grammatiken}
\subsection{PicoC}
Die PicoC Subsprache von C, verfolgt dieselben Präzidenzregeln, wie die Sprache C \footcite{noauthor_c_nodate}, welche in Tabelle~\ref{tab:reference_table} dargestellt sind.

\begin{table}[h]
  \center
  \begin{tabulary}{\linewidth}{|C|C|L|L|}
  \toprule
  Präzidenz &	Operator & Beschreibung &	Assoziativität \\
  \midrule
  1	& \verb|a()|	& Function call & \multirow{2}{=}{Left-to-right $\rightarrow$} \\
    & \verb|a[]|	& Subscript & \\
    & \verb|.| & 	Member access & \\
  \hline
  2	&	\verb|-a| & Unary minus & Right-to-left $\leftarrow$ \\
    & \verb|! ~|	& Logical NOT and bitwise NOT & \\
    & \verb|*a| & Indirection (dereference) & \\
    & \verb|&| & Address-of & \\
  \hline
  3	& \smalltt{a*b a/b a\%b} &	Multiplication, division, and remainder & Left-to-right $\rightarrow$ \\
  \cline{1-3}
  4	& \verb|a+b a-b|	& Addition and subtraction & \\
  \cline{1-3}
  5	& \verb|<< >>|	& Bitwise left shift and right shift & \\
  \cline{1-3}
  6	& \verb|< <= > >=|	& For relational operators $<$ and $\leq$ and $>$ and $\geq$ respectively & \\
  \cline{1-3}
  7 &	\verb|== !=|	& For equality operators $=$ and $\neq$ respectively & \\
  \cline{1-3}
  8 &	\verb|a&b| & Bitwise AND & \\
  \cline{1-3}
  9 &	\verb|^	&| & Bitwise XOR (exclusive or) & \\
  \cline{1-3}
  10 & \smalltt{a$\mid$b} & Bitwise OR (inclusive or) & \\
  \cline{1-3}
  11	& \verb|&&| &	Logical AND & \\
  \cline{1-3}
  12	& $\mid\mid$	& Logical OR & \\
  \hline
  13 & \verb|=| & Direct Assignment & Right-to-left $\leftarrow$ \\
  \hline
  14 &	\verb|,|& Comma	& Left-to-right $\rightarrow$ \\
  \bottomrule
\end{tabulary}
\label{tab:reference_table}
\caption{Präzidenzregeln von PicoC}
\end{table}

\subsection{RETI}
\subsection{Mehrdeutigkeit}
\subsection{Präzidenz und Assoziativität}
\subsection{Linksrekursivität}
\section{Lexikalische Analyse}
\subsection{Lark}
\subsection{LL(1) Recursive-Descent Lexer}
\section{Syntax Analyse}
\subsection{Lark}
\subsection{Normalized Heterogeneous ASTNode}
\subsection{Early Algorithmus}
Das LL(k) Recursive-Descent Parser \footcite{parr_language_2009} Pattern ist
\subsection{Visitor und Transformer}
\section{Code Generation}
\subsection{Symbol Table for Nested Scopes}
\subsection{PicoC-Shrink Pass}
\subsection{PicoC-Blocks Pass}
\subsection{PicoC-Mon Pass}
\subsection{RETI-Blocks Pass}
\subsection{RETI-Patch Pass}
\subsection{RETI Pass}
\section{Fehlermeldungen}
\subsection{Error Handler}
