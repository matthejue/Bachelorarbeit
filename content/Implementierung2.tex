%!Tex Root = ../Main.tex
% ./Packete_und_Deklarationen.tex
% ./Titlepage.tex
% ./Motivation.tex
% ./Einführung.tex
% ./Implementierung1.tex
% ./Ergebnisse_und_Ausblick.tex

\subsection{Umsetzung von Pointern}
\subsubsection{Referenzierung}
Die \colorbold{Referenzierung} \verb|&var| wird im Folgenden anhand des Beispiels in Code~\ref{code:picoc_code_für_pointer_referenzierung} erklärt.

\begin{code}
  \centering
  \numberedcodebox[minted language=c, minted options={highlightlines={3}}]{./code_examples/example_pntr_ref.picoc}
  \caption{PicoC Code für Pointer Referenzierung}
  \label{code:picoc_code_für_pointer_referenzierung}
\end{code}

Der Knoten \smalltt{Ref(Name('var')))} repräsentiert in der \colorbold{Abstrakten Syntax} in Code~\ref{code:abstract_syntax_tree_für_pointer_referenzierung} eine \colorbold{Referenzierung} \verb|&var|.

\begin{code}
  \centering
  \numberedcodebox[minted language=text, minted options={highlightlines={10}}]{./code_examples/example_pntr_ref.ast}
  \caption{Abstract Syntax Tree für Pointer Referenzierung}
  \label{code:abstract_syntax_tree_für_pointer_referenzierung}
\end{code}

Im \colorbold{PicoC-Mon Pass} in Code~\ref{code:picoc_mon_für_pointer_referenzierung} wird der Knoten \smalltt{Ref(Name('var')))} durch die Knoten \smalltt{Ref(GlobalRead(Num('0')))} und \smalltt{Assign(GlobalWrite(Num('1')), Tmp(Num('1')))} ersetzt. Im Fall, dass in \smalltt{Ref(exp))} das \smalltt{exp} vielleicht nicht direkt ein \smalltt{Name('var')} enthält und \smalltt{exp} z.B. ein \smalltt{Subscr(Attr(Name('var')))} ist, sind noch weitere Anweisungen zwischen den Zeilen \smalltt{11} und  \smalltt{12} nötig, die sich in diesem Beispiel um das Übersetzen von \smalltt{Subscr(exp)} und \smalltt{Attr(exp)} nach dem Schema in Subkapitel~\ref{mittelteil_für_die_verschiedenen_derived_datatypes} kümmern.

\begin{code}
  \centering
  \numberedcodebox[minted language=text, minted options={highlightlines={11-13}}]{./code_examples/example_pntr_ref.picoc_mon}
  \caption{PicoC Mon Pass für Pointer Referenzierung}
  \label{code:picoc_mon_für_pointer_referenzierung}
\end{code}

Im \colorbold{PicoC-Blocks Pass} in Code~\ref{code:reti_blocks_für_pointer_referenzierung} werden die \colorbold{PicoC-Knoten} \smalltt{ Ref(Global(Num('0')))} und \smalltt{Assign(Global(Num('1')), Stack(Num('1')))} durch ihre entsprechenden \colorbold{RETI-Knoten} ersetzt.

\begin{code}
  \centering
  \numberedcodebox[minted language=text, minted options={highlightlines={18-21,23-25}}]{./code_examples/example_pntr_ref.reti_blocks}
  \caption{RETI Blocks Pass für Pointer Referenzierung}
  \label{code:reti_blocks_für_pointer_referenzierung}
\end{code}
% Initialisierung eines Pointers
\subsubsection{Dereferenzierung durch Zugriff auf Arrayindex ersetzen}
Die \colorbold{Dereferenzierung} \smalltt{*var} wird im Folgenden anhand des Beispiels in Code~\ref{code:picoc_code_für_pointer_dereferenzierung} erklärt.

\begin{code}
  \centering
  \numberedcodebox[minted language=c, minted options={highlightlines={4}}]{./code_examples/example_pntr_deref.picoc}
  \caption{PicoC Code für Pointer Dereferenzierung}
  \label{code:picoc_code_für_pointer_dereferenzierung}
\end{code}

Der Knoten \smalltt{Deref(Name('var')))} repräsentiert in der \colorbold{Abstrakten Syntax} in Code~\ref{code:abstract_syntax_tree_für_pointer_dereferenzierung} eine \colorbold{Dereferenzierung} \smalltt{*var}.

\begin{code}
  \centering
  \numberedcodebox[minted language=text, minted options={highlightlines={11}}]{./code_examples/example_pntr_deref.ast}
  \caption{Abstract Syntax Tree für Pointer Dereferenzierung}
  \label{code:abstract_syntax_tree_für_pointer_dereferenzierung}
\end{code}

Im \colorbold{PicoC-Shrink Pass} in Code~\ref{code:picoc_shrink_für_pointer_dereferenzierung} wird ein Trikc angewandet, bei dem der Knoten \smalltt{Exp(Deref(Name('pntr'), Num('0')))} einfach durch den Knoten \smalltt{Exp(Subscr(Name('pntr'), Num('0')))} ersetzt wird. Der Trick besteht

\begin{code}
  \centering
  \numberedcodebox[minted language=text, minted options={highlightlines={11}}]{./code_examples/example_pntr_deref.picoc_shrink}
  \caption{PicoC Shrink Pass für Pointer Dereferenzierung}
  \label{code:picoc_shrink_für_pointer_dereferenzierung}
\end{code}

\subsection{Umsetzung von Arrays}
\subsubsection{Initialisierung von Arrays}
% Stack und Globale Statische Daten
\begin{code}
  \centering
  \numberedcodebox[minted language=c]{./code_examples/example_array_init.picoc}
  \caption{PicoC Code für Array Initialisierung}
  \label{code:picoc_code_für_array_initialisierung}
\end{code}

\begin{code}
  \centering
  \numberedcodebox[minted language=text]{./code_examples/example_array_init.ast}
  \caption{Abstract Syntax Tree für Array Initialisierung}
  \label{code:abstract_syntax_tree_für_array_initialisierung}
\end{code}

\begin{code}
  \centering
  \numberedcodebox[minted language=text]{./code_examples/example_array_init.st}
  \caption{Symboltabelle für Array Initialisierung}
  \label{code:symboltabelle_für_array_initialisierung}
\end{code}

\begin{code}
  \centering
  \numberedcodebox[minted language=text]{./code_examples/example_array_init.picoc_mon}
  \caption{PicoC Mon Pass für Array Initialisierung}
  \label{code:picoc_mon_für_array_initialisierung}
\end{code}

\begin{code}
  \centering
  \numberedcodebox[minted language=text]{./code_examples/example_array_init.reti_blocks}
  \caption{RETI Blocks Pass für Array Initialisierung}
  \label{code:reti_blocks_für_array_initialisierung}
\end{code}


% kleines Extra
\subsubsection{Zugriff auf Arrayindex}

Der Zugriff auf einen bestimmten  Index eines Arrays ist wie folgt umgesetzt:

\begin{code}
  \centering
  \numberedcodebox[minted language=c]{./code_examples/example_array_access.picoc}
  \caption{PicoC Code für Zugriff auf Arrayindex}
  \label{code:picoc_code_für_zugriff_auf_arrayindex}
\end{code}

\begin{code}
  \centering
  \numberedcodebox[minted language=text]{./code_examples/example_array_access.ast}
  \caption{Abstract Syntax Tree für Zugriff auf Arrayindex}
  \label{code:abstract_syntax_tree_für_zugriff_auf_arrayindex}
\end{code}

\begin{code}
  \centering
  \numberedcodebox[minted language=text]{./code_examples/example_array_access.picoc_mon}
  \caption{PicoC Mon Pass für Zugriff auf Arrayindex}
  \label{code:picoc_mon_für_zugriff_auf_arrayindex}
\end{code}

\begin{code}
  \centering
  \numberedcodebox[minted language=text]{./code_examples/example_array_access.reti_blocks}
  \caption{RETI Blocks Pass für Zugriff auf Arrayindex}
  \label{code:reti_blocks_für_zugriff_auf_arrayindex}
\end{code}

\subsubsection{Zuweisung an Arrayindex}
% Formel aus der Vorlesung, wo ist die hier?
\begin{code}
  \centering
  \numberedcodebox[minted language=c]{./code_examples/example_array_assignment.picoc}
  \caption{PicoC Code für Zuweisung an Arrayindex}
  \label{code:picoc_code_für_array_assignment}
\end{code}

\begin{code}
  \centering
  \numberedcodebox[minted language=text]{./code_examples/example_array_assignment.ast}
  \caption{Abstract Syntax Tree für Zuweisung an Arrayindex}
  \label{code:abstract_syntax_tree_für_array_assignment}
\end{code}

\begin{code}
  \centering
  \numberedcodebox[minted language=text]{./code_examples/example_array_assignment.picoc_mon}
  \caption{PicoC Mon Pass für Zuweisung an Arrayindex}
  \label{code:picoc_mon_für_array_assignment}
\end{code}

\begin{code}
  \centering
  \numberedcodebox[minted language=text]{./code_examples/example_array_assignment.reti_blocks}
  \caption{RETI Blocks Pass für Zuweisung an Arrayindex}
  \label{code:reti_blocks_für_array_assignment}
\end{code}

\subsection{Umsetzung von Structs}
\subsubsection{Deklaration von Structs}
\begin{code}
  \centering
  \numberedcodebox[minted language=c]{./code_examples/example_struct_decl.picoc}
  \caption{PicoC Code für Deklaration von Structs}
  \label{code:picoc_code_für_deklaration_von_structs}
\end{code}

\begin{code}
  \centering
  \numberedcodebox[minted language=text]{./code_examples/example_struct_decl.st}
  \caption{Symboltabelle für Deklaration von Structs}
  \label{code:symboltabelle_für_deklaration_von_structs}
\end{code}

\subsubsection{Initialisierung von Structs}
\begin{code}
  \centering
  \numberedcodebox[minted language=c]{./code_examples/example_struct_init.picoc}
  \caption{PicoC Code für Initialisierung von Structs}
  \label{code:picoc_code_für_initialisierung_von_structs}
\end{code}

\begin{code}
  \centering
  \numberedcodebox[minted language=text]{./code_examples/example_struct_init.ast}
  \caption{Abstract Syntax Tree für Initialisierung von Structs}
  \label{code:abstract_syntax_tree_für_initialisierung_von_structs}
\end{code}

\begin{code}
  \centering
  \numberedcodebox[minted language=text]{./code_examples/example_struct_init.st}
  \caption{Symboltabelle für Initialisierung von Structs}
  \label{code:symboltabelle_für_initialisierung_von_structs}
\end{code}

\begin{code}
  \centering
  \numberedcodebox[minted language=text]{./code_examples/example_struct_init.picoc_mon}
  \caption{PicoC Mon Pass für Initialisierung von Structs}
  \label{code:picoc_mon_pass_für_initialisierung_von_structs}
\end{code}

\begin{code}
  \centering
  \numberedcodebox[minted language=text]{./code_examples/example_struct_init.reti_blocks}
  \caption{RETI Blocks Pass für Initialisierung von Structs}
  \label{code:reti_blocks_pass_für_initialisierung_von_structs}
\end{code}

% Stack und Globale Statische Daten
\subsubsection{Zugriff auf Structattribut}
% Formel aus der Vorlesung, wo ist die hier?
\begin{code}
  \centering
  \numberedcodebox[minted language=c]{./code_examples/example_struct_attr_access.picoc}
  \caption{PicoC Code für Zugriff auf Structattribut}
  \label{code:picoc_code_für_zugriff_auf_structattribut}
\end{code}

\begin{code}
  \centering
  \numberedcodebox[minted language=text]{./code_examples/example_struct_attr_access.ast}
  \caption{Abstract Syntax Tree für Zugriff auf Structattribut}
  \label{code:abstract_syntax_tree_für_zugriff_auf_structattribut}
\end{code}

\begin{code}
  \centering
  \numberedcodebox[minted language=text]{./code_examples/example_struct_attr_access.picoc_mon}
  \caption{PicoC Mon Pass für Zugriff auf Structattribut}
  \label{code:picoc_mon_pass_für_zugriff_auf_structattribut}
\end{code}

\begin{code}
  \centering
  \numberedcodebox[minted language=text]{./code_examples/example_struct_attr_access.reti_blocks}
  \caption{RETI Blocks Pass für Zugriff auf Structattribut}
  \label{code:reti_blocks_pass_für_zugriff_auf_structattribut}
\end{code}

\subsubsection{Zuweisung an Structattribut}
\begin{code}
  \centering
  \numberedcodebox[minted language=c]{./code_examples/example_struct_attr_assignment.picoc}
  \caption{PicoC Code für Zuweisung an Structattribut}
  \label{code:picoc_code_für_zuweisung_an_structattribut}
\end{code}

\begin{code}
  \centering
  \numberedcodebox[minted language=text]{./code_examples/example_struct_attr_assignment.ast}
  \caption{Abstract Syntax Tree für Zuweisung an Structattribut}
  \label{code:abstract_syntax_tree_für_zuweisung_an_structattribut}
\end{code}

\begin{code}
  \centering
  \numberedcodebox[minted language=text]{./code_examples/example_struct_attr_assignment.picoc_mon}
  \caption{PicoC Mon Pass für Zuweisung an Structattribut}
  \label{code:picoc_mon_pass_für_zuweisung_an_structattribut}
\end{code}

\begin{code}
  \centering
  \numberedcodebox[minted language=text]{./code_examples/example_struct_attr_assignment.reti_blocks}
  \caption{RETI Blocks Pass für Zuweisung an Structattribut}
  \label{code:reti_blocks_pass_für_zuweisung_an_structattribut}
\end{code}

\subsection{Umsetzung der Derived Datatypes im Zusammenspiel}
\subsubsection{Einleitungsteil für Globale Statische Daten und Stackframe}
% Stack und Globale Statische Daten, unterschieldihe Berechnung der Adressen
% unterschiedliche Adressberechnung
\begin{code}
  \centering
  \numberedcodebox[minted language=c]{./code_examples/example_derived_dts_introduction_part.picoc}
  \caption{PicoC Code für den Einleitungsteil}
  \label{code:picoc_code_einleitungsteil}
\end{code}

% spezielles Vorgehen bei PntrDecl

\begin{code}
  \centering
  \numberedcodebox[minted language=text]{./code_examples/example_derived_dts_introduction_part.ast}
  \caption{Abstract Syntax Tree für den Einleitungsteil}
  \label{code:abstract_syntax_tree_einleitungsteil}
\end{code}

\begin{code}
  \centering
  \numberedcodebox[minted language=text]{./code_examples/example_derived_dts_introduction_part.picoc_mon}
  \caption{PicoC Mon Pass für den Einleitungsteil}
  \label{code:picoc_mon_pass_einleitungsteil}
\end{code}

\begin{code}
  \centering
  \numberedcodebox[minted language=text]{./code_examples/example_derived_dts_introduction_part.reti_blocks}
  \caption{RETI Blocks Pass für den Einleitungsteil}
  \label{code:reti_blocks_pass_einleitungsteil}
\end{code}

\subsubsection{Mittelteil für die verschiedenen Derived Datatypes}
\label{mittelteil_für_die_verschiedenen_derived_datatypes}

\begin{code}
  \centering
  \numberedcodebox[minted language=c]{./code_examples/example_derived_dts_main_part.picoc}
  \caption{PicoC Code für den Mittelteil}
  \label{code:picoc_code_mittelteil}
\end{code}

% spezielles Vorgehen bei PntrDecl

\begin{code}
  \centering
  \numberedcodebox[minted language=text]{./code_examples/example_derived_dts_main_part.ast}
  \caption{Abstract Syntax Tree für den Mittelteil}
  \label{code:abstract_syntax_tree_mittelteil}
\end{code}

\begin{code}
  \centering
  \numberedcodebox[minted language=text]{./code_examples/example_derived_dts_main_part.picoc_mon}
  \caption{PicoC Mon Pass für den Mittelteil}
  \label{code:picoc_mon_pass_mittelteil}
\end{code}

\begin{code}
  \centering
  \numberedcodebox[minted language=text]{./code_examples/example_derived_dts_main_part.reti_blocks}
  \caption{RETI Blocks Pass für den Mittelteil}
  \label{code:reti_blocks_pass_mittelteil}
\end{code}
% spezielles Vorgehen bei PntrDecl

\subsubsection{Schlussteil für die verschiedenen Derived Datatypes}
\begin{code}
  \centering
  \numberedcodebox[minted language=c]{./code_examples/example_derived_dts_final_part.picoc}
  \caption{PicoC Code für den Schlussteil}
  \label{code:picoc_code_schlussteil}
\end{code}

\begin{code}
  \centering
  \numberedcodebox[minted language=text]{./code_examples/example_derived_dts_final_part.ast}
  \caption{Abstract Syntax Tree für den Schlussteil}
  \label{code:abstract_syntax_tree_schlussteil}
\end{code}

\begin{code}
  \centering
  \numberedcodebox[minted language=text]{./code_examples/example_derived_dts_final_part.picoc_mon}
  \caption{PicoC Mon Pass für den Schlussteil}
  \label{code:picoc_mon_pass_schlussteil}
\end{code}

\begin{code}
  \centering
  \numberedcodebox[minted language=text]{./code_examples/example_derived_dts_final_part.reti_blocks}
  \caption{RETI Blocks Pass für den Schlussteil}
  \label{code:reti_blocks_pass_schlussteil}
\end{code}

% Umgang, wenn Datentyp abrubt aufhört am Ende

\subsection{Umsetzung von Funktionen}
\subsubsection{Funktionen auflösen zu RETI Code}
\begin{code}
  \centering
  \numberedcodebox[minted language=c]{./code_examples/example_3_funs.picoc}
  \caption{PicoC Code für 3 Funktionen}
  \label{code:picoc_code_für_3_Funktionen}
\end{code}

\begin{code}
  \centering
  \numberedcodebox[minted language=text]{./code_examples/example_3_funs.ast}
  \caption{Abstract Syntax Tree für 3 Funktionen}
  \label{code:abstract_syntax_tree_für_3_Funktionen}
\end{code}

\begin{code}
  \centering
  \numberedcodebox[minted language=text]{./code_examples/example_3_funs.picoc_blocks}
  \caption{PicoC Blocks Pass für 3 Funktionen}
  \label{code:picoc_blocks_pass_für_3_Funktionen}
\end{code}

\begin{code}
  \centering
  \numberedcodebox[minted language=text]{./code_examples/example_3_funs.picoc_mon}
  \caption{PicoC Mon Pass für 3 Funktionen}
  \label{code:picoc_mon_pass_für_3_Funktionen}
\end{code}

\begin{code}
  \centering
  \numberedcodebox[minted language=text]{./code_examples/example_3_funs.reti_blocks}
  \caption{RETI Blocks Pass für 3 Funktionen}
  \label{code:reti_blocks_pass_für_3_Funktionen}
\end{code}

% einfügen unsichtbarer Returns bei void
\newlineparagraph{Sprung zur Main Funktion}

\begin{code}
  \centering
  \numberedcodebox[minted language=c]{./code_examples/example_3_funs_main.picoc}
  \caption{PicoC Code für Funktionen, wobei die main Funktion nicht die erste Funktion ist}
  \label{code:picoc_code_für_funktionen_wobei_die_main_funktion_nicht_die_erste_Funktion_ist}
\end{code}

\begin{code}
  \centering
  \numberedcodebox[minted language=text]{./code_examples/example_3_funs_main.picoc_mon}
  \caption{PicoC Mon Pass für Funktionen, wobei die main Funktion nicht die erste Funktion ist}
  \label{code:picoc_mon_pass_für_funktionen_wobei_die_main_funktion_nicht_die_erste_Funktion_ist}
\end{code}

\begin{code}
  \centering
  \numberedcodebox[minted language=text]{./code_examples/example_3_funs_main.reti_blocks}
  \caption{PicoC Blocks Pass für Funktionen, wobei die main Funktion nicht die erste Funktion ist}
  \label{code:picoc_blocks_pass_für_funktionen_wobei_die_main_funktion_nicht_die_erste_Funktion_ist}
\end{code}

\begin{code}
  \centering
  \numberedcodebox[minted language=text]{./code_examples/example_3_funs_main.reti_patch}
  \caption{PicoC Patch Pass für Funktionen, wobei die main Funktion nicht die erste Funktion ist}
  \label{code:picoc_patch_pass_für_funktionen_wobei_die_main_funktion_nicht_die_erste_Funktion_ist}
\end{code}

\subsubsection{Funktionsdeklaration und -definition}
\begin{code}
  \centering
  \numberedcodebox[minted language=c]{./code_examples/example_3_funs_fun_decl.picoc}
  \caption{PicoC Code für Funktionen, wobei eine Funktion vorher deklariert werden muss}
  \label{code:picoc_code_für_funktionen_picoc_code_für_funktionen_wobei_eine_funktion_vorher_deklariert_werden_muss}
\end{code}

\begin{code}
  \centering
  \numberedcodebox[minted language=text]{./code_examples/example_3_funs_fun_decl.st}
  \caption{Symboltabelle für Funktionen, wobei eine Funktion vorher deklariert werden muss}
  \label{code:symboltabelle_für_funktionen_picoc_code_für_funktionen_wobei_eine_funktion_vorher_deklariert_werden_muss}
\end{code}

% Allocation von Variablen
% Stack und Globale Statische Daten
% die Sache mit Assign(Tmp, Global) und Assign(Global, Tmp)
% erwähnen, das Main Funktion keinen Stackframe hat
% zählen der Größe der lokalen Daten und Parameter
% TODO: Signatur zu Parameter umbenennen
\subsubsection{Funktionsaufruf}

\newlineparagraph{Ohne Rückgabewert}

% Unsichtbares return
\begin{code}
  \centering
  \numberedcodebox[minted language=c]{./code_examples/example_fun_call_no_return_value.picoc}
  \caption{PicoC Code für Funktionsaufruf ohne Rückgabewert}
  \label{code:picoc_code_für_funktionsaufruf_ohne_rückgabewert}
\end{code}

\begin{code}
  \centering
  \numberedcodebox[minted language=text]{./code_examples/example_fun_call_no_return_value.picoc_mon}
  \caption{PicoC Mon Pass für Funktionsaufruf ohne Rückgabewert}
  \label{code:picoc_mon_pass_für_funktionsaufruf_ohne_rückgabewert}
\end{code}

\begin{code}
  \centering
  \numberedcodebox[minted language=text]{./code_examples/example_fun_call_no_return_value.reti_blocks}
  \caption{RETI Blocks Pass für Funktionsaufruf ohne Rückgabewert}
  \label{code:reti_blocks_pass_für_funktionsaufruf_ohne_rückgabewert}
\end{code}

\begin{code}
  \centering
  \numberedcodebox[minted language=text]{./code_examples/example_fun_call_no_return_value.reti}
  \caption{RETI Pass für Funktionsaufruf ohne Rückgabewert}
  \label{code:reti_pass_für_funktionsaufruf_ohne_rückgabewert}
\end{code}

\newlineparagraph{Mit Rückgabewert}

\begin{code}
  \centering
  \numberedcodebox[minted language=c]{./code_examples/example_fun_call_with_return_value.picoc}
  \caption{PicoC Code für Funktionsaufruf mit Rückgabewert}
  \label{code:picoc_code_für_funktionsaufruf_mit_rückgabewert}
\end{code}

\begin{code}
  \centering
  \numberedcodebox[minted language=text]{./code_examples/example_fun_call_with_return_value.picoc_mon}
  \caption{PicoC Mon Pass für Funktionsaufruf mit Rückgabewert}
  \label{code:picoc_mon_pass_für_funktionsaufruf_mit_rückgabewert}
\end{code}

\begin{code}
  \centering
  \numberedcodebox[minted language=text]{./code_examples/example_fun_call_with_return_value.reti_blocks}
  \caption{RETI Blocks Pass für Funktionsaufruf mit Rückgabewert}
  \label{code:reti_blocks_pass_für_funktionsaufruf_mit_rückgabewert}
\end{code}

\begin{code}
  \centering
  \numberedcodebox[minted language=text]{./code_examples/example_fun_call_with_return_value.reti}
  \caption{RETI Pass für Funktionsaufruf mit Rückgabewert}
  \label{code:reti_pass_für_funktionsaufruf_mit_rückgabewert}
\end{code}

\newlineparagraph{Umsetzung von Call by Sharing für Arrays}

\begin{code}
  \centering
  \numberedcodebox[minted language=c, minted options={highlightlines={1,7}}]{./code_examples/example_fun_call_by_sharing_array.picoc}
  \caption{PicoC Code für Call by Sharing für Arrays}
  \label{code:picoc_code_für_call_by_sharing_für_arrays}
\end{code}

\begin{code}
  \centering
  \numberedcodebox[minted language=text, minted options={highlightlines={15-20}}]{./code_examples/example_fun_call_by_sharing_array.picoc_mon}
  \caption{PicoC Mon Pass für Call by Sharing für Arrays}
  \label{code:picoc_mon_pass_für_call_by_sharing_für_arrays}
\end{code}


% https://tex.stackexchange.com/questions/298383/how-to-highlight-color-draw-attention-to-a-particular-snippet-in-minted/498614#498614
\begin{code}
  \centering
  \numberedcodebox[minted language=text, minted options={highlightlines={15,24}}]{./code_examples/example_fun_call_by_sharing_array.st}
  \caption{Symboltabelle für Call by Sharing für Arrays}
  \label{code:symboltabelle_für_call_by_sharing_für_arrays}
\end{code}

\begin{code}
  \centering
  \numberedcodebox[minted language=text, minted options={highlightlines={13-20}}]{./code_examples/example_fun_call_by_sharing_array.reti_blocks}
  \caption{RETI Block Pass für Call by Sharing für Arrays}
  \label{code:reti_blocks_pass_für_call_by_sharing_für_arrays}
\end{code}

% die Sache mit dem erstetzen von ArryDecl durch PntrDecl

\newlineparagraph{Umsetzung von Call by Value für Structs}

\begin{code}
  \centering
  \numberedcodebox[minted language=c, minted options={highlightlines={8}}]{./code_examples/example_fun_call_by_value_struct.picoc}
  \caption{PicoC Code für Call by Value für Structs}
  \label{code:picoc_code_für_call_by_value_für_structs}
\end{code}

% argmode für Struct Call by Value

\begin{code}
  \centering
  \numberedcodebox[minted language=text, minted options={highlightlines={15-19}}]{./code_examples/example_fun_call_by_value_struct.picoc_mon}
  \caption{PicoC Mon Pass für Call by Value für Structs}
  \label{code:picoc_mon_pass_für_call_by_value_for_structs}
\end{code}

% hier könnte man anmerken, dass die Adressen unterschiedlich berechnet werden für Stack und Globale...

\begin{code}
  \centering
  \numberedcodebox[minted language=text, minted options={highlightlines={13-19}}]{./code_examples/example_fun_call_by_value_struct.reti_blocks}
  \caption{RETI Block Pass für Call by Value für Structs}
  \label{code:reti_blocks_pass_für_call_by_value_for_structs}
\end{code}

% Struct wird wirklich kopiert durch speziellen Argmode

\subsection{Umsetzung kleinerer Details}
% langen Sprüngen, großen Konstanten, Division durch 0
\section{Fehlermeldungen}
\subsection{Error Handler}
\subsection{Arten von Fehlermeldungen}
\subsubsection{Syntaxfehler}
\subsubsection{Laufzeitfehler}
% Fehlermeldung ist, wenn der Lexer (partielle Funktion) oder Parser nicht matcht
% Token und Nodes enthalten Position, im Transformer wird die Position von den Token auf die Nodes übertragen und auch die Symboltabelle speichert Position
