%!Tex Root = ../Main.tex
% ./Packete_und_Deklarationen.tex
% ./Titlepage.tex
% ./Motivation.tex
% ./Einführung.tex
% ./Implementierung1.tex
% ./Ergebnisse_und_Ausblick.tex

\subsection{Umsetzung von Pointern}
\subsubsection{Referenzierung}
Die \colorbold{Referenzierung} \verb|&<var>| wird im Folgenden anhand des Beispiels in Code~\ref{code:picoc_code_für_pointer_referenzierung} erklärt.

\begin{code}
  \centering
  \numberedcodebox[minted language=c, minted options={highlightlines={3}}]{./code_examples/example_pntr_ref.picoc}
  \caption{PicoC Code für Pointer Referenzierung}
  \label{code:picoc_code_für_pointer_referenzierung}
\end{code}

Der Knoten \smalltt{Ref(Name('var')))} repräsentiert im \colorbold{Abstrakt Syntax Tree} in Code~\ref{code:abstract_syntax_tree_für_pointer_referenzierung} eine \colorbold{Referenzierung} \verb|&<var>|.

\begin{code}
  \centering
  \numberedcodebox[minted language=text, minted options={highlightlines={10}}]{./code_examples/example_pntr_ref.ast}
  \caption{Abstract Syntax Tree für Pointer Referenzierung}
  \label{code:abstract_syntax_tree_für_pointer_referenzierung}
\end{code}

Im \colorbold{PicoC-Mon Pass} in Code~\ref{code:picoc_mon_für_pointer_referenzierung} wird der Knoten \smalltt{Ref(Name('var')))} durch die Knoten \smalltt{Ref(GlobalRead(Num('0')))} und \smalltt{Assign(GlobalWrite(Num('1')), Tmp(Num('1')))} ersetzt. Im Fall, dass in \smalltt{Ref(exp))} das \smalltt{exp} vielleicht nicht direkt ein \smalltt{Name('var')} enthält und \smalltt{exp} z.B. ein \smalltt{Subscr(Attr(Name('var')))} ist, sind noch weitere Anweisungen zwischen den Zeilen \smalltt{11} und  \smalltt{12} nötig, die sich in diesem Beispiel um das Übersetzen von \smalltt{Subscr(exp)} und \smalltt{Attr(exp)} nach dem Schema in Subkapitel~\ref{mittelteil_für_die_verschiedenen_derived_datatypes} kümmern.

\begin{code}
  \centering
  \numberedcodebox[minted language=text, minted options={highlightlines={11-13}}]{./code_examples/example_pntr_ref.picoc_mon}
  \caption{PicoC Mon Pass für Pointer Referenzierung}
  \label{code:picoc_mon_für_pointer_referenzierung}
\end{code}

Im \colorbold{PicoC-Blocks Pass} in Code~\ref{code:reti_blocks_für_pointer_referenzierung} werden die \colorbold{PicoC-Knoten} \smalltt{ Ref(Global(Num('0')))} und \smalltt{Assign(Global(Num('1')), Stack(Num('1')))} durch ihre entsprechenden \colorbold{RETI-Knoten} ersetzt.

\begin{code}
  \centering
  \numberedcodebox[minted language=text, minted options={highlightlines={18-21,23-25}}]{./code_examples/example_pntr_ref.reti_blocks}
  \caption{RETI Blocks Pass für Pointer Referenzierung}
  \label{code:reti_blocks_für_pointer_referenzierung}
\end{code}
% Initialisierung eines Pointers
\subsubsection{Dereferenzierung durch Zugriff auf Arrayindex ersetzen}
Die \colorbold{Dereferenzierung} \smalltt{*<var>} wird im Folgenden anhand des Beispiels in Code~\ref{code:picoc_code_für_pointer_dereferenzierung} erklärt.

\begin{code}
  \centering
  \numberedcodebox[minted language=c, minted options={highlightlines={4}}]{./code_examples/example_pntr_deref.picoc}
  \caption{PicoC Code für Pointer Dereferenzierung}
  \label{code:picoc_code_für_pointer_dereferenzierung}
\end{code}

Der Knoten \smalltt{Deref(Name('var')))} repräsentiert im \colorbold{Abstrakt Syntax Tree} in Code~\ref{code:abstract_syntax_tree_für_pointer_dereferenzierung} eine \colorbold{Dereferenzierung} \smalltt{*<var>}.

\begin{code}
  \centering
  \numberedcodebox[minted language=text, minted options={highlightlines={11}}]{./code_examples/example_pntr_deref.ast}
  \caption{Abstract Syntax Tree für Pointer Dereferenzierung}
  \label{code:abstract_syntax_tree_für_pointer_dereferenzierung}
\end{code}

Im \colorbold{PicoC-Shrink Pass} in Code~\ref{code:picoc_shrink_für_pointer_dereferenzierung} wird ein Trick angewandet, bei dem jeder Knoten \smalltt{Deref(Name('pntr'), Num('0'))} einfach durch den Knoten \smalltt{Subscr(Name('pntr'), Num('0'))} ersetzt wird. Der Trick besteht darin, dass der \colorbold{Dereferenzoperator} \smalltt{*(<var> + <i>)} sich identisch zum \colorbold{Operator für den Zugriff auf einen Arrayindex} \smalltt{<var>[<i>]} verhält\footnote{In der Sprache $L_{C}$ gibt es einen Unterschied bei der Initialisierung bei z.B. \smalltt{<datatype> *<var> = \dq string\dq} und \smalltt{<datatype> <var>[<i>] = \dq string\dq}, der allerdings nichts mit den beiden Operatoren zu tuen hat, sondern mit der \colorbold{Initialisierung}, bei der die Sprache $L_{C}$ verwirrenderweise die eckigen Klammern \smalltt{[]} genauso, wie beim \colorbold{Operator für den Zugriff auf einen Arrayindex}, vor den Bezeichner schreibt: \smalltt{<var>[<i>]}, obwohl es ein \colorbold{Derived Datatype} ist.}. Damit sparrt man sich viele vermeidbare \colorbold{Fallunterscheidungen} und \colorbold{doppelten Code} und kann die \colorbold{Derefenzierung} \smalltt{*(<var> + <i>)} einfach von den Routinen für einen \colorbold{Zugriff auf einen Arrayindex} \smalltt{<var>[<i>]} übernehmen lassen.

\begin{code}
  \centering
  \numberedcodebox[minted language=text, minted options={highlightlines={11}}]{./code_examples/example_pntr_deref.picoc_shrink}
  \caption{PicoC Shrink Pass für Pointer Dereferenzierung}
  \label{code:picoc_shrink_für_pointer_dereferenzierung}
\end{code}

\subsection{Umsetzung von Arrays}
\subsubsection{Initialisierung von Arrays}

Die \colorbold{Initialisierung} eines \colorbold{Arrays} (\smalltt{<datatype> <var>[2][1] = \{\{3+1\}, \{4\}\}}) wird im Folgenden anhand des Beispiels in Code~\ref{code:picoc_code_für_array_initialisierung} erklärt.

% Stack und Globale Statische Daten
\begin{code}
  \centering
  \numberedcodebox[minted language=c, minted options={highlightlines={2, 6}}]{./code_examples/example_array_init.picoc}
  \caption{PicoC Code für Array Initialisierung}
  \label{code:picoc_code_für_array_initialisierung}
\end{code}

Die \colorbold{Initialisierung} eines \colorbold{Arrays} \smalltt{<datatype> <var>[2][1] = \{\{3+1\}, \{4\}\}} wird im \colorbold{Abstrakt Syntax Tree} in Code~\ref{code:abstract_syntax_tree_für_array_initialisierung} mithilfe der Komposition \smalltt{Assign(Alloc(Writeable(), ArrayDecl([Num('2'), Num('1')], IntType('int')), Name('ar')), Array([Array([BinOp(Num('3'), Add('+'), Num('1'))]), Array([Num('4')])]))} dargestellt.

\begin{code}
  \centering
  \numberedcodebox[minted language=text, minted options={highlightlines={9, 16}}]{./code_examples/example_array_init.ast}
  \caption{Abstract Syntax Tree für Array Initialisierung}
  \label{code:abstract_syntax_tree_für_array_initialisierung}
\end{code}

Bei der \colorbold{Initialisierung} eines \colorbold{Arrays} wird zuerst \smalltt{Alloc(Writeable(), ArrayDecl([Num('2'), Num('1')], IntType('int')))} ausgewertet, da eine Variable zuerst definiert sein muss, bevor man sie verwenden kann\footnote{Das Widerspricht der üblichen Auswertungsreihenfolge beim \colorbold{Zuweisungsoperator} \smalltt{=}, der \colorbold{rechtsassoziativ} ist. Der \colorbold{Zuweisungsoperator} \smalltt{=} tritt allerdings erst später in Aktion.}. Das \colorbold{Definieren} der Variable \smalltt{ar} erfolgt mittels der \colorbold{Symboltabelle}, die in Code~\ref{code:symboltabelle_für_array_initialisierung} dargestellt ist.

Bei Variablen auf dem \colorbold{Stackframe} wird ein Array \colorbold{rückwärts} auf das Stackframe geschrieben und auch die \colorbold{Adresse des ersten Elements} als Adresse des Arrays genommen. Dies macht den \colorbold{Zugriff auf ein Arrayelement} in Subkapitel~\ref{sec:zugriff_auf_ein_arrayelement} deutlich unkomplizierter, da man so nicht mehr zwischen \colorbold{Stackframe} und \colorbold{Globalen Statischen Daten} beim \colorbold{Zugriff auf ein Arrayelement} unterscheiden muss, da es Probleme macht, dass ein \colorbold{Stackframe} in die Entgegengesetzt Richtung der \colorbold{Globalen Statischen Daten} wächst\footnote{Wenn man beim \colorbold{GCC}~\cite{noauthor_gcc_nodate} einen Stackframe mittels des \colorbold{GDB}~\cite{noauthor_gcc_nodate} beobachtet, sieht man, dass dieser es genauso macht.}.

\begin{code}
  \centering
  \numberedcodebox[minted language=text, minted options={highlightlines={14-19,32-37}}]{./code_examples/example_array_init.st}
  \caption{Symboltabelle für Array Initialisierung}
  \label{code:symboltabelle_für_array_initialisierung}
\end{code}

Im \colorbold{PiocC-Mon Pass} in Code~\ref{code:picoc_mon_für_array_initialisierung} werden zuerst die \colorbold{Ausdrücke} im \colorbold{Array-Initializer} \smalltt{Array([Array([BinOp(Num('3'), Add('+'), Num('1'))]), Array([Num('4')])])} nach dem \colorbold{Depth-First-Search} Schema, von \colorbold{links-nach-rechts} ausgewertet und auf den \colorbold{Stack} geschrieben.

Im finalen Schritt muss zwischen \colorbold{Globalen Statischen Daten} bei der \smalltt{main}-Funktion und \colorbold{Stackframe} bei der Funktion \smalltt{fun} unterschieden werden. Die auf den Stack ausgewerteten Expressions werden mittels der Komposition \smalltt{Assign(Global(Num('0')), Stack(Num('2')))} bzw. \smalltt{Assign(Stackframe(Num('3')), Stack(Num('4')))}, versetzt in der selben Reihenfolge zu den \colorbold{Globalen Statischen Daten} bzw. auf den \colorbold{Stackframe} geschrieben.

In die Knoten \smalltt{Global('0')} und  \smalltt{Stackframe('3')} wurde hierbei die \colorbold{Startadresse} des jeweiligen Arrays geschrieben, sodass man nach dem \colorbold{PicoC-Mon Pass} nie mehr Variablen in der  \colorbold{Symboltabelle} nachsehen muss und gleich weiß, ob sie bei den \colorbold{Globalen Statischen Daten} oder auf dem \colorbold{Stackframe} liegen.

\begin{code}
  \centering
  \numberedcodebox[minted language=text, minted options={highlightlines={8-12,19-23}}]{./code_examples/example_array_init.picoc_mon}
  \caption{PicoC Mon Pass für Array Initialisierung}
  \label{code:picoc_mon_für_array_initialisierung}
\end{code}

Im \colorbold{PicoC-Blocks Pass} in Code~\ref{code:reti_blocks_für_array_initialisierung} werden die \colorbold{PicoC-Knoten} für die Ausdrücke \smalltt{Exp(exp)} und \smalltt{Assign(Global(Num('0')), Stack(Num('2')))} bzw. \smalltt{Assign(Stackframe(Num('3')), Stack(Num('4')))} durch ihre entsprechenden \colorbold{RETI-Knoten} ersetzt.

\begin{code}
  \centering
  \numberedcodebox[minted language=text, minted options={highlightlines={9-11,13-15,17-21,23-25,27-31,40-42,44-46,48-50,52-54,56-64}}]{./code_examples/example_array_init.reti_blocks}
  \caption{RETI Blocks Pass für Array Initialisierung}
  \label{code:reti_blocks_für_array_initialisierung}
\end{code}


% kleines Extra
\subsubsection{Zugriff auf ein Arrayelement}
\label{sec:zugriff_auf_ein_arrayelement}

Der \colorbold{Zugriff auf ein Arrayelement} \smalltt{ar[0]} wird im Folgenden anhand des Beispiels in Code~\ref{code:picoc_code_für_zugriff_auf_arrayindex} erklärt.

\begin{code}
  \centering
  \numberedcodebox[minted language=c, minted options={highlightlines={3,8}}]{./code_examples/example_array_access.picoc}
  \caption{PicoC-Code für Zugriff auf ein Arrayelement}
  \label{code:picoc_code_für_zugriff_auf_arrayindex}
\end{code}

Der \colorbold{Zugriff auf ein Arrayelement} \smalltt{ar[0]} wird im  \colorbold{Abstract Syntx Tree} in Code~\ref{code:abstract_syntax_tree_für_zugriff_auf_arrayindex} mithilfe des \colorbold{Container-Knotens} \smalltt{Subscr(Name('ar'), Num('0'))} dargestellt.

\begin{code}
  \centering
  \numberedcodebox[minted language=text, minted options={highlightlines={10,18}}]{./code_examples/example_array_access.ast}
  \caption{Abstract Syntax Tree für Zugriff auf ein Arrayelement}
  \label{code:abstract_syntax_tree_für_zugriff_auf_arrayindex}
\end{code}

Im \colorbold{PicoC-Mon Pass} in Code~\ref{code:picoc_mon_für_zugriff_auf_arrayindex} wird beim \colorbold{Container-Knoten} \smalltt{Subscr(Name('ar'), Num('0'))} zuerst die \colorbold{Adresse} der Variable \smalltt{Name('ar')} auf den \colorbold{Stack} geschrieben. Bei den \colorbold{Globalen Statischen Daten} der \smalltt{main}-Funktion wird das durch die Komposition \smalltt{Ref(Global(Num('0')))} dargestellt und beim \colorbold{Stackframe} der Funktionm \smalltt{fun} wird das durch die Komposition \smalltt{Ref(Stackframe(Num('2')))} dargestellt.

In nächsten Schritt wird die Adresse des \colorbold{Index}, des Arrays auf das Zugegriffen werden soll berechnet. Da der \colorbold{Index} auf den Zugegriffen werden soll auch durch das Ergebnis eines \colorbold{komplexeren Ausdrucks}, z.B. \smalltt{<ar>[1 + <var2>]} bestimmt sein kann, indem auch \colorbold{Variablen} vorkommen können, kann dieser nicht während des \colorbold{Kompilierens} berechnet werden, sondern muss zur \colorbold{Laufzeit} berechnet werden. Daher muss zuerst der Wert des \colorbold{Index} dessen Adresse berechnet werden soll bestimmt werden, z.B. im einfachen Fall durch \smalltt{Exp(Num('0'))} und dann muss der Index berechnet werden, was durch die Komposition \smalltt{Ref(Subscr(Stack(Num('2')), Stack(Num('1'))))} dargestellt wird.

\begin{code}
  \centering
  \numberedcodebox[minted language=text, minted options={highlightlines={11-14,26}}]{./code_examples/example_array_access.picoc_mon}
  \caption{PicoC-Mon Pass für Zugriff auf ein Arrayelement}
  \label{code:picoc_mon_für_zugriff_auf_arrayindex}
\end{code}

\begin{code}
  \centering
  \numberedcodebox[minted language=text, minted options={highlightlines={18-21,23-25,27-32,34-36,66-69}}]{./code_examples/example_array_access.reti_blocks}
  \caption{RETI-Blocks Pass für Zugriff auf ein Arrayelement}
  \label{code:reti_blocks_für_zugriff_auf_arrayindex}
\end{code}

\subsubsection{Zuweisung an Arrayindex}
% Formel aus der Vorlesung, wo ist die hier?
\begin{code}
  \centering
  \numberedcodebox[minted language=c]{./code_examples/example_array_assignment.picoc}
  \caption{PicoC Code für Zuweisung an Arrayindex}
  \label{code:picoc_code_für_array_assignment}
\end{code}

\begin{code}
  \centering
  \numberedcodebox[minted language=text]{./code_examples/example_array_assignment.ast}
  \caption{Abstract Syntax Tree für Zuweisung an Arrayindex}
  \label{code:abstract_syntax_tree_für_array_assignment}
\end{code}

\begin{code}
  \centering
  \numberedcodebox[minted language=text]{./code_examples/example_array_assignment.picoc_mon}
  \caption{PicoC Mon Pass für Zuweisung an Arrayindex}
  \label{code:picoc_mon_für_array_assignment}
\end{code}

\begin{code}
  \centering
  \numberedcodebox[minted language=text]{./code_examples/example_array_assignment.reti_blocks}
  \caption{RETI Blocks Pass für Zuweisung an Arrayindex}
  \label{code:reti_blocks_für_array_assignment}
\end{code}

\subsection{Umsetzung von Structs}
\subsubsection{Deklaration von Structs}
\begin{code}
  \centering
  \numberedcodebox[minted language=c]{./code_examples/example_struct_decl.picoc}
  \caption{PicoC Code für Deklaration von Structs}
  \label{code:picoc_code_für_deklaration_von_structs}
\end{code}

\begin{code}
  \centering
  \numberedcodebox[minted language=text]{./code_examples/example_struct_decl.st}
  \caption{Symboltabelle für Deklaration von Structs}
  \label{code:symboltabelle_für_deklaration_von_structs}
\end{code}

\subsubsection{Initialisierung von Structs}
\begin{code}
  \centering
  \numberedcodebox[minted language=c]{./code_examples/example_struct_init.picoc}
  \caption{PicoC Code für Initialisierung von Structs}
  \label{code:picoc_code_für_initialisierung_von_structs}
\end{code}

\begin{code}
  \centering
  \numberedcodebox[minted language=text]{./code_examples/example_struct_init.ast}
  \caption{Abstract Syntax Tree für Initialisierung von Structs}
  \label{code:abstract_syntax_tree_für_initialisierung_von_structs}
\end{code}

\begin{code}
  \centering
  \numberedcodebox[minted language=text]{./code_examples/example_struct_init.st}
  \caption{Symboltabelle für Initialisierung von Structs}
  \label{code:symboltabelle_für_initialisierung_von_structs}
\end{code}

\begin{code}
  \centering
  \numberedcodebox[minted language=text]{./code_examples/example_struct_init.picoc_mon}
  \caption{PicoC Mon Pass für Initialisierung von Structs}
  \label{code:picoc_mon_pass_für_initialisierung_von_structs}
\end{code}

\begin{code}
  \centering
  \numberedcodebox[minted language=text]{./code_examples/example_struct_init.reti_blocks}
  \caption{RETI Blocks Pass für Initialisierung von Structs}
  \label{code:reti_blocks_pass_für_initialisierung_von_structs}
\end{code}

% Stack und Globale Statische Daten
\subsubsection{Zugriff auf Structattribut}
% Formel aus der Vorlesung, wo ist die hier?
\begin{code}
  \centering
  \numberedcodebox[minted language=c]{./code_examples/example_struct_attr_access.picoc}
  \caption{PicoC Code für Zugriff auf Structattribut}
  \label{code:picoc_code_für_zugriff_auf_structattribut}
\end{code}

\begin{code}
  \centering
  \numberedcodebox[minted language=text]{./code_examples/example_struct_attr_access.ast}
  \caption{Abstract Syntax Tree für Zugriff auf Structattribut}
  \label{code:abstract_syntax_tree_für_zugriff_auf_structattribut}
\end{code}

\begin{code}
  \centering
  \numberedcodebox[minted language=text]{./code_examples/example_struct_attr_access.picoc_mon}
  \caption{PicoC Mon Pass für Zugriff auf Structattribut}
  \label{code:picoc_mon_pass_für_zugriff_auf_structattribut}
\end{code}

\begin{code}
  \centering
  \numberedcodebox[minted language=text]{./code_examples/example_struct_attr_access.reti_blocks}
  \caption{RETI Blocks Pass für Zugriff auf Structattribut}
  \label{code:reti_blocks_pass_für_zugriff_auf_structattribut}
\end{code}

\subsubsection{Zuweisung an Structattribut}
\begin{code}
  \centering
  \numberedcodebox[minted language=c]{./code_examples/example_struct_attr_assignment.picoc}
  \caption{PicoC Code für Zuweisung an Structattribut}
  \label{code:picoc_code_für_zuweisung_an_structattribut}
\end{code}

\begin{code}
  \centering
  \numberedcodebox[minted language=text]{./code_examples/example_struct_attr_assignment.ast}
  \caption{Abstract Syntax Tree für Zuweisung an Structattribut}
  \label{code:abstract_syntax_tree_für_zuweisung_an_structattribut}
\end{code}

\begin{code}
  \centering
  \numberedcodebox[minted language=text]{./code_examples/example_struct_attr_assignment.picoc_mon}
  \caption{PicoC Mon Pass für Zuweisung an Structattribut}
  \label{code:picoc_mon_pass_für_zuweisung_an_structattribut}
\end{code}

\begin{code}
  \centering
  \numberedcodebox[minted language=text]{./code_examples/example_struct_attr_assignment.reti_blocks}
  \caption{RETI Blocks Pass für Zuweisung an Structattribut}
  \label{code:reti_blocks_pass_für_zuweisung_an_structattribut}
\end{code}

\subsection{Umsetzung der Derived Datatypes im Zusammenspiel}
\subsubsection{Einleitungsteil für Globale Statische Daten und Stackframe}
% Stack und Globale Statische Daten, unterschieldihe Berechnung der Adressen
% unterschiedliche Adressberechnung
\begin{code}
  \centering
  \numberedcodebox[minted language=c]{./code_examples/example_derived_dts_introduction_part.picoc}
  \caption{PicoC Code für den Einleitungsteil}
  \label{code:picoc_code_einleitungsteil}
\end{code}

% spezielles Vorgehen bei PntrDecl

\begin{code}
  \centering
  \numberedcodebox[minted language=text]{./code_examples/example_derived_dts_introduction_part.ast}
  \caption{Abstract Syntax Tree für den Einleitungsteil}
  \label{code:abstract_syntax_tree_einleitungsteil}
\end{code}

\begin{code}
  \centering
  \numberedcodebox[minted language=text]{./code_examples/example_derived_dts_introduction_part.picoc_mon}
  \caption{PicoC Mon Pass für den Einleitungsteil}
  \label{code:picoc_mon_pass_einleitungsteil}
\end{code}

\begin{code}
  \centering
  \numberedcodebox[minted language=text]{./code_examples/example_derived_dts_introduction_part.reti_blocks}
  \caption{RETI Blocks Pass für den Einleitungsteil}
  \label{code:reti_blocks_pass_einleitungsteil}
\end{code}

\subsubsection{Mittelteil für die verschiedenen Derived Datatypes}
\label{mittelteil_für_die_verschiedenen_derived_datatypes}

\begin{code}
  \centering
  \numberedcodebox[minted language=c]{./code_examples/example_derived_dts_main_part.picoc}
  \caption{PicoC Code für den Mittelteil}
  \label{code:picoc_code_mittelteil}
\end{code}

% spezielles Vorgehen bei PntrDecl

\begin{code}
  \centering
  \numberedcodebox[minted language=text]{./code_examples/example_derived_dts_main_part.ast}
  \caption{Abstract Syntax Tree für den Mittelteil}
  \label{code:abstract_syntax_tree_mittelteil}
\end{code}

\begin{code}
  \centering
  \numberedcodebox[minted language=text]{./code_examples/example_derived_dts_main_part.picoc_mon}
  \caption{PicoC Mon Pass für den Mittelteil}
  \label{code:picoc_mon_pass_mittelteil}
\end{code}

\begin{code}
  \centering
  \numberedcodebox[minted language=text]{./code_examples/example_derived_dts_main_part.reti_blocks}
  \caption{RETI Blocks Pass für den Mittelteil}
  \label{code:reti_blocks_pass_mittelteil}
\end{code}
% spezielles Vorgehen bei PntrDecl

\subsubsection{Schlussteil für die verschiedenen Derived Datatypes}
\begin{code}
  \centering
  \numberedcodebox[minted language=c]{./code_examples/example_derived_dts_final_part.picoc}
  \caption{PicoC Code für den Schlussteil}
  \label{code:picoc_code_schlussteil}
\end{code}

\begin{code}
  \centering
  \numberedcodebox[minted language=text]{./code_examples/example_derived_dts_final_part.ast}
  \caption{Abstract Syntax Tree für den Schlussteil}
  \label{code:abstract_syntax_tree_schlussteil}
\end{code}

\begin{code}
  \centering
  \numberedcodebox[minted language=text]{./code_examples/example_derived_dts_final_part.picoc_mon}
  \caption{PicoC Mon Pass für den Schlussteil}
  \label{code:picoc_mon_pass_schlussteil}
\end{code}

\begin{code}
  \centering
  \numberedcodebox[minted language=text]{./code_examples/example_derived_dts_final_part.reti_blocks}
  \caption{RETI Blocks Pass für den Schlussteil}
  \label{code:reti_blocks_pass_schlussteil}
\end{code}

% Umgang, wenn Datentyp abrubt aufhört am Ende

\subsection{Umsetzung von Funktionen}
\subsubsection{Funktionen auflösen zu RETI Code}
\begin{code}
  \centering
  \numberedcodebox[minted language=c]{./code_examples/example_3_funs.picoc}
  \caption{PicoC Code für 3 Funktionen}
  \label{code:picoc_code_für_3_Funktionen}
\end{code}

\begin{code}
  \centering
  \numberedcodebox[minted language=text]{./code_examples/example_3_funs.ast}
  \caption{Abstract Syntax Tree für 3 Funktionen}
  \label{code:abstract_syntax_tree_für_3_Funktionen}
\end{code}

\begin{code}
  \centering
  \numberedcodebox[minted language=text]{./code_examples/example_3_funs.picoc_blocks}
  \caption{PicoC Blocks Pass für 3 Funktionen}
  \label{code:picoc_blocks_pass_für_3_Funktionen}
\end{code}

\begin{code}
  \centering
  \numberedcodebox[minted language=text]{./code_examples/example_3_funs.picoc_mon}
  \caption{PicoC Mon Pass für 3 Funktionen}
  \label{code:picoc_mon_pass_für_3_Funktionen}
\end{code}

\begin{code}
  \centering
  \numberedcodebox[minted language=text]{./code_examples/example_3_funs.reti_blocks}
  \caption{RETI Blocks Pass für 3 Funktionen}
  \label{code:reti_blocks_pass_für_3_Funktionen}
\end{code}

% einfügen unsichtbarer Returns bei void
\newlineparagraph{Sprung zur Main Funktion}

\begin{code}
  \centering
  \numberedcodebox[minted language=c]{./code_examples/example_3_funs_main.picoc}
  \caption{PicoC Code für Funktionen, wobei die main Funktion nicht die erste Funktion ist}
  \label{code:picoc_code_für_funktionen_wobei_die_main_funktion_nicht_die_erste_Funktion_ist}
\end{code}

\begin{code}
  \centering
  \numberedcodebox[minted language=text]{./code_examples/example_3_funs_main.picoc_mon}
  \caption{PicoC Mon Pass für Funktionen, wobei die main Funktion nicht die erste Funktion ist}
  \label{code:picoc_mon_pass_für_funktionen_wobei_die_main_funktion_nicht_die_erste_Funktion_ist}
\end{code}

\begin{code}
  \centering
  \numberedcodebox[minted language=text]{./code_examples/example_3_funs_main.reti_blocks}
  \caption{PicoC Blocks Pass für Funktionen, wobei die main Funktion nicht die erste Funktion ist}
  \label{code:picoc_blocks_pass_für_funktionen_wobei_die_main_funktion_nicht_die_erste_Funktion_ist}
\end{code}

\begin{code}
  \centering
  \numberedcodebox[minted language=text]{./code_examples/example_3_funs_main.reti_patch}
  \caption{PicoC Patch Pass für Funktionen, wobei die main Funktion nicht die erste Funktion ist}
  \label{code:picoc_patch_pass_für_funktionen_wobei_die_main_funktion_nicht_die_erste_Funktion_ist}
\end{code}

\subsubsection{Funktionsdeklaration und -definition und Umsetzung von Scopes}
\begin{code}
  \centering
  \numberedcodebox[minted language=c]{./code_examples/example_3_funs_fun_decl.picoc}
  \caption{PicoC Code für Funktionen, wobei eine Funktion vorher deklariert werden muss}
  \label{code:picoc_code_für_funktionen_picoc_code_für_funktionen_wobei_eine_funktion_vorher_deklariert_werden_muss}
\end{code}

Bei mehreren Funktionen werden die \colorbold{Scopes} der unterschiedlichen \colorbold{Funktionen} mittels eines \colorbold{Suffix} \smalltt{\dq <fun\_name>@\dq} umgesetzt, der an den \colorbold{Variablennamen} \smalltt{<var>} drangehängt wird: \smalltt{<var>@<fun\_name>}. Dieser \colorbold{Suffix} wird geändert sobald beim \colorbold{Top-Down}\footnote{D.h. von der Wurzel zu den Blättern eines Baumes} Durchiterieren über den \colorbold{Abstract Syntax Tree} des aktuellen \colorbold{Passes} nach dem \colorbold{Depth-First-Search} Schema über den

\begin{code}
  \centering
  \numberedcodebox[minted language=text]{./code_examples/example_3_funs_fun_decl.st}
  \caption{Symboltabelle für Funktionen, wobei eine Funktion vorher deklariert werden muss}
  \label{code:symboltabelle_für_funktionen_picoc_code_für_funktionen_wobei_eine_funktion_vorher_deklariert_werden_muss}
\end{code}

% Allocation von Variablen
% Stack und Globale Statische Daten
% die Sache mit Assign(Tmp, Global) und Assign(Global, Tmp)
% erwähnen, das Main Funktion keinen Stackframe hat
% zählen der Größe der lokalen Daten und Parameter
% TODO: Signatur zu Parameter umbenennen
\subsubsection{Funktionsaufruf}

\newlineparagraph{Ohne Rückgabewert}

% Unsichtbares return
\begin{code}
  \centering
  \numberedcodebox[minted language=c]{./code_examples/example_fun_call_no_return_value.picoc}
  \caption{PicoC Code für Funktionsaufruf ohne Rückgabewert}
  \label{code:picoc_code_für_funktionsaufruf_ohne_rückgabewert}
\end{code}

\begin{code}
  \centering
  \numberedcodebox[minted language=text]{./code_examples/example_fun_call_no_return_value.picoc_mon}
  \caption{PicoC Mon Pass für Funktionsaufruf ohne Rückgabewert}
  \label{code:picoc_mon_pass_für_funktionsaufruf_ohne_rückgabewert}
\end{code}

\begin{code}
  \centering
  \numberedcodebox[minted language=text]{./code_examples/example_fun_call_no_return_value.reti_blocks}
  \caption{RETI Blocks Pass für Funktionsaufruf ohne Rückgabewert}
  \label{code:reti_blocks_pass_für_funktionsaufruf_ohne_rückgabewert}
\end{code}

\begin{code}
  \centering
  \numberedcodebox[minted language=text]{./code_examples/example_fun_call_no_return_value.reti}
  \caption{RETI Pass für Funktionsaufruf ohne Rückgabewert}
  \label{code:reti_pass_für_funktionsaufruf_ohne_rückgabewert}
\end{code}

\newlineparagraph{Mit Rückgabewert}

\begin{code}
  \centering
  \numberedcodebox[minted language=c]{./code_examples/example_fun_call_with_return_value.picoc}
  \caption{PicoC Code für Funktionsaufruf mit Rückgabewert}
  \label{code:picoc_code_für_funktionsaufruf_mit_rückgabewert}
\end{code}

\begin{code}
  \centering
  \numberedcodebox[minted language=text]{./code_examples/example_fun_call_with_return_value.picoc_mon}
  \caption{PicoC Mon Pass für Funktionsaufruf mit Rückgabewert}
  \label{code:picoc_mon_pass_für_funktionsaufruf_mit_rückgabewert}
\end{code}

\begin{code}
  \centering
  \numberedcodebox[minted language=text]{./code_examples/example_fun_call_with_return_value.reti_blocks}
  \caption{RETI Blocks Pass für Funktionsaufruf mit Rückgabewert}
  \label{code:reti_blocks_pass_für_funktionsaufruf_mit_rückgabewert}
\end{code}

\begin{code}
  \centering
  \numberedcodebox[minted language=text]{./code_examples/example_fun_call_with_return_value.reti}
  \caption{RETI Pass für Funktionsaufruf mit Rückgabewert}
  \label{code:reti_pass_für_funktionsaufruf_mit_rückgabewert}
\end{code}

\newlineparagraph{Umsetzung von Call by Sharing für Arrays}

\begin{code}
  \centering
  \numberedcodebox[minted language=c, minted options={highlightlines={1,7}}]{./code_examples/example_fun_call_by_sharing_array.picoc}
  \caption{PicoC Code für Call by Sharing für Arrays}
  \label{code:picoc_code_für_call_by_sharing_für_arrays}
\end{code}

\begin{code}
  \centering
  \numberedcodebox[minted language=text, minted options={highlightlines={15-20}}]{./code_examples/example_fun_call_by_sharing_array.picoc_mon}
  \caption{PicoC Mon Pass für Call by Sharing für Arrays}
  \label{code:picoc_mon_pass_für_call_by_sharing_für_arrays}
\end{code}


% https://tex.stackexchange.com/questions/298383/how-to-highlight-color-draw-attention-to-a-particular-snippet-in-minted/498614#498614
\begin{code}
  \centering
  \numberedcodebox[minted language=text, minted options={highlightlines={15,24}}]{./code_examples/example_fun_call_by_sharing_array.st}
  \caption{Symboltabelle für Call by Sharing für Arrays}
  \label{code:symboltabelle_für_call_by_sharing_für_arrays}
\end{code}

\begin{code}
  \centering
  \numberedcodebox[minted language=text, minted options={highlightlines={13-20}}]{./code_examples/example_fun_call_by_sharing_array.reti_blocks}
  \caption{RETI Block Pass für Call by Sharing für Arrays}
  \label{code:reti_blocks_pass_für_call_by_sharing_für_arrays}
\end{code}

% die Sache mit dem erstetzen von ArryDecl durch PntrDecl

\newlineparagraph{Umsetzung von Call by Value für Structs}

\begin{code}
  \centering
  \numberedcodebox[minted language=c, minted options={highlightlines={8}}]{./code_examples/example_fun_call_by_value_struct.picoc}
  \caption{PicoC Code für Call by Value für Structs}
  \label{code:picoc_code_für_call_by_value_für_structs}
\end{code}

% argmode für Struct Call by Value

\begin{code}
  \centering
  \numberedcodebox[minted language=text, minted options={highlightlines={15-19}}]{./code_examples/example_fun_call_by_value_struct.picoc_mon}
  \caption{PicoC Mon Pass für Call by Value für Structs}
  \label{code:picoc_mon_pass_für_call_by_value_for_structs}
\end{code}

% hier könnte man anmerken, dass die Adressen unterschiedlich berechnet werden für Stack und Globale...

\begin{code}
  \centering
  \numberedcodebox[minted language=text, minted options={highlightlines={13-19}}]{./code_examples/example_fun_call_by_value_struct.reti_blocks}
  \caption{RETI Block Pass für Call by Value für Structs}
  \label{code:reti_blocks_pass_für_call_by_value_for_structs}
\end{code}

% Struct wird wirklich kopiert durch speziellen Argmode

\subsection{Umsetzung kleinerer Details}
% langen Sprüngen, großen Konstanten, Division durch 0
\section{Fehlermeldungen}
\subsection{Error Handler}
\subsection{Arten von Fehlermeldungen}
\subsubsection{Syntaxfehler}
\subsubsection{Laufzeitfehler}
% Fehlermeldung ist, wenn der Lexer (partielle Funktion) oder Parser nicht matcht
% Token und Nodes enthalten Position, im Transformer wird die Position von den Token auf die Nodes übertragen und auch die Symboltabelle speichert Position
