%!Tex Root = ../Main.tex
% ./Packete_und_Deklarationen.tex
% ./Titlepage.tex
% ./Motivation.tex
% ./Einführung.tex
% ./Implementierung1_Tables_DT_AST.tex,
% ./Implementierung3_Struct_Derived.tex,
% ./Implementierung4_Fun.tex,
% ./Ergebnisse_und_Ausblick.tex

\subsection{Umsetzung von Pointern}
\label{sec:umsetzung_von_pointern}
\subsubsection{Referenzierung}
Die \colorbold{Referenzierung} (z.B. \verb|&var|) wird im Folgenden anhand des Beispiels in Code~\ref{code:picoc_code_für_pointer_referenzierung} erklärt.

\begin{code}
  \centering
  \numberedcodebox[minted language=c, minted options={highlightlines={3}}]{./code_examples/example_pntr_ref.picoc}
  \caption{PicoC-Code für Pointer Referenzierung}
  \label{code:picoc_code_für_pointer_referenzierung}
\end{code}

Der Knoten \smalltt{Ref(Name('var')))} repräsentiert im \colorbold{Abstract Syntax Tree} in Code~\ref{code:abstract_syntax_tree_für_pointer_referenzierung} eine \colorbold{Referenzierung} \verb|&var| und der Knoten \smalltt{PntrDecl(Num('1'), IntType('int'))} repräsentiert einen Pointer \smalltt{*pntr}.

\begin{code}
  \centering
  \numberedcodebox[minted language=text, minted options={highlightlines={10}}]{./code_examples/example_pntr_ref.ast}
  \caption{Abstract Syntax Tree für Pointer Referenzierung}
  \label{code:abstract_syntax_tree_für_pointer_referenzierung}
\end{code}

Bevor man einem \colorbold{Pointer} eine eine \colorbold{Adresse} (z.B. \smalltt{\&var}) zuweisen kann, muss dieser erstmal \colorbold{definiert} sein. Dafür braucht es einen Eintrag in der \colorbold{Symboltabelle} in Code~\ref{code:symboltabelle_für_pointer_referenzierung}.

\begin{Special_Paragraph}
Die \colorbold{Größe} eines Pointers (z.B. eines Pointers auf ein Array von \smalltt{int}: \smalltt{pntr = int *pntr[3]}), die ihm \smalltt{size}-Feld der \colorbold{Symboltabelle} eingetragen ist, ist dabei immer: $\mathtt{size(pntr) = 1}$.
\end{Special_Paragraph}

\begin{code}
  \centering
  \numberedcodebox[minted language=text, minted options={highlightlines={23-28}}]{./code_examples/example_pntr_ref.st}
  \caption{Symboltabelle für Pointer Referenzierung}
  \label{code:symboltabelle_für_pointer_referenzierung}
\end{code}

Im \colorbold{PicoC-Mon Pass} in Code~\ref{code:picoc_mon_für_pointer_referenzierung} wird der Knoten \smalltt{Ref(Name('var')))} durch die Knoten \smalltt{Ref(GlobalRead(Num('0')))} und \smalltt{Assign(GlobalWrite(Num('1')), Tmp(Num('1')))} ersetzt. Im Fall, dass in \smalltt{Ref(exp))} das \smalltt{exp} vielleicht nicht direkt ein \smalltt{Name('var')} enthält und \smalltt{exp} z.B. ein \smalltt{Subscr(Attr(Name('var')))} ist, sind noch weitere Anweisungen zwischen den Zeilen \smalltt{11} und  \smalltt{12} nötig, die sich in diesem Beispiel um das Übersetzen von \smalltt{Subscr(exp)} und \smalltt{Attr(exp)} nach dem Schema in Subkapitel~\ref{sec:mittelteil_für_die_verschiedenen_derived_datatypes} kümmern.

\begin{code}
  \centering
  \numberedcodebox[minted language=text, minted options={highlightlines={11-12}}]{./code_examples/example_pntr_ref.picoc_mon}
  \caption{PicoC-Mon Pass für Pointer Referenzierung}
  \label{code:picoc_mon_für_pointer_referenzierung}
\end{code}

Im \colorbold{RETI-Blocks Pass} in Code~\ref{code:reti_blocks_für_pointer_referenzierung} werden die \colorbold{PicoC-Knoten} \smalltt{ Ref(Global(Num('0')))} und \smalltt{Assign(Global(Num('1')), Stack(Num('1')))} durch ihre entsprechenden \colorbold{RETI-Knoten} ersetzt.

\begin{code}
  \centering
  \numberedcodebox[minted language=text, minted options={highlightlines={18-21,23-25}}]{./code_examples/example_pntr_ref.reti_blocks}
  \caption{RETI-Blocks Pass für Pointer Referenzierung}
  \label{code:reti_blocks_für_pointer_referenzierung}
\end{code}
% Initialisierung eines Pointers
\subsubsection{Dereferenzierung durch Zugriff auf Arrayindex ersetzen}
Die \colorbold{Dereferenzierung} (z.B. \smalltt{*var}) wird im Folgenden anhand des Beispiels in Code~\ref{code:picoc_code_für_pointer_dereferenzierung} erklärt.

\begin{code}
  \centering
  \numberedcodebox[minted language=c, minted options={highlightlines={4}}]{./code_examples/example_pntr_deref.picoc}
  \caption{PicoC-Code für Pointer Dereferenzierung}
  \label{code:picoc_code_für_pointer_dereferenzierung}
\end{code}

Der Container-Knoten \smalltt{Deref(Name('var'), Num('0')))} repräsentiert im \colorbold{Abstract Syntax Tree} in Code~\ref{code:abstract_syntax_tree_für_pointer_dereferenzierung} eine \colorbold{Dereferenzierung} \smalltt{*var}. Es gibt herbei \colorbold{zwei} Fälle. Bei der Anwendung von \colorbold{Pointer Arithmetik}, wie z.B. \smalltt{*(var + 2 - 1)} übersetzt sich diese zu \smalltt{Deref(Name('var'), BinOp(Num('2'), Sub(), BinOp(Num('1'))))}. Bei einer normalen \colorbold{Dereferenzierung}, wie z.B. \smalltt{*var}, übersetzt sich diese zu \smalltt{Deref(Name('var'), Num('0'))}.

\begin{code}
  \centering
  \numberedcodebox[minted language=text, minted options={highlightlines={11}}]{./code_examples/example_pntr_deref.ast}
  \caption{Abstract Syntax Tree für Pointer Dereferenzierung}
  \label{code:abstract_syntax_tree_für_pointer_dereferenzierung}
\end{code}

Im \colorbold{PicoC-Shrink Pass} in Code~\ref{code:picoc_shrink_für_pointer_dereferenzierung} wird ein Trick angewandet, bei dem jeder Knoten \smalltt{Deref(Name('pntr'), Num('0'))} einfach durch den Knoten \smalltt{Subscr(Name('pntr'), Num('0'))} ersetzt wird. Der Trick besteht darin, dass der \colorbold{Dereferenzoperator} (z.B. \smalltt{*(var + 1)}) sich identisch zum \colorbold{Operator für den Zugriff auf einen Arrayindex} (z.B. \smalltt{var[1]}) verhält\footnote{In der Sprache $L_{C}$ gibt es einen Unterschied bei der Initialisierung bei z.B. \smalltt{int *var = \dq string\dq} und z.B. \smalltt{int var[1] = \dq string\dq}, der allerdings nichts mit den beiden Operatoren zu tuen hat, sondern mit der \colorbold{Initialisierung}, bei der die Sprache $L_{C}$ verwirrenderweise die eckigen Klammern \smalltt{[]} genauso, wie beim \colorbold{Operator für den Zugriff auf einen Arrayindex}, vor den Bezeichner schreibt (z.B. \smalltt{var[1]}), obwohl es ein \colorbold{Derived Datatype} ist.}. Damit sparrt man sich viele vermeidbare \colorbold{Fallunterscheidungen} und \colorbold{doppelten Code} und kann die \colorbold{Derefenzierung} (z.B. \smalltt{*(var + 1)}) einfach von den Routinen für einen \colorbold{Zugriff auf einen Arrayindex} (z.B. \smalltt{var[1]}) übernehmen lassen.

\begin{code}
  \centering
  \numberedcodebox[minted language=text, minted options={highlightlines={11}}]{./code_examples/example_pntr_deref.picoc_shrink}
  \caption{PicoC-Shrink Pass für Pointer Dereferenzierung}
  \label{code:picoc_shrink_für_pointer_dereferenzierung}
\end{code}

\subsection{Umsetzung von Arrays}
\label{sec:umsetzung_von_arrays}
\subsubsection{Initialisierung von Arrays}
\label{sec:initialisierung_von_arrays}

Die \colorbold{Initialisierung} eines \colorbold{Arrays} (z.B. \smalltt{int ar[2][1] = \{\{3+1\}, \{4\}\}}) wird im Folgenden anhand des Beispiels in Code~\ref{code:picoc_code_für_array_initialisierung} erklärt.

% Stack und Globale Statische Daten
\begin{code}
  \centering
  \numberedcodebox[minted language=c, minted options={highlightlines={2, 6}}]{./code_examples/example_array_init.picoc}
  \caption{PicoC-Code für Array Initialisierung}
  \label{code:picoc_code_für_array_initialisierung}
\end{code}

Die \colorbold{Initialisierung} eines \colorbold{Arrays} \smalltt{int ar[2][1] = \{\{3+1\}, \{4\}\}} wird im \colorbold{Abstract Syntax Tree} in Code~\ref{code:abstract_syntax_tree_für_array_initialisierung} mithilfe der Komposition \smalltt{Assign(Alloc(Writeable(), ArrayDecl([Num('2'), Num('1')], IntType('int')), Name('ar')), Array([Array([BinOp(Num('3'), Add('+'), Num('1'))]), Array([Num('4')])]))} dargestellt.

\begin{code}
  \centering
  \numberedcodebox[minted language=text, minted options={highlightlines={9, 16}}]{./code_examples/example_array_init.ast}
  \caption{Abstract Syntax Tree für Array Initialisierung}
  \label{code:abstract_syntax_tree_für_array_initialisierung}
\end{code}

Bei der \colorbold{Initialisierung} eines \colorbold{Arrays} wird zuerst \smalltt{Alloc(Writeable(), ArrayDecl([Num('2'), Num('1')], IntType('int')))} ausgewertet, da eine Variable zuerst definiert sein muss, bevor man sie verwenden kann\footnote{Das Widerspricht der üblichen Auswertungsreihenfolge beim \colorbold{Zuweisungsoperator} \smalltt{=}, der \colorbold{rechtsassoziativ} ist. Der \colorbold{Zuweisungsoperator} \smalltt{=} tritt allerdings erst später in Aktion.}. Das \colorbold{Definieren} der Variable \smalltt{ar} erfolgt mittels der \colorbold{Symboltabelle}, die in Code~\ref{code:symboltabelle_für_array_initialisierung} dargestellt ist.

Bei Variablen auf dem \colorbold{Stackframe} wird ein Array \colorbold{rückwärts} auf das Stackframe geschrieben und auch die \colorbold{Adresse des ersten Elements} als Adresse des Arrays genommen. Dies macht den \colorbold{Zugriff auf einen Arrayindex} in Subkapitel~\ref{sec:zugriff_auf_arrayindex} deutlich unkomplizierter, da man so nicht mehr zwischen \colorbold{Stackframe} und \colorbold{Globalen Statischen Daten} beim \colorbold{Zugriff auf einen Arrayindex} unterscheiden muss, da es Probleme macht, dass ein \colorbold{Stackframe} in die entgegengesetzte Richtung wächst, verglichen mit den \colorbold{Globalen Statischen Daten}\footnote{Wenn man beim \colorbold{GCC}~\cite{noauthor_gcc_nodate} einen Stackframe mittels des \colorbold{GDB}~\cite{noauthor_gcc_nodate} beobachtet, sieht man, dass dieser es genauso macht.}.

\begin{Special_Paragraph}
  Das \colorbold{Größe} des Arrays $\mathtt{datatype \enspace ar[dim_1]\ldots[dim_k]}$, die ihm \smalltt{size}-Feld des \colorbold{Symboltabelleneintrags} eingetragen ist, berechnet sich dabei aus der \colorbold{Mächtigkeit} der einzelnen \colorbold{Dimensionen} des Arrays multipliziert mit der \colorbold{Größe} des \colorbold{grundlegenden Datentyps} der einzelnen \colorbold{Arrayelemente}: $\mathtt{size(datatype(ar)) = \left(\prod^n_{j=1} dim_j\right)\cdot size(datatype)}$\footnote{Die \colorbold{Funktion}  \smalltt{type} ordnet einer  \colorbold{Variable} ihren \colorbold{Datentyp} zu. Das ist notwendig, weil die \colorbold{Funktion} \smalltt{size} nur bei einem \colorbold{Datentyp} als \colorbold{Funktionsargument} die \colorbold{Größe dieses Datentyps} als \colorbold{Zielwert} liefert}.
\end{Special_Paragraph}

\begin{code}
  \centering
  \numberedcodebox[minted language=text, minted options={highlightlines={14-19,32-37}}]{./code_examples/example_array_init.st}
  \caption{Symboltabelle für Array Initialisierung}
  \label{code:symboltabelle_für_array_initialisierung}
\end{code}

Im \colorbold{PiocC-Mon Pass} in Code~\ref{code:picoc_mon_für_array_initialisierung} werden zuerst die \colorbold{Logischen Ausdrücke} in den Blättern des Teilbaums, der beim \colorbold{Array-Initializers} \colorbold{Container-Knoten} \smalltt{Array([Array([BinOp(Num('3'), Add('+'), Num('1'))]), Array([Num('4')])])} anfängt nach dem \colorbold{Depth-First-Search} Schema, von \colorbold{links-nach-rechts} ausgewertet und auf den \colorbold{Stack} geschrieben\footnote{Da der \colorbold{Zuweisungsoperator} \smalltt{=} \colorbold{rechtsassoziativ} ist und auch rein \colorbold{logisch}, weil man nichts zuweisen kann, was man noch nicht berechnet hat.}.

Im finalen Schritt muss zwischen \colorbold{Globalen Statischen Daten} bei der \smalltt{main}-Funktion und \colorbold{Stackframe} bei der Funktion \smalltt{fun} unterschieden werden. Die auf den Stack ausgewerteten Expressions werden mittels der Komposition \smalltt{Assign(Global(Num('0')), Stack(Num('2')))} bzw. \smalltt{Assign(Stackframe(Num('3')), Stack(Num('4')))}, die in Tabelle~\ref{tab:kompositionen_von_picoc_knoten_und_reti_knoten_mit_besonderer_bedeutung} genauer beschrieben ist, versetzt in der selben Reihenfolge zu den \colorbold{Globalen Statischen Daten} bzw. auf den \colorbold{Stackframe} geschrieben.

Der \colorbold{Trick} ist hier, dass egal wieviele Dimensionen und was für einen Datentyp das \colorbold{Array} hat, man letztendlich immer das gesamte Array erwischt, wenn man einfach die \colorbold{Größe des Arrays} viele \colorbold{Speicherzellen} mit z.B. der \colorbold{Komposition} \smalltt{Assign(Global(Num('0')), Stack(Num('2')))} verschiebt.

In die Knoten \smalltt{Global('0')} und  \smalltt{Stackframe('3')} wurde hierbei die \colorbold{Startadresse} des jeweiligen Arrays geschrieben, sodass man nach dem \colorbold{PicoC-Mon Pass} nie mehr Variablen in der  \colorbold{Symboltabelle} nachsehen muss und gleich weiß, ob sie in Bezug zu den \colorbold{Globalen Statischen Daten} oder dem \colorbold{Stackframe} stehen.

\begin{code}
  \centering
  \numberedcodebox[minted language=text, minted options={highlightlines={8-12,19-23}}]{./code_examples/example_array_init.picoc_mon}
  \caption{PicoC-Mon Pass für Array Initialisierung}
  \label{code:picoc_mon_für_array_initialisierung}
\end{code}

Im \colorbold{RETI-Blocks Pass} in Code~\ref{code:reti_blocks_für_array_initialisierung} werden die \colorbold{Kompositionen} \smalltt{Exp(exp)} und \smalltt{Assign(Global(Num('0')), Stack(Num('2')))} bzw. \smalltt{Assign(Stackframe(Num('3')), Stack(Num('4')))} durch ihre entsprechenden \colorbold{RETI-Knoten} ersetzt.

\begin{code}
  \centering
  \numberedcodebox[minted language=text, minted options={highlightlines={9-11,13-15,17-21,23-25,27-31,40-42,44-46,48-50,52-54,56-64}}]{./code_examples/example_array_init.reti_blocks}
  \caption{RETI-Blocks Pass für Array Initialisierung}
  \label{code:reti_blocks_für_array_initialisierung}
\end{code}


% kleines Extra
\subsubsection{Zugriff auf einen Arrayindex}
\label{sec:zugriff_auf_arrayindex}

Der \colorbold{Zugriff auf einen Arrayindex} (z.B. \smalltt{ar[0]}) wird im Folgenden anhand des Beispiels in Code~\ref{code:picoc_code_für_zugriff_auf_arrayindex} erklärt.

\begin{code}
  \centering
  \numberedcodebox[minted language=c, minted options={highlightlines={3,8}}]{./code_examples/example_array_access.picoc}
  \caption{PicoC-Code für Zugriff auf einen Arrayindex}
  \label{code:picoc_code_für_zugriff_auf_arrayindex}
\end{code}

Der \colorbold{Zugriff auf einen Arrayindex} \smalltt{ar[0]} wird im  \colorbold{Abstract Syntax Tree} in Code~\ref{code:abstract_syntax_tree_für_zugriff_auf_arrayindex} mithilfe des \colorbold{Container-Knotens} \smalltt{Subscr(Name('ar'), Num('0'))} dargestellt.

\begin{code}
  \centering
  \numberedcodebox[minted language=text, minted options={highlightlines={10,18}}]{./code_examples/example_array_access.ast}
  \caption{Abstract Syntax Tree für Zugriff auf einen Arrayindex}
  \label{code:abstract_syntax_tree_für_zugriff_auf_arrayindex}
\end{code}

Im \colorbold{PicoC-Mon Pass} in Code~\ref{code:picoc_mon_für_zugriff_auf_arrayindex} wird vom \colorbold{Container-Knoten} \smalltt{Subscr(Name('ar'), Num('0'))} zuerst im \colorbold{Anfangsteil}~\ref{sec:einleitungsteil_für_globale_statische_daten_und_stackframe} die \colorbold{Adresse} der Variable \smalltt{Name('ar')} auf den \colorbold{Stack} geschrieben. Bei den \colorbold{Globalen Statischen Daten} der \smalltt{main}-Funktion wird das durch die Komposition \smalltt{Ref(Global(Num('0')))} dargestellt und beim \colorbold{Stackframe} der Funktionm \smalltt{fun} wird das durch die Komposition \smalltt{Ref(Stackframe(Num('2')))} dargestellt.

In nächsten Schritt, dem \colorbold{Mittelteil}~\ref{sec:mittelteil_für_die_verschiedenen_derived_datatypes} wird die \colorbold{Adresse} ab der das \colorbold{Arrayelement} des Arrays auf das Zugegriffen werden soll anfängt berechnet. Dabei wurde im \colorbold{Anfangsteil} bereits die \colorbold{Anfangsadresse} des Arrays, in dem dieses \colorbold{Arrayelement} liegt auf den \colorbold{Stack} gelegt. Da ein \colorbold{Index} auf den Zugegriffen werden soll auch durch das Ergebnis eines \colorbold{komplexeren Ausdrucks}, z.B. \smalltt{ar[1 + var]} bestimmt sein kann, indem auch \colorbold{Variablen} vorkommen können, kann dieser nicht während des \colorbold{Kompilierens} berechnet werden, sondern muss zur \colorbold{Laufzeit} berechnet werden.

Daher muss zuerst der Wert des \colorbold{Index}, dessen Adresse berechnet werden soll bestimmt werden, z.B. im einfachen Fall durch \smalltt{Exp(Num('0'))} und dann muss die \colorbold{Adresse des Index} berechnet werden, was durch die Komposition \smalltt{Ref(Subscr(Stack(Num('2')), Stack(Num('1'))))} dargestellt wird. Die Bedeutung der Komposition \smalltt{\smalltt{Ref(Subscr(Stack(Num('2')), Stack(Num('1'))))}} ist in Tabelle~\ref{tab:kompositionen_von_picoc_knoten_und_reti_knoten_mit_besonderer_bedeutung} dokumentiert.

Zur \colorbold{Adressberechnung} ist es notwendig auf die \colorbold{Dimensionen} (z.B. \smalltt{[Num('3')]}) des Arrays, auf dessen \colorbold{Arrayelement} zugegriffen wird, zugreifen zu können. Daher ist der \colorbold{Arraydatentyp} (z.B. \smalltt{ArrayDecl([Num('3')], IntType('int'))}) dem \colorbold{Container-Knoten} \smalltt{Ref(exp, \textcolor{gray!90!black}{datatype})} als \textcolor{gray!90!black}{verstecktes Attribut} \smalltt{datatype} angehängt. Das \textcolor{gray!90!black}{versteckte Attribut} wird während des Kompiliervorgangs im \colorbold{PiocC-Mon Pass} dem \colorbold{Container-Knoten} \smalltt{Ref(exp, \textcolor{gray!90!black}{datatype})} angehängt.

Je nachdem, ob mehrere \smalltt{Subscr(exp, exp)} eine Komposition bilden (z.B. \smalltt{Subscr(Subscr(Name('var'), Num('1')), Num('1'))}) ist es notwendig mehrere \colorbold{Adressberechnungsschritte für den Index} \smalltt{Ref(Subscr(Stack(Num('2')), Stack(Num('1'))))} einzuleiten und es muss auch möglich sein, z.B. einen \colorbold{Attributzugriff} \smalltt{var.attr} und eine \colorbold{Zugriff auf einen Arryindex} \smalltt{var[1]} miteinander zu kombinieren, was in Subkapitel~\ref{sec:mittelteil_für_die_verschiedenen_derived_datatypes} allgemein erklärt ist.

Im letzten Schritt, dem \colorbold{Schlussteil}~\ref{sec:schlussteil_für_die_verschiedenen_derived_datatypes} wird der \colorbold{Inhalt} des \colorbold{Index}, dessen \colorbold{Adresse} in den vorherigen Schritten berechnet wurde, nun auf den \colorbold{Stack} geschrieben, wobei dieser die \colorbold{Adresse} auf dem Stack ersetzt, die es zum Finden des \colorbold{Index} brauchte. Dies wird durch den Knoten \smalltt{Exp(Stack(Num('1')))} dargestellt. Je nachdem, welchen \colorbold{Datentyp} die Variable \smalltt{ar} hat und auf welchen \colorbold{Unterdatentyp} folglich im \colorbold{Kontext} zuletzt zugegriffen wird, abhängig davon wird der \colorbold{Schlussteil} \smalltt{Exp(Stack(Num('1')))} auf eine andere Weise verarbeitet (siehe Subkapitel~\ref{sec:schlussteil_für_die_verschiedenen_derived_datatypes}). Der \colorbold{Unterdatentyp} ist dabei ein \textcolor{gray!90!black}{verstecktes Attribut} des \smalltt{Exp(Stack(Num('1')))}-Knoten.

Der einzige \colorbold{Unterschied}, je nachdem, ob der \colorbold{Zugriff auf einen Arrayindex} (z.B. \smalltt{ar[1]}) in der  \smalltt{main}-Funktion oder der Funktion \smalltt{fun} erfolgt, ist eigentlich nur beim \colorbold{Anfangsteil}, beim Schreiben der \colorbold{Adresse} der Variable \smalltt{ar} auf den \colorbold{Stack} zu finden, bei dem unterschiedliche \colorbold{RETI-Instructions} für eine Variable, die in den \colorbold{Globalen Statischen Daten} liegt und eine Variable, die auf dem \colorbold{Stackframe} liegt erzeugt werden müssen.

\begin{Special_Paragraph}
  Die Berechnung der \colorbold{Adresse}, ab der ein \colorbold{Arrayelement} eines Arrays $\mathtt{datatype\enspace ar[dim_1]\ldots[dim_n]}$ abgespeichert ist, kann mittels der Formel~\ref{eq:adresse_von_arrayelement}:

  \numberwithin{equation}{section}

  \begin{equation}
  \mathtt{ref(ar[idx_1]\ldots[idx_n]) = ref(ar) + \left(\sum_{i=1}^{n}\left(\prod_{j=i+1}^{n} dim_{j}\right) \cdot idx_{i}\right) \cdot \operatorname{size}(datatype)}
    \label{eq:adresse_von_arrayelement}
  \end{equation}
  aus der Betriebssysteme Vorlesung\footcite{scholl_betriebssysteme_2020} berechnet werden\footnote{\smalltt{ref(exp)} steht dabei für die Berechnung der \colorbold{Adresse} von \smalltt{exp}, wobei \smalltt{exp} z.B. \smalltt{ar[3][2]} sein könnte.}.

  Die Komposition \smalltt{Ref(Global(num))} bzw. \smalltt{Ref(Stackframe(num))} repräsentiert dabei den Summanden $\smalltt{ref(ar)}$ in der Formel.

  Die Komposition \smalltt{Exp(num)} repräsentiert dabei einen \colorbold{Subindex} (z.B. \smalltt{i} in \smalltt{a[i][j][k]}) beim \colorbold{Zugriff auf ein Arrayelement}, der als Faktor $\mathtt{idx_i}$ in der Formel auftaucht.

  Der Komposition \smalltt{Ref(Subscr(Stack(Num('2')), Stack(Num('1'))))} repräsentiert dabei einen ausmultiplizierten Summanden $\mathtt{\left(\prod_{j=i+1}^{n} dim_{j}\right) \cdot idx_{i} \cdot size(datatpye)}$ in der Formel.

Die Komposition \smalltt{Exp(Stack(Num('1')))} repräsentiert dabei das Lesen des \colorbold{Inhalts} $\mathtt{M\left[ref(ar[idx_1]\ldots[idx_n])\right]}$ der Speicherzelle an der finalen \colorbold{Adresse}  $\mathtt{ref(ar[idx_1]\ldots[idx_n])}$.
\end{Special_Paragraph}

\begin{code}
  \centering
  \numberedcodebox[minted language=text, minted options={highlightlines={11-14,26}}]{./code_examples/example_array_access.picoc_mon}
  \caption{PicoC-Mon Pass für Zugriff auf einen Arrayindex}
  \label{code:picoc_mon_für_zugriff_auf_arrayindex}
\end{code}

Im \colorbold{RETI-Blocks Pass} in Code~\ref{code:reti_blocks_für_zugriff_auf_arrayindex} werden die \colorbold{Kompositionen} \smalltt{Ref(Global(Num('0')))}, \smalltt{Ref(Subscr(Stack(Num('2')) und Stack(Num('1'))))} durch ihre entsprechenden \colorbold{RETI-Knoten} ersetzt.

\begin{code}
  \centering
  \numberedcodebox[minted language=text, minted options={highlightlines={18-21,23-25,27-32,34-36,66-69}}]{./code_examples/example_array_access.reti_blocks}
  \caption{RETI-Blocks Pass für Zugriff auf einen Arrayindex}
  \label{code:reti_blocks_für_zugriff_auf_arrayindex}
\end{code}

\subsubsection{Zuweisung an Arrayindex}
\label{sec:zuweisung_an_arrayindex}
% Formel aus der Vorlesung, wo ist die hier?

Die \colorbold{Zuweisung} eines Wertes an einen \colorbold{Arrayindex} (z.B. \smalltt{ar[2] = 42;}) wird im Folgenden anhand des Beispiels in Code~\ref{code:picoc_code_für_array_assignment} erläutert.

\begin{code}
  \centering
  \numberedcodebox[minted language=c, minted options={highlightlines={3}}]{./code_examples/example_array_assignment.picoc}
  \caption{PicoC-Code für Zuweisung an Arrayindex}
  \label{code:picoc_code_für_array_assignment}
\end{code}

Im \colorbold{Abstract Syntax Tree} in Code~\ref{code:abstract_syntax_tree_für_array_assignment} wird eine \colorbold{Zuweisung} an einen \colorbold{Arrayindex} \smalltt{ar[2] = 42;} durch die Komposition \smalltt{Assign(Subscr(Name('ar'), Num('2')), Num('42'))} dargestellt.

\begin{code}
  \centering
  \numberedcodebox[minted language=text, minted options={highlightlines={10}}]{./code_examples/example_array_assignment.ast}
  \caption{Abstract Syntax Tree für Zuweisung an Arrayindex}
  \label{code:abstract_syntax_tree_für_array_assignment}
\end{code}

Im \colorbold{PicoC-Mon Pass} in Code~\ref{code:picoc_mon_für_array_assignment} wird zuerst die \colorbold{rechte} Seite des \colorbold{rechtsassoziativen} Zuweisungsoperators \smalltt{=}, bzw. des \colorbold{Container-Knotens} der diesen darstellt ausgewertet: \smalltt{Exp(Num('42'))}.

Danach ist das Vorgehen, bzw. sind die Kompostionen, die dieses darauffolgende Vorgehen darstellen: \smalltt{Ref(Global(Num('0')))}, \smalltt{Exp(Num('2'))} und \smalltt{Ref(Subscr(Stack(Num('2')), Stack(Num('1'))))} identisch zum \colorbold{Anfangsteil} und \colorbold{Mittelteil} aus dem vorherigen Subkapitel~\ref{sec:zugriff_auf_arrayindex}. Es wird die \colorbold{Adresse} des \colorbold{Index}, dem das Ergebnis der Ausdrucks auf der rechten Seite des \colorbold{Zuweisungsoperators} \smalltt{=} zugewiesen wird berechet, wie in Subkapitel~\ref{sec:zugriff_auf_arrayindex}.

Zum Schluss stellt die \colorbold{Komposition} \smalltt{Assign(Stack(Num('1')), Stack(Num('2')))}\footnote{Ist in Tabelle~\ref{tab:kompositionen_von_picoc_knoten_und_reti_knoten_mit_besonderer_bedeutung} genauer beschrieben ist} die Zuweisung \smalltt{=} des Ergebnisses des Ausdrucks auf der \colorbold{rechten} Seite der Zuweisung zum \colorbold{Arrayindex}, dessen \colorbold{Adresse} im Schritt danach berechnet wurde dar.

\begin{code}
  \centering
  \numberedcodebox[minted language=text, minted options={highlightlines={9-13}}]{./code_examples/example_array_assignment.picoc_mon}
  \caption{PicoC-Mon Pass für Zuweisung an Arrayindex}
  \label{code:picoc_mon_für_array_assignment}
\end{code}

Im \colorbold{RETI-Blocks Pass} in Code~\ref{code:reti_blocks_für_array_assignment} werden die \colorbold{Kompositionen} \smalltt{Ref(Global(Num('0')))}, \smalltt{Ref(Subscr(Stack(Num('2')), Stack(Num('1'))))} und \smalltt{Assign(Stack(Num('1')), Stack(Num('2')))} durch ihre entsprechenden \colorbold{RETI-Knoten} ersetzt.

\begin{code}
  \centering
  \numberedcodebox[minted language=text, minted options={highlightlines={10-12,14-17,19-21,23-28,30-33}}]{./code_examples/example_array_assignment.reti_blocks}
  \caption{RETI-Blocks Pass für Zuweisung an Arrayindex}
  \label{code:reti_blocks_für_array_assignment}
\end{code}
