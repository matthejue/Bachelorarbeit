%!Tex Root = ../Main.tex
% ./Packete_und_Deklarationen.tex
% ./Titlepage.tex
% ./Einführung.tex
% ./Implementierung1.tex
% ./Implementierung2.tex
% ./Ergebnisse_und_Ausblick.tex

\chapter{Motivation}
\label{ch:motivation}

\section{RETI}
... basiert auf ... der Vorlesung~\cite{scholl_betriebssysteme_2020}.

% erweitern des Compilers aus dem Bachelorprojekt
\begin{Definition}{Caller-save Register}{caller_save_register}
  \footcite{thiemann_compilerbau_2021}
\end{Definition}

\begin{Definition}{Callee-save Register}{callee_save_register}
  \footcite{thiemann_compilerbau_2021}
\end{Definition}

\section{PicoC}
\section{Aufgabenstellung}
\section{Eigenheiten der Sprache C}
% Abhängigkeit von Typ von Variable usw. auch bei unvollständigem Ende von Datentyp
% Designfehler
% Spiral Rule
% Specifier, die ganzen Begriffe aus calcourse

\begin{Definition}{Deklaration}{deklaration}
  \footcite{scholl_einfuhrung_2021}
\end{Definition}

\begin{Definition}{Definition}{definition}
  \footcite{scholl_einfuhrung_2021}
\end{Definition}

\begin{Definition}{Call by value}{call_by_value}
  % call by sharing
  \footcite{bast_programmieren_2020}
\end{Definition}

\begin{Definition}{Call by reference}{call_by_reference}
  \footcite{bast_programmieren_2020}
\end{Definition}


\section{Richtlinien}
% Email an Scholl
% exakt gleiche Präzidenzregeln
% kein unnötiger Schnickschnack
